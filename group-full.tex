\documentclass{article}
\usepackage{CJKutf8}
\usepackage{amsmath}
\usepackage{amssymb}
\usepackage{amsfonts}
\usepackage{amsthm}
\usepackage{titlesec}
\usepackage{titletoc}
\usepackage{xCJKnumb}
\usepackage{tikz}
\usepackage{mathrsfs}
\usepackage{indentfirst}

\newtheorem{Def}{定义}
\newtheorem{Thm}{定理}
\newtheorem{Exercise}{练习}

\newtheorem*{Example}{例}


\begin{document}
\begin{CJK*}{UTF8}{gbsn}
  \title{第三讲 群的简单性质}
  \author{陈建文}
  \maketitle
  % \tableofcontents
  

\begin{Def}
  设$G$为一个非空集合,“$\circ$”为$G$上的一个二元代数运算。如果下列各个条件成立,则称$G$对“$\circ$”运算构成一个群(group):

  I. “$\circ$”运算满足结合律,即$\forall a,b,c \in G$ $(a\circ b)\circ c = a\circ(b\circ c)$;

  II. 对“$\circ$”运算,$G$中有一个左单位元$e$,即$\forall a\in G$ $e\circ a = a$;

  III. 对$G$中的每个元素,关于$\circ$运算有一个左逆元,
  即$\forall a\in G \exists b\in G b\circ a = e$,其中$e$为II中的同一个左单位元素。
\end{Def}
在群$(G,\circ)$中,$\forall a,b\in G, a\circ b$简写为$ab$。
\begin{Thm}
  设$G$为一个群,则$\forall a\in G$,$a$的左逆元也是$a$的右逆元。
\end{Thm}
\begin{proof}[证明]
  $\forall a\in G$,设$a_l$为$a$的一个左逆元,则
  \[a_la=e\]
  两边同时右乘以$a_l$得
  \[(a_la)a_l=ea_l\]
  从而
  \[a_l(aa_l)=a_l\]
  两边同时左乘以$a_l$的左逆元得
  \[aa_l=e\]
\end{proof}
\begin{Thm}
  设$G$为一个群,则$G$的左单位元$e$也是右单位元。
\end{Thm}
\begin{proof}[证明]
  $\forall a\in G$,设$a_l$为$a$的左逆元,则$ae=a(a_la)=(aa_l)a=ea=a$,所以$e$也是右单位元。
\end{proof}
\begin{Thm}
  设$a$与$b$为群$G$的任意两个元素,则$(a^{-1})^{-1}=a$,$(ab)^{-1}=b^{-1}a^{-1}$。
\end{Thm}
\begin{proof}[证明]
  由\[aa^{-1}=e\]
  得\[(a^{-1})^{-1}=a\]
由\[(b^{-1}a^{-1})(ab)=b^{-1}(a^{-1}a)b=b^{-1}eb=e\]
得\[(ab)^{-1}=b^{-1}a^{-1}\]
\end{proof}
\begin{Thm}
 在群$G$中,$\forall a, b \in G$,方程
 \begin{align*}
  ax&=b\\
  ya&=b
 \end{align*}
 关于未知量$x$与$y$都有唯一解。 
\end{Thm}

\begin{Thm}
  非空集合$G$对其二元代数运算$\circ$构成一个群的充分必要条件是下列两个条件同时成立:

  1. “$\circ$”运算满足结合律,即$\forall a,b,c\in G (a\circ b)\circ c=a\circ(b\circ c)$。

  2. $\forall a,b\in G$,方程
  \begin{align*}
    ax&=b\\
    ya&=b
   \end{align*}
   关于未知量$x$与$y$有解。
\end{Thm}
\begin{proof}[证明]

  $\Leftarrow$

  由$G$非空知$\exists b,b\in G$。方程$yb=b$有解,设$e$为一个解,则$eb=b$。$\forall a\in G$,方程$bx=a$有解,设$c$为一个解,则$bc=a$。
  于是
  \[ea=e(bc)=(eb)c=bc=a\]
  从而$e$为左单位元。

  $\forall a\in G$,方程$ya=e$有解,其解为$a$的左逆元。
\end{proof}
\begin{Thm}
  设$(G,\circ)$为一个群,则“$\circ$”运算满足消去律,即$\forall x, y, a\in G$,

  如果$ax = ay$,则$x=y$(左消去律)

  如果$xa = ya$, 则$x=y$(右消去律)
\end{Thm}

\begin{Thm}
  非空有限集合$G$对在其上定义的二元代数运算$\circ$构成一个群的充分必要条件是下列两个条件同时成立:

  1. “$\circ$”运算满足结合律。

  2. “$\circ$”运算满足左、右消去律。
\end{Thm}
\begin{proof}[证明]
$\Leftarrow$

先证$\forall a,b\in G$,方程$ax=b$有解。

令$f:G\to aG=\{ag|g\in G\}$,$\forall x\in G, f(x)=ax$。则$f$为单射,这是因为$\forall x_1,x_2\in G$,如果$f(x_1)=f(x_2)$,则$ax_1=ax_2$,由左消去律得$x_1=x_2$;
同时,$f$为满射,这是因为$\forall y\in aG$,$\exists x\in G$,$y=ax$,于是$f(x)=ax=y$。
此时必有$aG=G$,否则$aG\subseteq G$且$aG\neq G$,从而$aG$为$G$的真子集,于是$f$为有限集$G$与其真子集之间的一个双射,矛盾。
由$f:G\to aG=G$为双射知,$\forall b\in G$,$\exists c\in G$,$ac=b$。所以,方程$ax=b$在$G$中有解。

同理可证,  $\forall a,b\in G$,方程$ya=b$有解。
\end{proof}

\begin{Example}
  3阶群是交换群。
\end{Example}
\begin{Def}
  设$G$为一个群,$\forall a\in G$,定义$a^0=e$,$a^{n+1}=a^n\circ a$$(n\geq 0)$,$a^{-n}=(a^{-1})^n$$(n\geq 1)$。
\end{Def}
\begin{Thm}
设$G$为一个群,$a\in G$,$m$,$n$为任意整数,则$a^ma^n=a^{m+n}$,$(a^m)^n=a^{mn}$。
\end{Thm}
设$(G,+)$为一个阿贝尔群,$G$的单位元记为$0$。$\forall a\in G$,定义$0a=0$,$(n+1)a=na+a$$(n\geq 0)$,$(-n)a=n(-a)$$(n\geq 1)$。
对任意整数$m$,$n$,$ma+na=(m+n)a$,$(mn)a=m(na)$,$n(a+b)=na+nb$。
\begin{Def}
  设$(G,\circ)$为一个群,$a\in G$,使$a^n=e$的最小正整数$n$称为$a$的阶。如果不存在这样的正整数$n$,则称$a$的阶为无穷大。
\end{Def}
\begin{Thm}
  有限群的每个元素的阶不超过该有限群的阶。
\end{Thm}
\begin{proof}[证明]
  设群$G$的阶为$N$,则$a^0,a^1,a^2,\cdots,a^N$为$G$的$N+1$个元素,所以必有两个是相同的,设$a^k=a^l$,$0\leq k<l\leq N$。于是,$a^{l-k}=e$,$0<l-k\leq N$,从而$a$的阶不超过$N$。
\end{proof}
课后作业题:

\begin{Exercise}
  设$a$和$b$为群$G$的两个元素。如果$(ab)^2=a^2b^2$,试证:$ab=ba$。
\end{Exercise}
\begin{Exercise}
  设$G$为群。如果$\forall a\in G$,$a^2=e$,试证:$G$为交换群。
\end{Exercise}
\begin{Exercise}
  证明:四阶群是交换群。
\end{Exercise}
\begin{Exercise}
  证明:在任一阶大于2的非交换群里必有两个非单位元$a$和$b$,使得$ab=ba$。
\end{Exercise}
\begin{Exercise}
  有限阶群里阶大于2的元素的个数必为偶数。
\end{Exercise}
\begin{Exercise}
  证明:偶数阶群里,阶为2的元素的个数必为奇数。
\end{Exercise}
\begin{Exercise}
  设$a$为群$G$的一个元素,$a$的阶为$n$且$a^m=e$,试证$n$能整除$m$。
\end{Exercise}
\begin{Exercise}
  设$a_1,a_2,\cdots,a_n$为$n$阶群中的$n$个元素(它们不一定各不相同)。证明:存在整数$p$和$q$($1\leq p \leq q \leq n$),使得
  \[a_pa_{p+1}\cdots a_q=e\]
\end{Exercise}
\end{CJK*}
\end{document}





%%% Local Variables:
%%% mode: latex
%%% TeX-master: t
%%% End: