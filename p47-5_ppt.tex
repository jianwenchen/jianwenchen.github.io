\documentclass{beamer}
\usepackage{ragged2e}
\usepackage{CJKutf8}
\usepackage{tikz}
\setbeamertemplate{theorems}[numbered]
\justifying\let\raggedright\justifying
\begin{document}
\begin{CJK*}{UTF8}{gbsn}


  
\theoremstyle{definition}
\newtheorem{Def}{定义}
\theoremstyle{example}
\newtheorem*{Ex}{例:}
\newtheorem*{Exercise}{习题}

\date{}
\author{陈建文}
\title{习题讲解}

\begin{frame}
  \titlepage
\end{frame}
\begin{frame}

\begin{Exercise}
   设$f:X\to Y$。试证:$f$为满射当且仅当对任意的$E\in 2^Y$,$f(f^{-1}(E))=E$。
\end{Exercise}
\pause
\begin{proof}[证明]\justifying\let\raggedright\justifying
  设$f$为满射,对任意的$E\in 2^Y$往证$f(f^{-1}(E))=E$。
  \pause

  
  对任意的$y$,$y\in f(f^{-1}(E))$,\pause 则存在$x$,$x\in f^{-1}(E)$并且$y=f(x)$,\pause 于是存在$x$,$f(x) \in E$并且$y=f(x)$,\pause 从而$y\in E$。

 \pause  
  对任意的$y$,$y\in E$,\pause 由$f$为满射知存在$x\in X$,$y=f(x)$,\pause 从而$f(x)\in E$,\pause 即$x\in f^{-1}(E)$,\pause 由$y=f(x)$知$y\in f(f^{-1}(E))$。

  \pause
  设对任意的$E\in 2^Y$,$f(f^{-1}(E))=E$,往证$f$为满射。

  \pause
  对任意的$y\in Y$,\pause 则$f(f^{-1}(\{y\}))=\{y\}$,\pause 于是$f^{-1}(\{y\})\neq \phi$,\pause 从而存在$x\in X$,$x\in f^{-1}(\{y\})$, \pause 即$f(x)\in \{y\}$,\pause 等价的,$f(x)=y$,\pause 故$f$为满射。
\end{proof}
  
\end{frame}

\end{CJK*}
\end{document}
