\documentclass{article}
\usepackage{tikz}
\usepackage{CJKutf8}
\usepackage{amsmath}
\usepackage{amsthm}
\begin{document}
\title{第六章作业题}
\begin{CJK}{UTF8}{gbsn}
  \newtheorem{Exercise}{习题}
  \date{}
  \maketitle
  \begin{Exercise}
    画出具有$4$个顶点的所有无向图(同构的只算一个)。  
\end{Exercise}
 \vspace{5cm}
\begin{Exercise}
画出具有$4$个、$6$个、$8$个顶点的三次图。
\end{Exercise}
\vspace{5cm}
\begin{Exercise}
某次宴会上,许多人互相握手,证明:握过奇数次手的人数为偶数(注意,$0$为偶数)。
\end{Exercise}
\vspace{5cm}
\begin{Exercise}
设$u$与$v$为图$G$的两个不同的顶点,若$u$与$v$间有两条不同的通道(迹),则$G$中是否有圈?
\end{Exercise}
\vspace{5cm}
\begin{Exercise}
若$G$是一个$(p,q)$图,$q > \frac{1}{2}(p-1)(p-2)$,试证$G$是连通图。  
\end{Exercise}
\vspace{5cm}
\begin{Exercise}
在一个有$n$个人的宴会上,每个人至少有$m$个朋友($2\leq m < n$),试证:有不少于$m+1$个人,使得他们按照某种方法坐在一张圆桌旁,每人的左右均是他的朋友。
\end{Exercise}
\vspace{5cm}
\begin{Exercise}
设$G$为图。证明:若$\delta(G)\geq 2$,则$G$包含长度至少为$\delta(G)+1$的圈。  
\end{Exercise}
\vspace{5cm}
\begin{Exercise}
  证明:如果$G$不是连通图,则$G^c$是连通图。
\end{Exercise}
\vspace{5cm}
\begin{Exercise}
  每一个自补图有$4n$或$4n+1$个顶点。
\end{Exercise}
\vspace{5cm}
\begin{Exercise}
  给出一个$10$个顶点的非哈密顿图的例子,使得每一对不邻接的顶点的$u$和$v$,均有:$\deg u + \deg v \geq 9$。
\end{Exercise}
\vspace{5cm}
\begin{Exercise}
  试求$K_p$中不同的哈密顿圈的个数。
\end{Exercise}
\vspace{5cm}
\begin{Exercise}
  完全偶图$K_{m,n}$为哈密顿图的充分必要条件是什么?
\end{Exercise}
\vspace{5cm}
\begin{Exercise}
  证明:具有奇数顶点的偶图不是哈密顿图。
\end{Exercise}
\vspace{5cm}

\end{CJK}
\end{document}


%%% Local Variables:
%%% mode: latex
%%% TeX-master: t
%%% End:
