\documentclass{article}
\usepackage{CJKutf8}
\usepackage{amsmath}
\usepackage{amsthm}
\begin{document}
\begin{CJK}{UTF8}{gbsn}
\newtheorem{Exercise}{习题}
\begin{Exercise}
给出一个既不是自反的又不是反自反的二元关系?
\end{Exercise}
\vspace{3cm}
\begin{Exercise}
是否存在一个同时不满足自反性、反自反性、对称性、反对称性和传递性的二元关系?
\end{Exercise}
\vspace{3cm}
\begin{Exercise}
设$R$和$S$为集合$X$上的二元关系,下列命题哪些成立:

a)如果$R$与$S$为自反的,则$R\cup S$和$R\cap S$也为自反的;

b)如果$R$与$S$为反自反的,则$R\cup S$和$R\cap S$也为反自反的;

c)如果$R$与$S$为对称的,则$R\cup S$和$R\cap S$也为对称的;

d)如果$R$与$S$为反对称的,则$R\cup S$和$R\cap S$也为反对称的;

e)如果$R$与$S$为传递的,则$R\cup S$和$R\cap S$也为传递的;

f)如果$R$与$S$不是自反的,则$R\cup S$不是自反的;

g)如果$R$为自反的,则$R^c$为反自反的;

h)如果$R$与$S$为传递的,则$R\setminus S$为传递的。
\end{Exercise}
\vspace{2cm}
\begin{Exercise}
  设$R$与$S$为集合$X$上的二元关系,证明:

  a) $(R^{-1})^{-1}=R$;

  b)$(R\cup S)^{-1}=R^{-1}\cup S^{-1}$;

  c)$(R\cap S)^{-1}=R^{-1}\cap S^{-1}$;

  d)如果$R\subseteq S$,则$R^{-1}\subseteq S^{-1}$。
\end{Exercise}
\vspace{10cm}
\begin{Exercise}
  设$R$为集合$X$上的二元关系。证明:$R\cup R^{-1}$为集合$X$上对称的二元关系。
\end{Exercise}
\vspace{10cm}
\begin{Exercise}
  设$f:X\to Y$,$A\subseteq X$,$B\subseteq Y$。以下四个小题中,每个小题均有四个命题,这四个命题有且仅有一个正确。请找出正确的哪一个。

  (1)(a) 若$f(x)\in f(A)$,则$x$可能属于$A$,也可能不属于$A$;

  (b)若$f(x)\in f(A)$,则$x\in A$;

  (c)若$f(x)\in f(A)$,则$x\notin A$;

  (d)若$f(x)\in f(A)$,则$x\in A^c$。

  (2)(a) $f(f^{-1}(B))=B$;

  (b) $f(f^{-1}(B))\subseteq B$;

  (c) $f(f^{-1}(B))\supseteq B$;

  (d)$f(f^{-1}(B))= B^c$。

  (3)(a)$f^{-1}(f(A))=A$;

  (b) $f^{-1}(f(A))\subseteq A$;

  (c)$f^{-1}(f(A))\supseteq A$;

  (d)以上三个均不对。

  (4)(a)$f(A)\neq \phi$;

  (b)$f^{-1}(B) \neq \phi$;

  (c)若$y\in Y$,则$f^{-1}(\{y\})\in X$;

  (d)若$y\in Y$,则$f^{-1}(\{y\})\subseteq X$。
\end{Exercise}
\begin{Exercise}
  设$X=\{a,b,c\}$,$Y=\{0,1\}$,$Z=\{2,3\}$。$f:X\to Y$,$f(a)=f(b)=0$,$f(c)=1$;$g:Y\to Z$,$g(0)=2,g(1)=3$。试求$g\circ f$。
\end{Exercise}
\vspace{10cm}
\begin{Exercise}
  设$N=\{1,2,\cdots\}$,试构造两个从集合$N$到集合$N$的映射$f$与$g$,使得$fg=I_N$,但$gf\neq I_N$。
\end{Exercise}
\vspace{10cm}
\begin{Exercise}
设$f:X\to Y$。

(1)如果存在唯一的一个映射$g:Y\to X$,使得$gf=I_X$,那么$f$是否可逆呢?

(2)如果存在唯一的一个映射$g:Y\to X$,使得$fg=I_Y$,那么$f$是否可逆呢?
\end{Exercise}
\vspace{10cm}
\begin{Exercise}
设$f:X\to Y$,$X$与$Y$为有穷集合,

(1)如果$f$是左可逆的,那么$f$有多少个左逆映射?

(2)如果$f$是右可逆的,那么$f$有多少个右逆映射?
\end{Exercise}
\vspace{10cm}
\begin{Exercise}
是否有一个从$X$到$X$的一一对应$f$,使得$f=f^{-1}$,但$f\neq I_X$?
\end{Exercise}
\vspace{5cm}
\begin{Exercise}
 设$\sigma_1=\begin{pmatrix}1&2&3&4&5\\4&3&2&1&5\end{pmatrix}$,$\sigma_2=\begin{pmatrix}1&2&3&4&5\\3&2&5&1&4\end{pmatrix}$。求$\sigma_1\sigma_2$,$\sigma_2\sigma_1$,$\sigma_1^{-1}$,$\sigma_2^{-1}$。
\end{Exercise}
\vspace{10cm}
\begin{Exercise}
  将置换$\sigma=\begin{pmatrix}1&2&3&4&5&6&7&8&9\\7&9&1&6&5&2&3&4&8\end{pmatrix}$分解成对换的乘积。
\end{Exercise}



\end{CJK}
\end{document}


%%% Local Variables:
%%% mode: latex
%%% TeX-master: t
%%% End:
