\documentclass{article}
\usepackage{CJKutf8}
\usepackage{amsmath}
\usepackage{amsthm}
\begin{document}
\begin{CJK}{UTF8}{gbsn}
\newtheorem{Exercise}{习题}
\begin{Exercise}
给出一个既不是自反的又不是反自反的二元关系?
\end{Exercise}
\vspace{3cm}
\begin{Exercise}
是否存在一个同时不满足自反性、反自反性、对称性、反对称性和传递性的二元关系?
\end{Exercise}
\vspace{3cm}
\begin{Exercise}
设$R$和$S$为集合$X$上的二元关系,下列命题哪些成立:

a)如果$R$与$S$为自反的,则$R\cup S$和$R\cap S$也为自反的;

b)如果$R$与$S$为反自反的,则$R\cup S$和$R\cap S$也为反自反的;

c)如果$R$与$S$为对称的,则$R\cup S$和$R\cap S$也为对称的;

d)如果$R$与$S$为反对称的,则$R\cup S$和$R\cap S$也为反对称的;

e)如果$R$与$S$为传递的,则$R\cup S$和$R\cap S$也为传递的;

f)如果$R$与$S$不是自反的,则$R\cup S$不是自反的;

g)如果$R$为自反的,则$R^c$为反自反的;

h)如果$R$与$S$为传递的,则$R\setminus S$为传递的。
\end{Exercise}
\vspace{2cm}
\begin{Exercise}
  设$R$与$S$为集合$X$上的二元关系,证明:

  a) $(R^{-1})^{-1}=R$;

  b)$(R\cup S)^{-1}=R^{-1}\cup S^{-1}$;

  c)$(R\cap S)^{-1}=R^{-1}\cap S^{-1}$;

  d)如果$R\subseteq S$,则$R^{-1}\subseteq S^{-1}$。
\end{Exercise}
\vspace{10cm}
\begin{Exercise}
  设$R$为集合$X$上的二元关系。证明:$R\cup R^{-1}$为集合$X$上对称的二元关系。
\end{Exercise}
\vspace{10cm}
\begin{Exercise}
  “父子”关系的平方是什么关系?
\end{Exercise}
\vspace{10cm}
\begin{Exercise}
  设$R$与$S$为集合$X$上的二元关系,下列哪些命题为真?

  a)如果$R$,$S$都是自反的,则$R\circ S$也是自反的;

  b)如果$R$,$S$都是反自反的,则$R\circ S$也是反自反的;

  c)如果$R$,$S$都是对称的,则$R\circ S$也是对称的;

  d)如果$R$,$S$都是反对称的,则$R\circ S$也是反对称的;


  e)如果$R$,$S$都是传递的,则$R\circ S$也是传递的。
\end{Exercise}
\vspace{2cm}
\begin{Exercise}
设$R$,$S$为集合$X$上的两个满足$R\circ S\subseteq S\circ R$的对称关系。证明:$R\circ S\subseteq S\circ R$。
\end{Exercise}
  \vspace{10cm}
  \begin{Exercise}
    设集合$X = \{1,2,3\}$, $Y = \{1,2\}$,$S = \{f|f:X \to Y\}$。$S$上的二元关系$\cong$定义如下:$\forall f,g\in S$,$f \cong g$当且仅当\[f(1) + f(2) + f(3) = g(1) + g(2) + g(3)\]证明$\cong$为$S$上的等价关系,并求出等价类之集。    
  \end{Exercise}
  \vspace{10cm}
 \begin{Exercise}
  设集合$X = \{1,2,3\}$, $Y = \{1,2\}$,$S = \{f|f:X \to Y\}$。$S$上的二元关系$\cong$定义如下:$\forall f,g\in S$,$f \cong g$当且仅当\[\{f^{-1}(\{y\}) | y \in Y\} = \{g^{-1}(\{y\})|y \in Y\}\]证明$\cong$为$S$上的等价关系,并求出等价类之集。  
\end{Exercise}

\vspace{10cm}
\begin{Exercise}
由置换$\sigma=\begin{pmatrix}1&2&3&4&5&6&7&8\\3&6&5&8&1&2&4&7\end{pmatrix}$确定了集合$X=\{1,2,3,4,5,6,7,8\}$上的一个二元关系$\cong$:对任意的$i,j\in X$,$i\cong j$当且仅当$i$与$j$在$\sigma$的循环分解式的同一个循环置换中。
证明:$\cong$为集合$X$上的等价关系,求$X/\cong$。
\end{Exercise}
\vspace{5cm}
\begin{Exercise}
 给出集合$X=\{1,2,3,4\}$上的两个等价关系$R$与$S$,使得$R\circ S$不是等价关系。
\end{Exercise}
\vspace{10cm}
\begin{Exercise}
  设$R$为集合$X$上的一个二元关系,试证:$R$为一个等价关系,当且仅当以下两条成立(1)对任意的$x$,$xRx$;(2)如果$xRy$且$xRz$,则$yRz$。
\end{Exercise}
\vspace{10cm}

\begin{Exercise}
  设$X$为一个集合,$|X|=n$,试求:

  a)集合$X$上自反二元关系的个数;

  b)集合$X$上反自反二元关系的个数;

  c)集合$X$上对称二元关系的个数;

  d)集合$X$上反对称二元关系的个数。
\end{Exercise}
\vspace{10cm}
\begin{Exercise}
  是否存在一个偏序关系$\leq$,使$(X,\leq)$中有唯一极大元素,但没有最大元素?如果有,请给出一个具体例子;如果没有,请证明之。
\end{Exercise}
\vspace{5cm}
\begin{Exercise}
  令$S=\{1,2,\cdots,12\}$,画出偏序集$(S,|)$的Hasse图,其中$|$为整除关系。它有几
  个极大(小)元素?列出这些极大(小)元素。
\end{Exercise}
\end{CJK}
\end{document}


%%% Local Variables:
%%% mode: latex
%%% TeX-master: t
%%% End:
