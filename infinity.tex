\documentclass{book}[oneside]
\usepackage{CJKutf8}
\usepackage{amsmath}
\usepackage{amsfonts}
\usepackage{amsthm}
\usepackage{titlesec}
\usepackage{titletoc}
\usepackage{xCJKnumb}
\usepackage{clrscode3e}

\usepackage{tikz}
\titleformat{\chapter}{\centering\Huge\bfseries}{第\, \xCJKnumber{\thechapter}\,
    章}{1em}{}
  % \renewcommand{\chaptermark}[1]{\markboth{第 \thechapter 章}{}}
\usepackage{mathrsfs}

\newtheorem{Def}{定义}[chapter]
\newtheorem{Thm}{定理}[chapter]
\newtheorem{Cor}{推论}[chapter]
\newtheorem{Ax}{公理}[chapter]

\newtheorem{Exercise}{练习}[chapter]

\newtheorem{Example}{例}[chapter]


\begin{document}
\begin{CJK*}{UTF8}{gbsn}
  \title{离散数学讲义}
  \author{陈建文}
  \maketitle
  % \tableofcontents
  

  \setcounter{chapter}{3}
  \chapter{无穷集合}
  \begin{Def}
    如果从集合$X$到集合$Y$存在一个双射,则称$X$与$Y$对等,记为$X \sim Y$。
  \end{Def}
  \begin{Def}
    如果从自然数集$\mathbb{N}$到集合$X$存在一个一一对应$f:\mathbb{N}\to X$,则称
    集合$X$为可数无穷集合,简称可数集或可列集。如果$X$不是可数集且$X$不是有穷集合,则称$X$为不可数无穷集合,简称不可数集。
  \end{Def}
  \begin{Thm}
    集合$A$为可数集的充分必要条件为$A$的全部元素可以排成无重复项的序列
    \[a_1, a_2, \ldots, a_n, \cdots\]
    因此,$A$可写成$A = \{a_1, a_2, a_3, \cdots\}$。
  \end{Thm}
    \begin{Thm}
       可数集的任一无限子集也是可数集。
  \end{Thm}
  \begin{Thm}
   设$A$为可数集合,$B$为有穷集合,则$A\cup B$为可数集。
  \end{Thm}
  \begin{Thm}
    设$A$与$B$为两个可数集,则$A\cup B$为可数集。
  \end{Thm}

    \begin{Thm}
    设$A_1, A_2, \cdots, A_n, \cdots$为可数集合的一个无穷序列,则$\bigcup_{n=1}^{\infty}A_n$为可数集。即可数多个可数集之并为可数集。
  \end{Thm}
  \begin{Thm}
    设$A$与$B$为两个可数集,则$A\times B$为可数集。
  \end{Thm}
  \begin{Thm}
    全体有理数之集$\mathbb{Q}$为可数集。
  \end{Thm}
  \begin{Thm}
    区间$[0,1]$中的所有实数构成的集合为不可数集。
  \end{Thm}
  \begin{Def}
    凡与集合$[0,1]$存在一个一一对应的集合称为具有“连续统的势”的集合,简称连续统。
  \end{Def}
    \begin{Thm}
    无穷集合必包含有可数子集。
  \end{Thm}
  \begin{Thm}
    设$M$为一个无穷集合,$A$为至多可数集合,则$M \sim M \cup A$。
  \end{Thm}
  \begin{proof}[证明]
    因为$M$为一个无穷集合,所以$M$中必有一个可数子集$D$。令$P=M\setminus D$,则
    \[M=P\cup D, M\cup A = P\cup (D\cup (A\setminus P))\]
    由$P\sim P$,$D\sim D\cup (A\setminus P)$,得到$M\sim M\cup A$。
  \end{proof}
  \begin{Thm}
    设$M$为无穷集合,$A$为$M$的至多可数子集,$M\setminus A$为无穷集合,则$M \sim M\setminus A$。
  \end{Thm}
  \begin{Thm}
    设$A_1, A_2, \cdots, A_n$为$n$个两两不相交的连续统,则$\bigcup_{i=1}^nA_i$为连续统。
  \end{Thm}
  \begin{Thm}
    设$A_1, A_2, \cdots, A_n, \cdots$为两两不相交的集序列。如果$A_k \sim [0,1], k = 1, 2, \cdots$,则
    \[\bigcup_{n=1}^{\infty}A_n \sim [0,1]\]
  \end{Thm}
 \begin{Cor}
   全体实数之集是一个连续统。
 \end{Cor}
 \begin{Cor}
   全体无理数之集是一个连续统。
 \end{Cor}
  \begin{Def}
    集合$A$的基数为一个符号,凡与$A$对等的集合都赋以同一个记号。集合$A$的基数记为$|A|$。
  \end{Def}
  \begin{Def}
    所有与集合$A$对等的集合构成的集族称为$A$的基数。
  \end{Def}
    \begin{Def}
    集合$A$的基数与集合$B$的基数称为是相等的,当且仅当$A \sim B$。
  \end{Def}
  \begin{Def}
    设$\alpha$,$\beta$为任意两个基数,$A$,$B$为分别以$\alpha$,$\beta$为其基数的
    集合。如果$A$与$B$的一个真子集对等,但$A$却不能与$B$对等,则称基数$\alpha$小于基数$\beta$,记为$\alpha < \beta$。
  \end{Def}
  显然,
  $\alpha \leq \beta$当且仅当存在单射$f:A \to B$。

  $\alpha < \beta$当且仅当存在单射$f:A \to B$且不存在$A$到$B$的双射。

  \begin{Thm}[康托]
    对任一集合$M$,$|M| < |2^{M}|$。
  \end{Thm}
  \begin{proof}[证明]
    令$i:M\to
    2^M$,其定义为对任意的$m\in M$,$i(m)=\{m\}$。于
    是,$i$为从$M$到$2^M$的单射,故$|M|\leq |2^M|$。为了完成定理的证明,
    我们还需要证明:如果$f:M\to 2^M$为单射,则$f$一定不为满射。为此,令
    \[X=\{m\in M|m \notin f(m)\}\]显然,$X\in 2^M$。现在证明对任意
    的$x\in M$,$f(x)\neq X$。实际上,如果存在$x_0\in X$使得$f(x_0)=X$,
    则如果$x_0\in X$,那么由$X$的的定义知$x_0\notin
    f(x_0)$,即$x_0\notin X$;如果$x_0\notin X$,即$x_0\notin f(x_0)$,由$X$的定义可得$x_0\in X$。总之,$x_0\in X$与$x_0\notin X$都引出矛盾,从而不存在$x_0\in M$使得$f(x_0)=X$。因此,$f$不为满射,从而
    \[|M|<|2^M|\]
  \end{proof}
  \begin{Thm}[康托-伯恩斯坦]
    设$A$,$B$为两个集合。如果存在单射$f:A\to B$与单射$g:B\to A$,则$A$与$B$的基数相等。
  \end{Thm}

  \begin{proof}[证法一]
    设$f:A\to B$和$g:B\to A$都为单射。令$\psi:2^A\to 2^A$,对任意
    的$E\in 2^A$,\[\psi(E)=A\setminus g(B\setminus f(E))\]易见,如果$E\subseteq F\subseteq A$,则$\psi(E)\subseteq \psi(F)$。
    令\[\mathbb{D}=\{E\subseteq A|E\subseteq \psi(E)\}\],则$\phi\in \mathbb{D}$。又令
    \[D=\bigcup_{E\in \mathbb{D}}E,\]
    则对任意的$E\in \mathbb{D}$,由$E\subseteq D$知$E\subseteq \psi(E) \subseteq \psi(D)$,从而$D\subseteq \psi(D)$。
    于是$\psi(D)\subseteq \psi(\psi(D))$,故$\psi(D)\in \mathbb{D}$,因此,$\psi(D)\subseteq D$,所以
    \[D=\psi(D)=A\setminus g(B\setminus f(D))\]
    令$h:A\to B$,对任意的$x\in A$,定义
    \[h(x)=\begin{cases}
        f(x),&\text{如果}x\in D\\
        g^{-1}(x),&\text{如果}x\in A\setminus D
      \end{cases}
    \]
    其中$g^{-1}$为视$g$为$B$到$g(B)$的一一对应时$g$的逆,易见$h$为一一对应。所以$A$与$B$的基数相等。
  \end{proof}。

  \begin{proof}[证法二]
     We separate $A$ into two disjoint sets $A_1$ and $A_2$. We let $A_1$ consist of all $x\in A$ such that, when we lift back $x$ by a succession of inverse maps,
    \[x, g^{-1}(x), f^{-1}(g^{-1}(x)),g^{-1}(f^{-1}(g^{-1}(x)))\cdots\]
    then $x$ can be lifted indefinitely, or  at some stage we get stopped in A (i.e. reach an element of $A$ which has no inverse image in $B$ by $g$). We let $A_2$ be the complement of $A_1$, in other words, the set of $x\in A$ from which we get stopped in B by following the succession of inverse maps. We shall define a bijection $h$ of $A$ onto $B$.

    If $x\in A_1$, we define $h(x)=f(x)$.

    If $x\in A_2$, we define $h(x)=g^{-1}(x)$.

    Then trivially, $h$ is injective. We must prove that $h$ is surjective. Let $y\in B$. If, when we try to lift back $y$ by a succession of maps

    \[y, f^{-1}(y), g^{-1}(f^{-1}(y)),f^{-1}(g^{-1}(f^{-1}(y)))\cdots\]
    
    we can lift back indefinitely, or if we get stopped in $A$, then $f^{-1}(y)$ is defined, and $f^{-1}(y)$ lies in $A_1$. Consequently, $y = h(f^{-1}(y))$ is in the image of $h$. On the other hand, if we cannot lift back $y$ indefinitely, and get stopped in $B$, then $g(y)$ belongs to $A_2$. In this case, $y=h(g(y))$ is also in the image of $h$, as was to be shown.

  \end{proof}

  %   \begin{Def}
  %   设$\alpha$,$\beta$为两个基数,$A$与$B$为两个不相交集合,$|A|=\alpha$,$|B|=\beta$,则集合$A\cup B$的基数称为基数$\alpha$与$\beta$的和,记为$\alpha + \beta$。
  % \end{Def}
  % \begin{Def}
  %   设$\alpha$,$\beta$为两个基数,$A$与$B$为两个集合,$|A|=\alpha$,$|B|=\beta$,则集合$A\times B$的基数称为基数$\alpha$与$\beta$的积,记为$\alpha \cdot \beta$ 或者$\alpha \beta$。
  % \end{Def}
  % \begin{Def}
  %   设$\alpha$,$\beta$为两个基数,$A$与$B$为两个集合,$|A|=\alpha$,$|B|=\beta$,则集合$B^A=\{f|f:A\to B\}$的基数称为$\beta$的$\alpha$次幂,记为$\beta^{\alpha}$。
  % \end{Def}  

  %   \begin{Thm}
  % 设$a$为可数集的基数,$c$为连续统的基数,则
  % \begin{enumerate}
  % \item $\forall n\in N\cup \{0\}, n + a = a$.
  % \item $\forall n\in N, n \cdot a = a$.
  % \item $\forall n\in N, n \cdot c = c$.
  % \item $a\cdot c =c$.
  % \item $c\cdot c = c$.
  % \item $2^a=c$.
  % \item $(2^a)^a=c$.
  % \item $a^a=2^a$.
  % \end{enumerate}    
  % \end{Thm}

刻画集合的ZFC公理系统(Zermelo-Fraenkel-Choice axioms of set theory):
  \begin{Ax}[外延公理]
    \begin{equation*}
      \forall A \forall B (\forall x (x \in A \leftrightarrow x\in B)\rightarrow A = B)
    \end{equation*}
  \end{Ax}   
  \begin{Ax}[空集公理]
    \begin{equation*}
      \exists \phi \forall x (x \notin \phi)
    \end{equation*}
  \end{Ax}
  \begin{Ax}[对公理]
    \begin{equation*}
      \forall u \forall v \exists B \forall x (x \in B \leftrightarrow x = u \lor x = v)
    \end{equation*}
  \end{Ax}
  \begin{Ax}[并集公理]
    \begin{equation*}
     \forall A \exists B \forall x (x \in B \leftrightarrow (\exists b \in A) x \in b)
    \end{equation*}
  \end{Ax}
    \begin{Ax}[幂集公理]
    \begin{equation*}
      \forall a \exists B \forall x ( x \in B \leftrightarrow x \subseteq a)
    \end{equation*}
  \end{Ax}
  \begin{Ax}[子集公理]
    \begin{equation*}
      \forall c \exists B \forall x (x \in B \leftrightarrow x \in c \land \varphi(x))
    \end{equation*}
  \end{Ax}
  \begin{Ax}[无穷公理]
    \begin{equation*}
      \begin{split}
      \exists A ( \phi \in A \land (\forall a \in A) a^+ \in A)\\
      \text{其中} a^+ = a \cup \{a\}
      \end{split}
    \end{equation*}
  \end{Ax}
    \begin{Ax}[代换公理]
    \begin{equation*}
      \begin{split}
      \forall A ((\forall x \in A) \forall y_1 \forall y_2 (\varphi(x, y_1) \land \varphi(x, y2) \rightarrow y_1 = y_2)\\
      \rightarrow \exists B \forall y (y \in B \leftrightarrow (\exists x \in A) \varphi(x, y)))
    \end{split}
  \end{equation*}
  \end{Ax}
  \begin{Ax}[正则公理]
    \begin{equation*}
      (\forall A \neq \phi) (\exists m \in A) m \cap A = \phi
    \end{equation*}
  \end{Ax}
  \begin{Ax}[选择公理]
    \begin{equation*}
      (\forall \text{relation} R)
      (\exists \text{function} F)
      (F \subseteq R \land
      \text{dom} F
      = \text{dom} R)
    \end{equation*}
  \end{Ax}

  % 刻画自然数的Peano公理系统:
  % \begin{enumerate}
  % \item $0 \in \mathbb{N}$;
  % \item $n \in \mathbb{N} \rightarrow n ++ \in \mathbb{N}$;
  % \item $\forall n \in \mathbb{N} n ++ \neq 0$;
  % \item $\forall n \in \mathbb{N} \forall m \in \mathbb{N} n \neq m \rightarrow n ++ \neq m ++$;
  %   \item $(P(0) \land \forall n \in \mathbb{N} p(n) \rightarrow p(n++) )\rightarrow \forall n p(n)$。 
  % \end{enumerate}

  刻画实数的公理系统:

    设$x, y, z \in \mathbb{R}$,则
   \begin{enumerate}
   \item   $x + y = y + x$
   \item   $(x + y) + z = x + (y + z)$
   \item   $0 + x = x + 0 = x$
   \item   $(-x) + x =x + (-x) = 0$
   \item   $x * y = y * x$
   \item   $(x * y) * z = x * (y *z)$
   \item   $1 * x = x * 1 = x$
   \item   $x^{-1} * x = x * x^{-1} = 1 (x \neq 0)$
   \item   $x* (y + z) = x * y + x * z$
   \item   $(y + z) * x = y * x + z * x$
   \item 对任意的$x\in R$,$x\leq x$。
   \item 对任意的$x\in R$,$y\in R$,如果$x\leq y$并且$y\leq x$,则$x=y$。 
  \item 对任意的$x\in R$,$y\in R$,$z\in R$,如果$x\leq y$并且$y\leq z$,则$x\leq z$。
  
  我们用$x<y$表示$x\leq y$并且$x\neq y$,$x\geq y$表示$y\leq x$,$x > y$表示$x\geq y$并且$x\neq y$。
  
  \item 对任意的$x\in R$,$y\in R$,$z\in R$,如果$x<y$,则$x+z<y+z$。
  \item 对任意的$x\in R$,$y\in R$,如果$x>0$,$y>0$,则$xy>0$。
  \item 设$A_1$, $A_2$,$\cdots$,$A_i$,$\cdots$为实数集$R$上的闭区间,$A_1\supseteq A_2 \supseteq A_3 \supseteq \cdots \supseteq A_i \supseteq \cdots$,则$\bigcap_{i=1}^{\infty}A_i$非空。
  \end{enumerate}



      \chapter{}

\end{CJK*}
\end{document}





%%% Local Variables:
%%% mode: latex
%%% TeX-master: t
%%% End:



