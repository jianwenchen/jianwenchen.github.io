\documentclass{article}
\usepackage{tikz}
\usepackage{CJKutf8}
\usepackage{amsmath}
\usepackage{amsthm}
\begin{document}
\begin{CJK}{UTF8}{gbsn}
\newtheorem*{Ex}{习题}
\begin{Ex}
设$R,S$为集合$X$上的等价关系,证明:$(R\cup S)^+$为$X$上的等价关系。
\end{Ex}
\begin{proof}[证明]
  以下验证$(R\cup S)^+$为集合$X$上自反的,对称的和传递的二元关系。

  首先验证自反性:对任意的$x\in X$,由$R$为等价关系知$(x,x)\in R$,从而$(x,x)\in R\cup S\subseteq (R\cup S)^+$。

  其次验证对称性:对任意的$x\in X, y\in X$,如果$(x,y)\in (R\cup S)^+=\bigcup_{n=1}^{\infty}(R\cup S)^n$,则存在$m$使得$(x,y)\in R^m$。
  于是存在$x_1,x_2,\ldots,x_{m-1}\in X$使得$(x,x_1)\in R\cup S$,$(x_1,x_2)\in R\cup S$,$\ldots$,$(x_{m-1},y)\in R\cup S$。由$R$和$S$都为
  $X$上的等价关系知$R$和$S$都是对称的,从而易验证$(y,x_{m-1})\in R\cup S$,$\ldots$,$(x_2,x_1)\in R\cup S$,$(x_1,x)\in R\cup S$,从而$(y,x)\in (R\cup S)^m \subseteq (R\cup S)^+$。

  最后验证传递性:显然$(R\cup S)^+$为传递的。
\end{proof}
\end{CJK}
\end{document}


%%% Local Variables:
%%% mode: latex
%%% TeX-master: t
%%% End:
