\documentclass{article}
\usepackage{CJKutf8}
\usepackage{amsmath}
\usepackage{amssymb}
\usepackage{amsfonts}
\usepackage{amsthm}
\usepackage{titlesec}
\usepackage{titletoc}
\usepackage{xCJKnumb}
\usepackage{tikz}
\usepackage{mathrsfs}
\usepackage{indentfirst}

\newtheorem{Def}{定义}
\newtheorem{Thm}{定理}
\newtheorem{Exercise}{练习}

\newtheorem*{Example}{例}


\begin{document}
\begin{CJK*}{UTF8}{gbsn}
  \title{第六讲 循环群}
  \author{陈建文}
  \maketitle
  % \tableofcontents


课后作业题:
\begin{Exercise}
证明:$n$次单位根之集对数的通常乘法构成一个循环群。
\end{Exercise}
\begin{proof}[证明]
  $n$次单位根之集对数的通常乘法构成的群为$(cos(\frac{2\pi}{n})+isin(\frac{2\pi}{n}))$。
\end{proof}
\begin{Exercise}
找出模$12$的同余类加群的所有子群。
\end{Exercise}
\begin{proof}[解]
  $([0])=\{[0]\},([1])=Z_{12},([2])=\{[0],[2],[4],[6],[8],[10]\},([3])=\{[0],[3],[6],[9]\},([4])=\{[0],[4],[8]\},$
  $([6])=\{[0],[6]\}$。

\end{proof}
\begin{Exercise}
  设$G=(a)$为一个$n$阶循环群。证明:如果$(r,n)=1$,则$(a^r)=G$。
\end{Exercise}

\begin{proof}[证明]
  由$(r,n)=1$知存在$s,t\in Z$,使得$1=sr+tn$,从而$a^1=a^{sr+tn}=(a^r)^s(a^n)^t=(a^r)^se^t=(a^r)^s$,
  于是$a\in (a^r)$,从而$(a^r)=G$。
\end{proof}
\begin{Exercise}
  设群$G$中元素$a$的阶为$n$,$(r,n)=d$。证明:$a^r$的阶为$n/d$。
\end{Exercise}
\begin{proof}[证明]
  以下证明$(a^d)=(a^r)$,而$|(a^d)|=n/d$,于是$|(a^r)|=n/d$,从而$a^r$的阶为$n/d$。

  由$(r,n)=d$知存在$s,t\in Z$使得$d=sr+tn$,从而$a^d=a^{(sr+tn)}=(a^r)^s(a^n)^t=(a^r)^se^t=(a^r)^s$,
  于是$a^d\in (a^r)$,由此可得$(a^d)\subseteq (a^r)$。

  设$r=kd$,这里$k\in N$,于是$a^r=(a^d)^k$,从而$a^r\in (a^d)$,由此可得$(a^r)\subseteq (a^d)$。
\end{proof}
\end{CJK*}
\end{document}





%%% Local Variables:
%%% mode: latex
%%% TeX-master: t
%%% End:



