\documentclass{article}
\usepackage{CJKutf8}
\usepackage{amsmath}
\usepackage{amssymb}
\usepackage{amsfonts}
\usepackage{amsthm}
\usepackage{titlesec}
\usepackage{titletoc}
\usepackage{xCJKnumb}
\usepackage{tikz}
\usepackage{mathrsfs}
\usepackage{indentfirst}
\usepackage{enumitem}
\newtheorem{Def}{定义}
\newtheorem{Thm}{定理}
\newtheorem{Exercise}{练习}

\newtheorem*{Example}{例}
\setlist[enumerate,1]{label=(\arabic*)}

\begin{document}
\begin{CJK*}{UTF8}{gbsn}
  \title{第九讲 一阶谓词演算形式系统}
  \author{陈建文}
  \maketitle
一阶语言

    1. 字符集
    \begin{enumerate}
      \item 表示个体变元的符号:$v_1,v_2,\cdots,v_n,\cdots$
      \item 完备的联结词集合:$\{\lnot, \to\}$
      \item 量词:$\forall$
      \item 辅助符号:$()$
      \item 表示谓词的符号:$P,Q,\cdots$
      \item 表示函词的符号:$f,g,\cdots$
      \item 表示个体常元的符号:$a,b,c,\cdots$
    \end{enumerate}

    2. 公式:

      \begin{enumerate}
        \item 设$t_1,t_2,\cdots,t_n$为$n$个项,$P$为任意一个$n$元谓词符号,则$P(t_1,t_2,\cdots,t_n)$为合式公式;
        \item 如果$A,B$为合式公式,则$(\lnot A),(A\to B)$为合式公式;
        \item 如果$A$为合式公式,$x$为任意一个变量,则$\forall xA$为合式公式;     
        \item 有限次使用(1),(2)和(3)复合所得到的结果都是合式公式。
      \end{enumerate}

    合式公式简称公式。

    3. 公理  
  
    $A_1:A\to(B\to A)$

    $A_2:(A\to(B\to C))\to((A\to B)\to (A\to C))$

    $A_3:(\lnot A\to \lnot B)\to (B \to A)$

    $A_4:\forall v A\to A^v_t$(项$t$对$v$可代入)

    $A_5:\forall v(A\to B)\to (\forall A\to \forall B)$

    $A_6:A\to\forall v A$($v$在$A$中无自由出现)

    4. 推理规则 
    
    $r_{mp}:A,A\to B, B$

    5. 定理推导
    

    证明:称下列公式序列为谓词公式$A$在$FC$中的一个证明:

    \[A_1,A_2,\cdots,A_m(=A)\]

    其中$A_i(i=1,2,\cdots,m)$或为$FC$的公理,或为$A_j(j<i)$,或为$A_j,A_k(j,k<i)$使用$r_{mp}$导出的谓词公式。


    定理:如果谓词公式$A$在$PC$中有一个证明序列,则称$A$为$PC$的定理,记为$\vdash_{PC}A$,简记为$\vdash A$。


  \begin{Thm}$\forall xP(x)\vdash \forall yP(y)$
  \end{Thm}
  \begin{proof}[证明]$\quad$
    \begin{enumerate}
      \item $ \forall x P(x)\vdash \forall xP(x) \quad\quad $(前提)
      \item $ \forall x P(x)\vdash \forall xP(x)\to \forall y\forall xP(x) \quad\quad (A_6)$
      \item $ \forall x P(x)\vdash \forall y\forall xP(x) \quad\quad (1)(2)r_{mp}$ 
      \item $ \forall x P(x)\vdash \forall y(\forall xP(x)\to P(y))\quad\quad (A_4)$
      \item  $\forall x P(x)\vdash \forall y(\forall xP(x)\to P(y))\to (\forall y\forall xP(x)\to \forall yP(y))(A_5)$
      \item  $\forall x P(x)\vdash \forall y\forall xP(x)\to \forall yP(y)\quad\quad(4)(5)r_{mp}$
      \item $\forall xP(x)\vdash \forall yP(y)\quad (3)(6)r_{mp}$
    \end{enumerate}
  \end{proof}
  
\end{CJK*}
\end{document}





%%% Local Variables:
%%% mode: latex
%%% TeX-master: t
%%% End:
