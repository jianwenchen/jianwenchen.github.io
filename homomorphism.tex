\documentclass{article}
\usepackage{CJKutf8}
\usepackage{amsmath}
\usepackage{amssymb}
\usepackage{amsfonts}
\usepackage{amsthm}
\usepackage{titlesec}
\usepackage{titletoc}
\usepackage{xCJKnumb}
\usepackage{tikz}
\usepackage{mathrsfs}
\usepackage{indentfirst}

\newtheorem{Def}{定义}
\newtheorem{Thm}{定理}
\newtheorem{Exercise}{练习}

\newtheorem*{Example}{例}


\begin{document}
\begin{CJK*}{UTF8}{gbsn}
  \title{第九讲 同态基本定理}
  \author{陈建文}
  \maketitle
  % \tableofcontents
  

\begin{Def}
  设$(G_1,\circ)$与$(G_2,*)$为两个群,如果存在一个从$G_1$到$G_2$的映射$\phi$,使得$\forall a,b\in G_1$,\[\phi(a\circ b)=\phi(a)* \phi(b)\]
  则称$G_1$与$G_2$同态,$\phi$称为从$G_1$到$G_2$的一个同态(homomorphism)。如果同态$\phi$是满射,则称$\phi$为从$G_1$到$G_2$的一个满同态,此时称$G_1$与$G_2$为满同态,
  并记为$G_1\sim G_2$。类似的,如果同态$\phi$为单射,则称$\phi$为单同态。
\end{Def}

\begin{Thm}
  设$(G_1,\circ)$与$(G_2,*)$为两个群,$e_1$和$e_2$分别为其单位元,$\phi$为从$G_1$到$G_2$的同态,则
  \begin{align*}
    &\phi(e_1)=e_2\\
    &\forall a\in G_1 \phi(a^{-1})=(\phi(a))^{-1}\\
  \end{align*}
\end{Thm}
\begin{proof}[证明]
  由$\phi(e_1)=\phi(e_1\circ e_1)=\phi(e_1)*\phi(e_1)$知$\phi(e_1)=e_2$。
  $\forall a\in G_1, \phi(a^{-1})*\phi(a)=\phi(a^{-1}\circ a)=\phi(e_1)=e_2$,从而$(\phi(a))^{-1}=\phi(a^{-1})$。
\end{proof}
\begin{Thm}
  设$(G_1,\circ)$为一个群,$(G_2,*)$为一个代数系。如果存在一个满射$\phi:G_1\to G_2$使得$\forall a,b\in G_1$
  \[\phi(a\circ b)=\phi(a) * \phi(b)\]
  则$(G_2,*)$为一个群。
\end{Thm}
\begin{proof}[证明]
  验证$\forall x,y,z\in G_2,(x*y)*z=x*(y*z)$:由$\phi$为满射知$\exists a,b,c\in G_1$使得$\phi(a)=x,\phi(b)=y,\phi(c)=z$,从而
  $(x*y)*z=(\phi(a)*\phi(b))*\phi(c)=\phi(a\circ b)*\phi(c)=\phi((a\circ b)\circ c)$,$x*(y*z)=\phi(a)* (\phi(b)*\phi(c))=\phi(a)*\phi(b\circ c)=\phi(a\circ (b\circ c))$,$(x*y)*z=x*(y*z)$,这验证了在$G_2$中$*$运算满足结合律。

  $\forall x\in G_2$,由$\phi$为满射知$\exists a\in G_1$使得$\phi(a)=x$,于是$\phi(e)*x=\phi(e)*\phi(a)=\phi(e\circ a)=\phi(a)=x$。

  $\forall x\in G_2$,由$\phi$为满射知$\exists a\in G_1$使得$\phi(a)=x$,于是$\phi(a^{-1})*\phi(a)=\phi(a^{-1}\circ a)=\phi(e_1)$。
\end{proof}

\begin{Example}
  设$n\in Z^+$,$Z_n'=\{0,1,\cdots, n-1\}$,在$Z_n'$上定义运算"$\oplus$"如下:$i\oplus j=(i+j)\bmod n$,
  令$f:Z\to Z_n'$,$\forall m\in Z, f(m)= m \bmod n$,则$f$为从$Z$到$Z_n'$的满射,并且$\forall a,b\in Z$,$f(a+b)=f(a)\oplus f(b)$,从而$(Z_n',\oplus)$为一个群。
\end{Example}
\begin{proof}[证明]
  $f$显然为从$Z$到$Z_n'$的满射。要证$\forall a,b\in Z$,$f(a+b)=f(a)\oplus f(b)$,就是要证$(a+b)\bmod n = ((a\bmod n) + (b\bmod n))\bmod n$,此式显然成立。
\end{proof}
\begin{Thm}
  设$\phi$为从群$(G_1,\circ)$到群$(G_2,*)$的同态,则

  (1)如果$H$为$G_1$的子群,那么$\phi(H)$为$G_2$的子群;

  (2)如果$H$为$G_2$的子群,那么$\phi^{-1}(H)$为$G_1$的子群;

  (3)如果$N$为$G_1$的正规子群,那么$\phi(N)$为$\phi(G_1)$的正规子群;

  (4)如果$N$为$\phi(G_1)$的正规子群,那么$\phi^{-1}(N)$为$G_1$的正规子群;
\end{Thm}
\begin{proof}[证明]
  以下设$G_1$的单位元为$e_1$,$G_2$的单位元为$e_2$。

  (1)$e_2=\phi(e_1)\in \phi(H)$,从而$\phi(H)$非空。

  $\forall x, y\in \phi(H)$,$\exists a,b\in H$使得$x=\phi(a)$,$y=\phi(b)$,则$x*y^{-1}=\phi(a)*\phi(b)^{-1}=\phi(a)*\phi(b^{-1})=\phi(a\circ b^{-1})\in \phi(H)$。

  以上验证了$\phi(H)$为$G_2$的子群。

  (2)由$\phi(e_1)=e_2\in H$知$e_1\in \phi^{-1}(H)$,从而$\phi^{-1}(H)$非空。

  以下证明$\forall a,b\in \phi^{-1}(H)$,$a\circ b^{-1}\in \phi^{-1}(H)$,即$\phi(a\circ b^{-1})\in H$。
  
  $\forall a,b\in \phi^{-1}(H)$,则$\phi(a)\in H$,$\phi(b)\in H$,从而$\phi(a\circ b^{-1})=\phi(a)*\phi(b^{-1})=\phi(a)*\phi(b)^{-1}\in H$,于是$a\circ b^{-1}\in \phi^{-1}(H)$。

  这验证了$\phi^{-1}(H)$为$G_1$的子群。

  (3) $\phi(N)$显然为$\phi(G_1)$的子群。

  以下证明$\forall h\in \phi(N), \forall g\in \phi(G_1), g*h*g^{-1}\in \phi(N)$。

  $\forall h\in \phi(N)$,$\exists b\in N$使得$h=\phi(b)$,$\forall g\in \phi(G_1),\exists a\in G_1$,使得$g=\phi(a)$。于是,$g*h*g^{-1}=\phi(a)*\phi(b)*\phi(a)^{-1}=\phi(a\circ b)*\phi(a^{-1})=\phi(a\circ b\circ a^{-1})\in \phi(N)$,因此$\phi(N)$为$G_2$的正规子群。

  (4)由(2)知$\phi^{-1}(N)$为$G_1$的子群。

  要证$\phi^{-1}(N)$为$G_1$的正规子群,就是要证$\forall g\in \phi^{-1}(N), \forall a\in G_1, a\circ g\circ a^{-1}\in \phi^{-1}(N)$,而\phi(a\circ g\circ a^{-1})=\phi(a)*\phi(g)*\phi(a^{-1})=\phi(a)*\phi(g)*\phi(a)^{-1}\in N$,从而$a\circ g\circ a^{-1}\in \phi^{-1}(N)$,结论得证。

\end{proof}

\begin{Def}
设$\phi$为从群$(G_1,\circ)$到群$(G_2,*)$的一个同态,$e_2$为$G_2$的单位元,则$G_1$的子群$\phi^{-1}(e_2)$称为同态$\phi$的核,记为$Ker \phi$。$\phi(G_1)$称为$\phi$在$G_1$下的同态像。
\end{Def}

\begin{Thm}
  设$\phi$为从群$(G_1,\circ)$到群$(G_2,*)$的一个同态,则$Ker \phi$为群$G_1$的正规子群。
\end{Thm}

\begin{Thm}
设$N$为$G$的一个正规子群,$\phi$为从$G$到$G/N$的一个映射,$\forall x\in G \phi(x)=xN$,则$\phi$为从$G$到$G/N$的一个满同态,$Ker \phi=N$。
\end{Thm}

\begin{Thm}[群的同态基本定理]
设$\phi$为从群$(G_1,\circ)$到群$(G_2,*)$的同态,则$G_1/Ker G_1 \cong \phi(G_1)$。
\end{Thm}

课后作业题:
\begin{Exercise}
设$G$为$m$阶循环群,$\bar{G}$为$n$阶循环群,试证:$G \sim \bar{G}$当且仅当$n | m$。
\end{Exercise}

\begin{Exercise}
设$G$为一个循环群,$H$为群$G$的子群,试证:$G/H$也为循环群。
\end{Exercise}
\end{CJK*}
\end{document}





%%% Local Variables:
%%% mode: latex
%%% TeX-master: t
%%% End:



