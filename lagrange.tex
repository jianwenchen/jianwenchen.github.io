\documentclass{article}
\usepackage{CJKutf8}
\usepackage{amsmath}
\usepackage{amssymb}
\usepackage{amsfonts}
\usepackage{amsthm}
\usepackage{titlesec}
\usepackage{titletoc}
\usepackage{xCJKnumb}
\usepackage{tikz}
\usepackage{mathrsfs}
\usepackage{indentfirst}

\newtheorem{Def}{定义}
\newtheorem{Thm}{定理}
\newtheorem{Exercise}{练习}

\newtheorem*{Example}{例}

\newtheorem{Cor}{推论}
\begin{document}
\begin{CJK*}{UTF8}{gbsn}
  \title{第七讲 子群的陪集、拉格朗日定理}
  \author{陈建文}
  \maketitle
  % \tableofcontents
  \begin{Def}
    设$G$为一个群,$G$的任意子集称为群子集。在$2^G$中借助于$G$的乘法引入一个代数运算,称为群子集的乘法:$\forall A,B\in 2^G$,
    \[AB=\{ab|a\in A \text{且} b\in B\}\]
    $\forall g\in G,A\in 2^G,\{g\}A$简写为$gA$,即$gA=\{ga|a\in A\}$。
  \end{Def}
  \begin{Thm}
    设$G$为一个群,则$\forall A,B,C\in 2^G$,$(AB)C=A(BC)$。
  \end{Thm}
\begin{Def}
  设$H$为群$G$的一个子群,$a\in G$,则集合$aH$称为子群$H$的一个左陪集,$Ha$称为$H$的一个右陪集。
\end{Def}

\begin{Thm}
  设$H$为群$G$的一个子群,则$\forall a\in G$,$aH=H$的充分必要条件是$a\in H$。
\end{Thm}
  
\begin{Thm}
  设$H$为群$G$的一个子群,则$\forall a,b\in G$,$aH=bH$的充分必要条件是$a^{-1}b\in H$。
\end{Thm}

\begin{Thm}
  设$H$为群$G$的一个子群,则$\forall a,b\in G$,$aH=bH$或者$aH\cap bH=\phi$。
\end{Thm}

\begin{Thm}
  设$H$为群$G$的一个子群,则$\forall a,b\in G$,$|aH|=|bH|$。
\end{Thm}

\begin{Thm}
  设$H$为群$G$的一个子群,则$H$的所有左陪集构成的集合为$G$的一个划分。
\end{Thm}

\begin{Def}
  设$H$为群$G$的一个子群,如果$H$的所有不同的左陪集的个数为有限数$j$,则称$j$为$H$在$G$中的指数,记为$j=[G:H]$,否则称$H$在$G$中的指数为无穷大。
\end{Def}

\begin{Thm}
 设$G$为一个有限群,$H$为$G$的一个子群,则$|G|=|H|\cdot [G:H]$。 
\end{Thm}

\begin{Cor}
有限群中每个元素的阶都能整除该有限群的阶。
\end{Cor}

\begin{Cor}
  如果群$G$的阶为素数,则$G$为一个循环群。
\end{Cor}

\begin{Cor}
  设$G$为一个群,则$\forall a\in G$,$a^{|G|}=e$。
\end{Cor}

\begin{Example}
  阶小于等于$5$的群为交换群。
\end{Example}

\begin{Thm}
  设$H$为群$G$的一个子群,$S_l$为$H$的所有左陪集构成的集合,$S_r$为$H$的所有右陪集构成的集合,则$|S_l|=|S_r|$。
\end{Thm}

\begin{Thm}
  设$p$为素数,整数$a$与$p$互素,则$a^{p-1}\equiv 1 \pmod p$。
\end{Thm}
\begin{proof}[证明]
  以下证明$Z_p\setminus \{[0]\}=\{[1],[2],\cdots,[p-1]\}$对于乘法运算“$\cdot$”构成一个群。

  其中的乘法运算“$\cdot$”定义为:$\forall i,j\in Z$,$[i]\cdot [j]=[ij]$。

  $\forall i,j,i',j'\in Z$,如果$[i]=[i']$,$[j]=[j']$,则$[ij]=[i'j']$,这验证了“$\cdot$”为一个运算。

  $\forall i,j\in Z$,如果$[i]\neq [0]$,$[j]\neq 0$,则$p\nmid i$,$p\nmid j$,从而$p\nmid ij$,于是$[i]\cdot[j]=[ij]\neq [0]$,这验证了运算“$\cdot$”在$Z_p\setminus \{[0]\}$中封闭。

  $\forall i,j,k\in Z$,$([i]\cdot [j])\cdot [k]=[ij]\cdot [k]=[(ij)k]$,$[i]\cdot ([j]\cdot [k])=[i]\cdot [jk]=[i(jk)]$,$([i]\cdot [j])\cdot [k]=[i]\cdot ([j]\cdot [k])$,这验证了乘法运算“$\cdot$”满足结合律。

  $\forall i\in Z$,$[1]\cdot [i] =[i]$,这验证了$[1]$为左单位元。

  $\forall i\in Z$,$[i]\neq [0]$,则$(i,p)=1$,从而$\exists s,t\in Z,si+tp=1$,于是$p|(si-1)$,所以$[si]=[1]$,即$[s][i]=[1]$,这说明$[i]$有左逆元。

  以上验证了$Z_p\setminus \{[0]\}$对于乘法运算“$\cdot$”构成一个群。

  $\forall a\in Z$,如果$a$与$p$互素,则$[a]\in Z_p\setminus \{[0]\}$,从而$[a]^{p-1}=[1]$,$a^{p-1}\equiv 1 \pmod {p}$。
\end{proof}
RSA算法:

(1)随机选择两个大的素数$p$和$q$;

(2)计算$n=pq$;

(3)选择数$e$,使得$e$与$(p-1)(q-1)$互素;

(4)计算数$d$,使得对于某个整数$k$,$ed=1+k(p-1)(q-1)$;

(5)将$(e,n)$作为公钥发布,保留私钥$(d,n)$。

设待加密的明文为$M$,$M<n$。

加密过程:$C=M^e \mod n$;

解密过程:$M=C^d \mod n$。


\begin{Thm}
  在以上描述的RSA算法中,$(M^e \bmod n)^d \bmod n=M$。
\end{Thm}

\begin{proof}[证明]
  由于$(M^e \mod n)^d \bmod n &= (M^e)^d \bmod n=m^{ed}\bmod n$,
  因此只需证$M^{ed} \bmod n=M$。
  当$M$与$p$互素时,
\begin{align*}
  &M^{ed}\mod p\\
  =&M^{1+k(p-1)(q-1)} \bmod p\\
  =&M(M^{p-1})^{k(q-1)} \bmod p\\
  =&M(1)^{k(q-1)} \bmod p\\
  =&M \bmod p\\
\end{align*}
于是$M^{ed}\equiv M \pmod p$。当$p|M$时,该式显然也成立。

同理可证$M^{ed}\equiv M \pmod q$,进一步可得$M^{ed}\equiv M \pmod{pq}$,
即$M^{ed}\equiv M \pmod n$,从而$M^{ed} \bmod n = M\bmod n=M$。
\end{proof}
课后作业题:
\begin{Exercise}
证明:六阶群里必有一个三阶子群。
\end{Exercise}

\begin{Exercise}
设$p$为一个素数,证明:在阶为$p^m$的群里一定含有一个$p$阶子群,其中$m\geq 1$。
\end{Exercise}

\begin{Exercise}
在三次对称群$S_3$中,找一个子群$H$,使得$H$的左陪集不等于$H$的右陪集。
\end{Exercise}
\begin{Exercise}
设$H$为群$G$的一个子群,如果左陪集$aH$等于右陪集$Ha$,即$aH=Ha$,则$\forall h\in H, ah=ha$一定成立吗?
\end{Exercise}
\end{CJK*}
\end{document}





%%% Local Variables:
%%% mode: latex
%%% TeX-master: t
%%% End:



