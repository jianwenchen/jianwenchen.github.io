\documentclass{article}
\usepackage{CJKutf8}
\usepackage{amsmath}
\usepackage{amssymb}
\usepackage{amsfonts}
\usepackage{amsthm}
\usepackage{titlesec}
\usepackage{titletoc}
\usepackage{xCJKnumb}
\usepackage{tikz}
\usepackage{mathrsfs}
\usepackage{indentfirst}

\newtheorem{Def}{定义}
\newtheorem{Thm}{定理}
\newtheorem{Exercise}{练习}

\newtheorem*{Example}{例}

\newtheorem{Cor}{推论}
\begin{document}
\begin{CJK*}{UTF8}{gbsn}
  \title{第七讲 陪集、拉格朗日定理}
  \author{陈建文}
  \maketitle
  % \tableofcontents
  \begin{Def}
    设$G$为一个群,$G$的任意子集称为群子集。在$2^G$中借助于$G$的乘法引入一个代数运算,称为群子集的乘法:$\forall A,B\in 2^G$,
    \[AB=\{ab|a\in A \text{且} b\in B\}\]
    $\forall g\in G,A\in 2^G,\{g\}A$简写为$gA$,即$gA=\{ga|a\in A\}$。
  \end{Def}

\begin{Def}
  设$H$为群$G$的一个子群,$a\in G$,则集合$aH$称为子群$H$的一个左陪集,$Ha$称为$H$的一个右陪集。
\end{Def}

\begin{Thm}
  设$H$为群$G$的一个子群,则$\forall a\in G$,$aH=H$的充分必要条件是$a\in H$。
\end{Thm}
  
\begin{Thm}
  设$H$为群$G$的一个子群,则$\forall a,b\in G$,$aH=bH$的充分必要条件是$a^{-1}b\in H$。
\end{Thm}

\begin{Thm}
  设$H$为群$G$的一个子群,则$\forall a,b\in G$,$aH=bH$或者$aH\cap bH=\phi$。
\end{Thm}

\begin{Thm}
  设$H$为群$G$的一个子群,则$\forall a,b\in G$,$|aH|=|bH|$。
\end{Thm}

\begin{Thm}
  设$H$为群$G$的一个子群,则$H$的所有左陪集构成的集合为$G$的一个划分。
\end{Thm}

\begin{Def}
  设$H$为群$G$的一个子群,如果$H$的所有不同的左陪集的个数为有限数$j$,则称$j$为$H$在$G$中的指数,记为$j=[G:H]$,否则称$H$在$G$中的指数为无穷大。
\end{Def}

\begin{Thm}
 设$G$为一个有限群,$H$为$G$的一个子群,则$|G|=|H|\cdot [G:H]$。 
\end{Thm}

\begin{Cor}
有限群中每个元素的阶都能整除该有限群的阶。
\end{Cor}

\begin{Cor}
  如果群$G$的阶为素数,则$G$为一个循环群。
\end{Cor}

\begin{Cor}
  设$G$为一个群,则$\forall a\in G$,$a^{|G|}=e$。
\end{Cor}

\begin{Example}
  阶小于等于$5$的群为交换群。
\end{Example}

\begin{Thm}
  设$H$为群$G$的一个子群,$S_l$为$H$的所有左陪集构成的集合,$S_r$为$H$的所有右陪集构成的集合,则$|S_l|=|S_r|$。
\end{Thm}

\begin{Def}
  设$a,b,n\in Z$,$n>0$,如果$n|(a-b)$,则称$a$与$b$模$n$同余,记为$a \equiv b(\mod n)$。
\end{Def}

\begin{Thm}
  $\forall a,b\in Z, a\equiv b(\mod n)$等价于$a\mod n = b \mod n$。
\end{Thm}

\begin{Thm}
  1. $\forall a\in Z, a \equiv a (\mod n)$;

  2. $\forall a, b\in Z$,如果$a\equiv b (\mod n)$,则$b\equiv a(\mod n)$;

  3. $\forall a,b,c\in Z$,如果$a\equiv b(\mod n)$并且$b\equiv c(\mod n)$,则$a\equiv c(\mod n)$;

  4. $\forall a,b,k\in Z$,如果$a\equiv b(\mod n)$,则$a+k\equiv b+k(\mod n)$;

  5. $\forall a,b,c,d\in Z$,如果$a\equiv b(\mod n)$并且$c\equiv d(\mod n)$,则$a+c\equiv b+d (\mod n)$;

  6. $\forall a,b,k\in Z$,如果$a\equiv b(\mod n)$,则$ak\equiv bk(\mod n)$;

  7. $\forall a,b,c,d\in Z$,如果$a\equiv b(\mod n)$并且$c\equiv d(\mod n)$,则$ac\equiv bd (\mod n)$

  8. $\forall a,b\in Z$,$ab (\mod n)=(a\mod n)(b\mod n) (\mod n)$。
\end{Thm}

RSA算法:

(1)随机选择两个大的素数$p$和$q$;

(2)计算$n=pq$;

(3)选择数$e$,使得$e$与$(p-1)(q-1)$互素;

(4)计算数$d$,使得对于某个整数$k$,$ed=1+k(p-1)(q-1)$;

(5)将$(e,n)$作为公钥发布,保留私钥$(d,n)$。

设待加密的明文为$M$,$M<n$。

加密过程:$C=M^e \mod n$;

解密过程:$M=C^d \mod n$。


\begin{Thm}
  在以上描述的RSA算法中,$(M^e \mod n)^d \mod n=M$。
\end{Thm}

\begin{proof}[证明]
  由于$(M^e \mod n)^d \mod n &= (M^e)^d \mod n=m^{ed}\mod n$,
  因此只需证$M^ed \mod n=M$。
  当$M$与$p$互素时,
\begin{align*}
  &M^{ed}\mod p\\
  =&M^{1+k(p-1)(q-1)} \mod p\\
  =&M(M^{p-1})^{k(q-1)} \mod p\\
  =&M(1)^{k(q-1)} \mod p\\
  =&M \mod p\\
\end{align*}
于是$(M^{ed})\equiv M (\mod p)$。当$p|M$时,该式显然也成立。

同理可证$(M^{ed})\equiv M (\mod q)$,进一步可得$M^{ed}\equiv M (\mod pq)$,
即$M^{ed}\equiv M (\mod n)$,从而$M^{ed} \mod n = M\mod n=M$。
\end{proof}
课后作业题:
\begin{Exercise}
证明:六阶群里必有一个三阶子群。
\end{Exercise}

\begin{Exercise}
设$p$为一个素数,证明:在阶为$p^m$的群里一定含有一个$p$阶子群,其中$m\geq 1$。
\end{Exercise}

\begin{Exercise}
在三次对称群$S_3$中,找一个子群$H$,使得$H$的左陪集不等于$H$的右陪集。
\end{Exercise}
\begin{Exercise}
设$H$为群$G$的一个子群,如果左陪集$aH$等于右陪集$Ha$,即$aH=Ha$,则$\forall h\in H, ah=ha$一定成立吗?
\end{Exercise}
\end{CJK*}
\end{document}





%%% Local Variables:
%%% mode: latex
%%% TeX-master: t
%%% End:



