\documentclass{article}
\usepackage{CJKutf8}
\usepackage{amsmath}
\usepackage{amssymb}
\usepackage{amsfonts}
\usepackage{amsthm}
\usepackage{titlesec}
\usepackage{titletoc}
\usepackage{xCJKnumb}
\usepackage{tikz}
\usepackage{mathrsfs}
\usepackage{indentfirst}

\newtheorem{Def}{定义}
\newtheorem{Thm}{定理}
\newtheorem{Cor}{推论}
\newtheorem{Exercise}{练习}

\newtheorem*{Example}{例}


\begin{document}
\begin{CJK*}{UTF8}{gbsn}
  \title{第四讲 子群、生成子群}
  \author{陈建文}
  \maketitle
  % \tableofcontents
  


课后作业题:
\begin{Exercise}
举例说明两个子群的并可以不是子群。
\end{Exercise}
\begin{proof}[解]
  $S_6=\{[0],[1],[2],[3],[4],[5]\}$,

  $\{[0],[2],[4]\}$和$\{[0],[3]\}$为$S_6$的两个子群,
  但$\{[0],[2],[4]\}\cup \{[0],[3]\}=\{[0],[2],[3],[4]\}$不是$S_6$的子群,因为$[2]+[3]=[5]\notin \{[0],[2],[3],[4]\}$。
\end{proof}
\begin{Exercise}
  设$G_1$和$G_2$为群$G$的两个真子群,证明:$G_1\cup G_2$为$G$的子群的充分必要条件是$G_1\subseteq G_2$或者$G_2\subseteq G_1$。
\end{Exercise}
\begin{proof}[证明]
  如果$G_1\subseteq G_2$或者$G_2\subseteq G_1$,则$G_1\cup G_2=G_2$或$G_1$,此时显然$G_1\cup G_2$为$G$的子群。

  如果$G_1\cup G_2$为$G$的子群,以下用反证法证明$G_1\subseteq G_2$或者$G_2\subseteq G_1$。假设$G_1\nsubseteq G_2$并且$G_2\nsubseteq G_1$,则存在$g_2\in G_2$,但是$g_2\notin G_1$,同时存在$g_1\in G_1$,但是$g_1\notin G_2$。
  于是$G_1$为$G_1\cup G_2$的真子集,$G_2$为$G_1\cup G_2$的真子集,易得$G_1$和$G_2$为$G_1\cup G_2$的真子群,由于任一群不能是两个真子群的并,矛盾。
\end{proof}
\begin{Exercise}
  设$(G_1,\circ)$和$(G_2,*)$都是群,$\phi:G_1\to G_2$,$\forall a,b\in G_1$,$\phi(a\circ b)=\phi(a)*\phi(b)$,
  证明:$\phi^{-1}(e_2)$为$G_1$的子群,其中$e_2$为$G_2$的单位元素。
\end{Exercise}
\begin{proof}[证明]
  设$e_1$为$G_1$的单位元,则$\phi(e_1)=\phi(e_1\circ e_1)=\phi(e_1)*\phi(e_1)$,两边同时左乘$\phi(e_1)^{-1}$,得$\phi(e_1)=e_2$,
  从而$e_1\in \phi^{-1}(e_2)$,所以$\phi^{-1}(e_2)$非空。

  $\forall x,y\in \phi^{-1}(e_2)$,$\phi(x)=\phi(y)=e_2$,从而$\phi(x\circ y)=\phi(x)*\phi(y)=e_2*e_2=e_2$,故$x\circ y\in \phi^{-1}(e_2)$,于是$G$中的乘法在$\phi^{-1}(e_2)$中封闭。

$\forall x\in \phi^{-1}(e_2)$,$e_2=\phi(e_1)=\phi(x^{-1}\circ x)=\phi(x^{-1})*\phi(x)=\phi(x^{-1})*e_2=\phi(x^{-1})$,从而$x^{-1}\in \phi^{-1}(e_2)$。

以上证明了$\phi^{-1}(e_2)$为$G_1$的子群。
\end{proof}
\begin{Exercise}
  找出$3$次对称群的所有子群。
\end{Exercise}
\begin{proof}[解]
  $\{(1),(1,2)\}$,$\{(1),(1,3)\}$,$\{(1),(2,3)\}$,$\{(123),(132)\}$
\end{proof}

\begin{Exercise}
  令$P=\{(12),(123)\}\subseteq S_3$。写出由$P$生成的$S_3$的子群$(P)$。
\end{Exercise}
\begin{proof}[解]
  $(P)=S_3$。

  这可以由以下计算得到:

  $(12)(123)=(13)$,$(123)(12)=(23)$,$(123)(123)=(132)$。
\end{proof}
\end{CJK*}
\end{document}





%%% Local Variables:
%%% mode: latex
%%% TeX-master: t
%%% End:



