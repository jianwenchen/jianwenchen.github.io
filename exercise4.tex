\documentclass{article}
\usepackage{CJKutf8}
\usepackage{amsmath}
\usepackage{amsthm}
\begin{document}
\begin{CJK}{UTF8}{gbsn}
\newtheorem{Exercise}{习题}
\begin{Exercise}
设$A$为由序列$a_1,a_2,\cdots,a_n,\cdots$的所有项组成的集合,则$A$是否是可数的?为什么?
\end{Exercise}
\vspace{3cm}
\begin{Exercise}
  证明:直线上互不相交的开区间的全体所构成的集合至多是可数集。
\end{Exercise}
\vspace{3cm}
\begin{Exercise}
  证明:单调函数的不连续点的集合至多是可数集。
\end{Exercise}
\vspace{12cm}
\begin{Exercise}
  任一可数集$A$的所有有限子集构成的集族是可数集族。
\end{Exercise}
\vspace{10cm}
\begin{Exercise}
  判断下列命题之真伪:

 a) 若$f:X\to Y$且$f$是满射,则只要$X$是可数集,那么$Y$是至多可数的;

 b) 若$f:X\to Y$且$f$是单射,则只要$Y$是可数集,则$X$也是可数集;

 c) 可数集在任一映射下的像也是可数集。
\end{Exercise}
\vspace{10cm}
\begin{Exercise}
  设$\sum$为一个有限字母表,$\sum$上所有字(包括空字$\epsilon$)之集记为$\sum^*$。证明$\sum^*$是可数集。
  ($n$元组$(c_1,c_2,\cdots,c_n)$称为$\sum$上的一个字,这里$c_i\in \sum, 1\leq i\leq n$,$\epsilon=()$称为$\sum$上的一个空字)。
\end{Exercise}
\vspace{10cm}
\begin{Exercise}
  利用康托的对角线法证明所有0,1的无穷序列是不可数集。
\end{Exercise}
\vspace{5cm}
\begin{Exercise}
  证明:如果$A$是可数集,则$2^A$不是可数集。
\end{Exercise}
\end{CJK}
\end{document}


%%% Local Variables:
%%% mode: latex
%%% TeX-master: t
%%% End:
