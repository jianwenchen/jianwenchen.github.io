\documentclass{article}
\usepackage{CJKutf8}
\usepackage{amsmath}
\usepackage{amssymb}
\usepackage{amsfonts}
\usepackage{amsthm}
\usepackage{titlesec}
\usepackage{titletoc}
\usepackage{xCJKnumb}
\usepackage{tikz}
\usepackage{mathrsfs}
\usepackage{indentfirst}

\newtheorem{Def}{定义}
\newtheorem{Thm}{定理}
\newtheorem{Exercise}{练习}

\newtheorem*{Example}{例}
\newtheorem{Cor}{推论}

\begin{document}
\begin{CJK*}{UTF8}{gbsn}
  \title{第五讲 变换群、同构}
  \author{陈建文}
  \maketitle
  % \tableofcontents
 

  
课后作业题:
\begin{Exercise}
设$R$为实数集合,$G$为一切形如$f(x)=ax+b$的从$R$到$R$的函数之集,这里$a\in R$,$b\in R$,$a\neq 0$,试证:$G$为一个变换群。
\end{Exercise}
\begin{proof}[证明]
  显然$G$中的每个函数都为从$R$到$R$的双射。
  
  设$f(x)=ax+b$,$g(x)=cx+d$,$a,b,c,d\in R$,$a\neq 0$,$c\neq 0$,则
  $(f\circ g)(x)=a(cx+d)+b=acx+(ad+b)$,这里$ac\neq 0$,因此$f\circ g\in G$。
这验证了$G$中的函数关于函数的合成满足封闭性。

设$h:R\to R,h(x)=x$,则$h\in G$,$h$为$G$中的函数关于函数合成运算的单位元。

对任意的$f\in G$,设$f=ax+b$,$a,b\in R$,$a\neq 0$。

寻找$g\in G$,使得$g\circ f=h$。设$g(x)=cx+d$,$c,d\in R$,$c\neq 0$,

则$(g\circ f)(x)=c(ax+b)+d=cax+(cb+d)=x$,解方程组
\begin{equation*}
  \left\{
    \begin{array}{rl}
      &ca=1\\
      &cb+d=0
    \end{array}
  \right.
\end{equation*}
% \begin{cases}
%   &ca=1\\
%   &cb+d=0\\
% \end{cases}
得$c=\frac{1}{a},d=-\frac{b}{a}$,
易验证$g(x)=\frac{1}{a}x-\frac{b}{a}$满足$(g\circ f)(x)=x$。

\end{proof}
\begin{Exercise}
  设$R$为实数集合,$H$为一切形如$f(x)=x+b$的从$R$到$R$的函数之集,这里$b\in R$,试证:$H$为上题中$G$的一个子群。
\end{Exercise}
\begin{proof}[证明]
  显然$H$非空,例如$h:R\to R$,$\forall x\in R, h(x)=x$,则$h\in H$。

  $\forall f,g\in H$,$f(x)=x+b$,$g(x)=x+c$,$b,c\in R$,则$(f\circ g^{-1})(x)=(x-c)+b=x+(b-c)\in H$,因此$H$为上题中$G$的一个子群。
\end{proof}
\begin{Exercise}
设$R_+$为一切正实数之集,$R$为一切实数之集。$(R_+,\times)$,$(R,+)$都为群。令$\phi:R_+\to R,\forall x\in R_+,\phi(x)=log_p(x)$,其中$p$为任意一个正实数。证明$\phi$为同构。
\end{Exercise}
\begin{proof}[证明]
  显然$\phi$为从$R_+$到$R$的双射。

  其次,$\phi(x\times y)=log_p(x\times y)=\log_p(x)+\log_p(y)=\phi(x)+\phi(y)$。

  因此,$\phi$为从$(R_+,\times)$到$(R,+)$的同构。
\end{proof}
\end{CJK*}
\end{document}





%%% Local Variables:
%%% mode: latex
%%% TeX-master: t
%%% End:



