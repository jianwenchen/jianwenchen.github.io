\documentclass{article}
\usepackage{CJKutf8}
\usepackage{amsmath}
\usepackage{amssymb}
\usepackage{amsfonts}
\usepackage{amsthm}
\usepackage{titlesec}
\usepackage{titletoc}
\usepackage{xCJKnumb}
\usepackage{tikz}
\usepackage{mathrsfs}
\usepackage{indentfirst}

\newtheorem{Def}{定义}
\newtheorem{Thm}{定理}
\newtheorem{Cor}{推论}
\newtheorem{Exercise}{练习}

\newtheorem*{Example}{例}


\begin{document}
\begin{CJK*}{UTF8}{gbsn}
  \title{第十一讲 无零因子环的特征数}
  \author{陈建文}
  \maketitle
  % \tableofcontents
  

在初等代数中,如果$a\neq 0$,则$na=\underbrace{a+a+\cdots+a}_\text{n}\neq 0$,
这个结论在一般的域中成立吗?

\begin{Example}
  设$p$为一个素数,则模$p$同余类环$Z_p=\{[0],[1],\cdots,[p-1]\}$为一个域。在域$Z_p$中,
  同余类$[1]\neq [0]$,但是
  \[p[1]=\underbrace{[1]+[1]+\cdots+[1]}_\text{p个[1]}=[0]\]
  而且$\forall [i]\in Z_p$,$p[i]=[pi]=[0]$。$[0]$为域$Z_p$中的零元,在$Z_p$中任一非零元的$p$倍等于零元。
\end{Example}

在上例中,任意一个非零元对加法的阶都等于$p$,在一般的域中,这个结论成立吗?

下面的例子说明在一般的环中,该结论不成立。

\begin{Example}
  在环$Z_4=\{[0],[1],[2],[3]\}$中,$[1]$对加法的阶为$4$,$[2]$对加法的阶为$2$。
\end{Example}

\begin{Thm}
在一个无零因子环中,每个非零元素对加法的阶均相等。
\end{Thm}

\begin{Cor}
  体和域中每个非零元素对加法的阶均相等。
\end{Cor}

\begin{Def}
  无零因子环中非零元素对加法的阶称为该环的特征数,简称特征;域(体)中非零元素对加法的阶称为该域(体)的特征数,简称特征。
\end{Def}

\begin{Thm}
  如果一个无零因子环$R$的特征数为正整数$p$,则$p$为素数。
\end{Thm}

\begin{Thm}
  在特征为$p$的域中,
  \begin{align*}
    (a+b)^p=&a^p+b^p\\
    (a-b)^p=&a^p-b^p\\
  \end{align*}
\end{Thm}
\begin{proof}[证明]

由
  \[a^p=(a-b+b)^p=(a-b)^p+b^p\]
可得
\[(a-b)^p=a^p-b^p\]
\end{proof}
课后作业题:
\begin{Exercise}
设$F$为一个域,$|F|=4$,证明:

(1)$F$的特征数为$2$;

(2)$F$的任意一个非零元并且非单位元$1$的元素$x$均满足方程$x^2=x+1$;

(3)列出$F$的加法表和乘法表。
\end{Exercise}


\end{CJK*}
\end{document}





%%% Local Variables:
%%% mode: latex
%%% TeX-master: t
%%% End:



