\documentclass{article}
\usepackage{CJKutf8}
\usepackage{amsmath}
\usepackage{amssymb}
\usepackage{amsfonts}
\usepackage{amsthm}
\usepackage{titlesec}
\usepackage{titletoc}
\usepackage{xCJKnumb}
\usepackage{tikz}
\usepackage{mathrsfs}
\usepackage{indentfirst}

\newtheorem{Def}{定义}
\newtheorem{Thm}{定理}
\newtheorem{Exercise}{练习}

\newtheorem*{Example}{例}


\begin{document}
\begin{CJK*}{UTF8}{gbsn}
  \title{第一讲 若干基本概念}
  \author{陈建文}
  \maketitle
  % \tableofcontents
  

\section{近世代数的起源}


$ax+b=0$

$ax^2+bx+c=0$

$ax^3+bx^2+cx+d=0$

$ax^4+bx^3+cx^2+dx+e=0$

$ax^5+bx^4+cx^3+dx^2+ex+f=0$

Abel(1802-1829):证明了一般的次数$\geq 5$的一元方程没有用$+,-,*,/,\sqrt[n]{\quad}$表示的求根公式。

Crelle

Galois(1811-1832):

$x^5=1$

解决了哪些次数$\geq 5$的一元方程有求根公式,哪些没有的问题,构思了群的概念

Liouville

群的概念来源于三个主要的数学领域:代数方程论,几何,数论

Cantor(1845-1918):创立集合论

Noether(1882-1935):现代代数学之母

van der Waerden:Modern Algebra

Abstract Algebra

Basic Algebra

Algebra



\section{运算}
\begin{Def}
  设$X$为一个非空集合,一个从$X\times X$到$X$的映射$\phi$称为集合$X$上的一个二元代数运算。
\end{Def}
  
注:设$X,Y,Z$为任意三个非空集合,一个从$X\times Y$到$Z$的映射$\phi$称为从$X$与$Y$到$Z$的一个二元代数运算。

\begin{Def}
  设$X$为一个非空集合,一个从$X$到$X$的映射$\phi$称为集合$X$上的一个一元运算。
\end{Def}

注:设$X,Y$为任意两个非空集合,一个从$X$到$Y$的映射$\phi$称为从$X$到$Y$的一个一元运算。

\begin{Def}
  设“$\circ$”为非空集合$S$上的一个二元代数运算,则称二元组$(S,\circ)$为一个(有一个代数运算的)代数系。
\end{Def}
类似的,可以定义具有两个代数运算的代数系$(S,\circ,*)$,具有三个代数运算的代数系$(S,\circ,*,+)$,等等。

我们熟知的实数集$R$,与其上的加法运算"$+$"和乘法运算"$*$"一起构成了一个代数系,满足如下性质:
   \begin{enumerate} 
   \item   对任意的$x\in R$,$y\in R$,$z\in R$,$(x + y) + z = x + (y + z)$
   \item   对任意的$x\in R$,$0 + x = x + 0 = x$
   \item   对任意的$x\in R$,$(-x) + x =x + (-x) = 0$
   \item   对任意的$x\in R$,$y\in R$,$x + y = y + x$ 
   \item   对任意的$x\in R$,$y\in R$,$z\in R$,$(x * y) * z = x * (y *z)$
   \item   对任意的$x\in R$,$1 * x = x * 1 = x$
   \item   对任意的$x\in R$,$x\neq 0 \to x^{-1} * x = x * x^{-1} = 1$
   \item   对任意的$x\in R$,$y\in R$,$x * y = y * x$
   \item   对任意的$x\in R$,$y\in R$,$z\in R$,$x* (y + z) = x * y + x * z$
   \item   对任意的$x\in R$,$y\in R$,$z\in R$,$(y + z) * x = y * x + z * x$
   \item 对任意的$x\in R$,$x\leq x$。
   \item 对任意的$x\in R$,$y\in R$,如果$x\leq y$并且$y\leq x$,则$x=y$。 
  \item 对任意的$x\in R$,$y\in R$,$z\in R$,如果$x\leq y$并且$y\leq z$,则$x\leq z$。
  \item 对任意的$x\in R$,$y\in R$,$x\leq y$和$y\leq x$两者中必有其一成立。
  
  我们用$x<y$表示$x\leq y$并且$x\neq y$,$x\geq y$表示$y\leq x$,$x > y$表示$x\geq y$并且$x\neq y$。
  
  \item 对任意的$x\in R$,$y\in R$,$z\in R$,如果$x<y$,则$x+z<y+z$。
  \item 对任意的$x\in R$,$y\in R$,如果$x>0$,$y>0$,则$xy>0$。
  \item 设$A_1$, $A_2$,$\cdots$,$A_i$,$\cdots$为实数集$R$上的闭区间,$A_1\supseteq A_2 \supseteq A_3 \supseteq \cdots \supseteq A_i \supseteq \cdots$,则$\bigcap_{i=1}^{\infty}A_i$非空。
  \end{enumerate}


  \begin{Def}
    设“$\circ$”为集合$S$上的一个二元代数运算。如果$\forall a, b, c \in S$,$(a \circ b) \circ c = a \circ (b \circ c)$, 则称二元代数运算“$\circ$”满足结合律。
  \end{Def}
\begin{Thm}
  设$(S,\circ)$为一个代数系,如果二元代数运算“$\circ$”满足结合律,则$\forall a_i\in S$,$i=1,2,\cdots,n$,$n$个元素$a_1,a_2,\cdots,a_n$的乘积由它们的次序唯一确定。
\end{Thm}
\begin{proof}[证明]
  用$a_1\circ a_2\circ \cdots \circ a_n$表示按照$a_1,a_2, \cdots, a_n$的次序进行“$\circ$”运算时任意加括号所得到的运算结果。

  以下用数学归纳法证明$a_1\circ a_2\circ \cdots \circ a_n=(((a_1\circ a_2)\circ a_3)\circ \cdots )\circ a_n$。

  当$n=1$时结论显然成立。

  假设当$n<k$时结论成立,往证当$n=k$时结论也成立。

  对$k$个元素按$a_1,a_2,\cdots,a_k$的次序不论用什么方法加括号确定计算方案,最后一步必是两个元素的乘积,不妨设为$b_1\circ b_2$,这里$b_1$为前$i$个元素$a_1,a_2,\cdots,a_i$之积,
  而$b_2$为后$k-i$个元素$a_{i+1},\cdots,a_k$之积。

  \begin{align*}
    b_1\circ b_2 = &((((a_1\circ a_2)\circ a_3)\circ \cdots )\circ a_i)\circ ((((a_{i+1}\circ a_{i+2})\circ a_{i+3})\circ \cdots )\circ a_k)\\
                = &(((((a_1\circ a_2)\circ a_3)\circ \cdots )\circ a_i)\circ ((((a_{i+1}\circ a_{i+2})\circ a_{i+3})\circ \cdots )\circ a_{k-1})))\circ a_k\\
                =&((((a_1\circ a_2)\circ a_3)\circ \cdots )\circ a_{k-1})\circ a_k\\
  \end{align*}
\end{proof}

Scala: Martin Ordersky

C++ STL: Alexander Stepanov
 \begin{Def}
    设“$\circ$”为集合$S$上的一个二元代数运算。如果$\forall a, b \in S$,\\$a \circ b = b \circ a$, 则称二元代数运算“$\circ$”满足交换律。
  \end{Def}
  \begin{Thm}
    设$(S,\circ)$为一个代数系,如果二元代数运算“$\circ$”满足结合律和交换律,则$\forall a_i\in S$,$i=1,2,\cdots,n$,$n$个元素$a_1,a_2,\cdots,a_n$的乘积仅与这$n$个元素有关而与它们的次序无关。
  \end{Thm}
  \begin{proof}[证明]
留作课后作业题。    
  \end{proof}
  \begin{Def}
    设“$+$”与“$\circ$”为集合$S$上的两个二元代数运算。\\如果$\forall a, b, c \in S$,\[a \circ (b + c) = a \circ b + a \circ c,\] 则称二元代数运算“$\circ$”对“$+$”满足左分配律。
    如果$\forall a, b, c \in S$,\[(b + c)\circ a = b \circ a + c \circ a,\] 则称二元代数运算“$\circ$”对“$+$”满足右分配律。
  \end{Def}
  \begin{Thm}
    设$(S,+,\circ)$为具有两个二元代数运算的代数系,“$+$”满足结合律。如果“$\circ$”对“$+$”满足左分配律,则对任意的$a,a_i\in S$,$i=1,2,\cdots,n$,有
    \[a\circ (a_1+a_2+\cdots+a_n) = a\circ a_1 + a\circ a_2 + \cdots + a\circ a_n\]
    如果“$\circ$”对“$+$”满足右分配律,则对任意的$a,a_i\in S$,$i=1,2,\cdots,n$,有
    \[(a_1+a_2+\cdots+a_n)\circ a  = a_1\circ a + a_2\circ a + \cdots + a_n\circ a\]
  \end{Thm}
  \begin{Def}
    设$(S,\circ)$为一个代数系。如果存在一个元素$e_l\in S$,使得$\forall a\in S$,
    \[e_l\circ a = a\]
    则称$e_l$为“$\circ$”运算的左单位元素;如果存在一个元素$e_r\in S$,使得$\forall a\in S$,
    \[a\circ e_r = a\]
    则称$e_r$为“$\circ$”运算的右单位元素;如果存在一个元素$e\in S$,使得$\forall a\in S$,
    \[e\circ a = a\circ e = a\]
    则称$e$为“$\circ$”运算的单位元素。
  \end{Def}
  \begin{Thm}
    设$(S,\circ)$为一个代数系,如果二元代数运算$\circ$既有左单位元$e_l$,又有右单位元$e_r$,则$e_l=e_r$,从而有单位元。
  \end{Thm}
  \begin{proof}[证明]
    $e_r = e_l\circ e_r = e_l$
  \end{proof}
  \section{课后作业题}
  \begin{Exercise}
    设$(S,\circ)$为一个代数系,如果二元代数运算“$\circ$”满足结合律和交换律,则$\forall a_i\in S$,$i=1,2,\cdots,n$,$n$个元素$a_1,a_2,\cdots,a_n$的乘积仅与这$n$个元素有关而与它们的次序无关。
  \end{Exercise}


\end{CJK*}
\end{document}





%%% Local Variables:
%%% mode: latex
%%% TeX-master: t
%%% End:



