\documentclass{article}
\usepackage{CJKutf8}
\usepackage{amsmath}
\usepackage{amsthm}
\begin{document}
\begin{CJK}{UTF8}{gbsn}
\newtheorem{Exercise}{习题}
\begin{Exercise}
设$A$为由序列$a_1,a_2,\cdots,a_n,\cdots$的所有项组成的集合,则$A$是否是可数的?为什么?
\end{Exercise}
解:$A$不一定可数,例如当$a_1=a_2=\cdots=1$时,$A=\{1\}$为有穷集合。
\begin{Exercise}
  证明:直线上互不相交的开区间的全体所构成的集合至多是可数集。
\end{Exercise}
\begin{proof}[证明]
  设$A$为直线上某个互不相交的开区间构成的集合,在每个开区间中取一个有理数,
  则$A$与有理数集合的一个子集之间存在一一对应的关系,于是$A$为至多可数集。
\end{proof}
\begin{Exercise}
  证明:单调函数的不连续点的集合至多是可数集。
\end{Exercise}

首先来看结论中涉及的一些基本概念。

设$f:R\to R$为一个函数。如果对任意的$x_1\in R$,$x_2\in R$,$x_1< x_2$,那么$f(x_1) \leq f(x_2)$,则称$f$为单调函数。

设$x_0\in R$,$L\in R$,如果对任意的$\varepsilon\in R$, $\varepsilon> 0$,存在$\delta \in R$, $\delta > 0$,只要$|x-x_0|<\delta$,就有$|f(x) - L|<\varepsilon$,则称函数$f$在$x_0$处的极限为$L$,记为$\lim_{x\to x_0}f(x)=L$。

如果$\lim_{x\to x_0}f(x)=f(x_0)$,则称函数$f$在$x_0$处连续,$x_0$为函数$f$的连续点;如果函数$f$在$x_0$处不连续,则称$x_0$为函数$f$的不连续点。


\begin{proof}[证明]
  对任意的$x_0\in R$,由单调函数的定义知,集合$\{f(x)|x<x_0\}$有上界$f(x_0)$,从而有上确界,定义$L(x_0)=sup \{f(x)|x<x_0\}$;集合$\{f(x)|x>x_0\}$有下界$f(x_0)$,从而有下确界,定义$U(x_0)=inf \{f(x)|x>x_0\}$。如果$x_1<x_2$,那么$U(x_1)\leq f(\frac{x_1+x_2}{2}) \leq L(x_2)$。另外,如果$x_0$为$f$的不连续点,可以证明$L(x_0) <  U(x_0)$。因此,集合$S=\{(L(x),U(x))|x\text{为函数}$f$\text{的不连续点}\}$中的开区间两两不相交。在$S$中的每个开区间中取一个有理数,则所有这些有理数的集合与函数$f$的所有不连续点构成的集合是对等的,从而$f$的所有不连续点所构成的集合为至多可数的。

  设$x_0$为$f$的不连续点,以下证明$L(x_0) <  U(x_0)$。由$L(x_0)\leq f(x_0) \leq U(x_0)$知 $L(x_0) \leq U(x_0)$,因此只需证$L(x_0)\neq U(x_0)$。用反证法,假设$L(x_0)=U(x_0)$,则$L(x_0)=U(x_0)=f(x_0)$。对任意的$\varepsilon >0$,由$L(x_0)$的定义知存在$x'<x_0$使得$f(x')>L(x_0)-\varepsilon=f(x_0)-\varepsilon$;由$U(x_0)$的定义知存在$x''>x_0$使得$f(x'')<U(x_0) + \varepsilon=f(x_0) + \varepsilon$。设$\delta = \min (|x'-x_0|, |x''-x_0|)$,那么当$|x-x_0|< \delta$时,就有$|f(x)-f(x_0)|<\varepsilon$,从而$\lim_{x\to x_0}=f(x_0)$,函数$f$在$x_0$处连续,这与$x_0$为$f$的不连续点矛盾。
\end{proof}

\clearpage
\begin{Exercise}
  任一可数集$A$的所有有限子集构成的集族是可数集族。
\end{Exercise}
\begin{proof}[证明]
  因为A为可数集合,所以A中元素可以排成无重复项的序列:

  $a_1,a_2,\ldots,a_n,\ldots$

令$S=\{B|B\subseteq A,B\text{为有穷集}\}$,Q为有理数集,
定义映射$\phi:S\to Q$,对任意的$B\in S$,$\phi(B)=0.b_1b_2\ldots b_n\ldots$,
这里
\[
b_i=\begin{cases}
  1&\text{如果}a_i\in B\\
  0&\text{如果}a_i\notin B\\
\end{cases}  
\]
则对任意的$B_1\in S, B_2\in S$,如果$B_1\neq B_2$,则$\phi(B_1)\neq \phi(B_2)$,即$\phi$为从S到Q的单射。$\phi(S)$为可数集Q的无限子集,从而也为可数集。$\phi$为从S到$\phi(S)$的双射,因此S为可数集。
\end{proof}
\begin{Exercise}
  判断下列命题之真伪:

 a) 若$f:X\to Y$且$f$是满射,则只要$X$是可数集,那么$Y$是至多可数的;

 b) 若$f:X\to Y$且$f$是单射,则只要$Y$是可数集,则$X$也是可数集;

 c) 可数集在任一映射下的像也是可数集。
\end{Exercise}
\begin{proof}[解]
a)真,b)c)伪。
\end{proof}
\begin{Exercise}
  设$\sum$为一个有限字母表,$\sum$上所有字(包括空字$\epsilon$)之集记为$\sum^*$。证明$\sum^*$是可数集。
  ($n$元组$(c_1,c_2,\cdots,c_n)$称为$\sum$上的一个字,这里$c_i\in \sum, 1\leq i\leq n$,$\epsilon=()$称为$\sum$上的一个空字)。
\end{Exercise}
\begin{proof}[证明]
$\sum^*=\bigcup_{i=0}^{\infty}\sum^n$,其中$\sum^0=\{\epsilon\}$。对任意的自然数$i$,$\sum^i$为有穷集合。
于是$\sum^*$可以排成无重复项的序列:
先排$\sum^0$中的元素,再排$\sum^1$中的元素,再排$\sum^2$中的元素,$\cdots$。
\end{proof}
\begin{Exercise}
  用对角线法证明:如果$A$是可数集,则$2^A$是不可数集。
\end{Exercise}
\begin{proof}[证明]
  由$A$为可数集知$A$中的元素可以排成无重复项的序列
 \[a_1,a_2,a_3,\cdots\]

  $2^A$与$A$的所有特征函数构成的集合$Ch(A)$对等。进一步,$Ch(A)$与所有的0,1序列构成的集合对等,对任意的$f\in Ch(A)$,$f$对应$0,1$序列$f(a_1),f(a_2),f(a_3),\cdots$。
  
  以下用对角线法证明所有的$0,1$序列构成的集合不可数。

  用反证法,假设所有0,1的无穷序列构成的集合$B$为可数集,
  则$B$中元素可以排成无重复项的序列:
  \begin{align*}
    b_{11}b_{12}b_{13}\cdots\\
    b_{21}b_{22}b_{23}\cdots\\
    b_{31}b_{32}b_{33}\cdots\\
    \cdots\\
    b_{n1}b_{n2}b_{n3}\cdots\\
    \cdots
  \end{align*}
 其中$b_{ij}=0$或$1$。
  
 构造0,1序列
\[d_1,d_2,d_3,\cdots\]

其中
\[d_n=
\begin{cases}
  0&\text{如果}b_{nn}=1\\
  1&\text{如果}b_{nn}=0
\end{cases}\]
则所构造的0,1序列$d_1,d_2,d_3,\cdots$与前述序列中的任意一个$0,1$序列都不相同,矛盾。

\end{proof}  

\begin{Exercise}
  利用康托的对角线法证明所有0,1的无穷序列是不可数集。
\end{Exercise}
\begin{proof}[证明]
  用反证法。设所有0,1的无穷序列构成的集合$A$为可数集,
  则$A$中元素可以排成无重复项的序列:

  \begin{align*}
    a_{11}a_{12}a_{13}\cdots\\
    a_{21}a_{22}a_{23}\cdots\\
    a_{31}a_{32}a_{33}\cdots\\
    \cdots\\
    a_{n1}a_{n2}a_{n3}\cdots\\
    \cdots
  \end{align*}
  其中$a_{ij}=0$或$1$。

  构造0,1序列
\[b_1,b_2,b_3,\cdots\]

其中
\[b_n=
\begin{cases}
  0&\text{如果}a_{nn}=1\\
  1&\text{如果}a_{nn}=0
\end{cases}\]
则所构造的0,1序列$b_1,b_2,b_3,\cdots$与前述序列中的任意一个$0,1$序列都不相同,矛盾。
\end{proof}
% \begin{Exercise}
%   证明:如果$A$是可数集,则$2^A$是不可数集。
% \end{Exercise}
% \begin{proof}[证明]
%   由$A$为可数集知$A$中的元素可以排成无重复项的序列
%   \[a_1,a_2,a_3,\cdots\]
%   构造从$2^A$到所有的0,1序列构成的集合之间的映射f:

%   对任意的$B\in 2^A$,f(B)为如下的0,1序列:
%   \[b_1,b_2,b_3,\cdots\]

%   其中

%   \[b_i=\begin{cases}
%     1& \text{如果}a_i\in B\\
%     0& \text{如果}a_i\notin B
%   \end{cases}\]
  
%   则$f$为从$2^A$到所有的0,1序列构成的集合之间的双射,
%   而所有的0,1序列构成的集合为不可数集,从而$2^A$为不可数集。
% \end{proof}

\end{CJK}
\end{document}


%%% Local Variables:
%%% mode: latex
%%% TeX-master: t
%%% End:
