\documentclass{article}
\usepackage{CJKutf8}
\usepackage{amsmath}
\usepackage{amssymb}
\usepackage{amsfonts}
\usepackage{amsthm}
\usepackage{titlesec}
\usepackage{titletoc}
\usepackage{xCJKnumb}
\usepackage{tikz}
\usepackage{mathrsfs}
\usepackage{indentfirst}
\usepackage{enumitem}
\newtheorem{Def}{定义}
\newtheorem{Thm}{定理}
\newtheorem{Exercise}{练习}

\newtheorem*{Example}{例}
\setlist[enumerate,1]{label=(\arabic*)}

\begin{document}
\begin{CJK*}{UTF8}{gbsn}
  \title{第七讲 一阶谓词形式系统的语义}
  \author{陈建文}
  \maketitle

  \begin{Example}$\quad$

    \begin{enumerate}
      \item $\forall x(M(x)\to D(x))$
      \item $M(a)\to D(a)$
      \item $M(a)$
      \item $D(a)$
    \end{enumerate}


    谓词$M(x):x$是人

    谓词$D(x):x$是要死的

    个体常元$a$表示“苏格拉底”

    % \begin{enumerate}
    %   \item 所有的人都是要死的;
    %   \item 苏格拉底是人;
    %   \item 所以苏格拉底是要死的。
    % \end{enumerate}

    谓词$M(x):x$是有两个角相等的三角形

    谓词$D(x):x$是等腰三角形

    个体常元$a$表示一个具体的三角形
    

    
  \end{Example}
  \begin{Example}
    $(G,\circ, e)$称为一个群,如果
    \begin{enumerate}
      \item $\forall x\forall y\forall z (x\circ y)\circ z=x\circ(y\circ z);$
      \item $\forall x e\circ x = x;$
      \item $\forall x \exists y y\circ x=e.$
    \end{enumerate}

    $(R,+,0)$为一个群

    $(Z_n,+,[0])$为一个群,这里$Z_n=\{[0],[1],\cdots,[n-1]\}$
  \end{Example}
  
\begin{Def}
  设$\Sigma$为一个符号集,$D$为任意一个非空集合,称为论域;I为一个定义在由$\Sigma$中的常量符号、函词符号和谓词符号所构成的集合上的一个映射,
  称为一个解释,
  \begin{enumerate}
    \item 对任意一个常元$a$,$I(a)$为论域$D$中的一个个体,简记为$\bar{a}$;
    \item 对任意一个$n$元函词$f$,$I(f)$为$D$上的一个$n$元函数,简记为$\bar{f}$;
    \item 对任意一个$n$元谓词$P$,$I(P)$为$D$上的一个$n$元关系,简记为$\bar{P}$。
  \end{enumerate}
  $(D,I)$称为一个结构。
\end{Def}
\begin{Def}
  映射$s:\{v_1,v_2,\cdots\}\to D$称为一个指派,对任意一个变元$v_i$,$s(v_i)\in D$为$v_i$在指派$s$下的值。
\end{Def}
\begin{Def}
  指派$s$可以扩展为从项集合到论域的映射$\bar{s}$:对任意的项$t$,

  \begin{equation*}
    \bar{s}(t)=\begin{cases}
      s(v)&\text{当}t\text{为变元}$v$\text{时}\\
      \bar{a}&\text{当}t\text{为常元}$a$\text{时}\\
      \bar{f}(\bar{s}(t_1),\bar{s}(t_2),\cdots,\bar{s}(t_n))&\text{当}t\text{为}n\text{元函词}$f$\text{时}
    \end{cases}
  \end{equation*}
\end{Def}
  \begin{Example}
  在结构$(R,I)$中,设$I(\circ)=+,I(e)=0$,则结构$(R,I)$又可以简记为$(R,+,0)$。设$s(x)=1,s(y)=2,s(z)=3$,则$\bar{s}(x)=1,\bar{s}(e)=0,\bar{s}(e\circ x)=\bar{s}(e)+\bar{s}(x)=0+1=1,\bar{s}(x\circ y)=\bar{s}(x)+\bar{s}(y)=1+2=3$,$\bar{s}((x\circ y)\circ z)=\bar{s}(x\circ y)+\bar{s}(z)=3+3=6$。  
  \end{Example}
\begin{Def}
  公式$A$在结构$U=(D,I)$和指派$s$下的取值为真记为$\vDash_UA[s]$,定义如下:
  \begin{enumerate}
    \item 当$A$为原子公式$P(t_1,t_2,\cdots,t_n)$时,$\vDash_UA[s]$ iff $(\bar{s}(t_1),\bar{s}(t_2),\cdots,\bar{s}(t_n))\in \bar{P}$;
    \item 当$A$为公式$\lnot B$时,$\vDash_UA[s]$ iff $\nvDash_UB[s]$;
    \item 当$A$为公式$B\land C$时,$\vDash_UA[s]$ iff $\vDash_UB[s]$并且$\vDash_UC[s]$;
    \item 当$A$为公式$B\lor C$时,$\vDash_UA[s]$ iff $\vDash_UB[s]$或者$\vDash_UC[s]$;
    \item 当$A$为公式$B\to C$时,$\vDash_UA[s]$ iff 如果$\vDash_UB[s]$,那么$\vDash_UC[s]$;
    \item 当$A$为公式$B\leftrightarrow C$时,$\vDash_UA[s]$ iff 如果$\vDash_UB[s]$,那么$\vDash_UC[s]$,并且如果$\vDash_UC[s]$,那么$\vDash_UB[s]$;
    \item 当$A$为公式$\forall vB$时,$\vDash_UA[s]$ iff 对任意的$d\in D$,$\vDash_UB[s(v|d)]$,这里$s(v|d)$表示除了对变元$v$用指定元素$d$赋值外,对其他变元的赋值与$s$相同的指派;
    \item 当$A$为公式$\exists vB$时,$\vDash_UA[s]$ iff 存在$d\in D$使得$\vDash_UB[s(v|d)]$。
  \end{enumerate}

  $\vDash_UA$表示在结构$U$中,对任意的指派$s$,公式$A$均为真。

  $\vDash A$表示公式$A$在任意结构和任意指派下均为真。

  \begin{Example}
    $\vDash \forall x_1 \forall x_2(\lnot(x_1=x_2)\to \lnot (f(x_1)=f(x_2)))\leftrightarrow \forall x_1\forall x_2 ((f(x_1)=f(x_2))\to (x_1=x_2))$
  \end{Example}
\end{Def}
  % \begin{Example}
  %   设论域$D$为自然数集$N$,一元函数$\bar{f}(x)=x+1$,二元谓词$\bar{P}(x,y)$为$x\leq y$,常元$\bar{a}=0$,则有如下结论:
  %   \begin{enumerate}
  %     \item 当公式$A=P(a,f(x))$时,则$\vDash_UA$;
  %     \item 当公式$A=P(f(x),a)$时,则$\nvDash_UA$;
  %     \item 当公式$A=\forall xP(a,x)$时,则$\vDash_UA$;
  %     \item 当公式$A=\forall x\exists yP(f(x),y)$时,则$\vDash_UA$;
  %     \item 当公式$A=\exists yP(f(y),y)$时,则$\nvDash_UA$;
  %   \end{enumerate}
  % \end{Example}
\begin{Def}
  设$\Gamma$为任意一个公式集,$B$为任意一个公式,如果对任意的使得$\Gamma$中每个公式均为真的结构$U$和指派$s$,$B$也为真,则称$\Gamma$逻辑蕴含$B$,记为$\Gamma \vDash B$。如果$\Gamma=\{A\}$,则$\Gamma \vDash B$简记为$A\vDash B$,称为$A$逻辑蕴含$B$,即对任意的结构$U$和任意的指派$s$,如果$\vDash_UA[s]$,则$\vDash_UB[s]$。对任意的两个公式$A$和$B$,如果$A\vDash B$并且$B\vDash A$,则称$A$与$B$逻辑等价。
\end{Def}
\begin{Example}
  $\forall x_1 \forall x_2(\lnot(x_1=x_2)\to \lnot (f(x_1)=f(x_2)))\vDash \forall x_1\forall x_2 ((f(x_1)=f(x_2))\to (x_1=x_2))$

  $\forall x_1\forall x_2 ((f(x_1)=f(x_2))\to (x_1=x_2))\vDash \forall x_1 \forall x_2(\lnot(x_1=x_2)\to \lnot (f(x_1)=f(x_2)))$
\end{Example}
课后作业题
\begin{Exercise}证明:
  \begin{enumerate}
    \item $P(v)\nvDash \forall v P(v)$
    \item $\nvDash P(v)\to \forall vP(v)$
    \item $\vDash \exists v(P(v)\to \forall v P(v))$
    \item $\lnot P(v)\to \forall P(v)$是可满足的。(设$A$为任意一个谓词公式,如果存在结构$U$和指派$s$,使得$\vDash_UA[s]$,则称$A$为可满足的。)
  \end{enumerate}
\end{Exercise}

\begin{Exercise}
  给出下列公式为真的解释和指派。

  \begin{enumerate}
    \item $\forall x(P(x)\to Q(x))$
    \item $\exists x(F(x)\lor G(x))$
    \item $\forall x(F(x)\land G(x))$
  \end{enumerate}
\end{Exercise}
\end{CJK*}
\end{document}





%%% Local Variables:
%%% mode: latex
%%% TeX-master: t
%%% End:
