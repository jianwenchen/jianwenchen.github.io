\documentclass{article}
\usepackage{CJKutf8}
\usepackage{amsmath}
\usepackage{amssymb}
\usepackage{amsfonts}
\usepackage{amsthm}
\usepackage{titlesec}
\usepackage{titletoc}
\usepackage{xCJKnumb}
\usepackage{tikz}
\usepackage{mathrsfs}
\usepackage{indentfirst}

\newtheorem{Def}{定义}
\newtheorem{Thm}{定理}
\newtheorem{Exercise}{练习}

\newtheorem*{Example}{例}


\begin{document}
\begin{CJK*}{UTF8}{gbsn}
  \title{第九讲 同态基本定理}
  \author{陈建文}
  \maketitle
  % \tableofcontents
  

\begin{Def}
  设$(G_1,\circ)$与$(G_2,*)$为两个群,如果存在一个从$G_1$到$G_2$的映射$\phi$,使得$\forall a,b\in G_1$,\[\phi(a\circ b)=\phi(a)* \phi(b)\]
  则称$G_1$与$G_2$同态,$\phi$称为从$G_1$到$G_2$的一个同态(homomorphism)。如果同态$\phi$是满射,则称$\phi$为从$G_1$到$G_2$的一个满同态,此时称$G_1$与$G_2$为满同态,
  并记为$G_1\sim G_2$。类似的,如果同态$\phi$为单射,则称$\phi$为单同态。
\end{Def}

\begin{Thm}
  设$(G_1,\circ)$与$(G_2,*)$为两个群,$e_1$和$e_2$分别为其单位元,$\phi$为从$G_1$到$G_2$的同态,则
  \begin{align*}
    &\phi(e_1)=e_2\\
    &\forall a\in G \phi(a^{-1})=(\phi(a))^{-1}\\
  \end{align*}
\end{Thm}
\begin{proof}[证明]
  由$\phi(e_1)=\phi(e_1\circ e_1)=\phi(e_1)*\phi(e_1)$知$\phi(e_1)=e_2$。
  $\forall a\in G, \phi(a^{-1})*\phi(a)=\phi(a^{-1}\circ a)=\phi(e_1)=e_2$知$(\phi(a))^{-1}=\phi(a^{-1})$。
\end{proof}
\begin{Thm}
  设$(G_1,\circ)$为一个群,$G_2$为一个具有二元代数运算$*$的代数系。如果存在一个满射$\phi:G_1\to G_2$使得$\forall a,b\in G_1$
  \[\phi(a\circ b)=\phi(a) * \phi(b)\]
  则$(G_2,*)$为一个群。
\end{Thm}
\begin{proof}[证明]
  $\forall x,y,z\in G_2$,由$\phi$为满射知$\exists a,b,c\in G_1$使得$\phi(a)=x,\phi(b)=y,\phi(c)=z$,从而
  $(x*y)*z=(\phi(a)*\phi(b))*\phi(c)=\phi(a\circ b)*\phi(c)=\phi((a\circ b)\circ c)$,$x*(y*z)=\phi(a)* (\phi(b)*\phi(c))=\phi(a)*\phi(b\circ c)=\phi(a\circ (b\circ c))$,$(x*y)*z=x*(y*z)$,这验证了在$G_2$中$*$运算满足结合律。

  $\forall x\in G_2$,由$\phi$为满射知$\exists a\in G_1$使得$\phi(a)=x$,于是$\phi(e)*x=\phi(e)*\phi(a)=\phi(e\circ a)=\phi(a)=x$。

  $\forall x\in G_2$,由$\phi$为满射知$\exists a\in G_1$使得$\phi(a)=x$,于是$\phi(a^{-1})*\phi(a)=\phi(a^{-1}\circ a)=\phi(e_1)$。
\end{proof}
\begin{Thm}
  设$\phi$为从群$(G_1,\circ)$到群$(G_2,*)$的同态,则

  (1)如果$H$为$G_1$的子群,那么$\phi(H)$为$G_2$的子群;

  (2)如果$H$为$G_2$的子群,那么$\phi^{-1}(H)$为$G_1$的子群;

  (3)如果$N$为$G_2$的正规子群,那么$\phi^{-1}(N)$为$G_1$的正规子群。
\end{Thm}
\begin{proof}[证明]
  以下设$G_1$的单位元为$e_1$,$G_2$的单位元为$e_2$。

  (1)$e_2=\phi(e_1)\in \phi(H)$,从而$\phi(H)$非空。

  $\forall x, y\in \phi(H)$,$\exists a,b\in H$使得$x=\phi(a)$,$y=\phi(b)$,则$x*y^{-1}=\phi(a)*\phi(b)^{-1}=\phi(a)*\phi(b^{-1})=\phi(a\circ b^{-1})\in \phi(H)$。

  以上验证了$\phi(H)$为$G_2$的子群。

  (2)由$\phi(e_1)=e_2$知$e_1\in \phi^{-1}(H)$,从而$\phi^{-1}(H)$非空。

  $\forall a,b\in \phi^{-1}(H)$,则$\phi(a)\in H$,$\phi(b)\in H$,从而$\phi(ab^{-1})=\phi(a)*\phi(b^{-1})=\phi(a)*\phi(b)^{-1}\in H$,于是$ab^{-1}\in \phi^{-1}(H)$。

  这验证了$\phi^{-1}(H)$为$G_1$的子群。

  (3)由(2)知$\phi^{-1}(N)$为$G_1$的子群。

  $\forall g\in \phi^{-1}(N), a\in G, \phi(aga^{-1})=\phi(a)\phi(g)\phi(a^{-1})=\phi(a)\phi(g)\phi(a)^{-1}\in N$,从而$aga^{-1}\in \phi^{-1}(N)$,于是$a\phi^{-1}(N)a\subseteq \phi^{-1}(N)$,因此$\phi^{-1}(N)$为$G_1$的正规子群。
\end{proof}
\begin{Thm}
设$\phi$为从群$G_1$到群$G_2$的满同态,$N$为$G_1$的正规子群,则$\phi(N)$为$G_2$的正规子群。
\end{Thm}
\begin{proof}[证明]
  $\phi(N)$显然为$G_2$的子群。

  $\forall g\in \phi(N)$,$\exists b\in G_1$使得$g=\phi(b)$,$\forall h\in G_2,\exists a\in G_1$,使得$h=\phi(a)$。于是,$hgh^{-1}=\phi(a)\phi(b)\phi(a)^{-1}=\phi(a\circ b)*\phi(a^{-1})=\phi(a\circ b\circ a^{-1})\in \phi(N)$,从而$h\phi(N)h^{-1}\subseteq \phi(N)$,因此$\phi(N)$为$G_2$的正规子群。
\end{proof}
\begin{Def}
设$\phi$为群$(G_1,\circ)$到群$(G_2,*)$的一个同态,$e_2$为$G_2$的单位元,则$G_1$的子群$\phi^{-1}(e_2)$称为同态$\phi$的核,记为$Ker \phi$。$\phi(G_1)$称为$\phi$在$G_1$下的同态像。
\end{Def}

\begin{Thm}
  设$\phi$为从群$(G_1,\circ)$到群$(G_2,*)$的一个同态,则$Ker \phi$为群$G_1$的正规子群。
\end{Thm}

\begin{Thm}
设$N$为$G$的一个正规子群,$\phi$为从$G$到$G/N$的一个映射,$\forall x\in G \phi(x)=xN$,则$\phi$为从$G$到$G/N$的一个满同态,$Ker \phi=N$。
\end{Thm}
\begin{proof}[证明]
  $\forall x,y\in G, \phi(xy)=(xy)N=(xN)(yN)=\phi(x)\phi(y)$,这验证了$\phi$为从$G$到$G/N$的一个同态。

  $\forall g\in G, g\in Ker\phi \Leftrightarrow \phi(g)=N \Leftrightarrow gN=N \Leftrightarrow g\in N$。
\end{proof}
\begin{Thm}[群的同态基本定理]
设$\phi$为从群$G_1$到群$G_2$的同态,则$G_1/Ker G_1 \cong \phi(G_1)$。
\end{Thm}
记$K=KerG_1$。令$f:G_1/K\to \phi(G_1)$,$\forall gK\in G_1/K, f(gK)=\phi(g)$。

$\forall g_1,g_2\in G_1$,如果$g_1K=g_2K$,则$g_1^{-1}g_2\in K$,从而$\phi(g_1^{-1}g_2)=e_2$,即$\phi(g_1)^{-1}\phi(g_2)=e_2$,于是$\phi(g_1)=\phi(g_2)$,所以$f(g_1K)=f(g_2K)$,这验证了$f$为映射。

$f$为单射,这是因为$\forall g_1K,g_2K\in G_1/K$,如果$f(g_1K)=f(g_2K)$,则$\phi(g_1)=\phi(g_2)$,从而$\phi(g_1^{-1}\circ g_2)=e_2$,于是$g_1^{-1}g_2\in K$,所以$g_1K=g_2K$。

$f$为满射,这是因为$\forall g_2 \in \phi(G_1)$,$\exists g_1\in G$使得$\phi(g_1)=g_2$,于是$f(g_1K)=\phi(g_1)=g_2$。

$\forall g_1K,g_2K\in G_1/K$,$f((g_1K)(g_2K))=f(g_1g_2K)=\phi(g_1g_2)=\phi(g_1)\phi(g_2)=f(g_1K)f(g_2K)$,因此$f$为从$G_1/K$到$\phi(G_1)$的同构。

课后作业题:
\begin{Exercise}
设$G$为$m$阶循环群,$\bar{G}$为$n$阶循环群,试证:$G \sim \bar{G}$当且仅当$n | m$。
\end{Exercise}

\begin{Exercise}
设$G$为一个循环群,$H$为群$G$的子群,试证:$G/H$也为循环群。
\end{Exercise}
\end{CJK*}
\end{document}





%%% Local Variables:
%%% mode: latex
%%% TeX-master: t
%%% End:



