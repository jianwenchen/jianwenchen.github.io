\documentclass{article}
\usepackage{CJKutf8}
\usepackage{amsmath}
\usepackage{amssymb}
\usepackage{amsfonts}
\usepackage{amsthm}
\usepackage{titlesec}
\usepackage{titletoc}
\usepackage{xCJKnumb}
\usepackage{tikz}
\usepackage{mathrsfs}
\usepackage{indentfirst}

\newtheorem{Def}{定义}
\newtheorem{Thm}{定理}
\newtheorem{Exercise}{练习}

\newtheorem*{Example}{例}


\begin{document}
\begin{CJK*}{UTF8}{gbsn}
  \title{第九讲 同态基本定理}
  \author{陈建文}
  \maketitle
  % \tableofcontents
  

\begin{Def}
  设$(G,\circ)$与$(\bar{G},\cdot)$为两个群,如果存在一个从$G$到$\bar{G}$的映射$\phi$,使得$\forall a,b\in G$,\[\phi(a\circ b)=\phi(a)\circ \phi(b)\]
  则称$\phi$为从$G$到$\bar{G}$的一个同态(homomorphism),而称$G$与$\bar{G}$同态。如果同态$\phi$是满射,则称$\phi$为从$G$到$\bar{G}$的一个满同态,此时称$G$与$\bar{G}$为满同态,
  并记为$G\sim \bar{G}$。类似的,如果同态$\phi$为单射,则称$\phi$为单同态。
\end{Def}

\begin{Thm}
  设$(G,\circ)$与$(\bar{G},\cdot)$为两个群,$e$和$\bar{e}$分别为其单位元,$\phi$为从$G$到$\bar{G}$的同态,则,
  \begin{align*}
    &\phi(e)=\bar{e}\\
    &\forall a\in G \phi(a^{-1})=(\phi(a))^{-1}\\
  \end{align*}
\end{Thm}

\begin{Thm}
  设$(G,\circ)$为一个群,$\bar{G}$为一个具有二元代数运算$\cdot$的代数系。如果存在一个满射$\phi:G\to \bar{G}$使得$\forall a,b\in G$
  \[\phi(a\circ b)=\phi(a) \cdot \phi(b)\]
  则$(\bar{G},\cdot)$为一个群。
\end{Thm}

\begin{Thm}
  设$\phi$为从群$G$到群$\bar{G}$的同态,则

  (1)如果$H$为$G$的子群,那么$\phi(H)$为$\bar{G}$的子群;

  (2)如果$\bar{H}$为$\bar{G}$的子群,那么$\phi^{-1}(\bar{H})$为$G$的子群;

  (3)如果$\bar{N}$为$\bar{G}$的正规子群,那么$\phi^{-1}(\bar{N})$为$G$的正规子群。
\end{Thm}

\begin{Thm}
设$\phi$为从群$G$到群$\bar{G}$的满同态,$N$为$G$的正规子群,则$\phi(N)$为$\bar{G}$的正规子群。
\end{Thm}

\begin{Def}
设$\phi$为群$(G,\circ)$到群$(\bar{G},\cdot)$的同态,$\bar{e}$为$\bar{G}$的单位元,则$G$的子群$\phi^{-1}(\bar{e})$称为同态$\phi$的核,记为$Ker \phi$。$\phi(G)$称为$\phi$在$G$下的同态像。
\end{Def}

\begin{Thm}
  设$\phi$为从群$(G,\circ)$到群$(\bar{G},\cdot)$的同态,则$Ker \phi$为群$G$的正规子群。
\end{Thm}

\begin{Thm}
设$N$为$G$的一个正规子群,$\phi$为从$G$到$G/N$的一个映射,$\forall x\in G \phi(x)=xN$,则$\phi$为从$G$到$G/N$的一个满同态,$Ker \phi=N$。
\end{Thm}
\begin{proof}[证明]
  $\forall x,y\in G, \phi(xy)=(xy)N=(xN)(yN)=\phi(x)\phi(y)$,这验证了$\phi$为从$G$到$G/N$的一个同态。

  $\forall g\in G, g\in Ker\phi \Leftrightarrow \phi(g)=N \Leftrightarrow gN=N \Leftrightarrow g\in N$。
\end{proof}
\begin{Thm}[群的同态基本定理]
设$\phi$为从群$G$到群$\bar{G}$的同态,则$G/Ker G \cong \phi(G)$。
\end{Thm}
记$K=KerG$。令$f:G/K\to \phi(G)$,$\forall gK\in G/K, f(gK)=\phi(g)$。

$\forall g_1,g_2\in G$,如果$g_1K=g_2K$,则$g_1^{-1}g_2\in K$,从而$\phi(g_1^{-1}g_2)=\bar{e}$,即$\phi(g_1)^{-1}\phi(g_2)=\bar{e}$,于是$\phi(g_1)=\phi(g_2)$,所以$f(g_1K)=f(g_2K)$,这验证了$f$为映射。

$f$为单射,这是因为$\forall g_1K,g_2K\in G/K$,如果$f(g_1K)=f(g_2K)$,则$\phi(g_1)=\phi(g_2)$,从而$\phi(g_1^{-1}g_2)=\bar{e}$,于是$g_1^{-1}g_2\in K$,所以$g_1K=g_2K$。

$f$为满射,这是因为$\forall \bar{g}\in \phi(G)$,$\exists g\in G$使得$\phi(g)=\bar{g}$,于是$f(gK)=\phi(g)=\bar{g}$。

$\forall g_1K,g_2K\in G/K$,$f((g_1K)(g_2K))=f(g_1g_2K)=\phi(g_1g_2)=\phi(g_1)\phi(g_2)=f(g_1K)f(g_2K)$,因此$f$为从$G/K$到$\phi(G)$的同构。

课后作业题:
\begin{Exercise}
设$G$为$m$阶循环群,$\bar{G}$为$n$阶循环群,试证:$G \sim \bar{G}$当且仅当$n | m$。
\end{Exercise}

\begin{Exercise}
设$G$为一个循环群,$H$为群$G$的子群,试证:$G/H$也为循环群。
\end{Exercise}
\end{CJK*}
\end{document}





%%% Local Variables:
%%% mode: latex
%%% TeX-master: t
%%% End:



