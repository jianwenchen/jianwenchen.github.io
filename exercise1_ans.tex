\documentclass{article}
\usepackage{CJKutf8}
\usepackage{amsmath}
\usepackage{amssymb}
\usepackage{amsthm}
\begin{document}
\begin{CJK}{UTF8}{gbsn}
\newtheorem{Exercise}{习题}
\begin{Exercise}
  写出方程$x^2+2x+1=0$的根所构成的集合。
\end{Exercise}
\begin{proof}[解]
  \{-1\}
\end{proof}
\begin{Exercise}
  设有$n$个集合$A_1$,$A_2$,$\ldots$,$A_n$,且$A_1\subseteq A_2\subseteq \ldots \subseteq A_n \subseteq A_1$,试证
  \[A_1 = A_2 = \ldots = A_n\]
\end{Exercise}
\begin{proof}[证明]
  用数学归纳法证明,施归纳于$n$。
  
  1.当$n=2$时,$A_1\subseteq A_2 \subseteq A_1$,$A_1=A_2$显然成立。

  2.假设当$n=k(k\geq 2)$时结论成立,往证当$n=k+1$时结论也成立。

  设$A_1\subseteq A_2\subseteq \ldots \subseteq A_{k} \subseteq A_{k+1} \subseteq A_1$,则由$A_{k} \subseteq A_{k+1} \subseteq A_1$知$A_{k} \subseteq A_1$,
  于是$A_1\subseteq A_2\subseteq \ldots \subseteq A_{k} \subseteq A_1$,由归纳假设,$A_1 = A_2 = \ldots = A_k$。

  由$A_1=A_k$,$A_{k+1}\subseteq A_1$知$A_{k+1} \subseteq A_k$,再由$A_{k}\subseteq A_{k+1}$知,$A_k=A_{k+1}$。
  
  于是$A_1=A_2=\ldots = A_k = A_{k+1}$,结论得证。
\end{proof}


\begin{Exercise}
  设集合$S=\{\phi, \{\phi\}\}$,则$2^S=\underline{\{\phi, \{\phi\},\{\{\phi\}\},\{\phi, \{\phi\}\}\}}$。
\end{Exercise}
\begin{Exercise}
  设集合$S$有$n$个元素,证明$2^S$有$2^n$个元素。
\end{Exercise}
\begin{proof}[证明]
  用数学归纳法证明,施归纳于$n$:

  1.当$n=0$时,$S=\phi$,$2^{\phi} = \{\phi\}$有$1$个元素,结论成立。

  2.假设当$n=k(k\geq
  0)$时结论成立,往证当$n=k+1$时结论也成立。设集合$S$中有$k+1$个元
  素$s_1$,$s_2$,$\ldots$,$s_k$,$s_{k+1}$,记
  \begin{align*}
    S_1&=2^{S\setminus \{s_{k+1}\}}\\
    S_2&=\{X\cup \{s_{k+1}\}|X\subseteq S\setminus \{s_{k+1}\}\}
  \end{align*}
  则$2^S = S_1\cup S_2$。

  考虑映射$f:S_1\to S_2$,对任意的$X\in S_1$,$f(X)=X\cup \{s_{k+1}\}$。易验证$f$为从$S_1$到$S_2$的双射,从而$|S_1|=|S_2|$。

  显然$S_1\cap S_2=\phi$,再由归纳假设,$|S_1|=2^k$,从而$|2^S|=|S_1|+|S_2|=2^k+2^k=2^{k+1}$。

\end{proof}
\begin{Exercise}
  设$A$,$B$为集合,试证
  \[(A\setminus B)\cup B = (A\cup B)\setminus B \Leftrightarrow B = \phi\]
\end{Exercise}
\begin{proof}[证明]
  如果$B = \phi$,显然$(A\setminus B)\cup B = (A\cup B)\setminus B$。

  由$(A\setminus B)\cup B = (A\cup B)\setminus B$,往证$B=\phi$。用反证法,假设$B\neq \phi$,则存在$x\in B$,于是$x\in(A\setminus B)\cup B$,但是$x\notin (A\cup B)\setminus B$,这与$(A\setminus B)\cup B = (A\cup B)\setminus B $矛盾。
\end{proof}
\begin{Exercise}
  设$A$,$B$为集合,试证$A=\phi \Leftrightarrow B= A\bigtriangleup B$。
\end{Exercise}
\begin{proof}[证法一]
  当$A=\phi$时,显然$B= A\bigtriangleup B$。

  设$B= A\bigtriangleup B$,往证$A=\phi$。用反证法。设$A\neq \phi$,则存在$x\in A$。此时,如果$x\in B$,则$x\notin A\bigtriangleup B = B$,矛盾;如果$x\notin B$,则$x\in A\bigtriangleup B=B$,也矛盾。
\end{proof}

\begin{proof}[证法二]
  \begin{equation*}
    \begin{split}
      &B= A\bigtriangleup B \\
      \Leftrightarrow& B \bigtriangleup B = (A\bigtriangleup B)\bigtriangleup B\\
      \Leftrightarrow& \phi = A\bigtriangleup (B\bigtriangleup B)\\
            \Leftrightarrow& \phi = A\bigtriangleup \phi   \\
      \Leftrightarrow& \phi = A   
    \end{split}
  \end{equation*}
\end{proof}

\begin{Exercise}
  设$A$,$B$为集合,证明$A\setminus (B\cup C) = A\setminus B \setminus C$。
\end{Exercise}
\begin{proof}[证明]
  先证$A\setminus (B\cup C) \subseteq A\setminus B \setminus C$。

  对任意的$x\in A\setminus (B\cup C)$,则$x\in A$并且$x\notin B\cup C$,即$x\in A$并且$x\notin B$,$x\notin C$,故$x\in A\setminus B \setminus C$。
  
  再证$A\setminus B \setminus C\subseteq A\setminus (B\cup C) $。

  对任意的$x\in A\setminus B \setminus C$,则$x\in A$并且$x\notin B$,$x\notin C$,于是$x\in A$并且$x\notin B\cup C$,故$A\setminus (B\cup C)$。

\end{proof}
\begin{Exercise}
  设$A,B,C$为集合,证明$(A\cup B)\setminus C = (A\setminus C) \cup (B\setminus C)$。
\end{Exercise}
\begin{proof}[证明]
  先证$(A\cup B)\setminus C \subseteq (A\setminus C) \cup (B\setminus C)$。

  对任意的$x\in (A\cup B)\setminus C$,则$x\in A\cup B$并且$x\notin C$,从而$x\in A$或者$x\in B$,并且$x\notin C$,于是,$x\in A$并且$x\notin C$,或者$x\in B$并且$x\notin C$,
  即$x\in A\setminus C$或者$x\in B\setminus C$,因此$x\in (A\setminus C) \cup (B\setminus C)$。

  再证$(A\cup B)\setminus C \subseteq (A\setminus C) \cup (B\setminus C)$。

  对任意的$(A\setminus C) \cup (B\setminus C)$,则$x\in A\setminus C$或者$x\in B\setminus C$,即$x\in A$并且$x\notin C$,或者$x\in B$并且$x\notin C$,从而$x\in A$或者$x\in B$,并且$x\notin C$,于是$x\in A\cup B$并且$x\notin C$,因此$x\in (A\cup B)\setminus C$。
\end{proof}
\begin{Exercise}
  设$A,B,C$为集合,证明$(A\cap B)\setminus C = (A\setminus C) \cap (B\setminus C)$。
\end{Exercise}
\begin{proof}[证明]
  先证$(A\cap B)\setminus C \subseteq (A\setminus C) \cap (B\setminus C)$。

对任意的$x\in (A\cap B)\setminus C$,则$x\in A\cap B$并且$x\notin C$,于是$x\in A$,$x\in B$,并且$x\notin C$,从而$x\in A\setminus C$并且$x\in B\setminus C$,因此$x\in (A\setminus C) \cap (B\setminus C)$。


  再证$(A\setminus C) \cap (B\setminus C) \subseteq (A\cap B)\setminus C$。

  对任意的$x\in (A\setminus C) \cap (B\setminus C)$,则$x\in A\setminus C$并且$x\in B\setminus C$,于是$x\in A$,$x\in B$,并且$x\notin C$,从而$x\in A\cap B$并且$x\notin C$,因此$x\in (A\cap B)\setminus C$。

\end{proof}
\begin{Exercise}
  设$A,B,C$都是集合,若$A\cup B = A\cup C$且$A\cap B = A\cap C$,试证$B=C$。
\end{Exercise}
\begin{proof}[证法一]
  先证$B\subseteq C$。

  对任意的$x \in B$,分两种情况讨论:

  1)若$x \in A$:此时$x \in A\cap B$,由$A\cap B = A\cap C$知$x \in A\cap C$,从而$x \in C$。

  2)若$x \notin A$:此时由$x\in B$知$x\in A\cup B$,再由$A\cup B = A\cup C$知$x \in A\cup C$,再由$x \notin A$知$x \in C$。

  综合以上两种情况知对任意的$x$,当$x\in B$时$x\in C$,即$B \subseteq C$。
  

  由$B$和$C$的对称性知$C \subseteq B$,因此$B=C$。
\end{proof}

\begin{proof}[证法二]
  $B= B\cap (A\cup B) = B \cap (A \cup C) = (B \cap A) \cup (B \cap C) = (A \cap B) \cup (B \cap C) = (A \cap C) \cup (B \cap C) = (C \cap A) \cup (C \cap B) = C \cap (A \cup B) = C \cap (A \cup C) = C$
  
\end{proof}

\begin{proof}[证法三]
由已知条件知$(A\cup B)\setminus (A\cap B) = (A\cup C)\setminus (A\cap C)$,从而$A\bigtriangleup B = A \bigtriangleup C$,于是$A \bigtriangleup (A\bigtriangleup B) =A \bigtriangleup (A \bigtriangleup C)$, 由对称差运算的结合律知$(A \bigtriangleup A)\bigtriangleup B =(A \bigtriangleup A) \bigtriangleup C$,即$\phi \bigtriangleup B = \phi \bigtriangleup C$,从而$B = C$。
\end{proof}

\begin{Exercise}
  下列等式是否成立?如果成立,请给出证明;如果不成立,请说明理由。

  a) $(A\setminus B)\cup C = A\setminus (B\setminus C)$;

  b)$A\cup(B\setminus C) = (A\cup B)\setminus C$;

  c)$A\setminus (B\cup C) = (A\cup B)\setminus C$。
\end{Exercise}
\begin{proof}[解]
  a)结论不成立。这是因为设$A=\phi$,$B=\phi$,$C=\{1\}$,则$(A\setminus B)\cup C=\{1\}$,而$A\setminus (B\setminus C)=\phi$,$(A\setminus B)\cup C \neq A\setminus (B\setminus C)$。

  b)结论不成立。这是因为设$A=\{1\}$,$B=\phi$,$C=\{1\}$,则$A\cup(B\setminus C)=\{1\}$,而$(A\cup B)\setminus C=\phi$,$A\cup(B\setminus C) \neq (A\cup B)\setminus C$。

  c)结论不成立。这是因为设$A=\phi$,$B=\{1\}$,$C=\phi$,则$A\setminus (B\cup C)=\phi$,$(A\cup B)\setminus C=\{1\}$,$A\setminus (B\cup C) \neq (A\cup B)\setminus C$
\end{proof}
\begin{Exercise}
  下列命题中哪个是真的?(  B  )

A. 对任意集合$A$,$B$,$2^{A\cup B} = 2^A \cup 2^B$。

B. 对任意集合$A$,$B$,$2^{A\cap B} = 2^A \cap 2^B$。

C. 对任意集合$A$,$B$,$2^{A\setminus B} = 2^A \setminus 2^B$。

D. 对任意集合$A$,$B$,$2^{A\bigtriangleup B} = 2^A \bigtriangleup 2^B$。

\end{Exercise}
\begin{Exercise}
  填空:设$A$,$B$为两个集合。

  a) $x\notin A\cup B\Leftrightarrow \underline{x\notin A \land x\notin B}$

  b) $x\notin A\cap B\Leftrightarrow \underline{x\notin A \lor x\notin B}$

  c) $x\notin A\setminus B\Leftrightarrow \underline{x\notin A \lor x\in B}$

  d) $x\notin A\bigtriangleup B\Leftrightarrow \underline{(x\in A \land x\in B)\lor(x\notin A \lor x\notin B)}$
\end{Exercise}
\begin{Exercise}
  设$A$,$B$,$C$为任意三个集合,下列集合表达式中哪一个等于$A\setminus (B\cap C)$?( B  )

  A. $(A\setminus B)\cap (A\setminus C)$

  B. $(A\setminus B)\cup (A\setminus C)$

  C. $(A\cap B)\setminus (A\cap C)$

  D. $(A\cup B)\setminus (A\cup C)$
\end{Exercise}


\begin{Exercise}
  设$A,B,C$为集合,并且$A\cup B=A\cup C$,则下列哪个等式成立?(D)

  A.$B=C$

  B.$A\cap B = A\cap C$

  C.$A\cap B^c = A\cap C^c$

  D.$A^c\cap B = A^c\cap C$
\end{Exercise}



\begin{Exercise}
  设$A,B,C$为集合,化简:

  \begin{equation*}
    \begin{split}
     & (A\cap B \cap C)\cup(A^c\cap B \cap C)\cup(A\cap B^c \cap C)\cup \\
     & (A\cap B \cap C^c)\cup(A^c\cap B^c \cap C)\cup(A\cap B^c \cap C^c)\cup \\
     & (A^c\cap B \cap C^c)
    \end{split}
  \end{equation*}
\end{Exercise}
\begin{proof}[解法一]
设全集为$S$,则

原式$\cup (A^c\cap B^c \cap C^c)=S$

原式$\cap (A^c\cap B^c \cap C^c)=\phi$


从而原式$=(A^c\cap B^c \cap C^c)^c =A\cup B\cup C$。
\end{proof}
\begin{proof}[解法二]
  \begin{align*}
    \text{原式}&=((A\cap B)\cap(C\cup C^c))\cup ((A^c\cap B)\cap (C\cup C^c)) \cup ((A\cap B^c)\cap (C\cup C^c)) \cup (A^c\cap B^c \cap C)\\
  &=(A\cap B)\cup (A^c\cap B) \cup (A\cap B^c) \cup (A^c\cap B^c \cap C)\\
    &=(A\cap (B\cup B^c))\cup (A^c\cap B)  \cup (A^c\cap B^c \cap C)\\
    &=A\cup (A^c\cap B)  \cup (A^c\cap B^c \cap C)\\
    &=((A\cup A^c)\cap (A\cup B))  \cup (A^c\cap B^c \cap C)\\
    &=(A\cup B)  \cup (A^c\cap B^c \cap C)\\
    &=(A\cup B\cup A^c)  \cap  (A\cup B\cup B^c)\cap (A\cup B \cup C)\\
    &=A\cup B\cup C
\end{align*}
\end{proof}
\begin{Exercise}
  设$V$为一个集合,证明:$\forall S,T,W \in 2^V$有$S \subseteq T \subseteq W$当且仅当$S \bigtriangleup T \subseteq S \bigtriangleup W$且$S \subseteq W$。
\end{Exercise}
\begin{proof}[证明]
  首先,$\forall S,T,W \in 2^V$由$S \subseteq T \subseteq W$往证$S \bigtriangleup T \subseteq S \bigtriangleup W$且$S \subseteq W$。

  由$S \subseteq T \subseteq W$知$S \bigtriangleup T = T\setminus S$,$S \bigtriangleup W = W\setminus S$,由$T\subseteq W$知$T\setminus S \subseteq W\setminus S$,从而$S \bigtriangleup T \subseteq S \bigtriangleup W$。$S \subseteq W$显然成立。

  接下来,$\forall S,T,W \in 2^V$由$S \bigtriangleup T \subseteq S \bigtriangleup W$且$S \subseteq W$往证$S \subseteq T \subseteq W$。

  由$S\subseteq W$知$S\bigtriangleup W = W\setminus S$。
  
  先证$S\subseteq T$。用反证法,假设$S\subseteq T$不成立,则存在$x$,$x\in S$但$x \notin T$,于是$x\in S\setminus T\subseteq S\bigtriangleup T \subseteq S\bigtriangleup W = W\setminus S$,这与$x\in S$矛盾。

  再证$T\subseteq W$。用反证法,假设$T\subseteq W$不成立,则存在$x$,$x\in T$但$x \notin W$,由$S\subseteq W$知$x\notin S$,于是$x \in T\setminus S \subseteq S\bigtriangleup T \subseteq S\bigtriangleup W = W\setminus S$,这与$x\notin W$矛盾。
\end{proof}

\begin{Exercise}
  设$A=\{a,b,c\}, B=\{e,f,g,h\}, C=\{x,y,z\}$。求$A\times B, B\times A, A\times C, A\times B \times C, A^2\times B$。
\end{Exercise}
\begin{proof}[解]
  \begin{align*}
    &A\times B\\
    =\{&(a,e),(a,f),(a,g),(a,h),\\
    &(b,e),(b,f),(b,g),(b,h)\\
    &(c,e),(c,f),(c,g),(c,h)\\
      &\}\\ 
      &B\times A\\
      =\{&(e,a),(e,b),(e,c),
      (f,a),(f,b),(f,c),\\
      &(g,a),(g,b),(g,c),
      (h,a),(h,b),(h,c)\\
        &\}\\   
        &A\times C\\
        =\{&(a,x),(a,y),(a,z),
        (b,x),(b,y),(b,z),  
      (c,x,),(c,y),(c,z) 
        \}\\ 
        &A\times B\times C\\ 
        =\{ & (a,e,x),(a,e,y),(a,e,z),\\ 
        &(a,f,x),(a,f,y),(a,f,z),\\ 
        &(a,g,x),(a,g,y),(a,g,z),\\ 
        &(a,h,x),(a,h,y),(a,h,z),\\ 
        & (b,e,x),(b,e,y),(b,e,z),\\ 
        &(b,f,x),(b,f,y),(b,f,z),\\ 
        &(b,g,x),(b,g,y),(b,g,z),\\ 
        &(b,h,x),(b,h,y),(b,h,z),\\ 
        & (c,e,x),(c,e,y),(c,e,z),\\ 
        &(c,f,x),(c,f,y),(c,f,z),\\ 
        &(c,g,x),(c,g,y),(c,g,z),\\ 
        &(c,h,x),(c,h,y),(c,h,z),\\ 
        &\}\\
        &A^2\times B\\ 
        =\{ & ((a,a),x), ((a,a), y), ((a,a),z),\\ 
        &((a,b),x), ((a,b), y), ((a,b), z), \\ 
        &((a,c), x), ((a,c), y), ((a, c), z), \\ 
        & ((b,a),x), ((b,a), y), ((b,a),z),\\ 
        &((b,b),x), ((b,b), y), ((b,b), z), \\ 
        &((b,c), x), ((b,c), y), ((b, c), z), \\ 
        & ((c,a),x), ((c, a), y), ((c,a),z),\\ 
        &((c,b),x), ((c,b), y), ((c,b), z), \\ 
        &((c,c), x), ((c,c), y), ((c, c), z) \\
        &\}\\
  \end{align*}
\end{proof}
\begin{Exercise}
  设$A,B$为集合,试证:$A\times B= B\times A$的充分必要条件是下列三个条件至少一个成立:

  (1) $A=\phi$;(2) $B=\phi$;(3) $A=B$。
\end{Exercise}
\begin{proof}[证明]
  如果 (1) $A=\phi$;(2) $B=\phi$;(3) $A=B$中的一条成立,易验证$A\times B= B\times A$成立。

  设$A\times B= B\times A$成立。如果$A\neq \phi$并且$B\neq \phi$,以下证明必有$A=B$成立。
  由$A\neq \phi$并且$B\neq \phi$知存在$a\in A$,$b\in B$。对任意的$x\in A$,则$(x,b)\in A\times B = B\times A$,从而$x\in B$;对任意的$x\in B$,则$(x,a)\in B\times A = A\times B$,从而$x\in A$。
\end{proof}
\begin{Exercise}
  设$A$,$B$,$C$,$D$为任意四个集合,证明
  \[(A\cap B) \times (C \cap D) = (A\times C)\cap (B \times D)\]
\end{Exercise}
先在草稿纸上分析如下:
\begin{equation*}
  \begin{split}
  \forall x \forall y&(x,y) \in (A\cap B) \times (C \cap D)\\
  \Leftrightarrow&x \in A \cap B \land y \in C \cap D\\
  \Leftrightarrow&x \in A \land x \in B \land y \in C \land y \in D\\
  \Leftrightarrow&x \in A \land y \in C \land x \in B \land y \in D\\
  \Leftrightarrow&(x,y) \in A \times C \land (x,y) \in B \times D\\
  \Leftrightarrow&(x,y) \in (A\times C)\cap (B \times D)
  \end{split}
\end{equation*}
然后转换称用自然语言描述的证明过程如下:
\begin{proof}[证明]
  先证$(A\times C)\cap (B \times D)\subseteq (A\cap B) \times (C \cap D)$。

  对任意的$x$和$y$,如果$(x,y)\in (A\times C)\cap (B \times D)$,则$(x,y)\in A\times C$,并且$(x,y)\in B \times D$,从而$x\in A,y\in C, x \in B, y\in D$,即$x\in A$, $x \in B$,$ y\in C$, $y\in D$,于是$x \in A\cap B$并且$y \in C \cap D$,因此$(x,y) \in (A\cap B) \times (C \cap D)$。
  
  再证$(A\cap B) \times (C \cap D) \subseteq (A\times C)\cap (B \times D)$。
  
    对任意的$x$和$y$,$(x,y) \in (A\cap B) \times (C \cap D)$,则$x\in A\cap B$并且$y \in C \cap D$,从而$x\in A$, $x \in B$,$ y\in C$, $y\in D$, 即$x\in A,y\in C, x \in B, y\in D$,
  于是$(x,y)\in A\times C$,并且$(x,y)\in B \times D$,因此$(x,y)\in (A\times C)\cap (B \times D)$。
  
\end{proof}
\begin{Exercise}
  设$A,B,C$为集合,证明:$A\times(B\bigtriangleup C) = (A\times B)\bigtriangleup(A\times C)$。
\end{Exercise}
\begin{proof}[证明]
  \begin{align*}
    &A\times(B\bigtriangleup C)\\
    =&A\times((B\setminus C) \cup (C\setminus B))\\
    =&(A\times (B\setminus C)) \cup (A\times (C\setminus B))\\
    =&((A\times B)\setminus (A\times C)) \cup ((A\times C)\setminus (A\times B))\\
    =&(A\times B)\bigtriangleup(A\times C)\\
  \end{align*}
\end{proof}
\begin{Exercise}
  设$A$有$m$个元素,$B$有$n$个元素,则$A\times B$是多少个序对组成的?$A\times B$有多少个不同的子集?
\end{Exercise}
\begin{proof}[解]
  $A\times B$是$mn$个序对组成的,$A\times B$有$2^{mn}$个不同的子集。
\end{proof}
\begin{Exercise}
  设$A,B$为集合,$B\neq \phi$。试证:如果$A\times B= B\times B$,则$A=B$。
\end{Exercise}
\begin{proof}[证明]
  由$B\neq \phi$知,存在$b$,$b\in B$。对任意的$x\in A$,则$(x,b)\in A\times B=B\times B$,从而$x\in B$;对任意的$x\in B$,则$(x,b)\in B\times B = A\times B$,从而$x\in A$。这证明了$A=B$。
\end{proof}
\begin{Exercise}
  某班学生中有$45\%$正在学德文,$65\%$正在学法文,问此班中至少有百分之几的学生正在同时学德文和法文?
\end{Exercise}
\begin{proof}[解]
  此班中至少有$10\%$的学生正在同时学德文和法文。
\end{proof}
\begin{Exercise}
  设$A,B$为两个有穷集合,则$|2^{2^{A\times B}}|=\underline{2^{2^{|A|\dot|B|}}}$。
\end{Exercise}

\begin{Exercise}
  毕业舞会上,小伙子与姑娘跳舞。已知每个小伙子至少与一个姑娘跳过舞,但未能与所有的姑娘跳过舞。同样的,每个姑娘也至少与一个小伙子跳过舞,但也未能与所有的小伙子跳过舞。证明:在所有参加舞会的小伙子与姑娘中,必可找到两个小伙子与两个姑娘,这两个小伙子中的每一个只与这两个姑娘中的一个跳过舞,而这两个姑娘中的每一个也只与这两个小伙子中的一个跳过舞。
\end{Exercise}
\begin{proof}[证法一]设小伙子的集合为$B=\{b_1,b_2,\cdots,b_m\}$,姑娘的集合为$G=\{g_1,g_2,\cdots,g_n\}$, $G_{b_i}$为与小伙子$b_i$跳过舞的姑娘的集合,则由假设$G_{b_i}\neq G$,$i=1,2,\cdots,m$。如果存在$i, j$,$i \neq j$,使得$G_{b_i}\nsubseteq G_{b_j}$且$G_{b_j}\nsubseteq G_{b_i}$,则问题得证。否则,对任意的$i$,$j$,或者$G_{b_i}\subseteq G_{b_j}$,或者$G_{b_j}\subseteq G_{b_i}$,于是有\[G_{b_{i_1}}\subseteq G_{b_{i_2}}\subseteq \cdots \subseteq G_{b_{i_m}}\]但由假设,$\bigcup_{i=1}^mG_{b_i} = G$,即$G_{b_{i_m}}=G$,所以$b_{i_m}$与所有的姑娘都跳过舞,矛盾。
  
\end{proof}
\begin{proof}[证法二]设$b_1$为与姑娘跳舞最多的小伙子。由$b_1$未能与所有的姑娘跳过舞知,存在一个姑娘$g_2$,$b_1$未能与$g_2$跳过舞。由每个姑娘至少与一个小伙子跳过舞知,存在一个小伙子$b_2$与$g_2$跳过舞。在与小伙子$b_1$跳过舞的姑娘中,必存在一个姑娘$g_1$未能与小伙子$b_2$跳过舞,否则与$b_1$为与姑娘跳舞最多的小伙子矛盾。于是,$b_1$与$g_1$跳过舞,但未与$g_2$跳过舞;$b_2$与$g_2$跳过舞,但未与$g_1$跳过舞,结论得证。  
\end{proof}

\end{CJK}
\end{document}


%%% Local Variables:
%%% mode: latex
%%% TeX-master: t
%%% End:
