\documentclass{article}
\usepackage{CJKutf8}
\usepackage{amsmath}
\usepackage{amssymb}
\usepackage{amsfonts}
\usepackage{amsthm}
\usepackage{titlesec}
\usepackage{titletoc}
\usepackage{xCJKnumb}
\usepackage{tikz}
\usepackage{mathrsfs}
\usepackage{indentfirst}

\newtheorem{Def}{定义}
\newtheorem{Thm}{定理}
\newtheorem{Exercise}{练习}

\newtheorem*{Example}{例}


\begin{document}
\begin{CJK*}{UTF8}{gbsn}
  \title{第六讲 循环群}
  \author{陈建文}
  \maketitle
  % \tableofcontents
  
\begin{Def}
  群$G$称为循环群,如果$G$是由其中的某个元素$a$生成的,即$G=(a)=\{\cdots,a^{-2},a^{-1},e,a,a^2,\cdots\}$。
\end{Def}

\begin{Example}
  整数加法群$(Z,+)$为循环群,其生成元为$1$。
\end{Example}

\begin{Example}
  模$n$同余类加群$Z_n=\{[0],[1],\cdots,[n-1]\}$为一个阶为$n$的有限循环群,其生成元为$[1]$。
\end{Example}

\begin{Thm}
  (1)循环群$G=(a)$为无穷循环群的充分必要条件是$a$的阶为无穷大,此时$G=\{\cdots,a^{-2},a^{-1},e,a,a^2,\cdots\}$;
  
  (2)循环群$G=(a)$为$n$阶循环群的充分必要条件是$a$的阶为$n$,此时$G=\{e,a,a^2,\cdots,a^{n-1}\}$。
\end{Thm}

\begin{Thm}
  (1)无穷循环群同构于整数加群$(Z,+)$,即如果不计同构,无穷循环群只有一个,就是整数加群;
  
  (2)阶为$n$的有限循环群同构于模$n$同余类加群$(Z_n,+)$,即如果不计同构,$n$阶循环群只有一个,就是模$n$同余类加群。
\end{Thm}

\begin{Thm}
  设$G=(a)$为由$a$生成的循环群,则

  (1)循环群的子群仍为循环群;

  (2)如果$G$为无限循环群,则$H_0=\{e\},H_m=(a^m),m=1,2,\cdots$为$G$的所有子群,这里$H_m,m=1,2,\cdots$都同构于$G$;

  (3)如果$G$为阶为$n$的循环群,则$H_0=\{e\},H_m=(a^m),m|n$为$G$的所有子群。每个子群$H_m,m=1,2,\cdots$的阶为$n/m$。
\end{Thm}


课后作业题:
\begin{Exercise}
证明:$n$次单位根之集对数的通常乘法构成一个循环群。
\end{Exercise}

\begin{Exercise}
找出模$12$的同余类加群的所有子群。
\end{Exercise}

\begin{Exercise}
  设$G=(a)$为一个$n$阶循环群。证明:如果$(r,n)=1$,则$(a^r)=G$。
\end{Exercise}


\begin{Exercise}
  设群$G$中元素$a$的阶为$n$,$(r,n)=d$。证明:$a^r$的阶为$n/d$。
\end{Exercise}
\end{CJK*}
\end{document}





%%% Local Variables:
%%% mode: latex
%%% TeX-master: t
%%% End:



