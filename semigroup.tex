\documentclass{article}
\usepackage{CJKutf8}
\usepackage{amsmath}
\usepackage{amssymb}
\usepackage{amsfonts}
\usepackage{amsthm}
\usepackage{titlesec}
\usepackage{titletoc}
\usepackage{xCJKnumb}
\usepackage{tikz}
\usepackage{mathrsfs}
\usepackage{indentfirst}

\newtheorem{Def}{定义}
\newtheorem{Thm}{定理}
\newtheorem{Exercise}{练习}

\newtheorem*{Example}{例}


\begin{document}
\begin{CJK*}{UTF8}{gbsn}
  \title{第二讲 半群、幺半群与群}
  \author{陈建文}
  \maketitle
  % \tableofcontents
  


\begin{Def}
设“$\circ$”为非空集合$S$上的一个二元代数运算。如果$\forall a,b,c \in S$,
\[(a\circ b) \circ c = a \circ (b\circ c)\]
则称集合$S$对“$\circ$”运算形成一个半群(semigroup),并记为$(S,\circ)$。  
\end{Def}

\begin{Example}
  正整数集合$Z_+$对“$+$”运算构成一个半群。
\end{Example}

$\forall a,b,c \in Z_+ (a+b)+c = a + (b + c)$

\begin{Def}
  如果一个半群中的二元代数运算满足交换律,则称此半群为交换半群。
\end{Def}

\begin{Example}
  设$S$为一切形如
  \[\begin{bmatrix}
    a&b\\
    0&0
  \end{bmatrix},a,b\in N\]
的$2\times 2$矩阵之集,则$S$对矩阵的乘法构成一个不可交换的半群。

$\forall d\in N$,$2\times 2$矩阵
\[\begin{bmatrix}1&d\\0&0\end{bmatrix}\]
为左单位元素。于是,$(S,*)$有无穷多个左单位元素,然而它却没有右单位元素。
\end{Example}

\begin{Thm}
  设$(S,\circ)$为一个代数系,如果二元代数运算$\circ$既有左单位元$e_l$,又有右单位元$e_r$,则$e_l=e_r$,从而有单位元且单位元是唯一的。
\end{Thm}

\begin{Def}
  有单位元素的半群称为独异点(monoid),或称为幺半群。
\end{Def}

\begin{Example}
  自然数集合$N$对加法运算“$+$”构成幺半群,单位元为$0$。正整数集合$Z_+$对乘法运算“$\times$”构成幺半群,单位元为$1$。
\end{Example}

\begin{Example}
  设$S$为任意一个非空集合,则$(2^S,\cup,\phi)$和$(2^S,\cap,S)$都为幺半群。
\end{Example}

\begin{Def}
  如果一个幺半群中的二元代数运算满足交换律,则称此幺半群为交换幺半群。
\end{Def}

\begin{Example}
  设$S$为非空集合,$M(S)=\{f|f:S\to S\}$,则$M(S)$对映射的合成构成了一个以$I_S$为单位元的幺半群$(M(S),\circ,I_S)$,它是不可交换幺半群。
\end{Example}
\begin{Example}
  设$M_n$为所有$n\times n$实矩阵构成的集合,则$M_n$对矩阵的乘法构成了一个以$I_n$为单位元的幺半群$(M_n,*,I_n)$。
\end{Example}


\begin{Def}
  设$(S,\circ,e)$为一个幺半群,$a\in S$。如果存在$a_l\in S$使得$a_l\circ a=e$,则称$a_l$为$a$的左逆元素;如果存在$a_r
  \in S$使得$a\circ a_r=e$,则称$a_r$为$a$的右逆元素;如果存在$b\in S$使得$b\circ a=a\circ b=e$,则称$b$为$a$的逆元素。
\end{Def}
\begin{Thm}
  如果幺半群$(S,\circ,e)$中的元素$a$既有左逆元素$a_l$,又有右逆元素$a_r$,则$a_l=a_r$。于是,$a$有逆元素且$a$的逆元素是唯一的,记为$a^{-1}$。
\end{Thm}
\begin{Def}
  每个元素都有逆元素的幺半群称为群。
\end{Def}

\begin{Def}
  设$G$为一个非空集合,“$\circ$”为$G$上的一个二元代数运算。如果下列各个条件成立,则称$G$对“$\circ$”运算构成一个群(group):

  I. “$\circ$”运算满足结合律,即$\forall a,b,c \in G$ $(a\circ b)\circ c = a\circ(b\circ c)$;

  II. 对“$\circ$”运算,$G$中有一个单位元$e$,即$\forall a\in G$ $e\circ a =a\circ e= a$;

  III. 对$G$中的每个元素,关于$\circ$运算有一个逆元,
  即$\forall a\in G \exists b\in G b\circ a = a\circ b= e$。
\end{Def}
\begin{Example}
  整数集合$Z$,有理数集合$Q$,实数集合$R$,复数集合$C$对通常的加法运算构成群;
  非零有理数集合$Q^*$,非零实数集合$R^*$,非零复数集合$C^*$对通常的乘法运算构成群。
\end{Example}
\begin{Def}
  如果一个群中的二元代数运算满足交换律,则称此群为交换群,又称为Abel群。
\end{Def}
\begin{Example}
设$S$为一个非空集合,从$S$到$S$的所有双射构成的集合对映射的合成构成一个群,称为$S$上的对称群,记为$Sym(S)$。当$S=\{1,2,\cdots,n\}$时,$Sym(S)=S_n$。
\end{Example}
  \begin{Example}
    设$M_n$为所有可逆$n\times n$实矩阵构成的集合,则$M_n$对矩阵的乘法构成了一个以$I_n$为单位元的群$(M_n,*,I_n)$。
  \end{Example}
\begin{Def}
  群$(G,\circ)$称为有限群,如果$G$为有限集。$G$的基数称为群$G$的阶。如果$G$含有无穷多个元素,则称$G$为无限群。
\end{Def}

\begin{Def}
  设$a,n\in Z$,$n>0$,$a=bn+r$,$0\leq r<n$,则称$r$为$a$除以$n$所得到的余数,记为$a\mod n$。
\end{Def}
\begin{Def}
  设$a,b,n\in Z$,$n>0$,如果$a\mod n=b\mod n$,则称$a$与$b$模$n$同余,记为$a \equiv b(\mod n)$。
\end{Def}

\begin{Thm}
  $\forall a,b\in Z, a\equiv b(\mod n)$等价于$n|(a-b)$。
\end{Thm}

\begin{Thm}
  1. $\forall a\in Z, a \equiv a (\mod n)$;

  2. $\forall a, b\in Z$,如果$a\equiv b (\mod n)$,则$b\equiv a(\mod n)$;

  3. $\forall a,b,c\in Z$,如果$a\equiv b(\mod n)$并且$b\equiv c(\mod n)$,则$a\equiv c(\mod n)$;

  4. $\forall a,b,k\in Z$,如果$a\equiv b(\mod n)$,则$a+k\equiv b+k(\mod n)$;

  5. $\forall a,b,c,d\in Z$,如果$a\equiv b(\mod n)$并且$c\equiv d(\mod n)$,则$a+c\equiv b+d (\mod n)$;

  6. $\forall a,b,k\in Z$,如果$a\equiv b(\mod n)$,则$ak\equiv bk(\mod n)$;

  7. $\forall a,b,c,d\in Z$,如果$a\equiv b(\mod n)$并且$c\equiv d(\mod n)$,则$ac\equiv bd (\mod n)$

  8. $\forall a,b\in Z$,$ab (\mod n)=(a\mod n)(b\mod n) (\mod n)$。
\end{Thm}

\begin{Def}
  设$n\in Z$,$n>0$,$\forall x\in Z$,定义$[x]=\{y|y\equiv x (\mod n)\}$。
\end{Def}
\begin{Thm}
  设$n\in Z$,$n>0$,$\forall x\in z$,$[x]=[y]$当且仅当$x\equiv y(\mod n)$。
\end{Thm}

\begin{Example}
  设$Z_n=\{[0],[1],\cdots,[n-1]\}$为整数集$Z$上在模$n$同余的等价关系下所有等价类之集,
  在$Z_n$上定义加法运算“$+$”如下:$\forall [i],[j]\in Z_n,[i]+[j]=[i+j]$,则$(Z_n,+)$构成一个交换群;

  在$Z_n$上定义加法运算“$*$”如下:$\forall [i],[j]\in Z_n,[i]*[j]=[i*j]$,则$(Z_n,*)$构成一个交换幺半群;
\end{Example}

  课后作业题:

\begin{Exercise}
  给出一个半群,它有无穷多个右单位元素。
\end{Exercise}

\begin{Exercise}
  设$(S,\circ)$为一个半群,$a\in S$称为左消去元素,如果$\forall x, y\in S$,有$a\circ x=a\circ y$,则一定有$x=y$。试证:
  如果$a$和$b$均为左消去元,则$a\circ b$也是左消去元。
\end{Exercise}

\begin{Exercise}
  设$Z$为整数集合,$M=Z\times Z$。在$M$上定义二元运算$\circ$如下:

  $\forall (x_1,x_2), (y_1,y_2)\in M, (x_1,x_2)\circ (y_1,y_2)=(x_1y_1+2x_2y_2,x_1y_2+x_2y_1)$

  试证:

  (1)$M$对上述定义的代数运算构成一个幺半群。

  (2)如果$(x_1,x_2)\neq (0,0)$,则$(x_1,x_2)$是左消去元。

  (3)运算“$\circ$”满足交换率。
\end{Exercise}


\begin{Exercise}
  证明:有限半群中一定有一个元素$a$使得$a\circ a=a$。
\end{Exercise}


\begin{Exercise}
  设$R$为实数集合,$S=\{(a,b)|a\neq 0,a,b\in R\}$。在$S$上利用通常的加法和乘法定义二元运算“$\circ$”如下:
  \[(a,b)\circ (c,d) = (ac, ad + b)\]
  验证$(S,\circ)$为群。
\end{Exercise}

\begin{Exercise}
  $n$次方程$x^n=1$的根称为$n$次单位根,所有$n$次单位根之集记为$U_n$。证明:$U_n$对通常的复数乘法构成一个群。
\end{Exercise}


\begin{Exercise}
 令
 \[G=\bigg\{\begin{bmatrix}
  1&0\\0&1
 \end{bmatrix},
 \begin{bmatrix}
  -1&0\\0&-1
 \end{bmatrix},
 \begin{bmatrix}
  1&0\\0&-1
 \end{bmatrix},
 \begin{bmatrix}
  -1&0\\0&1
 \end{bmatrix}\bigg\}\] 
 试证:$G$对矩阵乘法构成一个群。
\end{Exercise}



%%% Local Variables:
%%% mode: latex
%%% TeX-master: "chapter1"
%%% End:

\end{CJK*}
\end{document}





%%% Local Variables:
%%% mode: latex
%%% TeX-master: t
%%% End:



