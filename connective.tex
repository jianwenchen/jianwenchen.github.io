\documentclass{article}
\usepackage{CJKutf8}
\usepackage{amsmath}
\usepackage{amssymb}
\usepackage{amsfonts}
\usepackage{amsthm}
\usepackage{titlesec}
\usepackage{titletoc}
\usepackage{xCJKnumb}
\usepackage{tikz}
\usepackage{mathrsfs}
\usepackage{indentfirst}
\usepackage{enumitem}
\newtheorem{Def}{定义}
\newtheorem{Thm}{定理}
\newtheorem{Exercise}{练习}

\newtheorem*{Example}{例}
\setlist[enumerate,1]{label=(\arabic*)}

\begin{document}
\begin{CJK*}{UTF8}{gbsn}
  \title{第三讲 联结词的扩充与规约}
  \author{陈建文}
  \maketitle
  % \tableofcontents

  \section{联结词的扩充}

  一元逻辑运算:

  \begin{tabular}{c|c}
    p& $\lnot$ p\\
    \hline
    T&F\\
    F&T\\
  \end{tabular}

  $\lnot: \{T,F\}\to\{T,F\}$

  \begin{tabular}{c|cccc}
    p& $f_1(p)$ & $f_2(p)$ & $f_3(p)$ & $f_4(p)$ \\
    \hline
    T&T&T&F&F\\
    F&T&F&T&F\\
  \end{tabular}

  $f_i:\{T,F\}\to \{T,F\}$

  二元逻辑运算:

  \begin{tabular}{cc|c}
    p& q& p $\land$ q\\
    \hline
    T&T&T\\
    T&F&F\\
    F&T&F\\
    F&F&F\\
  \end{tabular}

  $\land : \{T,F\}\times \{T,F\}\to \{T,F\}$


  \begin{tabular}{cc|c}
    p& q& p $\downarrow$ q\\
    \hline
    T&T&F\\
    T&F&F\\
    F&T&F\\
    F&F&T\\
  \end{tabular}

  $p\downarrow q \Leftrightarrow \lnot (p\lor q)$


  \begin{tabular}{cc|c}
    p& q& p $\uparrow$ q\\
    \hline
    T&T&F\\
    T&F&T\\
    F&T&T\\
    F&F&T\\
  \end{tabular}

  $p\uparrow q \Leftrightarrow \lnot (p\land q)$

  \begin{tabular}{cc|c}
    p& q& p $\oplus $ q\\
    \hline
    T&T&F\\
    T&F&T\\
    F&T&T\\
    F&F&F\\
  \end{tabular}

  $p\oplus q \Leftrightarrow \lnot (p\leftrightarrow q)$

  \section{联结词的规约}

  \begin{Def}[联结词的可表示性]
    设$h$为一个$n$元联结词,$A$为由$m$个联结词$g_1,g_2,\cdots, g_m$构成的命题公式,如果$h(p_1,p_2,\ldots,p_n)\Leftrightarrow A$,则称联结词$h$可由联结词$g_1,g_2,\cdots, g_m$表示。
  \end{Def}
  \begin{Def}[完备的联结词集合]
    设$C$为一个联结词的集合,如果任意一个$n$元联结词都与由$C$中的联结词表示的一个命题公式逻辑等价,则称$C$为完备的联结词集合。
  \end{Def}
  \begin{Thm}
    $\{\lnot, \land, \lor\}$是完备的联结词集合。
  \end{Thm}

  \begin{tabular}{ccc|c}
    $p$& $q$& $r$& $h(p,q,r)$\\
    \hline
   T& T&T&T\\
    T&T&F&T\\
    T&F&T&T\\
     T& F&F&F\\
    F&T&T&T\\
    F&T&F&F\\
   F& F&T&F\\
    F&  F&F&F\\      
  \end{tabular}

主合取范式:$(\lnot p\lor q\lor r)\land (p\lor \lnot q\lor r)\land (p\lor q\lor \lnot r) \land (p\lor q\lor r)$

主析取范式:$(p\land q\land r)\lor (p\land q\land \lnot r)\lor (p\land \lnot q\land r) \lor (\lnot p\land q\land r)$


  \begin{Thm}
    $\{\lnot, \lor\}$是完备的联结词集合。
  \end{Thm}
  \begin{proof}[证明]
    $p\land q\Leftrightarrow \lnot (\lnot p\lor \lnot q)$
  \end{proof}
  \begin{Thm}
    $\{\lnot, \land\}$是完备的联结词集合。
  \end{Thm}
  \begin{proof}[证明]
    $p\lor q\Leftrightarrow \lnot (\lnot p\land \lnot q)$
  \end{proof}
  \begin{Thm}
    $\{\lnot, \to\}$是完备的联结词集合。
  \end{Thm}
  \begin{proof}[证明]
    $p\lor q\Leftrightarrow \lnot \lnot p \lor q \Leftrightarrow \lnot p \to q$
  \end{proof}
  \begin{Thm}
    $\{\downarrow\}$是完备的联结词集合。
  \end{Thm}
  \begin{proof}[证明]
    $\lnot p \Leftrightarrow \lnot (p\lor p) \Leftrightarrow p \downarrow p$

    $p\lor q \Leftrightarrow \lnot \lnot (p\lor q) \Leftrightarrow \lnot (p\downarrow q) \Leftrightarrow (p\downarrow q)\downarrow (p\downarrow q)$
  \end{proof}
  \begin{Thm}
    $\{\uparrow\}$是完备的联结词集合。
  \end{Thm}
  \begin{proof}[证明]
    $\lnot p \Leftrightarrow \lnot (p\land p) \Leftrightarrow p \uparrow p$

    $p\land q \Leftrightarrow \lnot \lnot (p\land q) \Leftrightarrow \lnot (p\uparrow q) \Leftrightarrow (p\uparrow q)\uparrow (p\uparrow q)$
  \end{proof}
  \begin{Example}
    用$\uparrow$表示公式$(p\to \lnot q)\to \lnot r$。
  \end{Example}
  \begin{proof}[解]
    \begin{align*}
      &(p\to \lnot q)\to \lnot r\Leftrightarrow (\lnot p \lor \lnot q)\to \lnot r\\
      &\Leftrightarrow \lnot (\lnot p \lor \lnot q)\lor \lnot r \Leftrightarrow\lnot ((\lnot p \lor \lnot q)\land r)\\
      &\Leftrightarrow (\lnot p \lor \lnot q)\uparrow r \Leftrightarrow (\lnot (p \land q))\uparrow r\Leftrightarrow (p\uparrow q)\uparrow r\\
    \end{align*}
  \end{proof}
  课后作业题:
  % \begin{Exercise}
  %   用$\{\lnot,\to\}$等价表示下列公式:
  %   \[(p\lor(p\land q))\leftrightarrow p\]
  % \end{Exercise}
  \begin{Exercise}
    用$\{\downarrow, \uparrow \}$分别等价表示下列公式:
    \begin{enumerate}
      \item $\lnot p\lor q$
      \item $p\land \lnot q$
      \item $\lnot p \lor \lnot q$
      \item $p\leftrightarrow q$
    \end{enumerate}

    
  \end{Exercise}
\end{CJK*}
\end{document}





%%% Local Variables:
%%% mode: latex
%%% TeX-master: t
%%% End: