\documentclass{article}
\usepackage{CJKutf8}
\usepackage{amsmath}
\usepackage{amssymb}
\usepackage{amsfonts}
\usepackage{amsthm}
\usepackage{titlesec}
\usepackage{titletoc}
\usepackage{xCJKnumb}
\usepackage{tikz}
\usepackage{mathrsfs}
\usepackage{indentfirst}

\newtheorem{Def}{定义}
\newtheorem{Thm}{定理}
\newtheorem{Exercise}{练习}

\newtheorem*{Example}{例}


\begin{document}
\begin{CJK*}{UTF8}{gbsn}
  \title{第九讲 同态基本定理}
  \author{陈建文}
  \maketitle
  % \tableofcontents
  


课后作业题:
\begin{Exercise}
设$(G,\circ)$为$m$阶循环群,$(\bar{G},\cdot)$为$n$阶循环群,试证:$G \sim \bar{G}$当且仅当$n | m$。
\end{Exercise}
\begin{proof}[证明]
由$G\sim \bar{G}$往证$n|m$:

设$\phi$为从$G$到$\bar{G}$的一个满同态,由群同态基本定理,$G/Ker \phi\cong \bar{G}$,于是$|G/Ker \phi|=|\bar{G}|$。由拉格朗日定理,$|G|=|G/Ker \phi||Ker \phi|$,这说明$|G/Ker \phi|||G|$,从而$|\bar{G}|||G|$,即$n|m$。

设$n|m$,往证$G\sim \bar{G}$:

设$G=(a)$,$\bar{G}=(b)$。

令$\phi:G\to \bar{G}$,$\forall i\in Z$,$\phi(a^i)=b^i$。

$\forall i,j\in Z, i\neq j$,如果$a^i=a^j$,则$a^{j-i}=e$,从而$m|(j-i)$,由$n|m$知$n|(j-i)$,于是$b^{j-i}=e$,所以$b^i=b^j$,这验证了映射定义的合理性。

$\forall i,j\in Z$,$\phi(a^i\circ a^j)=\phi(a^{i+j})=b^{i+j} =b^i\cdot b^j=\phi(a^i)\cdot \phi(a^j)$。 
这证明了$\phi$为从$G$到$\bar{G}$的同态,$\phi$显然为满同态,于是$G\sim \bar{G}$。
\end{proof}
\begin{Exercise}
设$G$为一个循环群,$H$为群$G$的子群,试证:$G/H$也为循环群。
\end{Exercise}
\begin{proof}[证明]
由$G$为循环群知$G$为交换群,从而$H$为$G$的正规子群。设$G=(a)$,以下证明$G/H=(aH)$,从而$G/H$为循环群。

$\forall x\in G/H$,存在整数$i$使得$x=a^iH=(aH)^i\in (aH)$,于是$G/H\subseteq (aH)$;

$\forall x\in (aH)$,存在整数$i$使得$x=(aH)^i=a^iH\in G/H$,于是$(aH)\subseteq G/H$。

以上证明了$G/H=(aH)$。
\end{proof}
\end{CJK*}
\end{document}





%%% Local Variables:
%%% mode: latex
%%% TeX-master: t
%%% End:



