\documentclass{article}
\usepackage{tikz}
\usepackage{CJKutf8}
\usepackage{amsmath}
\usepackage{amsthm}
\usepackage{enumitem}
\newtheorem{Def}{定义}
\newtheorem{Exercise}{练习}
\newtheorem*{Thm}{定理}
\newtheorem*{Example}{例}
\setlist[enumerate,1]{label=(\arabic*)}
\begin{document}
\begin{CJK}{UTF8}{gbsn}
  \title{第一讲 命题与联结词}
  \author{陈建文}
  \maketitle
  \section{命题}
  命题:可以判断真假的陈述句。通常,我们用$T$表示真,用$F$表示假。
  \begin{Example}\quad
    
    \begin{enumerate}
    \item 对任意的自然数$a,b,c$,$(a + b) + c = a + (b + c)$。(真命题)
    \item $\sqrt{2}$是无理数。(真命题)
    \item $\sqrt{2}$是有理数。(假命题)
    \item 设$f:[a,b] \to R$为一个Riemann可积函数,$F:[a,b] \to R$在$[a,b]$上满足$F'(x)=f(x)$,那么$\int_{a}^{b}f(x)dx = F(b) - F(a)$。(真命题)
    % \item 任何一幅地图都可以用四种颜色进行着色,使得相邻的区域着以不同的颜色。
     \end{enumerate}
  \end{Example}
  谓词:命题的谓语部分。
  
  \begin{Example}\quad
    
    \begin{description}
    \item     [$P(x): x$ 是偶数] 这里$P$为一元谓词,表示“是偶数”。当$x$为某个确定的数字时,$P(x)$则对应一个命题。例如$P(2)$为真命题,$P(1)$为假命题。这里,$P$之所以被称为一元谓词,是因为$P(x)$只包含一个变量$x$。
    \item     [$P(x,y): x >y$]  这里$P$为二元谓词,表示$>$。当$x$和$y$为确定的数字时,$P(x,y)$则对应一个命题。例如$1>0$为真命题,$0>1$为假命题。这里,$P$之所以被称为二元谓词,是因为$P(x,y)$包含两个变量$x$和$y$。
    \end{description}
相应的,有三元谓词,四元谓词,......
\end{Example}

我们还可以用如下方式由谓词得到命题:

\begin{description}
\item [$\forall x P(x)$:] 对任意的$x$,$P(x)$。For All中的$A$上下颠倒可以得到$\forall$。
\item [$\exists x P(x)$:] 存在$x$,$P(x)$。There Exists中的$E$左右颠倒可以得到$\exists$。
\end{description}
 
\section{命题联结词及真值表}
命题可以由联结词$\lnot$,$\land$,$\lor$,$\to$,$\leftrightarrow$联结而构成复合命题。

设$p$为命题,则$\lnot p$表示“$p$不成立”。

 \begin{tabular}{c|c}
    p& $\lnot$ p\\
    \hline
    T&F\\
    F&T\\
  \end{tabular}

  设$p$和$q$为两个命题,则$p\land q$表示“$p$成立,并且$q$成立”。
  
  \begin{tabular}{cc|c}
    p& q& p $\land$ q\\
    \hline
    T&T&T\\
    T&F&F\\
    F&T&F\\
    F&F&F\\
  \end{tabular}

  设$p$和$q$为两个命题,则$p\lor q$表示“$p$成立,或者$q$成立”。
  
  \begin{tabular}{cc|c}
    p& q& p $\lor$ q\\
    \hline
    T&T&T\\
    T&F&T\\
    F&T&T\\
    F&F&F\\
  \end{tabular}

设$p$和$q$为两个命题,则$p\to q$表示“如果$p$成立,那么$q$成立”。  

    \begin{tabular}{cc|c}
    p& q& p $\to$ q\\
    \hline
    T&T&T\\
    T&F&F\\
    F&T&T\\
    F&F&T\\
    \end{tabular}\hspace{0.87cm}

    这里需要注意的是,当$p$为假时,则$p\to q$一定为真,这是所有数学家共同的约定。
    下面的例子可以帮助大家更好的理解其实我们已经用到了这个约定。

    对任意的实数$x$,当$x>1$时,$x^2 > 1$。该命题显然是真命题,可以符号化为$\forall x \; x > 1 \to x^2 > 1$。那么,既然对于任意的$x$,$x>1 \to x^2>1$成立,则

    1)当$x=2$时, $2 > 1 \to 2^2 >1$成立,这对应于以上真值表的第一行;

    2)当$x=0$时,$0 > 1 \to 0^2 > 1$成立,这对应于以上真值表的第四行;

    3)当$x=-2$时,$-2>1 \to (-2)^2 > 1$成立,这对应于以上真值表的第三行。
    
设$p$和$q$为两个命题,则$p\leftrightarrow q$表示“$p$等价于$q$”。  

  \begin{tabular}{cc|c}
    p& q& p $\leftrightarrow$ q\\
    \hline
    T&T&T\\
    T&F&F\\
    F&T&F\\
    F&F&T\\
  \end{tabular}

  \section{命题公式及真值}
  \begin{Def}
    \begin{enumerate}
      \item 任意一个表示命题的符号为命题公式(这样的命题公式又称为原子公式);
      \item 如果$A,B$是命题公式,则$(\lnot A),(A\land B),(A\lor B),(A\to B),(A\leftrightarrow B)$是命题公式;
      \item 有限次使用(1)与(2)复合所得到的结果都是命题公式。
    \end{enumerate}
  \end{Def}
\begin{Example}
  $((\lnot p)\lor q)$,$(\lnot (p\lor q))$,$((p\lor q)\to (r\land s))$都是命题公式。
\end{Example}

最外层的括号可以省略;

五个逻辑联结词优先级从高到低依次为$\lnot, (\land, \lor), \to, \leftrightarrow$;

$\land, \lor$左结合。
\begin{Example}
  $((\lnot p)\lor q)$,$(\lnot (p\lor q))$,$((p\lor q)\to (r\land s))$可以简写为$\lnot p\lor q$,$\lnot (p\lor q)$,$p\lor q\to r\land s$。
\end{Example}
  \begin{Def}
    设$A(p_1,p_2,\cdots,p_n)$为一个含有$n$个命题变元$p_1,p_2,\cdots, p_n$的命题公式,$\alpha:\{p_1,p_2,\cdots, p_n\}\to \{T,F\}$称为一个真值指派。$\forall i, 1\leq i \leq n$,当$p_i$的真值为$\alpha(p_i)$时,如果公式$A$的真值为$T$,记为$\alpha(A)=T$;如果公式$A$的真值为$F$,记为$\alpha(A)=F$。
  \end{Def}
\begin{Example}
  设$\alpha(p)=T,\alpha(q)=F$,则$\alpha(\lnot p \lor q)=F$;设$\alpha(p)=T, \alpha(q)=T, \alpha(\lnot p \lor q)=T$。
\end{Example}
  \begin{Def}
    如果公式$A$对任一真值指派其真值均为真,则称之为重言式(永真式)。
  \end{Def}
  \begin{Example}
    $A\to(B\to A)$,$(A\to(B\to C))\to((A\to B)\to (A\to C))$,$(\lnot A\to \lnot B)\to (B \to A)$
都是重言式。
  \end{Example}
  \begin{Def}
    如果公式$A$对任一真值指派其真值均为假,则称之为永假式。
  \end{Def}
  \begin{Example}
    $P\land \lnot P$为永假式。
  \end{Example}
  \begin{Def}
    如果公式$A$存在一个真值指派使其真值为真,则称之为可满足式。
  \end{Def}
  \begin{Example}
    $A\to (B\to C)$,$A\lor B$都是可满足式。
  \end{Example}
\section{逻辑蕴含与逻辑等价}
\begin{Def}
设$A,B$为任意两个公式,对任意的指派$\alpha$,如果$\alpha(A)=T$,那么$\alpha(B)=T$,则称$A$逻辑蕴含$B$,记为$A\Rightarrow B$。
设$\Gamma=\{A_1,A_2,\cdots,A_n\}$为一个公式集,$B$为任意一个公式,对任意的指派$\alpha$,如果$\forall i, 1\leq i \leq n, \alpha(A_i)=T$,那么$\alpha(B)=T$,则称$\Gamma$逻辑蕴含$B$,记为$\Gamma \Rightarrow B$。
\end{Def}
\begin{Example}
  判定下列逻辑蕴含是否成立。
  \begin{enumerate}
    \item $\lnot A\Rightarrow A\to B$
    \item $\Gamma \Rightarrow A\to C$,其中公式集$\Gamma=\{A\to (B\to C), B\}$
  \end{enumerate}
 
    (1)成立。
    \begin{proof}[解法一]$\quad$

    \begin{tabular}{cc|cc}
      $A$& $B$&$\lnot A$&$A\to B$\\
      \hline
      T&T&F&T\\
      T&F&F&F\\
      F&T&T&T\\
      F&F&T&T\\      
    \end{tabular}

    从以上真值表可以看出使得$\lnot A$为真的指派也使得$A\to B$为真。
  \end{proof}

  \begin{proof}[解法二]
    对任意的指派$\alpha$,如果$\alpha(\lnot A)=T$,则$\alpha(A)=F$,此时必有$\alpha(A\to B)=T$。
  \end{proof}
   
    (2)成立。
  \begin{proof}[解法一]
    使得$\Gamma$中的每个公式为真的指派分别为

    $\alpha_1(B)=T, \alpha_1(A)=F, \alpha_1(C)=T$,此时$\alpha_1(A\to C)=T$

    $\alpha_2(B)=T, \alpha_2(A)=F, \alpha_2(C)=F$,此时$\alpha_2(A\to C)=T$

    $\alpha_3(B)=T, \alpha_3(A)=T, \alpha_3(C)=T$,此时$\alpha_3(A\to C)=T$

    故$\Gamma \Rightarrow A\to C$成立。
  \end{proof}
\begin{proof}[解法二]
    对任意的指派$\alpha$,如果$\alpha(A\to C)=F$,则$\alpha(A)=T$,$\alpha(C)=F$,此时如果$\alpha(B)=T$,则$\alpha(A\to(B\to C))=F$。于是,对任意的指派$\alpha$,如果$\alpha(A\to(B\to C))=T$,$\alpha(B)=T$,必有$\alpha(A\to C)=T$,从而$\Gamma\Rightarrow A\to C$成立。
\end{proof}   

\end{Example}
\begin{Def}
  设$A,B$为任意两个公式,如果$A\Rightarrow B$并且$B\Rightarrow A$,则称$A$与$B$逻辑等价,记为$A\Leftrightarrow B$。
\end{Def}

  
$p\land (q\lor r)$与$(p\land q)\lor (p \land r)$逻辑等价。

    \begin{tabular}{ccc|cc}
    $p$& $q$& $r$& $p\land (q\lor r)$&$(p\land q)\lor (p \land r)$\\
    \hline
   T& T&T&T&T\\
    T&T&F&T&T\\
    T&F&T&T&T\\
     T& F&F&F&F\\
    F&T&T&F&F\\
    F&T&F&F&F\\
   F& F&T&F&F\\
    F&  F&F&F&F\\      
  \end{tabular}

  

  $p\lor (q\land r)$与$(p\lor q)\land (p \lor r)$逻辑等价。

      \begin{tabular}{ccc|cc}
    $p$& $q$& $r$&$p\lor (q\land r)$ &$(p\lor q)\land (p \lor r)$\\
    \hline
    T&T&T&T&T\\
    T&T&F&T&T\\
    T&F&T&T&T\\
     T& F&F&T&T\\
    F&T&T&T&T\\
    F&T&F&F&F\\
    F&F&T&F&F\\
     F&F&F&F&F\\
  \end{tabular}

  $p \lor (p\land q)$与$p$逻辑等价。 

  $p \land (p\lor q)$与$p$逻辑等价。

  $\lnot (p\land q)$与$\lnot p \lor \lnot q$逻辑等价。

      \begin{tabular}{cc|cc}
    $p$& $q$&$\lnot (p\land q)$ &$\lnot p \lor \lnot q$\\
    \hline
    T&T&F&F\\
    T&F&T&T\\
    F&T&T&T\\
      F&F&T&T\\      
  \end{tabular}

   

  $\lnot (p \lor q)$与$\lnot p \land \lnot q$逻辑等价。
  
      \begin{tabular}{cc|cc}
    $p$& $q$&$\lnot (p \lor q)$&$\lnot p \land \lnot q$\\
    \hline
    T&T&F&F\\
    T&F&F&F\\
    F&T&F&F\\
    F&F&T&T\\      
  \end{tabular}

  $p \to q$与$\lnot p \lor q$逻辑等价。
  
      \begin{tabular}{cc|cc}
    $p$& $q$&$p \to q$&$\lnot p \lor q$\\
    \hline
    T&T&T&T\\
    T&F&F&F\\
    F&T&T&T\\
    F&F&T&T\\      
  \end{tabular}
  
  \begin{Thm}
“如果$p$成立,那么q成立”等价于“$p$不成立或者$q$成立”。
  \end{Thm}
  \begin{proof}[证明]$\quad$


    $\Rightarrow$如果$p$成立,那么$q$成立,此时“$p$不成立或者$q$成立”成立;如果$p$不成立,此时亦有“$p$不成立或者$q$成立”成立。

$\Leftarrow$如果$p$成立,此时$p$不成立不可能,从而$q$成立。
  \end{proof}

  

  $p\leftrightarrow q$与$(p\to q) \land (q \to p)$逻辑等价。

  \begin{tabular}{cc|cc}
    $p$& $q$&$p \leftrightarrow q$&$(p\to q) \land (q \to p)$\\
    \hline
    T&T&T&T\\
    T&F&F&F\\
    F&T&F&F\\
    F&F&T&T\\      
  \end{tabular}
  
  代入原理:设$A$为含命题变元$p$的重言式,将$A$中$p$的所有出现均代换为命题公式$B$,所得到的公式仍为重言式。


  \begin{Example}
    $p\to (q\to p)$为重言式,将其中命题变元$p$的所有出现均用公式$r\lor s$代换所得到的公式$(r\lor s)\to (q\to (r\lor s))$仍为重言式。
  \end{Example}
  替换原理:设$C$为命题公式$A$中的子命题公式,$C\Leftrightarrow D$,将$C$用$D$替换(未必对所有的子公式$C$均作替换)后得到公式$B$,则$A\Leftrightarrow B$。

  \begin{Example}
    由于$P\to Q \Leftrightarrow \lnot P \lor Q$,
    \begin{align*}
      &(P\to Q)\land ((R\to (P\to Q))\lor(\lnot S \land (P\to Q)))\\
      \Leftrightarrow&(\lnot P \lor Q)\land ((R\to (P\to Q))\lor(\lnot S \land (\lnot P\lor Q)))
    \end{align*}

  \end{Example}

  课后作业题:
  \begin{Exercise}
    判定下列逻辑蕴含和逻辑等价是否成立,其中$A,B,C,D$为任意公式。
    \begin{enumerate}
      \item $A\Rightarrow  B\to A$
      \item $A\to (B\to C)\Rightarrow (A\to B)\to (A\to C)$
      \item $\lnot A\to \lnot B\Leftrightarrow B\to A$
      \item $A\to (B\to C)\Leftrightarrow (A\land B)\to C$
      \item $(A\lor B)\to C\Leftrightarrow (A\to C)\land (B\to C)$
      \item $\lnot A\lor B,A\to (B\land C),D\to B\Rightarrow \lnot B\to C$   
    \end{enumerate}
  \end{Exercise}
  % 我们还可以利用真值表检验$(p\to q) \land (p\to \lnot q) \to \lnot p$是永真的。
  
  %     \begin{tabular}{cc|cc}
  %   $p$& $q$&$(p\to q) \land (p\to \lnot q) \to \lnot p$\\
  %   \hline
  %   T&T&T\\
  %   T&F&T\\
  %   F&T&T\\
  %   F&F&T\\      
  %     \end{tabular}

  %     假设我约定"$\to$"的真值表如下:

  %   \begin{tabular}{cc|c}
  %   p& q& p $\to$ q\\
  %   \hline
  %   T&T&T\\
  %   T&F&F\\
  %   F&T&F\\
  %   F&F&T\\
  %   \end{tabular}\hspace{0.87cm}

  %   我们会发现复合命题$(p\to q) \land (p\to \lnot q) \to \lnot p$不是永真的,这将与我们关于“蕴含”的思维不相符。

  %   同时我们还会发现$p \to q$和$q\to p$在逻辑上是等价的。

  %   \begin{tabular}{cc|cc}
  %   p& q& p $\to$ q&q $\to$ p\\
  %   \hline
  %   T&T&T&T\\
  %   T&F&F&F\\
  %   F&T&F&F\\
  %   F&F&T&T\\
  %   \end{tabular}\hspace{0.87cm}

  %   这也与我们的思维习惯不相符。

    

  
  % 有些逻辑术语从外文翻译成中文时产生了不同的称谓,在本门课程中关于逻辑术语我们做如下的约定:


  % The negation of a proposition $P$: $\lnot P$

  % 命题$P$的否定:$\lnot P$

  % The converse of $P\to Q$: $Q \to P$

  % 命题$P\to Q$的逆命题:$Q\to P$



  % The inverse of $P\to Q$: $\lnot P \to \lnot Q$
  
  % 在较深入的探讨数理逻辑的教材中,该概念用的很少,因此我们不给出具体的翻译称谓,在需要表达该概念时明确说明为$\lnot P \to \lnot Q$即可。

  
  % The contrapositive of $P\to Q$: $\lnot Q \to \lnot P$

  % 命题$P\to Q$的逆否命题:$\lnot Q \to \lnot P$

  % 需要特别说明的是,命题$P\to Q$的否定为$\lnot (P \to Q)$,而不是$P \to \lnot Q$。 

  
  

  
\end{CJK}
\end{document}


%%% Local Variables:
%%% mode: latex
%%% TeX-master: t
%%% End:
