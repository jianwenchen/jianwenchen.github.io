\documentclass{article}
\usepackage{CJKutf8}
\usepackage{amsmath}
\usepackage{amssymb}
\usepackage{amsfonts}
\usepackage{amsthm}
\usepackage{titlesec}
\usepackage{titletoc}
\usepackage{xCJKnumb}
\usepackage{tikz}
\usepackage{mathrsfs}
\usepackage{indentfirst}

\newtheorem{Def}{定义}
\newtheorem{Thm}{定理}
\newtheorem{Exercise}{练习}

\newtheorem*{Example}{例}


\begin{document}
\begin{CJK*}{UTF8}{gbsn}
  \title{第三讲 群的简单性质}
  \author{陈建文}
  \maketitle
  % \tableofcontents
  
  \begin{Def}
    设$G$为一个非空集合,“$\circ$”为$G$上的一个二元代数运算。如果下列各个条件成立,则称$G$对“$\circ$”运算构成一个群(group):
  
    I. “$\circ$”运算满足结合律,即$\forall a,b,c \in G$ $(a\circ b)\circ c = a\circ(b\circ c)$;
  
    II. 对“$\circ$”运算,$G$中有一个左单位元$e$,即$\forall a\in G$ $e\circ a = a$;
  
    III. $\forall a\in G \exists b\in G b\circ a = e$,其中$e$为II中的同一个左单位元素。
  \end{Def}
  在群$(G,\circ)$中,$\forall a,b\in G, a\circ b$简写为$ab$。
  \begin{Thm}
    设$G$为一个群,则$\forall a,b\in G$,如果$ba=e$,则$ab=e$。
  \end{Thm}
  \begin{proof}[证明]
    在\[ba=e\]的
    两边同时右乘以$b$得
    \[(ba)b=eb\]
    从而
    \[b(ab)=b\]
    在$G$中存在$c$使得$cb=e$,于是
    \[c(b(ab))=cb\]
    所以
    \[ab=e\]
  \end{proof}
  \begin{Thm}
    设$G$为一个群,则$G$的左单位元$e$也是右单位元。
  \end{Thm}
  \begin{proof}[证明]
    $\forall a\in G$,设$b\in G$,$ba=e$,则$ae=a(ba)=(ab)a=ea=a$,所以$e$也是右单位元。
  \end{proof}
  \begin{Thm}
    设$a$与$b$为群$G$的任意两个元素,则$(a^{-1})^{-1}=a$,$(ab)^{-1}=b^{-1}a^{-1}$。
  \end{Thm}
  \begin{proof}[证明]
    由\[aa^{-1}=e\]
    得\[(a^{-1})^{-1}=a\]
  由\[(b^{-1}a^{-1})(ab)=b^{-1}(a^{-1}a)b=b^{-1}eb=e\]
  得\[(ab)^{-1}=b^{-1}a^{-1}\]
  \end{proof}
  \begin{Thm}
   在群$G$中,$\forall a, b \in G$,方程
   \begin{align*}
    ax&=b\\
    ya&=b
   \end{align*}
   关于未知量$x$与$y$都有唯一解。 
  \end{Thm}
  
  \begin{Thm}
    非空集合$G$对其二元代数运算$\circ$构成一个群的充分必要条件是下列两个条件同时成立:
  
    1. “$\circ$”运算满足结合律,即$\forall a,b,c\in G (a\circ b)\circ c=a\circ(b\circ c)$。
  
    2. $\forall a,b\in G$,方程
    \begin{align*}
      ax&=b\\
      ya&=b
     \end{align*}
     关于未知量$x$与$y$有解。
  \end{Thm}
  \begin{proof}[证明]
  
    $\Leftarrow$
  
    由$G$非空知$\exists b,b\in G$。方程$yb=b$有解,设$e$为一个解,则$eb=b$。$\forall a\in G$,方程$bx=a$有解,设$c$为一个解,则$bc=a$。
    于是
    \[ea=e(bc)=(eb)c=bc=a\]
    从而$e$为左单位元。
  
    $\forall a\in G$,方程$ya=e$有解,其解为$a$的左逆元。
  \end{proof}
  \begin{Thm}
    设$(G,\circ)$为一个群,则“$\circ$”运算满足消去律,即$\forall x, y, a\in G$,
  
    如果$ax = ay$,则$x=y$(左消去律)
  
    如果$xa = ya$, 则$x=y$(右消去律)
  \end{Thm}
  
  \begin{Thm}
    非空有限集合$G$对在其上定义的二元代数运算$\circ$构成一个群的充分必要条件是下列两个条件同时成立:
  
    1. “$\circ$”运算满足结合律。
  
    2. “$\circ$”运算满足左、右消去律。
  \end{Thm}
  \begin{proof}[证明]
  $\Leftarrow$
  
  先证$\forall a,b\in G$,方程$ax=b$有解。
  
  令$f:G\to aG=\{ag|g\in G\}$,$\forall x\in G, f(x)=ax$。则$f$为单射,这是因为$\forall x_1,x_2\in G$,如果$f(x_1)=f(x_2)$,则$ax_1=ax_2$,由左消去律得$x_1=x_2$;
  同时,$f$为满射,这是因为$\forall y\in aG$,$\exists x\in G$,$y=ax$,于是$f(x)=ax=y$。
  此时必有$aG=G$,否则$aG\subseteq G$且$aG\neq G$,从而$aG$为$G$的真子集,于是$f$为有限集$G$与其真子集之间的一个双射,矛盾。
  由$f:G\to aG=G$为双射知,$\forall b\in G$,$\exists c\in G$,$ac=b$。所以,方程$ax=b$在$G$中有解。
  
  同理可证,  $\forall a,b\in G$,方程$ya=b$有解。
  \end{proof}
  
  \begin{Example}
    3阶群是交换群。
  \end{Example}
  \begin{Def}
    设$G$为一个群,$\forall a\in G$,定义$a^0=e$,$a^{n+1}=a^n\circ a$$(n\geq 0)$,$a^{-n}=(a^{-1})^n$$(n\geq 1)$。
  \end{Def}
  \begin{Thm}
  设$G$为一个群,$a\in G$,$m$,$n$为任意整数,则$a^ma^n=a^{m+n}$,$(a^m)^n=a^{mn}$。
  \end{Thm}
  \begin{proof}[证明]
    1. $a^ma^n=a^{m+n}$
  
  $a^2a^3=a^5:(aa)(aaa)=a^5$
  
  $a^2a^{-2}=e:(aa)(a^{-1}a^{-1})=e$
  
  $a^{-2}a^{2}=e:(a^{-1}a^{-1})(aa)=e$
  
  $a^2a^{-3}=aa(a^{-1}a^{-1}a^{-1})=a^{-1}$
  
  $a^{-2}a^3=(a^{-1}a^{-1})aaa=a$
  
  
  
  $a^{-2}a^{-3}=(a^{-1}a^{-1})(a^{-1}a^{-1}a^{-1})=a^{-5}$
  
  $m\geq 0,n\geq 0:$
  
  对$n$归纳:
  
  (1)当$n=0$时,$a^ma^0=a^{m+0}$
  
  (2)当$n=k+1$时,$a^{m}a^{k+1}=a^m(a^ka)=(a^ma^k)a=a^{m+k}a=a^{m+k+1}$
  
  $m\geq 0,n \leq 0:$
  
  
  $m=s,n=-t, s\geq 0,t\geq 0:$
  
  当$s=t$时,要证$a^{s}a^{-s}=a^{s+(-s)}=a^0=e$,对$s$归纳:
  
  (1)当$s=0$时,$a^0a^{-0}=a^{0+(-0)}$。
  
  (2)当$s=k+1$时,$a^{k+1}a^{-(k+1)}=(a^ka)(a^{-1})^{k+1}=(a^ka)a^{-1}(a^{-1})^k=a^ka^{-k}=e$
  
  当$s>t$时,$a^sa^{-t}=a^{s-t}a^{t}a^{-t}=a^{s-t}$
  
  当$s<t$时,$a^sa^{-t}=a^s(a^{-1})^t=a^s(a^{-1})^s(a^{-1})^{t-s}=a^sa^{-s}a^{-(t-s)}=a^{s-t}$
  
  $m\leq 0, n\geq 0:$
  
  $m=-s,n=t,s\geq 0, t\geq 0:$
  
  $a^{-s}a^t=(a^{-1})^s((a^{-1})^{-1})^t=(a^{-1})^s(a^{-1})^{-t}=(a^{-1})^{s-t}=a^{-(s-t)}=a^{t-s}$
  
  $m<0, n<0:$
  
  $m=-s,n=-t, s>0,t>0$
  
  $a^{-s}a^{-t}=(a^{-1})^s(a^{-1})^t=(a^{-1})^{s+t}=a^{-(s+t)}$
  \end{proof}
  设$(G,+)$为一个阿贝尔群,$G$的单位元记为$0$。$\forall a\in G$,$a$的逆元记为$-a$。$\forall a\in G$,定义$0a=0$,$(n+1)a=na+a$$(n\geq 0)$,$(-n)a=n(-a)$$(n\geq 1)$。
  对任意整数$m$,$n$,$ma+na=(m+n)a$,$(mn)a=m(na)$,$n(a+b)=na+nb$。
  \begin{Def}
    设$(G,\circ)$为一个群,$a\in G$,使$a^n=e$的最小正整数$n$称为$a$的阶。如果不存在这样的正整数$n$,则称$a$的阶为无穷大。
  \end{Def}
  \begin{Thm}
    有限群的每个元素的阶不超过该有限群的阶。
  \end{Thm}
  \begin{proof}[证明]
    设群$G$的阶为$N$,则$a^0,a^1,a^2,\cdots,a^N$为$G$的$N+1$个元素,所以必有两个是相同的,设$a^k=a^l$,$0\leq k<l\leq N$。于是,$a^{l-k}=e$,$0<l-k\leq N$,从而$a$的阶不超过$N$。
  \end{proof}


课后作业题:

\begin{Exercise}
  设$a$和$b$为群$G$的两个元素。如果$(ab)^2=a^2b^2$,试证:$ab=ba$。
\end{Exercise}
\begin{proof}[证明]
  由已知条件知$abab=aabb$,两边同时左乘$a^{-1}$,右乘$b^{-1}$,得$ab=ba$。
\end{proof}
\begin{Exercise}
  设$G$为群。如果$\forall a\in G$,$a^2=e$,试证:$G$为交换群。
\end{Exercise}
\begin{proof}[证明]
  $\forall a,b\in G$,由已知条件知$a^2=e$,$b^2=e$,同时$(ab)^2=e$,即$abab=e$,两边同时左乘$a$,右乘$b$,得$ba=ab$,这证明了$G$为交换群。
\end{proof}
\begin{Exercise}
  证明:四阶群是交换群。
\end{Exercise}
\begin{proof}[证明]
  设在四阶群$(G,\circ)$中,$G=\{e,a,b,c\}$,

  其乘法表为:

\begin{tabular}{c|cccc}
  $\circ$&e&a&b&c\\
  \hline
e&e&a&b&c\\
a&a&aa&ab&ac\\
b&b&ba&bb&bc\\
c&c&ca&cb&cc\\
\end{tabular}

$ab\neq a$,否则$b=e$,矛盾;$ab\neq b$,否则$a=e$,也矛盾。于是$ab=e$或$c$。

当$ab=e$时,$a$为$b$的逆元,因此$ba=e$,此时$ab=ba$。

当$ab=c$时,此时亦有$ba\neq b$并且$ba\neq a$。如果$ba=e$,则$b$为$a$的逆元,于是$ab=e$,与$ab=c$矛盾。因此,必有$ba=c$,于是,$ab=ba$。


% \hspace{1cm}\begin{tabular}{c|cccc}
%   $\circ$&e&a&b&c\\
%   \hline
% e&e&a&b&c\\
% a&a&e&c&b\\
% b&b&c&a&e\\
% c&c&b&e&a\\
% \end{tabular}

% \vspace{1cm}
% \begin{tabular}{c|cccc}
%   $\circ$&e&a&b&c\\
%   \hline
% e&e&a&b&c\\
% a&a&b&c&e\\
% b&b&c&e&a\\
% c&c&e&a&b\\
% \end{tabular}\hspace{1cm}\begin{tabular}{c|cccc}
%   $\circ$&e&a&b&c\\
%   \hline
% e&e&a&b&c\\
% a&a&c&e&b\\
% b&b&e&c&a\\
% c&c&b&a&e\\
% \end{tabular}


同理可证$ac=ca$,$bc=cb$。因此,$(G,\circ)$一定为交换群。
\end{proof}
\begin{Exercise}
  证明:在任一阶大于2的非交换群里必有两个非单位元$a$和$b$,使得$ab=ba$。
\end{Exercise}
\begin{proof}[证明]
  设$G$为任一阶大于2的非交换群,$a\in G$且$a$不是$G$的单位元。令$b=a^{-1}$,$b$不是单位元,$ab=ba=e$。
\end{proof}
\begin{Exercise}
  有限阶群里阶大于2的元素的个数必为偶数。
\end{Exercise}
\begin{proof}[证明]
  设$G$为一个有限阶群,阶大于2的元素必成对出现,设$a\in G$,$a$的阶为$n(n>2)$,则$a^{-1}$的阶也为$n$。这里$a\neq a^{-1}$。
\end{proof}
\begin{Exercise}
  证明:偶数阶群里,阶为2的元素的个数必为奇数。
\end{Exercise}
\begin{proof}[证明]
  在偶数阶群里,阶大于2的元素的个数为偶数,单位元的阶为1,其余元素的阶的2,显然阶为2的元素的个数为奇数。
\end{proof}
\begin{Exercise}
  设$a$为群$G$的一个元素,$a$的阶为$n$且$a^m=e$,试证$n$能整除$m$。
\end{Exercise}
\begin{proof}[证明]
  设$m=nq+r(0\leq r <n)$,则$a^m=(a^n)^qa^r$,由$a^n=e$且$a^m=e$得$a^r=e$,再由$n$为$a$的阶知$r=0$(否则将存在比$n$更小的正整数$r$,$a^r=e$,与$a$的阶为$n$矛盾),这证明了$n$能整除$m$。
\end{proof}
\begin{Exercise}
  设$a_1,a_2,\cdots,a_n$为$n$阶群中的$n$个元素(它们不一定各不相同)。证明:存在整数$p$和$q$($1\leq p \leq q \leq n$),使得
  \[a_pa_{p+1}\cdots a_q=e\text{。}\]
\end{Exercise}
\begin{proof}[证明]
  考虑以下表达式:

  \begin{align*}
    &a_1\\
    &a_1a_2\\
    &\cdots\\
    &a_1a_2\cdots a_i\\
    &\cdots\\
    &a_1a_2\cdots a_n\\
  \end{align*}

  以上表达式中如果存在某个表达式计算结果为$e$,则结论成立。
如果以上表达式中任意一个计算结果都不为$e$,则其中必有两个表达式计算结果相等,不妨设$a_1a_2\cdots a_{p-1} = a_1a_2\cdots a_{p-1}a_p a_{p+1}\cdots a_q$,
两边依次同时左乘$a_1^{-1}$,$a_2^{-1}$,$\cdots$,$a_{p-1}^{-1}$,可得$a_pa_{p+1}\cdots a_q=e$。

\end{proof}
\begin{Exercise}
  设$a$和$b$为群$G$的两个元素,$ab=ba$,$a$的阶为$m$,$b$的阶为$n$。试证:乘积$ab$的阶为$m$与$n$的最小公倍数的约数。何时$ab$的阶为$mn$?
\end{Exercise}
\begin{proof}[证明]
  设$m$和$n$的最小公倍数为$k$,则$m|k$,$n|k$。设$k=xm$,$k=yn$,则$(ab)^k=a^kb^k=(a^m)^x(b^n)^y=e$,于是$ab$的阶整除$k$,即$ab$的阶为$m$与$n$的最小公倍数的约数。

  当$m$与$n$互素时,$ab$的阶为$mn$。


  设$ab$的阶为$t$,则$e=(ab)^{mt}=(a^m)^tb^{mt}=b^{mt}$,从而$n|mt$,由$n$与$m$互素知$n|t$。同理,$e=(ab)^{nt}=a^{nt}(b^n)^t=a^{nt}$,从而$m|nt$,由$m$与$n$互素知$m|t$。由$n|t$知$\exists s\in Z,t=ns$,再由$m|t$知$m|ns$,进一步,由$m$与$n$互素知$m|s$,从而$\exists p\in Z,s=pm$,于是$t=ns=n(pm)=p(mn)$,即$mn|t$。

  由$(ab)^{mn}=a^{mn}b^{mn}=e$知$t|mn$,所以$t=mn$。

  当$ab$的阶为$mn$时,必有$m$与$n$互素,否则设$d=(m,n)$,$d>1$,则$m$与$n$的最小公倍数$<mn$,而$ab$的阶为$m$与$n$的最小公倍数的约数,从而$ab$的阶$<mn$,矛盾。
\end{proof}

\end{CJK*}
\end{document}





%%% Local Variables:
%%% mode: latex
%%% TeX-master: t
%%% End: