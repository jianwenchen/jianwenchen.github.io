\documentclass{article}
\usepackage{CJKutf8}
\usepackage{amsmath}
\usepackage{amssymb}
\usepackage{amsfonts}
\usepackage{amsthm}
\usepackage{titlesec}
\usepackage{titletoc}
\usepackage{xCJKnumb}
\usepackage{tikz}
\usepackage{mathrsfs}
\usepackage{indentfirst}

\newtheorem{Def}{定义}
\newtheorem{Thm}{定理}
\newtheorem{Exercise}{练习}

\newtheorem*{Example}{例}


\begin{document}
\begin{CJK*}{UTF8}{gbsn}
  \title{第三讲 群的简单性质}
  \author{陈建文}
  \maketitle
  % \tableofcontents
  
  \begin{Exercise}
    设$R$为实数集合,$S=\{(a,b)|a\neq 0,a,b\in R\}$。在$S$上利用通常的加法和乘法定义二元运算“$\circ$”如下:
    \[(a,b)\circ (c,d) = (ac, ad + b)\]
    验证$(S,\circ)$为群。
  \end{Exercise}
  \begin{proof}[证明]

    $\forall (a,b),(c,d)\in S$,

    \[(a,b)\circ (c,d) = (ac, ad + b)\]

    当$a\neq 0$,$c\neq 0$时,$ac\neq 0$,因此运算封闭。


    $\forall (a,b),(c,d),(e,f)\in S$,
    \begin{align*}
      &((a,b)\circ (c,d))\circ (e,f) = (ac,ad+b)\circ (e,f)\\
      =&(ace,acf+ad+b)\\
      &(a,b)\circ ((c,d)\circ (e,f)) = (a,b)\circ (ce,cf+d)\\
      =&(ace,acf+ad+b)\\
      &((a,b)\circ (c,d))\circ (e,f) = (a,b)\circ ((c,d)\circ (e,f))\\
    \end{align*}
    这验证了“$\circ$”运算满足结合律。
  
    $\forall (c,d)\in S, (1,0)\circ (c,d) = (1\cdot c, 1\cdot d + 0) = (c,d)$,
  这验证了$(1,0)$为$\circ$运算的左单位元。
  
  $\forall (c,d)\in S,(1/c,-d/c)\circ (c,d) = (1,0)$,这验证了$(1/c,-d/c)$为$(c,d)$关于左单位元$(1,0)$的左逆元。
  \end{proof}
  \begin{Exercise}
    $n$次方程$x^n=1$的根称为$n$次单位根,所有$n$次单位根之集记为$U_n$。证明:$U_n$对通常的复数乘法构成一个群。
  \end{Exercise}
  \begin{proof}[证明]
    对任意的$x_1\in U_n$,$x_2\in U_n$,则$x^n_1=1$,$x_2^n=1$,从而$(x_1x_2)^n=1$,即$x_1x_2\in U_n$。
  
    结合律显然成立。
  
    $1^n=1$,从而$1\in U_n$。 $\forall x\in U_n, 1\cdot x = x$,这说明$1$为左单位元。
  
    $\forall x\in U_n$,则$x^n=1$,从而$(\frac{1}{x})^n=1$,即$\frac{1}{x}\in U_n$。$\frac{1}{x}\cdot x =1$说明$\frac{1}{x}$为$x$关于左单位元$1$的左逆元。
  
    以上验证了$U_n$对通常的复数乘法构成一个群。
  \end{proof}
  
  \begin{Exercise}
   令
   \[G=\bigg\{\begin{bmatrix}
    1&0\\0&1
   \end{bmatrix},
   \begin{bmatrix}
    -1&0\\0&-1
   \end{bmatrix},
   \begin{bmatrix}
    1&0\\0&-1
   \end{bmatrix},
   \begin{bmatrix}
    -1&0\\0&1
   \end{bmatrix}\bigg\}\] 
   试证:$G$对矩阵乘法构成一个群。
  \end{Exercise}
  
  \begin{proof}[证明]
    易验证$G$对矩阵的乘法满足封闭性。
  
    结合律显然成立。
  
    $\begin{bmatrix}
      1&0\\0&1
     \end{bmatrix}$为左单位元。
  
     $\begin{bmatrix}
      1&0\\0&1
     \end{bmatrix}$的左逆元为
     $\begin{bmatrix}
      1&0\\0&1
     \end{bmatrix}$,
     $\begin{bmatrix}
      -1&0\\0&-1
     \end{bmatrix}$的左逆元为
     $\begin{bmatrix}
      -1&0\\0&-1
     \end{bmatrix}$,
     $\begin{bmatrix}
      1&0\\0&-1
     \end{bmatrix}$的左逆元为$\begin{bmatrix}
      1&0\\0&-1
     \end{bmatrix}$,
     $\begin{bmatrix}
      -1&0\\0&1
     \end{bmatrix}$的左逆元为$\begin{bmatrix}
      -1&0\\0&1
     \end{bmatrix}$。
  
     以上验证了$G$对矩阵的乘法构成一个群。
    
  \end{proof}
  \begin{Exercise}
    设$a$和$b$为群$G$的两个元素。如果$(ab)^2=a^2b^2$,试证:$ab=ba$。
  \end{Exercise}
  \begin{proof}[证明]
    由已知条件知$abab=aabb$,两边同时左乘$a^{-1}$,右乘$b^{-1}$,得$ab=ba$。
  \end{proof}
  \begin{Exercise}
    设$G$为群。如果$\forall a\in G$,$a^2=e$,试证:$G$为交换群。
  \end{Exercise}
  \begin{proof}[证明]
    $\forall a,b\in G$,由已知条件知$a^2=e$,$b^2=e$,同时$(ab)^2=e$,即$abab=e$,两边同时左乘$a$,右乘$b$,得$ba=ab$,这证明了$G$为交换群。
  \end{proof}

\begin{Exercise}
  设$a$和$b$为群$G$的两个元素。如果$(ab)^2=a^2b^2$,试证:$ab=ba$。
\end{Exercise}
\begin{proof}[证明]
  由已知条件知$abab=aabb$,两边同时左乘$a^{-1}$,右乘$b^{-1}$,得$ab=ba$。
\end{proof}
\begin{Exercise}
  设$G$为群。如果$\forall a\in G$,$a^2=e$,试证:$G$为交换群。
\end{Exercise}
\begin{proof}[证明]
  $\forall a,b\in G$,由已知条件知$a^2=e$,$b^2=e$,同时$(ab)^2=e$,即$abab=e$,两边同时左乘$a$,右乘$b$,得$ba=ab$,这证明了$G$为交换群。
\end{proof}
\begin{Exercise}
  证明:四阶群是交换群。
\end{Exercise}
\begin{proof}[证明]
  设在四阶群$(G,\circ)$中,$G=\{e,a,b,c\}$,

  其乘法表为:

\begin{tabular}{c|cccc}
  $\circ$&e&a&b&c\\
  \hline
e&e&a&b&c\\
a&a&aa&ab&ac\\
b&b&ba&bb&bc\\
c&c&ca&cb&cc\\
\end{tabular}

$ab\neq a$,否则$b=e$,矛盾;$ab\neq b$,否则$a=e$,也矛盾。于是$ab=e$或$c$。

当$ab=e$时,$a$为$b$的逆元,因此$ba=e$,此时$ab=ba$。

当$ab=c$时,此时亦有$ba\neq b$并且$ba\neq a$。如果$ba=e$,则$b$为$a$的逆元,于是$ab=e$,与$ab=c$矛盾。因此,必有$ba=c$,于是,$ab=ba$。


% \hspace{1cm}\begin{tabular}{c|cccc}
%   $\circ$&e&a&b&c\\
%   \hline
% e&e&a&b&c\\
% a&a&e&c&b\\
% b&b&c&a&e\\
% c&c&b&e&a\\
% \end{tabular}

% \vspace{1cm}
% \begin{tabular}{c|cccc}
%   $\circ$&e&a&b&c\\
%   \hline
% e&e&a&b&c\\
% a&a&b&c&e\\
% b&b&c&e&a\\
% c&c&e&a&b\\
% \end{tabular}\hspace{1cm}\begin{tabular}{c|cccc}
%   $\circ$&e&a&b&c\\
%   \hline
% e&e&a&b&c\\
% a&a&c&e&b\\
% b&b&e&c&a\\
% c&c&b&a&e\\
% \end{tabular}


同理可证$ac=ca$,$bc=cb$。因此,$(G,\circ)$一定为交换群。
\end{proof}
\begin{Exercise}
  证明:在任一阶大于2的非交换群里必有两个非单位元$a$和$b$,使得$ab=ba$。
\end{Exercise}
\begin{proof}[证明]
  设$G$为任一阶大于2的非交换群,必存在$a\in G$,$a^2\neq e$。令$b=a^{-1}$,则$b\neq a$,$ab=ba=e$。
\end{proof}
\begin{Exercise}
  有限阶群里阶大于2的元素的个数必为偶数。
\end{Exercise}
\begin{proof}[证明]
  设$G$为一个有限阶群,阶大于2的元素必成对出现,设$a\in G$,$a$的阶为$n(n>2)$,则$a^{-1}$的阶也为$n$。这里$a\neq a^{-1}$。
\end{proof}
\begin{Exercise}
  证明:偶数阶群里,阶为2的元素的个数必为奇数。
\end{Exercise}
\begin{proof}[证明]
  在偶数阶群里,阶大于2的元素的个数为偶数,单位元的阶为1,其余元素的阶的2,显然阶为2的元素的个数为奇数。
\end{proof}
\begin{Exercise}
  设$a$为群$G$的一个元素,$a$的阶为$n$且$a^m=e$,试证$n$能整除$m$。
\end{Exercise}
\begin{proof}[证明]
  设$m=nq+r(0\leq r <n)$,则$a^m=(a^n)^qa^r$,由$a^n=e$且$a^m=e$得$a^r=e$,再由$n$为$a$的阶知$r=0$(否则将存在比$n$更小的正整数$r$,$a^r=e$,与$a$的阶为$n$矛盾),这证明了$n$能整除$m$。
\end{proof}
\begin{Exercise}
  设$a_1,a_2,\cdots,a_n$为$n$阶群中的$n$个元素(它们不一定各不相同)。证明:存在整数$p$和$q$($1\leq p \leq q \leq n$),使得
  \[a_pa_{p+1}\cdots a_q=e\text{。}\]
\end{Exercise}
\begin{proof}[证明]
  考虑以下表达式:

  \begin{align*}
    &a_1\\
    &a_1a_2\\
    &\cdots\\
    &a_1a_2\cdots a_i\\
    &\cdots\\
    &a_1a_2\cdots a_n\\
  \end{align*}

  以上表达式中如果存在某个表达式计算结果为$e$,则结论成立。
如果以上表达式中任意一个计算结果都不为$e$,则其中必有两个表达式计算结果相等,不妨设$a_1a_2\cdots a_{p-1} = a_1a_2\cdots a_{p-1}a_p a_{p+1}\cdots a_q$,
两边依次同时左乘$a_1^{-1}$,$a_2^{-1}$,$\cdots$,$a_{p-1}^{-1}$,可得$a_pa_{p+1}\cdots a_q=e$。

\end{proof}
\begin{Exercise}
  设$a$和$b$为群$G$的两个元素,$ab=ba$,$a$的阶为$m$,$b$的阶为$n$。试证:乘积$ab$的阶为$m$与$n$的最小公倍数的约数。何时$ab$的阶为$mn$?
\end{Exercise}
\begin{proof}[证明]
  设$m$和$n$的最小公倍数为$k$,则$m|k$,$n|k$。设$k=xm$,$k=yn$,则$(ab)^k=a^kb^k=(a^m)^x(b^n)^y=e$,于是$ab$的阶整除$k$,即$ab$的阶为$m$与$n$的最小公倍数的约数。

  当$m$与$n$互素时,$ab$的阶为$mn$。


  设$ab$的阶为$t$,则$e=(ab)^{mt}=(a^m)^tb^{mt}=b^{mt}$,从而$n|mt$,由$n$与$m$互素知$n|t$。同理,$e=(ab)^{nt}=a^{nt}(b^n)^t=a^{nt}$,从而$m|nt$,由$m$与$n$互素知$m|t$。由$n|t$知$\exists s\in Z,t=ns$,再由$m|t$知$m|ns$,进一步,由$m$与$n$互素知$m|s$,从而$\exists p\in Z,s=pm$,于是$t=ns=n(pm)=p(mn)$,即$mn|t$。

  由$(ab)^{mn}=a^{mn}b^{mn}=e$知$t|mn$,所以$t=mn$。

  当$ab$的阶为$mn$时,必有$m$与$n$互素,否则设$d=(m,n)$,$d>1$,则$m$与$n$的最小公倍数$<mn$,而$ab$的阶为$m$与$n$的最小公倍数的约数,从而$ab$的阶$<mn$,矛盾。
\end{proof}

\end{CJK*}
\end{document}





%%% Local Variables:
%%% mode: latex
%%% TeX-master: t
%%% End: