\documentclass{article}
\usepackage{CJKutf8}
\usepackage{amsmath}
\usepackage{amssymb}
\usepackage{amsfonts}
\usepackage{amsthm}
\usepackage{titlesec}
\usepackage{titletoc}
\usepackage{xCJKnumb}
\usepackage{tikz}
\usepackage{mathrsfs}
\usepackage{indentfirst}

\newtheorem{Def}{定义}
\newtheorem{Thm}{定理}
\newtheorem{Cor}{推论}
\newtheorem{Exercise}{练习}

\newtheorem*{Example}{例}


\begin{document}
\begin{CJK*}{UTF8}{gbsn}
  \title{第四讲 子群、生成子群}
  \author{陈建文}
  \maketitle
  % \tableofcontents
 \begin{Def}
  设$S$为群$G$的非空子集,如果$G$的乘法在$S$中封闭且$S$对此乘法也构成一个群,则称$S$为$G$的一个子群。
  如果$S\neq G$,则称$S$为$G$的真子群。
 \end{Def} 

 \begin{Thm}
  设$G_1$为$G$的子群,则$G_1$的单位元必为$G$的单位元;$G_1$的元素$a$在$G_1$中的逆元素也是$a$在$G$中的逆元素。
 \end{Thm}

 \begin{Thm}
  群$G$的任意多个子群的交还是$G$的子群。
 \end{Thm}

 \begin{Thm}
  任一群不能是其两个真子群的并。
 \end{Thm}

 \begin{Thm}
  群$G$的非空子集$S$为$G$的子群的充分必要条件是

  (1)$\forall a,b\in S, ab\in S$且

  (2)$\forall a\in S, a^{-1}\in S$。
 \end{Thm}

 \begin{Thm}
  群$G$的非空子集$S$为$G$的子群的充分必要条件是$\forall a,b\in S, ab^{-1}\in S$。
 \end{Thm}


 \begin{Thm}
  群$G$的有限非空子集$F$为$G$的子群的充分必要条件是$\forall a,b\in F, ab\in F$。
 \end{Thm}

 \begin{Def}
  群$G$的元素$a$称为$G$的中心元素,如果$a$与$G$的每个元素可交换,即$\forall x\in G, ax=xa$。$G$的所有中心元素构成的集合$C$称为$G$的中心。
 \end{Def}
 \begin{Thm}
  群$G$的中心$C$是$G$的可交换子群。
 \end{Thm}

\begin{Example}
  设$G$为一个群,$a\in G$,$\{\cdots,a^{-2},a^{-1},e,a,a^2,\cdots\}$为$G$的一个子群。
\end{Example}

\begin{Example}
  设$G$为一个有限群,$a\in G$,$\{e,a,a^2,\cdots\}$为$G$的一个子群。
\end{Example}

\begin{Example}
  设$G$为一个交换群,$a,b\in G$,则$\{a^mb^n|m,n\in Z\}$为$G$的一个子群。
\end{Example}

\begin{Def}
  设$M$为$G$的一个子集,$G$的包含$M$的所有子群的交称为由$M$生成的子群,记为$(M)$。
\end{Def}
课后作业题:
\begin{Exercise}
举例说明两个子群的并可以不是子群。
\end{Exercise}
\begin{Exercise}
  设$G_1$和$G_2$为群$G$的两个真子群,证明:$G_1\cup G_2$为$G$的子群的充分必要条件是$G_1\subseteq G_2$并且$G_2\subseteq G_1$。
\end{Exercise}

\begin{Exercise}
  设$(G_1,\circ)$和$(G_2,*)$都是群,$\phi:G_1\to G_2$,$\forall a,b\in G_1$,$\phi(a\circ b)=\phi(a)*\phi(b)$,
  证明:$\phi^{-1}(e_2)$为$G_1$的子群,其中$e_2$为$G_2$的单位元素。
\end{Exercise}

\begin{Exercise}
  找出$3$次对称群的所有子群。
\end{Exercise}
\end{CJK*}
\end{document}





%%% Local Variables:
%%% mode: latex
%%% TeX-master: t
%%% End:



