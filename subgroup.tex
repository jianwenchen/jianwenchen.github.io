\documentclass{article}
\usepackage{CJKutf8}
\usepackage{amsmath}
\usepackage{amssymb}
\usepackage{amsfonts}
\usepackage{amsthm}
\usepackage{titlesec}
\usepackage{titletoc}
\usepackage{xCJKnumb}
\usepackage{tikz}
\usepackage{mathrsfs}
\usepackage{indentfirst}

\newtheorem{Def}{定义}
\newtheorem{Thm}{定理}
\newtheorem{Cor}{推论}
\newtheorem{Exercise}{练习}

\newtheorem*{Example}{例}


\begin{document}
\begin{CJK*}{UTF8}{gbsn}
  \title{第四讲 子群、生成子群}
  \author{陈建文}
  \maketitle
  % \tableofcontents
 \begin{Def}
  设$S$为群$G$的非空子集,如果$G$的乘法在$S$中封闭且$S$对此乘法也构成一个群,则称$S$为$G$的一个子群。
  如果$S\neq G$,则称$S$为$G$的真子群。
 \end{Def} 
\begin{Thm}
设$G$为一个群,则$\{e\}$为$G$的子群,$G$为$G$的子群。
\end{Thm}
\begin{Example}
  $(Z,+)$为$(Q,+)$的子群,$(Q,+)$为$(R,+)$的子群,$(R,+)$为$(C,+)$的子群;
  $(Q^*,\times)$为$(R^*,\times)$的子群,$(R^*,\times)$为$(C^*,\times)$的子群。
  集合$\{1,-1\}$对通常的乘法构成一个群,但它不是$(Q,+)$的子群,因为它们的运算不一样。
\end{Example}

 \begin{Thm}
  设$G_1$为$G$的子群,则$G_1$的单位元必为$G$的单位元;$G_1$的元素$a$在$G_1$中的逆元素也是$a$在$G$中的逆元素。
 \end{Thm}
\begin{proof}[证明]
  设$G_1$的单位元为$e_1$,$G$的单位元为$e$,则$e_1e_1=e_1e$,由消去律得$e_1=e$。

  设$b$为$a$在$G_1$的逆元,则$ba=e$,该式在$G$中也成立,于是$b$也是$a$在$G$中的逆元。
\end{proof}
 \begin{Thm}
  群$G$的任意多个子群的交还是$G$的子群。
 \end{Thm}
\begin{proof}[证明]
  设$H$为$G$的一些子群的交,则$e\in H$,从而$H\neq \phi$。其次,$\forall a,b\in H$,$ab$在每个参加交运算的子群中,从而$ab\in H$。
  所以,$G$的乘法在$H$中封闭。最后,$\forall a\in H$,由$a$在每个参加交运算的子群中知$a^{-1}$在每个参加交运算的子群中,故$a^{-1}\in H$。因此,$H$为$G$的子群。
\end{proof}
 \begin{Thm}
  任一群不能是其两个真子群的并。
 \end{Thm}
\begin{proof}[证明]
  用反证法。设$G_1$和$G_2$为$G$的两个真子群,且$G_1\cup G_2=G$。由于$G_1$和$G_2$为$G$的真子群,所以$\exists a,b\in G$,$a\notin G_1$,$b\notin G_2$。
  于是$a\in G_2$,$b\in G_1$,从而$ab\in G$,但$ab\notin G_1$且$ab\notin G_2$,这与$G=G_1\cup G_2$矛盾。
\end{proof}
 \begin{Thm}
  群$G$的非空子集$S$为$G$的子群的充分必要条件是

  (1)$\forall a,b\in S, ab\in S$且

  (2)$\forall a\in S, a^{-1}\in S$。
 \end{Thm}
\begin{proof}[证明]
  $\Leftarrow$ 显然。

  $\Rightarrow$
  运算的封闭性显然成立;
  运算的结合律显然成立;
  由$G$非空知$\exists a\in G$,从而$a^{-1}\in G$,于是$e=a^{-1}a\in G$。
\end{proof}
 \begin{Thm}
  群$G$的非空子集$S$为$G$的子群的充分必要条件是$\forall a,b\in S, ab^{-1}\in S$。
 \end{Thm}
 \begin{proof}[证明]
  $\Leftarrow$ 显然。

  $\Rightarrow$
  由$G$非空知$\exists a\in G$,从而$e=aa^{-1}\in S$;

  $\forall g\in G$,$g^{-1}=eg^{-1}\in S$;

  $\forall a,b\in G$,$b^{-1}\in S$,从而$ab=a(b^{-1})^{-1}\in S$。
\end{proof}

 \begin{Thm}
  群$G$的有限非空子集$F$为$G$的子群的充分必要条件是$\forall a,b\in F, ab\in F$。
 \end{Thm}

 $\forall A,B\in 2^G$,定义$AB=\{ab|a\in A,b\in B\}$,则以上定理可以写成

 \begin{Thm}
  群$G$的有限非空子集$F$为$G$的子群的充分必要条件是$FF\subseteq F$。
 \end{Thm}
 \begin{Def}
  群$G$的元素$a$称为$G$的中心元素,如果$a$与$G$的每个元素可交换,即$\forall x\in G, ax=xa$。$G$的所有中心元素构成的集合$C$称为$G$的中心。
 \end{Def}
 \begin{Thm}
  群$G$的中心$C$是$G$的可交换子群。
 \end{Thm}
\begin{proof}[证明]
$\forall x\in G, ex=ex=x$,所以$e\in C$,故$C\neq \phi$。

$\forall a,b\in C$,$\forall x\in G$,

$(ab^{-1})x=a(b^{-1}x)=a(x^{-1}b)^{-1}=a(bx^{-1})^{-1}=a(xb^{-1})=(ax)b^{-1}=(xa)b^{-1}=x(ab^{-1})$。

从而$ab^{-1}\in C$,故$C$为$G$的子群。$C$显然是可交换的。
\end{proof}
\begin{Example}
  设$G$为一个群,$a\in G$,$\{\cdots,a^{-2},a^{-1},e,a,a^2,\cdots\}$为$G$的一个子群。
\end{Example}

\begin{Example}
  设$G$为一个有限群,$a\in G$,$\{e,a,a^2,\cdots\}$为$G$的一个子群。
\end{Example}

\begin{Example}
  设$G$为一个交换群,$a,b\in G$,则$\{a^mb^n|m,n\in Z\}$为$G$的一个子群。
\end{Example}

\begin{Def}
  设$M$为$G$的一个子集,$G$的包含$M$的所有子群的交称为由$M$生成的子群,记为$(M)$。
\end{Def}
课后作业题:
\begin{Exercise}
举例说明两个子群的并可以不是子群。
\end{Exercise}
\begin{Exercise}
  设$G_1$和$G_2$为群$G$的两个真子群,证明:$G_1\cup G_2$为$G$的子群的充分必要条件是$G_1\subseteq G_2$并且$G_2\subseteq G_1$。
\end{Exercise}

\begin{Exercise}
  设$(G_1,\circ)$和$(G_2,*)$都是群,$\phi:G_1\to G_2$,$\forall a,b\in G_1$,$\phi(a\circ b)=\phi(a)*\phi(b)$,
  证明:$\phi^{-1}(e_2)$为$G_1$的子群,其中$e_2$为$G_2$的单位元素。
\end{Exercise}

\begin{Exercise}
  找出$3$次对称群的所有子群。
\end{Exercise}
\begin{Exercise}
  令$P=\{(12),(123)\}\subseteq S_3$。写出由$P$生成的$S_3$的子群$(P)$。
\end{Exercise}
\end{CJK*}
\end{document}





%%% Local Variables:
%%% mode: latex
%%% TeX-master: t
%%% End:



