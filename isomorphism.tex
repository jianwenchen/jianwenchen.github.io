\documentclass{article}
\usepackage{CJKutf8}
\usepackage{amsmath}
\usepackage{amssymb}
\usepackage{amsfonts}
\usepackage{amsthm}
\usepackage{titlesec}
\usepackage{titletoc}
\usepackage{xCJKnumb}
\usepackage{tikz}
\usepackage{mathrsfs}
\usepackage{indentfirst}

\newtheorem{Def}{定义}
\newtheorem{Thm}{定理}
\newtheorem{Exercise}{练习}

\newtheorem*{Example}{例}
\newtheorem{Cor}{推论}

\begin{document}
\begin{CJK*}{UTF8}{gbsn}
  \title{第五讲 变换群、同构}
  \author{陈建文}
  \maketitle
  % \tableofcontents
  
  \begin{Def}
    设$(G_1,\circ)$,$(G_2,*)$为两个群。如果存在一个双射$\phi:G_1\to G_2$,使得$\forall a,b\in G_1$,
    \[\phi(a\circ b)=\phi(a)* \phi(b),\]
    则称群$G_1$与$G_2$同构,记为$G_1\cong G_2$。$\phi$称为从$G_1$到$G_2$的一个同构。
  \end{Def}

  \begin{Def}
    设$S$为一个非空集合,从$S$到$S$的所有双射构成的集合对映射的合成构成一个群,称为$S$上的对称群,记为$Sym(S)$。当$S=\{1,2,\cdots,n\}$时,$Sym(S)=S_n$。
  \end{Def}

\begin{Def}
  $Sym(S)$的任意一个子群称为$S$上的一个变换群。$S_n$的任意一个子群称为一个置换群。
\end{Def}

\begin{Thm}
  任何一个群都同构于某个变换群。
\end{Thm}
\begin{proof}[证明]
  设$(G,\circ)$为一个群。$\forall a\in G$,令$f_a:G\to G$,$\forall x\in G$,$f_a(x)=ax$,则$f_a$为从$G$到$G$的双射。
  ($f_a$为单射,这是因为$\forall x_1,x_2\in G$,如果$f_a(x_1)=f_a(x_2)$,则$ax_1=ax_2$,从而$x_1=x_2$;$f_a$为满射,这是因为$\forall y\in G$,$f_a(a^{-1}y)=a(a^{-1}y)=y$。)
设$L(G)=\{f_a|f_a:G\to G,\forall x\in G,f_a(x)=ax,a\in G\}$,则$L(G)$对映射的合成构成一个群。实际上,$\forall f_a,f_b\in L(G)$,$\forall x\in G$,$f_a\circ f_b(x)=f_a(f_b(x))=f_a(bx)=abx=f_{ab}(x)$,所以$f_{ab}=f_a\circ f_b$,
即$f_a\circ f_b\in L(G)$。因此,合成运算在$L(G)$中封闭。显然,合成运算满足结合律。$G$上的恒等映射$I_G=f_e\in L(G)$为$L(G)$中的单位元素($\forall x\in G,f_e(x)=ex=x$)。又因为$\forall a\in G$,$\forall x\in G$,
\[f_{a^{-1}}\circ f_a(x)=(a^{-1}a)x=x=f_e(x)\]
所以$f_{a^{-1}}f_a=f_e$,$f_{a^{-1}}$为$f_a$的左逆元。因此,$L(G)$为一个群。

令$\phi:G\to L(G)$,$\forall a\in G$,$\phi(a)=f_a$,则$\phi$为双射($\phi$为单射,这是因为如果$\forall a,b\in G$,如果$\phi(a)=\phi(b)$,则$f_a=f_b$,从而$f_a(e)=f_b(e)$,即$ae=be$,于是$a=b$;$\phi$为满射,这是因为对任意的$f\in L(G)$,$\exists a\in G$使得$f=f_a$,从而$\phi(a)=f_a=f$)。

$\forall a,b\in G, \phi(ab)=f_{ab}=f_a\circ f_b=\phi(a)\circ \phi(b)$,因此$\phi$为从$G$到$L(G)$的一个同构,即$G\cong L(G)$。
\end{proof}
设$(G,\circ)$为一个$n$阶群,$G=\{a_1,a_2,\cdots,a_n\}$,则$G\cong L(G)$,这里

\[L(G)=\Big\{\begin{pmatrix}a_1&a_2&\cdots&a_n\\a_ia_1&a_ia_2&\cdots&a_ia_n\end{pmatrix}|a_i\in G\Big\}\]
为一个置换群。
\begin{Cor}
  任意一个$n$阶有限群同构于$n$次对称群$S_n$的一个$n$阶子群,亦即任意一个有限群同构于某个置换群。
\end{Cor}
课后作业题:
\begin{Exercise}
设$R$为实数集合,$G$为一切形如$f(x)=ax+b$的从$R$到$R$的函数之集,这里$a\in R$,$b\in R$,$a\neq 0$,试证:$G$为一个变换群。
\end{Exercise}

\begin{Exercise}
  设$R$为实数集合,$H$为一切形如$f(x)=x+b$的从$R$到$R$的函数之集,这里$b\in R$,试证:$H$为上题中$G$的一个子群。
\end{Exercise}
\begin{Exercise}
设$R_+$为一切正实数之集,$R$为一切实数之集。$(R_+,\times)$,$(R,+)$都为群。令$\phi:R_+\to R,\forall x\in R_+,\phi(x)=log_p(x)$,其中$p$为任意一个正实数。证明$\phi$为同构。
\end{Exercise}
\end{CJK*}
\end{document}





%%% Local Variables:
%%% mode: latex
%%% TeX-master: t
%%% End:


