\documentclass{beamer}
\usepackage{ragged2e}

%\usepackage{beamerthemesplit}
\usepackage{ragged2e}
\usepackage{CJKutf8}
\usepackage{tikz}
\usepackage{clrscode3e}
\setbeamertemplate{theorems}[numbered]

\begin{document}
\begin{CJK*}{UTF8}{gbsn}

\newtheorem*{Thm}{定理}
\newtheorem*{Cor}{推论}
\newtheorem{Ax}{公理}[section]
\theoremstyle{definition}
\newtheorem*{Def}{定义}
\newtheorem*{Defofset}{集合的定义}

\theoremstyle{example}
\newtheorem*{Ex}{例:}
\newtheorem*{Exercise}{习题:}
\date{}
\author{陈建文}

\title{第四章 无穷集合及其基数}
\begin{frame}
  \titlepage
\end{frame}  
\begin{frame}
  设集合$X=\{1,2,3\}$,$Y=\{4,5,6\}$,则下列不是双射的是?

  A. $\{(1,4),(2,5),(3,6)\}$

  B. $\{(1,6),(2,4),(3,5)\}$

  C. $\{(1,4),(2,4),(3,6)\}$

  D. $\{(1,5),(2,4),(3,6)\}$
\end{frame}

\begin{frame}
  下列说法错误的是?

  A. 所有的$n$次奇置换构成的集合与所有的$n$次偶置换构成的集合之间存在一个双射。

  B. 设$X$为集合,则$X$上的所有等价关系构成的集合与$X$的所有划分构成的集合之间存在一个双射。

  C. 整数集合与偶数集合之间存在一个双射。

  D. 设$A$与$B$为两个互不相交的集合,则在$A$与$A\cup B$之间不可能存在双射。
\end{frame}


\section{可数集}
\begin{frame}[t]
  \frametitle{1. 可数集}
  \begin{Def}
    如果从集合$X$到集合$Y$存在一个双射,则称$X$与$Y$\alert{对等},记为$X \sim Y$。
  \end{Def}\pause
  \begin{Def}
    如果从自然数集$\mathbb{N}$到集合$X$存在一个一一对应$f:\mathbb{N}\to X$,则称
    集合$X$为可数无穷集合,简称\alert{可数集}或\alert{可列集}。如果$X$不是可数集且$X$不是有穷集合,则称$X$为不可数无穷集合,简称\alert{不可数集}。
  \end{Def}\pause
  \begin{Thm}
    集合$A$为可数集的充分必要条件是$A$的全部元素可以排成无重复项的序列
    \[a_1, a_2, \ldots, a_n, \cdots\]
  \end{Thm}
\end{frame}

\begin{frame}[t]
  \frametitle{1. 可数集}
  \begin{Thm}
    可数集的任一无限子集也是可数集。
  \end{Thm}
\end{frame}
\begin{frame}[t]
  \frametitle{1. 可数集}
  \begin{Thm}
   设$A$为可数集合,$B$为有穷集合,则$A\cup B$为可数集。
  \end{Thm}
\end{frame}

\begin{frame}[t]
  \frametitle{1. 可数集}
  \begin{Thm}
    设$A$与$B$为两个可数集,则$A\cup B$为可数集。
  \end{Thm}
\end{frame}

\begin{frame}[t]
  \frametitle{1. 可数集}
  \begin{Thm}
    设$A_1, A_2, \cdots, A_n, \cdots$为可数集合的一个无穷序列,则$\bigcup_{n=1}^{\infty}A_n$是可数集。即可数多个可数集之并是可数集。
  \end{Thm}
\end{frame}

\begin{frame}[t]
  \frametitle{1. 可数集}
  \begin{Thm}
    设$A$与$B$为两个可数集,则$A\times B$为可数集。
  \end{Thm}
\end{frame}

\begin{frame}[t]
  \frametitle{1. 可数集}
  \begin{Thm}
    全体有理数之集$\mathbb{Q}$为可数集。
  \end{Thm}
\end{frame}


\section{连续统集}
\begin{frame}[t]
  \frametitle{2 连续统集}
  \begin{Thm}
    区间$[0,1]$中的所有实数构成的集合为不可数集。
  \end{Thm}
\end{frame}
\begin{frame}[t]
  \frametitle{2 连续统集}
   设$x, y, z \in \mathbb{R}$,则
   \begin{enumerate}
   \item   $x + y = y + x$
   \item   $(x + y) + z = x + (y + z)$
   \item   $0 + x = x + 0 = x$
   \item   $(-x) + x = x + (-x) = 0$
   \item   $x * y = y * x$
   \item   $(x * y) * z = x * (y *z)$
   \item   $1 * x = x * 1 = x$
   \item   $\forall x \in \mathbb{R} x \neq 0 \to x^{-1} * x = x * x^{-1} = 1$
   \item   $x* (y + z) = x * y + x * z$
   \item   $(y + z) * x = y * x + z * x$
    \end{enumerate}
  \end{frame}
  \begin{frame}[t]
  \frametitle{2 连续统集}
    \begin{enumerate}
      \item 对任意的$x\in R$,$x\leq x$。
      \item 对任意的$x\in R$,$y\in R$,如果$x\leq y$并且$y\leq x$,则$x=y$。 
     \item 对任意的$x\in R$,$y\in R$,$z\in R$,如果$x\leq y$并且$y\leq z$,则$x\leq z$。
     \item 对任意的$x\in R$,$y\in R$,$x\leq y$和$y\leq x$两者中必有其一成立。
     
     我们用$x<y$表示$x\leq y$并且$x\neq y$,$x\geq y$表示$y\leq x$,$x > y$表示$x\geq y$并且$x\neq y$。
     
     \item 对任意的$x\in R$,$y\in R$,$z\in R$,如果$x<y$,则$x+z<y+z$。
     \item 对任意的$x\in R$,$y\in R$,如果$x>0$,$y>0$,则$xy>0$。 
  \end{enumerate}
\end{frame}
\begin{frame}[t]
  \frametitle{2 连续统集}
    另外,实数集还具有如下性质:

  设$A_1$, $A_2$,$\cdots$,$A_i$,$\cdots$为实数集$R$上的闭区间,$A_1\supseteq A_2 \supseteq A_3 \supseteq \cdots \supseteq A_i \supseteq \cdots$,则$\bigcap_{i=1}^{\infty}A_i$非空。
\end{frame}

\begin{frame}[t]
  \frametitle{2 连续统集}
  \begin{Def}
    凡与集合$[0,1]$存在一个一一对应的集合称为具有“连续统的势”的集合,简称\alert{连续统}。
  \end{Def}
\end{frame}

\begin{frame}[t]
  \frametitle{2 连续统集}
  \begin{Thm}
    无穷集合必包含有可数子集。
  \end{Thm}
\end{frame}
\begin{frame}[t]
  \begin{Thm}
    设$M$为一个无穷集合,$A$为至多可数集合,则$M \sim M \cup A$。
  \end{Thm}\pause
  \begin{proof}[证明]
    因为$M$为一个无穷集合,所以$M$中必有一个可数子集$D$。令$P=M\setminus D$,则
    \[M=P\cup D, M\cup A = P\cup (D\cup A)\]
    由$P\sim P$,$D\sim D\cup A$,得到$M\sim M\cup A$。
  \end{proof}
\end{frame}
\begin{frame}[t]
  \begin{Thm}
    设$M$为一个无穷集合,$A$为至多可数集合,则$M \sim M \cup A$。
  \end{Thm}
  \begin{proof}[证明]
    先考虑$A\cap M=\phi$的情况。
    因为$M$为一个无穷集合,所以$M$中必有一个可数子集$D$。令$P=M\setminus D$,则
    \[M=P\cup D, M\cup A = P\cup (D\cup A)\]
    由$P\sim P$,$D\sim D\cup A$,得到$M\sim M\cup A$。

    再考虑$A\cap M\neq \phi$的情况,此时$A\setminus M$为至多可数集合,从而$M\sim M\cup(A\setminus M)=M\cup A$。
  \end{proof}
\end{frame}

\begin{frame}[t]
  \begin{Thm}
    设$M$为无穷集合,$A$为$M$的至多可数子集,$M\setminus A$为无穷集合,则$M \sim M\setminus A$。
  \end{Thm}
\end{frame}
\begin{frame}[t]
  \frametitle{2 连续统集}
  \begin{Thm}
    设$A_1, A_2, \cdots, A_n$为$n$个两两不相交的连续统,则$\bigcup_{i=1}^nA_i$是连续统。
  \end{Thm}
\end{frame}

\begin{frame}
  \frametitle{2 连续统集}
  \begin{Thm}
    设$A_1, A_2, \cdots, A_n, \cdots$为两两不相交的集序列。如果$A_k \sim [0,1], k = 1, 2, \cdots$,则
    \[\bigcup_{n=1}^{\infty}A_n \sim [0,1]\]
  \end{Thm}\pause
  \begin{Cor}
    全体实数之集是一个连续统。
  \end{Cor}\pause
  \begin{Cor}
    全体无理数之集是一个连续统。
  \end{Cor}
\end{frame}

% \begin{frame}
%   \frametitle{2 连续统集}
% 设$A_1$,$A_2$均为连续统,则$A_1\times A_2$为连续统。  
% \end{frame}
% \begin{frame}
%   \frametitle{2 连续统集}
%     \begin{codebox}
%     \Procname{$\proc{K}(P)$}
%     \li \If $H(P,P) == 1$ 
%     \li  $\quad\quad$\Return
%     \li \ElseNoIf Loop forever
%     \End
%   \end{codebox}  
% \end{frame}

\section{基数及其比较}
\begin{frame}
  \frametitle{3 基数及其比较}
  \begin{Def}
    集合$A$的基数是一个符号,凡与$A$对等的集合都赋以同一个记号。集合$A$的基数记为$|A|$。
  \end{Def}\pause
  \begin{Def}
    所有与集合$A$对等的集合构成的集族称为$A$的基数。
  \end{Def}
  %   \begin{Def}
  %   集合$A$的基数与集合$B$的基数称为是相等的,当且仅当$A \sim B$。
  % \end{Def}
\end{frame}
\begin{frame}[t] \justifying\let\raggedright\justifying
  设集合$X=\{1,2,3\}$,在$2^X$上定义二元关系$R$,对任意的$A\in 2^X$,$B\in 2^X$,$(A,B)\in R$当且仅当在$A$与$B$之间存在一个双射,
  则$|2^X/R|=?$
\end{frame}
\begin{frame}[t] \justifying\let\raggedright\justifying
 以下结论是否正确?

 设$N$为自然数集,在$2^N$上定义二元关系$R$,对任意的$A\in 2^X$,$B\in 2^X$,$(A,B)\in R$当且仅当在$A$与$B$之间存在一个双射,
 则$2^N/R$为可数集。
\end{frame}

\begin{frame}
  \frametitle{3. 基数及其比较}
  设$A$,$B$为两个集合,
  \begin{description}
  \pause\item[$|A|=|B|:$]在集合$A$与集合$B$之间存在一个双射。
  \pause\item[$|A|\leq |B|:$]在集合$A$与集合$B$之间存在一个单射。
  \pause\item[$|A|< |B|:$]在集合$A$与集合$B$之间存在一个单射,但不存在从集合$A$到集合$B$的双射。
  \end{description}
\end{frame}

% \begin{frame}
%   \frametitle{3 基数及其比较}
%   \begin{Def}
%     设$\alpha$,$\beta$为任意两个基数,$A$,$B$为分别以$\alpha$,$\beta$为其基数的
%     集合。如果$A$与$B$的一个真子集对等,但$A$却不能与$B$对等,则称基数$\alpha$小于基数$\beta$,记为$\alpha < \beta$。
%   \end{Def}\pause
%   显然,

%   $\alpha \leq \beta$当且仅当存在单射$f:A \to B$。

%   $\alpha < \beta$当且仅当存在单射$f:A \to B$且不存在$A$到$B$的双射。
% \end{frame}




\begin{frame}[t]
  \frametitle{3 基数及其比较}
  \begin{Thm}[康托]
    对任一集合$M$,$|M| < |2^{M}|$。
  \end{Thm}
\pause
  设$M=\{1,2,3\}$,则$2^{M}=\{\phi,\{1\},\{2\},\{3\},\{1,2\},\{1,3\},\{2,3\},\{1,2,3\}\}$。
\end{frame}
\begin{frame}[t]
  \frametitle{3 基数及其比较}
  \begin{Thm}[康托]
    对任一集合$M$,$|M| < |2^{M}|$。
  \end{Thm}
  \begin{proof}[证明]\justifying\let\raggedright\justifying
    \pause令$i:M\to 2^M$,\pause其定义为对任意的$m\in M$,\pause$i(m)=\{m\}$。\pause于
    是,\pause$i$为从$M$到$2^M$的单射,\pause故$|M|\leq |2^M|$。\pause为了完成定理的证明,
    \pause我们还需要证明:\pause如果$f:M\to 2^M$为单射,\pause则$f$一定不为满射。\pause为此,\pause令
    \[X=\{m\in M|m \notin f(m)\}\]\pause显然,\pause$X\in 2^M$。\pause现在证明对任意
    的$x\in M$,\pause$f(x)\neq X$。\pause实际上,\pause如果存在$x_0\in M$使得$f(x_0)=X$,
    \pause则如果$x_0\in X$,\pause那么由$X$的的定义知$x_0\notin
    f(x_0)$,\pause即$x_0\notin X$;\pause如果$x_0\notin X$,\pause即$x_0\notin f(x_0)$,\pause由$X$的定义可得$x_0\in X$。\pause总之,\pause$x_0\in X$与$x_0\notin X$都引出矛盾,\pause从而不存在$x_0\in M$使得$f(x_0)=X$。\pause因此,\pause$f$不为满射,\pause从而
    \[|M|<|2^M|\]
  \end{proof}
\end{frame}

\section{康托-伯恩斯坦定理}
\begin{frame}
  \frametitle{4 康托-伯恩斯坦定理}
  \begin{Thm}[康托-伯恩斯坦]
    设$A$,$B$为两个集合。如果存在单射$f:A\to B$与单射$g:B\to A$,则存在从$A$到$B$的双射。
  \end{Thm}\pause
  \begin{proof}[证明]{\small
    \pause如果可以找到$A$的子集$D$使得$D=A\setminus g(B\setminus f(D))$,
    \pause令$h:A\to B$,\pause对任意的$x\in A$,\pause定义\pause
    \[h(x)=\begin{cases}
        f(x),&\text{如果}x\in D\\
        g^{-1}(x),&\text{如果}x\in A\setminus D
      \end{cases}
    \]
    \pause其中$g^{-1}$为视$g$为$B$到$g(B)$的一一对应时$g$的逆,\pause易见$h$为一一对应。\pause所以$A$与$B$的基数相等。
    }
  \end{proof}
\end{frame}

\begin{frame}
  \frametitle{4 康托-伯恩斯坦定理}
  \begin{Thm}[康托-伯恩斯坦]
    设$A$,$B$为两个集合。如果存在单射$f:A\to B$与单射$g:B\to A$,则存在从$A$到$B$的双射。
  \end{Thm}\pause
  \begin{proof}[证明]{\small
    \pause令$\psi:2^A\to 2^A$,\pause对任意
    的$E\in 2^A$,\pause\[\psi(E)=A\setminus g(B\setminus f(E))\]\pause易见,\pause如果$E\subseteq F\subseteq A$,\pause则$\psi(E)\subseteq \psi(F)$。
    \pause令\[\mathbb{D}=\{E\subseteq A|E\subseteq \psi(E)\}\],\pause则$\phi\in \mathbb{D}$。\pause又令
    \[D=\bigcup_{E\in \mathbb{D}}E,\]
    \pause则对任意的$E\in \mathbb{D}$,\pause由$E\subseteq D$知$E\subseteq \psi(E) \subseteq \psi(D)$,\pause从而$D\subseteq \psi(D)$。
    \pause于是$\psi(D)\subseteq \psi(\psi(D))$,\pause故$\psi(D)\in \mathbb{D}$,\pause因此,\pause$\psi(D)\subseteq D$,\pause所以
    \pause\[D=\psi(D)=A\setminus g(B\setminus f(D))\]}
  \end{proof}
\end{frame}
\begin{frame}[t]
  \frametitle{1. 集合的概念}
  \begin{Defofset}
    通常把一些互不相同的东西放在一起所形成的整体叫做一个\alert{集合}。
  \end{Defofset}
\end{frame}

% \begin{frame}
%   \frametitle{4 康托-伯恩斯坦定理}
%   \begin{proof}
%     We separate $A$ into two disjoint sets $A_1$ and $A_2$. We let $A_1$ consist of all $x\in A$ such that, when we lift back $x$ by a succession of inverse maps,
%     \[x, g^{-1}(x), f^{-1}(g^{-1}(x)),g^{-1}(f^{-1}(g^{-1}(x)))\cdots\]
%     then $x$ can be lifted indefinitely, or  at some stage we get stopped in A (i.e. reach an element of $A$ which has no inverse image in $B$ by $g$). We let $A_2$ be the complement of $A_1$, in other words, the set of $x\in A$ from which we get stopped in B by following the succession of inverse maps. We shall define a bijection $h$ of $A$ onto $B$.

%     If $x\in A_1$, we define $h(x)=f(x)$.

%     If $x\in A_2$, we define $h(x)=g^{-1}(x)$.
    
%   \end{proof}
% \end{frame}

% \begin{frame}
%   \begin{proof}
%     Then trivially, $h$ is injective. We must prove that $h$ is surjective. Let $y\in B$. If, when we try to lift back $y$ by a succession of maps

%     \[y, f^{-1}(y), g^{-1}(f^{-1}(y)),f^{-1}(g^{-1}(f^{-1}(y)))\cdots\]
    
%     we can lift back indefinitely, or if we get stopped in $A$, then $f^{-1}(y)$ is defined, and $f^{-1}(y)$ lies in $A_1$. Consequently, $y = h(f^{-1}(y))$ is in the image of $h$. On the other hand, if we cannot lift back $y$ indefinitely, and get stopped in $B$, then $g(y)$ belongs to $A_2$. In this case, $y=h(g(y))$ is also in the image of $h$, as was to be shown.
%   \end{proof}
% \end{frame}
% \begin{frame}
%   \frametitle{4 康托-伯恩斯坦定理}
%   \begin{Thm}[塔斯基不动点定理]
%     设$(A, \leq)$为一个完备格($A$的任一非空子集均有上确界和下确界),$f:A \to A$为单调函数($\forall x, y \in A, x \leq y \rightarrow f(x) \leq f(y)$),则$\exists z \in A$,使得$f(z)=z$。
%   \end{Thm}
%   \begin{Thm}[巴拿赫映射分解定理]
%     设$A$, $B$为任意两个集合,$f$为从$A$到$B$的映射,$g$为从$B$到$A$的映射,则存在$A$的划分$\{A_1,A_2\}$和$B$的$\{B_1,B_2\}$,使得\[f(A_1) = B_1,g(A_2) = B_2.\]
%   \end{Thm}
% \end{frame}

% \begin{frame}
%   \begin{Def}
%     设$\alpha$,$\beta$为两个基数,$A$与$B$为两个不相交集合,$|A|=\alpha$,$|B|=\beta$,则集合$A\cup B$的基数称为基数$\alpha$与$\beta$的和,记为$\alpha + \beta$。
%   \end{Def}
%   \begin{Def}
%     设$\alpha$,$\beta$为两个基数,$A$与$B$为两个集合,$|A|=\alpha$,$|B|=\beta$,则集合$A\times B$的基数称为基数$\alpha$与$\beta$的积,记为$\alpha \cdot \beta$ 或者$\alpha \beta$。
%   \end{Def}
%   \begin{Def}
%     设$\alpha$,$\beta$为两个基数,$A$与$B$为两个集合,$|A|=\alpha$,$|B|=\beta$,则集合$B^A=\{f|f:A\to B\}$的基数称为$\beta$的$\alpha$次幂,记为$\beta^{\alpha}$。
%   \end{Def}  
% \end{frame}
% \begin{frame}
%   \begin{Thm}
%   设$a$为可数集的基数,$c$为连续统的基数,则
%   \begin{enumerate}
%   \item $\forall n\in N\cup \{0\}, n + a = a$.
%   \item $\forall n\in N, n \cdot a = a$.
%   \item $\forall n\in N, n \cdot c = c$.
%   \item $a\cdot c =c$.
%   \item $c\cdot c = c$.
%   \item $2^a=c$.
%   \item $(2^a)^a=c$.
%   \item $a^a=2^a$.
%   \end{enumerate}    
%   \end{Thm}
% \end{frame}
\begin{frame}
  设$X=\{1,2,3\}$,以下为从$X$到$X$的映射的是()。

  A. $\{(1,1),(2,3)\}$

  B. $\{(1,1),(1,2),(2,3),(3,1)\}$

  C. $\{(1,1),(2,3),(3,2)\}$

  D. $\{(1,2),(2,3),(3,1),(3,2)\}$
\end{frame}
\begin{frame}
  判断题:
  映射是关系。
  
\end{frame}
\section{公理集合论}
\begin{frame}[t]
  \frametitle{5 公理集合论}
  \begin{Ax}[外延公理]
    \begin{equation*}
      \forall A \forall B (\forall x (x \in A \leftrightarrow x\in B)\rightarrow A = B)
    \end{equation*}
  \end{Ax}
\end{frame}
\begin{frame}[t]
  \begin{Ax}[空集公理]
    \begin{equation*}
      \exists \phi \forall x (x \notin \phi)
    \end{equation*}
  \end{Ax}
\end{frame}
\begin{frame}[t]
  \begin{Ax}[对公理]
    \begin{equation*}
      \forall u \forall v \exists B \forall x (x \in B \leftrightarrow x = u \lor x = v)
    \end{equation*}
  \end{Ax}
\end{frame}
\begin{frame}[t]
  \begin{Ax}[并集公理]
    \begin{equation*}
     \forall A \exists B \forall x (x \in B \leftrightarrow (\exists b \in A) x \in b)
    \end{equation*}
  \end{Ax}
\end{frame}
\begin{frame}[t]
  \frametitle{5 公理集合论}
    \begin{Ax}[幂集公理]
    \begin{equation*}
      \forall a \exists B \forall x ( x \in B \leftrightarrow x \subseteq a)
    \end{equation*}
  \end{Ax}
\end{frame}
\begin{frame}[t]
  \begin{Ax}[子集公理]
    \begin{equation*}
      \forall c \exists B \forall x (x \in B \leftrightarrow x \in c \land \varphi(x))
    \end{equation*}
  \end{Ax}
\end{frame}
\begin{frame}[t]
  \begin{Ax}[无穷公理]
    \begin{equation*}
      \begin{split}
      \exists A ( \phi \in A \land (\forall a \in A) a^+ \in A)\\
      \text{其中} a^+ = a \cup \{a\}
      \end{split}
    \end{equation*}
  \end{Ax}
\end{frame}
\begin{frame}[t]
  \frametitle{5 公理集合论}
    \begin{Ax}[代换公理]
    \begin{equation*}
      \begin{split}
      \forall A ((\forall x \in A) \forall y_1 \forall y_2 (\varphi(x, y_1) \land \varphi(x, y2) \rightarrow y_1 = y_2)\\
      \rightarrow \exists B \forall y (y \in B \leftrightarrow (\exists x \in A) \varphi(x, y)))
    \end{split}
  \end{equation*}
  \end{Ax}
\end{frame}
\begin{frame}[t]
  \begin{Ax}[正则公理]
    \begin{equation*}
      (\forall A \neq \phi) (\exists m \in A) m \cap A = \phi
    \end{equation*}
  \end{Ax}
\end{frame}
\begin{frame}[t]
  \begin{Ax}[选择公理]
    \begin{equation*}
      (\forall \text{relation} R)
      (\exists \text{function} F)
      (F \subseteq R \land
      \text{dom} F
      = \text{dom} R)
    \end{equation*}
  \end{Ax}
\end{frame}
\begin{frame}[t]
  \begin{Exercise}
    设\[
    \begin{cases}
      x_0&=2\\
      x_n&=\frac{1}{2}(x_{n-1}+\frac{2}{x_{n-1}})\\
    \end{cases}
    \]
  
  (1)令$[a,b]$表示$\{x\in R|a\leq x \leq b\}$,这里$R$为实数集,则$\bigcap_{n=0}^{\infty}[\frac{2}{x_n},x_n]=\underline{\quad\quad\quad\quad}$。
  
  (2)令$[a,b]$表示$\{x\in Q|a\leq x \leq b\}$,这里$Q$为有理数集,则$\bigcap_{n=0}^{\infty}[\frac{2}{x_n},x_n]=\underline{\quad\quad\quad\quad}$。
  \end{Exercise}
\end{frame}
\begin{frame}
  \begin{Exercise}
    设\[
    \begin{cases}
      x_0&=2\\
      x_n&=\frac{1}{2}(x_{n-1}+\frac{2}{x_{n-1}})\\
    \end{cases}
    \]
  \end{Exercise}
  (1)令$[a,b]$表示$\{x\in R|a\leq x \leq b\}$,这里$R$为实数集,则$\bigcap_{n=0}^{\infty}[\frac{2}{x_n},x_n]=\underline{\quad\quad\quad\quad}$。
  \pause
  \begin{proof}[解]\pause
   (1)$x_n=\frac{1}{2}(x_{n-1}+\frac{2}{x_{n-1}})\geq \frac{1}{2}*2\sqrt{x_{n-1}*\frac{2}{x_{n-1}}}=\sqrt{2}$\pause
    \begin{align*}
      &x_n-x_{n-1}\\
      =&\frac{1}{2}(x_{n-1}+\frac{2}{x_{n-1}})-x_{n-1}\\
      =&\frac{1}{2}(x_{n-1}+\frac{2}{x_{n-1}}-2x_{n-1})\\
      =&\frac{1}{2}(\frac{2}{x_{n-1}}-x_{n-1})=\frac{2-x_{n-1}^2}{2x_{n-1}}\leq 0\\
    \end{align*}
 \end{proof}  
\end{frame}
\begin{frame}[t]
  \begin{Exercise}
    设\[
    \begin{cases}
      x_0&=2\\
      x_n&=\frac{1}{2}(x_{n-1}+\frac{2}{x_{n-1}})\\
    \end{cases}
    \]
  \end{Exercise} 
\begin{proof}[解]\pause\justifying\let\raggedright\justifying
  (1)这说明序列$x_0,x_1,\cdots,x_n,\cdots$单调下降且有下界,\pause因此收敛,\pause设极限为$x$,
  \pause则\[x=\frac{1}{2}(x+\frac{2}{x})\]
  \pause解得$x=\sqrt{2}$。

  \pause由$x_n\geq \sqrt{2}$\pause知$\frac{2}{x_n}\leq \sqrt{2}$,\pause由序列$x_0,x_1,\cdots,x_n,\cdots$单调下降知\pause序列$\frac{2}{x_0},\frac{2}{x_1},\cdots,\frac{2}{x_n},\cdots$单调上升,
\pause由$\lim_{n\to \infty}x_n=\sqrt{2}$知$\lim_{n\to \infty}\frac{2}{x_{n}}=\sqrt{2}$。

\pause综上,$\bigcap_{n=0}^{\infty}[\frac{2}{x_n},x_n]=\{\sqrt{2}\}$。
\end{proof}  

\end{frame}

\begin{frame}[t]
  \begin{Exercise}
    设\[
    \begin{cases}
      x_0&=2\\
      x_n&=\frac{1}{2}(x_{n-1}+\frac{2}{x_{n-1}})\\
    \end{cases}
    \] 
  (2)令$[a,b]$表示$\{x\in Q|a\leq x \leq b\}$,这里$Q$为有理数集,则$\bigcap_{n=0}^{\infty}[\frac{2}{x_n},x_n]=\underline{\quad\quad\quad\quad}$。
  \end{Exercise}
\end{frame}

\begin{frame}[t]
  \begin{Exercise}
    设\[
    \begin{cases}
      x_0&=2\\
      x_n&=\frac{1}{2}(x_{n-1}+\frac{2}{x_{n-1}})\\
    \end{cases}
    \]
  (2)令$[a,b]$表示$\{x\in Q|a\leq x \leq b\}$,这里$Q$为有理数集,则$\bigcap_{n=0}^{\infty}[\frac{2}{x_n},x_n]=\underline{\quad\quad\quad\quad}$。
  \end{Exercise}
  \begin{proof}[解]\vspace{-1cm}
  \begin{align*}
     &x_n^2\\
     =&(\frac{1}{2}(x_{n-1}+\frac{2}{x_{n-1}}))^2\\
     =&\frac{1}{4}(x_{n-1}^2+\frac{4}{x_{n-1}^2}+4)\\
   \geq&\frac{1}{4}(2*x_{n-1}*\frac{2}{x_{n-1}}+4)=2\\
  \end{align*}
\end{proof}
\end{frame}
\begin{frame}[t]
  \begin{Exercise}
    设\[
    \begin{cases}
      x_0&=2\\
      x_n&=\frac{1}{2}(x_{n-1}+\frac{2}{x_{n-1}})\\
    \end{cases}
    \]
  (2)令$[a,b]$表示$\{x\in Q|a\leq x \leq b\}$,这里$Q$为有理数集,则$\bigcap_{n=0}^{\infty}[\frac{2}{x_n},x_n]=\underline{\quad\quad\quad\quad}$。
  \end{Exercise}
   \begin{proof}[解]\justifying\let\raggedright\justifying\vspace{-1cm}
  \begin{align*}
    &x_n-x_{n-1}\\
    =&\frac{1}{2}(x_{n-1}+\frac{2}{x_{n-1}})-x_{n-1}\\
    =&\frac{1}{2}(x_{n-1}+\frac{2}{x_{n-1}}-2x_{n-1})
    =\frac{1}{2}(\frac{2}{x_{n-1}}-x_{n-1})=\frac{2-x_{n-1}^2}{2x_{n-1}}\leq 0\\
  \end{align*}
   这说明序列$x_0,x_1,\cdots,x_n,\cdots$单调下降,从而序列$\frac{2}{x_0},\frac{2}{x_1},\cdots,\frac{2}{x_n},\cdots$单调上升。
 \end{proof}
\end{frame}
\begin{frame}[t]
  \begin{Exercise}
    设\[
    \begin{cases}
      x_0&=2\\
      x_n&=\frac{1}{2}(x_{n-1}+\frac{2}{x_{n-1}})\\
    \end{cases}
    \]
  (2)令$[a,b]$表示$\{x\in Q|a\leq x \leq b\}$,这里$Q$为有理数集,则$\bigcap_{n=0}^{\infty}[\frac{2}{x_n},x_n]=\underline{\quad\quad\quad\quad}$。
  \end{Exercise}
  \begin{proof}[解]\vspace{-1cm}
    \begin{align*}
      &x_n-\frac{2}{x_n}=\frac{x_n^2-2}{x_n}=\frac{(\frac{1}{2}(x_{n-1}+\frac{2}{x_{n-1}}))^2-2}{x_n}\\
    =&\frac{\frac{1}{4}(x_{n-1}^2+\frac{4}{x_{n-1}^2}+4)-2}{x_n}=\frac{\frac{1}{4}(x_{n-1}^2+\frac{4}{x_{n-1}^2}-4)}{x_n}\\
    =&\frac{x_{n-1}-\frac{2}{x_{n-1}}}{4x_n}(x_{n-1}-\frac{2}{x_{n-1}})\leq \frac{1}{4}(x_{n-1}-\frac{2}{x_{n-1}})
    \end{align*} 
    这说明$\lim_{n\to\infty}(x_n-\frac{2}{x_n})=0$。
\end{proof}
\end{frame}
\begin{frame}[t]
  \begin{Exercise}
    设\[
    \begin{cases}
      x_0&=2\\
      x_n&=\frac{1}{2}(x_{n-1}+\frac{2}{x_{n-1}})\\
    \end{cases}
    \]
  (2)令$[a,b]$表示$\{x\in Q|a\leq x \leq b\}$,这里$Q$为有理数集,则$\bigcap_{n=0}^{\infty}[\frac{2}{x_n},x_n]=\underline{\quad\quad\quad\quad}$。
  \end{Exercise}
  \begin{proof}[解]\justifying\let\raggedright\justifying
    \pause以下证明$\bigcap_{n=0}^{\infty}[\frac{2}{x_n},x_n]=\phi$。

    \pause用反证法。\pause设存在$x\in \bigcap_{n=0}^{\infty}[\frac{2}{x_n},x_n]$,\pause由$x$为有理数知$x^2\neq 2$。
    
    \pause如果$x^2>2$,则$\frac{1}{x^2}<\frac{1}{2}$,从而$(\frac{2}{x})^2<2<x^2$,于是$\frac{2}{x}<x$。对任意的自然数$n$,由$x\leq x_n$知$\frac{2}{x}\geq \frac{2}{x_n}$,
    从而$x_n-\frac{2}{x_n}\geq x-\frac{2}{x}$,这与$\lim_{n\to\infty}(x_n-\frac{2}{x_n})=0$矛盾。
    
    \pause如果$x^2<2$,则$\frac{1}{x^2}>\frac{1}{2}$,从而$(\frac{2}{x})^2>2>x^2$,于是$\frac{2}{x}>x$。对任意的自然数$n$,由$x\geq \frac{2}{x_n}$知$\frac{2}{x}\leq x_n$,
    从而$x_n-\frac{2}{x_n}\geq \frac{2}{x}-x$,这也与$\lim_{n\to\infty}(x_n-\frac{2}{x_n})=0$矛盾。    
\end{proof}
\end{frame}
\begin{frame}
  \begin{Exercise}
    设$f:X\to Y$。试证:$f$为满射当且仅当对任意的$E\in 2^Y$,$f(f^{-1}(E))=E$。
 \end{Exercise}
 \pause
 \begin{proof}[证明]\justifying\let\raggedright\justifying
   设$f$为满射,对任意的$E\in 2^Y$往证$f(f^{-1}(E))=E$。
   \pause
 
   
   对任意的$y$,$y\in f(f^{-1}(E))$,\pause 则存在$x$,$x\in f^{-1}(E)$并且$y=f(x)$,\pause 于是存在$x$,$f(x) \in E$并且$y=f(x)$,\pause 从而$y\in E$。
 
  \pause  
   对任意的$y$,$y\in E$,\pause 由$f$为满射知存在$x\in X$,$y=f(x)$,\pause 从而$f(x)\in E$,\pause 即$x\in f^{-1}(E)$,\pause 由$y=f(x)$知$y\in f(f^{-1}(E))$。
 
   \pause
   设对任意的$E\in 2^Y$,$f(f^{-1}(E))=E$,往证$f$为满射。
 
   \pause
   对任意的$y\in Y$,\pause 则$f(f^{-1}(\{y\}))=\{y\}$,\pause 于是$f^{-1}(\{y\})\neq \phi$,\pause 从而存在$x\in X$,$x\in f^{-1}(\{y\})$, \pause 即$f(x)\in \{y\}$,\pause 等价的,$f(x)=y$,\pause 故$f$为满射。
 \end{proof}
   
\end{frame}
\begin{frame}
  \begin{Exercise}
    设$f:X\to Y$。
    
    (1)如果存在唯一的一个映射$g:Y\to X$,使得$gf=I_X$,那么$f$是否可逆呢?
    
    (2)如果存在唯一的一个映射$g:Y\to X$,使得$fg=I_Y$,那么$f$是否可逆呢?
    \end{Exercise}
    \begin{proof}[解]{\small
    
      \pause(1)\pause当$|X|=1$时,\pause$f$不一定可逆,\pause举例如下:
    
      \pause设集合$X=\{1\}$,\pause$Y=\{1,2\}$,\pause$f:X\to Y$,\pause$f(1)=1$。\pause则存在唯一的一个映射$g:Y\to X$,\pause$g(1)=1,g(2)=1$,\pause使得$gf=I_X$,\pause但$f$不可逆。
    
      \pause当$|X|>1$时,\pause$f$一定可逆,\pause证明如下:
    
      \pause由$gf=I_X$知$f$为单射,\pause以下证明$f$为满射。\pause用反证法,\pause假设$f$不为满射,\pause则存在$y_0\in Y$,\pause对任意的$x\in X$,\pause$f(x)\neq y_0$。
      \pause由于$|X|>1$,\pause可取$x_0\in X$,\pause使得$g(y_0)\neq x_0$。
    
      \pause令$h:Y\to X$,
      \[h(y)=\begin{cases}
          g(y) &\text{如果} y \neq y_0,\\
          x_0 &\text{如果} y = y_0\\
        \end{cases}
      \]
      \pause则$hf=I_X$,\pause且$h\neq g$,\pause与存在唯一的一个映射$g:Y\to X$使得$gf=I_X$矛盾。}
      \end{proof}

\end{frame}
\begin{frame}
  \begin{Exercise}
    是否存在一个偏序关系$\leq$,使$(X,\leq)$中有唯一极大元素,但没有最大元素?如果存在,请给出一个具体例子;如果不存在,请证明之。
  \end{Exercise}\pause
  \begin{proof}[解]
    存在。偏序集$(R\cup \{i\},\leq)$上有唯一极大元素$i$,但没有最大元素。
  
    这里$\leq$为实数集上的小于等于关系,复数$i$与任意实数都不可比较,因此没有元素比它大,它就是极大元。
  \end{proof}  
\end{frame}

\begin{frame}
  \begin{Exercise}
    设有穷偏序集$(X,\leq)$中有唯一极大元素$x$,则$x$为$X$的最大元素。
    \end{Exercise}\pause
  \begin{proof}[证明]\justifying\let\raggedright\justifying
    \pause用数学归纳法证明,\pause施归纳于$X$中元素的个数$n$。

    \pause(1)当$n=1$时,\pause$X=\{x\}$,\pause则显然$x$为$X$的最大元素。
    
    \pause(2)假设当$n=k(k\geq 1)$时结论成立,\pause往证当$n=k+1$时结论也成立。
    \pause设$|X|=k+1$,\pause$x$为$X$的唯一极大元素,\pause以下证明$x$为$X$的最大元素。
    \pause对任意的$y\in X$,\pause如果$y=x$,\pause则显然$y\leq x$。\pause当$y\neq x$时,\pause由$x$为$X$的唯一极大元素知$y$不是$X$的极大元素,\pause从而存在$z\in Z$,\pause$z>y$。
    \pause关系$\{(x,y)|x\in X\setminus \{y\}, y\in X\setminus \{y\}, x\leq y\}$\pause构成集合$X\setminus \{y\}$上的一个偏序关系。\pause$|X\setminus \{y\}|=k$。
    
    \pause此时,\pause$x$必为$X\setminus \{y\}$的唯一极大元。\pause否则,\pause如果存在元素$a$为$X\setminus \{y\}$的极大元,\pause$a\neq x$,\pause由$a$不是$X$的极大元知$a\leq y$,\pause再由$z>y$知$z>a$,\pause矛盾。
    \pause由归纳假设,\pause$x$为$X\setminus \{y\}$的最大元。\pause由$z\leq x$及$z>y$知,\pause$y\leq x$。
  \end{proof}  
\end{frame}


% \begin{frame}
%   \frametitle{5 公理集合论}
%   \begin{enumerate}
%   \item $0 \in \mathbb{N}$;
%   \item $n \in \mathbb{N} \rightarrow n ++ \in \mathbb{N}$;
%   \item $\forall n \in \mathbb{N} n ++ \neq 0$;
%   \item $\forall n \in \mathbb{N} \forall m \in \mathbb{N} n \neq m \rightarrow n ++ \neq m ++$;
%     \item $(P(0) \land \forall n \in \mathbb{N} p(n) \rightarrow p(n++) )\rightarrow \forall n p(n)$。 
%   \end{enumerate}
% \end{frame}

\begin{frame}


  \begin{Exercise}[P20-5]
    设$X$为一个非空集合,$A_n\subseteq X$, $A_{n+1}\subseteq A_n$,$n=1,2,3,\cdots$。试证对任意的自然数$n$,
    \[A_n=\bigcup_{m=n}^{\infty}(A_m\cap A_{m+1}^c)\cup \bigcap_{m=n}^{\infty}A_m\]
  \end{Exercise}
\end{frame}
\begin{frame}
  \begin{Exercise}[P47-5]
    设$f:X\to Y$。试证:$f$为满射当且仅当对任意的$E\in 2^Y$,$f(f^{-1}(E))=E$。
  \end{Exercise}
\end{frame}
\begin{frame}
  \begin{Exercise}[P126-6]
    设$R$为集合$X$上的自反且传递的二元关系。

    a)给出$R$的一个实例。

    b)在$X$上定义二元关系$\sim$如下:$x\sim y$当且仅当$x R y$且$y R x$。 证明$\sim$为$X$上的等价关系。

    c)在商集$X/\sim$上定义二元关系$\leq$:$[a]\leq [b]$当且仅当$aRb$。
    证明$\leq$为$X/\sim$上的偏序关系。
  \end{Exercise}
\end{frame}
\end{CJK*}
\end{document}

%%% Local Variables:
%%% mode: latex
%%% TeX-master: t
%%% End:
