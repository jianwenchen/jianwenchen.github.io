\documentclass{article}
\usepackage{CJKutf8}
\usepackage{amsmath}
\usepackage{amssymb}
\usepackage{amsfonts}
\usepackage{amsthm}
\usepackage{titlesec}
\usepackage{titletoc}
\usepackage{xCJKnumb}
\usepackage{tikz}
\usepackage{mathrsfs}
\usepackage{indentfirst}
\usepackage{enumitem}
\newtheorem{Def}{定义}
\newtheorem{Thm}{定理}
\newtheorem{Exercise}{练习}

\newtheorem*{Example}{例}
\setlist[enumerate,1]{label=(\arabic*)}

\begin{document}
\begin{CJK*}{UTF8}{gbsn}
  \title{第六讲 一阶谓词逻辑演算基本概念}
  \author{陈建文}
  \maketitle

  \begin{Example}$\quad$

    \begin{enumerate}
      \item 所有的人都是要死的;
      \item 苏格拉底是人;
      \item 所以苏格拉底是要死的。
    \end{enumerate}

    谓词$M(x):x$是人

    谓词$D(x):x$是要死的

    个体常元$a$表示“苏格拉底”

    谓词:表示研究对象的性质或研究对象之间关系的词称为谓词。

    \begin{enumerate}
      \item $\forall x(M(x)\to D(x))$
      \item $M(a)\to D(a)$
      \item $M(a)$
      \item $D(a)$
    \end{enumerate}
  \end{Example}
  \begin{Example}
    $(G,\circ, e)$称为一个群,如果
    \begin{enumerate}
      \item $\forall x\forall y\forall z (x\circ y)\circ z=x\circ(y\circ z);$
      \item $\forall x e\circ x = x;$
      \item $\forall x \exists y y\circ x=e.$
    \end{enumerate}
  \end{Example}
  
  字符集:
    \begin{enumerate}
      \item 表示个体变元的符号:$v_1,v_2,\cdots,v_n,\cdots$
      \item 完备的联结词集合:$\{\lnot, \land, \lor, \to, \leftrightarrow \}$
      \item 量词:$\forall,\exists$
      \item 辅助符号:$()$
      \item 表示谓词的符号:$P,Q,\cdots$
      \item 表示函词的符号:$f,g,\cdots$
      \item 表示个体常元的符号:$a,b,c,\cdots$
    \end{enumerate}


  \begin{Def}$\quad$
    \begin{enumerate}
      \item 个体常元和个体变元都是项;
      \item 如果$f$为一个$n$元函词,$t_1,t_2,\cdots,t_n$是项,则$f(t_1,t_2,\cdots,t_n)$也是项;
      \item 由(1)(2)有限次复合所产生的结果都是项。
    \end{enumerate}
  \end{Def}
  \begin{Def}$\quad$
    \begin{enumerate}
      \item 设$t_1,t_2,\cdots,t_n$为$n$个项,$P$为任意一个$n$元谓词符号,则$P(t_1,t_2,\cdots,t_n)$为合式公式(这样的合式公式又称为原子谓词公式);
      \item 如果$A,B$为合式公式,则$(\lnot A),(A\land B),(A\lor B),(A\to B),(A\leftrightarrow B)$为合式公式;
      \item 如果$A$为合式公式,$x$为任意一个变量,则$\forall xA$和$\exists xA$为合式公式;    
      \item 有限次使用(1),(2)和(3)复合所得到的结果都是合式公式。
    \end{enumerate}
  \end{Def}
  合式公式简称公式。

  \begin{Example}
    $\forall xP(x)\to P(x)$和$\forall y\forall xP(x)$都是公式。
  \end{Example}
  \begin{Def}$\quad$
  \begin{enumerate}
    \item 受量词约束的变元称为约束变元;
    \item 不受量词约束的变元称为自由变元。
  \end{enumerate}
\end{Def}

\begin{Example}
  在公式$\forall xP(x)\to P(x)$中,$\forall xP(x)$中的$x$是约束变元,$\to$之后的$P(x)$中的$x$为自由变元。
\end{Example}
\end{CJK*}
\end{document}





%%% Local Variables:
%%% mode: latex
%%% TeX-master: t
%%% End:
