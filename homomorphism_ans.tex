\documentclass{article}
\usepackage{CJKutf8}
\usepackage{amsmath}
\usepackage{amssymb}
\usepackage{amsfonts}
\usepackage{amsthm}
\usepackage{titlesec}
\usepackage{titletoc}
\usepackage{xCJKnumb}
\usepackage{tikz}
\usepackage{mathrsfs}
\usepackage{indentfirst}

\newtheorem{Def}{定义}
\newtheorem{Thm}{定理}
\newtheorem{Exercise}{练习}

\newtheorem*{Example}{例}


\begin{document}
\begin{CJK*}{UTF8}{gbsn}
  \title{第八讲 同态基本定理}
  \author{陈建文}
  \maketitle
  % \tableofcontents
  


课后作业题:
\begin{Exercise}
设$(G,\circ)$为$m$阶循环群,$(\bar{G},\cdot)$为$n$阶循环群,试证:$G \sim \bar{G}$当且仅当$n | m$。
\end{Exercise}
\begin{proof}[证明]
由$G\sim \bar{G}$往证$n|m$:

设$\phi$为从$G$到$\bar{G}$的一个满同态,由群同态基本定理,$G/Ker \phi\cong \bar{G}$,于是$|G/Ker \phi|=|\bar{G}|$。由拉格朗日定理,$|G|=|G/Ker \phi||Ker \phi|$,这说明$|G/Ker \phi|||G|$,从而$|\bar{G}|||G|$,即$n|m$。

设$n|m$,往证$G\sim \bar{G}$:

设$G=\{a^0,a,a^2,\cdots,a^{m-1}\}$,$\bar{G}=\{b^0,b,b^2,\cdots,b^{n-1}\}$。

令$\phi:G\to \bar{G}$,$\phi(a^i)=b^{i \mod n}$,则$\forall i,j, 0\leq i \leq m-1, 0\leq j \leq m-1,$\phi(a^i\circ a^j)=\phi(a^{i+j})=b^{(i+j) \mod n} = b^{((i\mod n) + (j\mod n))\mod n}=b^{i\mod n}\cdot b^{j\mod n}=\phi(a^i)\cdot \phi(a^j)$。 
这证明了$\phi$为从$G$到$\bar{G}$的同态,$\phi$显然为满同态,于是$G\sim \bar{G}$。
\end{proof}
\begin{Exercise}
设$G$为一个循环群,$H$为群$G$的子群,试证:$G/H$也为循环群。
\end{Exercise}
\begin{proof}[证明]
设$G=(a)$,$\forall x\in G/H$,存在自然数$i$使得$x=a^iH=(aH)^i$,于是$G/H=(aH)$,即$G/H$为循环群。
\end{proof}
\end{CJK*}
\end{document}





%%% Local Variables:
%%% mode: latex
%%% TeX-master: t
%%% End:



