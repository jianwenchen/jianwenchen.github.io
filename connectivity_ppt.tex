\documentclass{beamer}
\usepackage{ragged2e}
\justifying\let\raggedright\justifying
\usepackage{clrscode3e}
\usepackage{CJKutf8}
\usepackage{tikz}
\setbeamertemplate{theorems}[numbered]
\newtheorem{Thm}{定理}[section]
\newtheorem*{Thm1.1}{定理1.1}
\newtheorem*{Thm1.4}{定理1.4}
\newtheorem*{Thm3.1}{定理3.1}

\theoremstyle{definition}
\newtheorem{Def}{定义}[section]
\theoremstyle{example}
\newtheorem*{Ex}{例:}
\newtheorem*{Exercise}{练习:}

\begin{document}
\begin{CJK}{UTF8}{gbsn}

\date{}
\author{陈建文}

\begin{frame}
  \begin{Thm}
    设$G$为一个有$p$个顶点的图,如果对$G$的每一对不临接的顶点$u$和$v$,均有
    \begin{equation*}
      \deg u + \deg v \geq p - 1,
    \end{equation*}
则$G$为连通的。
\end{Thm}
\begin{proof}[证明]\justifying\let\raggedright\justifying
  \pause用反证法。\pause假设$G$不连通,\pause则至少有两个连通分量。\pause设$G_1=(V_1,E_1)$为其中的一个连通分量,\pause其他各连通分量构成的子图为$G_2=(V_2,E_2)$。\pause取$V_1$中的任意一个顶点$u$和$V_2$中的任意一个顶点$v$,\pause则顶点$u$和顶点$v$不邻接并且
  \[\deg u + \deg v \leq (|V_1| - 1 ) + (|V_2| - 1) = p - 2\]
  \pause矛盾。
\end{proof}
\end{frame}

\begin{frame}
  \begin{Exercise}
    若$G$是一个$(p,q)$图,$q > \frac{1}{2}(p-1)(p-2)$,试证$G$是连通图。  
    \end{Exercise}
    \begin{proof}[证明]\justifying\let\raggedright\justifying
      \pause用反证法。\pause假设$G$不连通,\pause则至少有两个连通分量。\pause设$G_1=(V_1,E_1)$为其中的一个连通分量,\pause其他各连通分量构成的子图为$G_2=(V_2,E_2)$。\pause取$G_1$中的一个顶点$u$和$G_2$中的一个顶点$v$,\pause将$G_1$中与$u$相关联的边替换为与$v$相关联的边(边的另一个顶点保持不变)所得到的图为$G'$,\pause则$G$中的边数等于$G'$中的边数。\pause显然$G'$中的边数小于等于$K_{p-1}$中的边数,\pause从而$G$中的边数小于等于$K_{p-1}$中的边数,\pause即
       \[q\leq \frac{1}{2}(p-1)(p-2)\]
       矛盾。    
       \end{proof}
\end{frame}

\title{第八章 连通度和匹配}
\begin{frame}
  \titlepage
\end{frame}
\begin{frame}
\centering
  \begin{minipage}{0.24\linewidth}
    \centering
    \begin{tikzpicture}[auto,
    specification/.style ={circle, draw, thick}]
   \node[specification] (A) at (0,0)  {};
   \node[specification] (B)  at (0,1)  {};
   \node[specification] (C)  at (1,1)  {};
   \node[specification] (D) at (1,0)  {};
 \end{tikzpicture}\\
 \vspace*{0.1cm}
 A
\end{minipage}\hfill 
  \begin{minipage}{0.24\linewidth}
    \centering
    \begin{tikzpicture}[auto,
    specification/.style ={circle, draw, thick}]
   \node[specification] (A) at (0,0)  {};
   \node[specification] (B) at (0,1)  {};
   \node[specification] (C) at (1,1)  {};
   \node[specification] (D) at (1,0)  {};
   \draw[thick] (B) to  (C);
 \end{tikzpicture}\\
 \vspace*{0.1cm}
 B
\end{minipage}\hfill 
  \begin{minipage}{0.24\linewidth}
    \centering
    \begin{tikzpicture}[auto,
    specification/.style ={circle, draw, thick}]
   \node[specification] (A) at (0,0)  {};
   \node[specification] (B) at (0,1)  {};
   \node[specification] (C) at (1,1)  {};
   \node[specification] (D) at (1,0)  {};
   \draw[thick] (A) to  (B);
   \draw[thick] (B) to  (C);
 \end{tikzpicture}\\
 \vspace*{0.1cm}
 C
\end{minipage}\hfill 
  \begin{minipage}{0.24\linewidth}
    \centering
    \begin{tikzpicture}[auto,
    specification/.style ={circle, draw, thick}]
   \node[specification] (A)  at (0,0)  {};
   \node[specification] (B)  at (0,1)  {};
   \node[specification] (C)  at (1,1)  {};
   \node[specification] (D) at (1,0)  {};
   \draw[thick] (B) to  (C);
   \draw[thick] (D) to  (A);
 \end{tikzpicture}\\
 \vspace*{0.1cm}
 D
\end{minipage}\hfill

\vspace*{0.5cm}
  \begin{minipage}{0.24\linewidth}
    \centering
    \begin{tikzpicture}[auto,
    specification/.style ={circle, draw, thick}]
   \node[specification] (A) at (0,0)  {};
   \node[specification] (B)  at (0,1)  {};
   \node[specification] (C)  at (1,1)  {};
   \node[specification] (D) at (1,0)  {};
   \draw[thick] (A) to (B);
   \draw[thick] (B) to (C);
      \draw[thick] (B) to (D);
 \end{tikzpicture}\\
 \vspace*{0.1cm}
 E
\end{minipage}\hfill
  \begin{minipage}{0.24\linewidth}
    \centering
    \begin{tikzpicture}[auto,
    specification/.style ={circle, draw, thick}]
   \node[specification] (A) at (0,0)  {};
   \node[specification] (B) at (0,1)  {};
   \node[specification] (C) at (1,1)  {};
   \node[specification] (D) at (1,0)  {};
   \draw[thick] (A) to  (B);
   \draw[thick] (B) to (C);
   \draw[thick] (C) to (A);
 \end{tikzpicture}\\
 \vspace*{0.1cm}
 F
\end{minipage}\hfill
  \begin{minipage}{0.24\linewidth}
    \centering
    \begin{tikzpicture}[auto,
    specification/.style ={circle, draw, thick}]
   \node[specification] (A) at (0,0)  {};
   \node[specification] (B) at (0,1)  {};
   \node[specification] (C) at (1,1)  {};
   \node[specification] (D) at (1,0)  {};
   \draw[thick] (A) to  (B);
   \draw[thick] (B) to  (C);
      \draw[thick] (C) to (D);
 \end{tikzpicture}\\
 \vspace*{0.1cm}
 G
\end{minipage}\hfill 
  \begin{minipage}{0.24\linewidth}
    \centering
    \begin{tikzpicture}[auto,
    specification/.style ={circle, draw, thick}]
   \node[specification] (A)  at (0,0)  {};
   \node[specification] (B)  at (0,1)  {};
   \node[specification] (C)  at (1,1)  {};
   \node[specification] (D) at (1,0)  {};
   \draw[thick] (A) to  (C);
   \draw[thick] (C) to  (D);
   \draw[thick] (D) to (A);
   \draw[thick] (B) to (D);
 \end{tikzpicture}\\
 \vspace*{0.1cm}
 H
\end{minipage}\hfill 

\vspace*{0.5cm}
\flushleft
  \begin{minipage}{0.24\linewidth}
    \centering
    \begin{tikzpicture}[auto,
    specification/.style ={circle, draw, thick}]
   \node[specification] (A) at (0,0)  {};
   \node[specification] (B)  at (0,1)  {};
   \node[specification] (C)  at (1,1)  {};
   \node[specification] (D) at (1,0)  {};
   \draw[thick] (A) to (B);
   \draw[thick] (B) to (D);
   \draw[thick] (D) to (C);
      \draw[thick] (C) to (A);
 \end{tikzpicture}\\
 \vspace*{0.1cm}
 I
\end{minipage}
  \begin{minipage}{0.24\linewidth}
    \centering
    \begin{tikzpicture}[auto,
    specification/.style ={circle, draw, thick}]
   \node[specification] (A) at (0,0)  {};
   \node[specification] (B) at (0,1)  {};
   \node[specification] (C) at (1,1)  {};
   \node[specification] (D) at (1,0)  {};
   \draw[thick] (A) to  (B);
      \draw[thick] (C) to (D);
   \draw[thick] (D) to (A);
   \draw[thick] (A) to (C);
   \draw[thick] (B) to (D);
 \end{tikzpicture}\\
 \vspace*{0.1cm}
 J
\end{minipage} 
  \begin{minipage}{0.24\linewidth}
    \centering
    \begin{tikzpicture}[auto,
    specification/.style ={circle, draw, thick}]
   \node[specification] (A) at (0,0)  {};
   \node[specification] (B) at (0,1)  {};
   \node[specification] (C) at (1,1)  {};
   \node[specification] (D) at (1,0)  {};
   \draw[thick] (A) to  (B);
   \draw[thick] (B) to  (C);
      \draw[thick] (C) to (D);
   \draw[thick] (D) to (A);
   \draw[thick] (A) to (C);
   \draw[thick] (B) to (D);
 \end{tikzpicture}\\
 \vspace*{0.1cm}
 K
\end{minipage}  
  
\end{frame}
\section{顶点连通度和边连通度}
\begin{frame}
  \frametitle{1. 顶点连通度和边连通度}
  \begin{Def}
    图$G$的\alert{顶点连通度}是指为了产生一个不连通图或平凡图所需要从$G$中去掉的最少顶点数目, 记为$\kappa (G)$。
  \end{Def}\pause
  \centering
  \begin{minipage}{0.24\linewidth}
    \centering
    \begin{tikzpicture}[auto,
    specification/.style ={circle, draw, thick}]
   \node[specification] (A) at (0,0)  {};
   \node[specification] (B)  at (0,1)  {};
   \node[specification] (C)  at (1,1)  {};
   \node[specification] (D) at (1,0)  {};
 \end{tikzpicture}\\
 A
\end{minipage}\hfill 
  \begin{minipage}{0.24\linewidth}
    \centering
    \begin{tikzpicture}[auto,
    specification/.style ={circle, draw, thick}]
   \node[specification] (A) at (0,0)  {};
   \node[specification] (B) at (0,1)  {};
   \node[specification] (C) at (1,1)  {};
   \node[specification] (D) at (1,0)  {};
   \draw[thick] (B) to  (C);
 \end{tikzpicture}\\
 B
\end{minipage}\hfill 
  \begin{minipage}{0.24\linewidth}
    \centering
    \begin{tikzpicture}[auto,
    specification/.style ={circle, draw, thick}]
   \node[specification] (A) at (0,0)  {};
   \node[specification] (B) at (0,1)  {};
   \node[specification] (C) at (1,1)  {};
   \node[specification] (D) at (1,0)  {};
   \draw[thick] (A) to  (B);
   \draw[thick] (B) to  (C);
 \end{tikzpicture}\\
 C
\end{minipage}\hfill 
  \begin{minipage}{0.24\linewidth}
    \centering
    \begin{tikzpicture}[auto,
    specification/.style ={circle, draw, thick}]
   \node[specification] (A)  at (0,0)  {};
   \node[specification] (B)  at (0,1)  {};
   \node[specification] (C)  at (1,1)  {};
   \node[specification] (D) at (1,0)  {};
   \draw[thick] (B) to  (C);
   \draw[thick] (D) to  (A);
 \end{tikzpicture}\\
 D
\end{minipage}\hfill

\vspace*{0.5cm}
  \begin{minipage}{0.24\linewidth}
    \centering
    \begin{tikzpicture}[auto,
    specification/.style ={circle, draw, thick}]
   \node[specification] (A) at (0,0)  {};
   \node[specification] (B)  at (0,1)  {};
   \node[specification] (C)  at (1,1)  {};
   \node[specification] (D) at (1,0)  {};
   \draw[thick] (A) to (B);
   \draw[thick] (B) to (C);
      \draw[thick] (B) to (D);
 \end{tikzpicture}\\
 E
\end{minipage}\hfill
  \begin{minipage}{0.24\linewidth}
    \centering
    \begin{tikzpicture}[auto,
    specification/.style ={circle, draw, thick}]
   \node[specification] (A) at (0,0)  {};
   \node[specification] (B) at (0,1)  {};
   \node[specification] (C) at (1,1)  {};
   \node[specification] (D) at (1,0)  {};
   \draw[thick] (A) to  (B);
   \draw[thick] (B) to (C);
   \draw[thick] (C) to (A);
 \end{tikzpicture}\\
 F
\end{minipage}\hfill
  \begin{minipage}{0.24\linewidth}
    \centering
    \begin{tikzpicture}[auto,
    specification/.style ={circle, draw, thick}]
   \node[specification] (A) at (0,0)  {};
   \node[specification] (B) at (0,1)  {};
   \node[specification] (C) at (1,1)  {};
   \node[specification] (D) at (1,0)  {};
   \draw[thick] (A) to  (B);
   \draw[thick] (B) to  (C);
      \draw[thick] (C) to (D);
 \end{tikzpicture}\\
 G
\end{minipage}\hfill 
  \begin{minipage}{0.24\linewidth}
    \centering
    \begin{tikzpicture}[auto,
    specification/.style ={circle, draw, thick}]
   \node[specification] (A)  at (0,0)  {};
   \node[specification] (B)  at (0,1)  {};
   \node[specification] (C)  at (1,1)  {};
   \node[specification] (D) at (1,0)  {};
   \draw[thick] (A) to  (C);
   \draw[thick] (C) to  (D);
   \draw[thick] (D) to (A);
   \draw[thick] (B) to (D);
 \end{tikzpicture}\\
 H
\end{minipage}\hfill 

\vspace*{0.5cm}
\flushleft
  \begin{minipage}{0.24\linewidth}
    \centering
    \begin{tikzpicture}[auto,
    specification/.style ={circle, draw, thick}]
   \node[specification] (A) at (0,0)  {};
   \node[specification] (B)  at (0,1)  {};
   \node[specification] (C)  at (1,1)  {};
   \node[specification] (D) at (1,0)  {};
   \draw[thick] (A) to (B);
   \draw[thick] (B) to (D);
   \draw[thick] (D) to (C);
      \draw[thick] (C) to (A);
 \end{tikzpicture}\\
 I
\end{minipage}
  \begin{minipage}{0.24\linewidth}
    \centering
    \begin{tikzpicture}[auto,
    specification/.style ={circle, draw, thick}]
   \node[specification] (A) at (0,0)  {};
   \node[specification] (B) at (0,1)  {};
   \node[specification] (C) at (1,1)  {};
   \node[specification] (D) at (1,0)  {};
   \draw[thick] (A) to  (B);
      \draw[thick] (C) to (D);
   \draw[thick] (D) to (A);
   \draw[thick] (A) to (C);
   \draw[thick] (B) to (D);
 \end{tikzpicture}\\
 J
\end{minipage} 
  \begin{minipage}{0.24\linewidth}
    \centering
    \begin{tikzpicture}[auto,
    specification/.style ={circle, draw, thick}]
   \node[specification] (A) at (0,0)  {};
   \node[specification] (B) at (0,1)  {};
   \node[specification] (C) at (1,1)  {};
   \node[specification] (D) at (1,0)  {};
   \draw[thick] (A) to  (B);
   \draw[thick] (B) to  (C);
      \draw[thick] (C) to (D);
   \draw[thick] (D) to (A);
   \draw[thick] (A) to (C);
   \draw[thick] (B) to (D);
 \end{tikzpicture}\\
 K
\end{minipage}  

%   \begin{minipage}{0.33\linewidth}
    
%       \begin{tikzpicture}[auto,
%     specification/.style ={circle, draw, thick, inner sep = 0pt, minimum size=2mm}]
%    \node[specification] (A)  at (0,-1)  {};
%    \node[specification] (B)  at (0,1)  {};
%    \node[specification] (C)  at (1,0)  {};
%    \node[specification] (D) at (2,-1)  {};
%    \node[specification] (E)  at (2,1)  {};
%    \node[specification] (F)  at (3,0)  {};
   
   
%    \draw[thick] (A) to  (B);
%    \draw[thick] (B) to  (C);
%    \draw[thick] (C) to  (A);
   
%    \draw[thick] (D) to  (E);
%    \draw[thick] (E) to  (F);
%    \draw[thick] (F) to  (D);
   
%    \draw[thick] (A) to  (D);
%    \draw[thick] (B) to  (E);
%    \draw[thick] (C) to  (F);
%  \end{tikzpicture}  \end{minipage}
%  \begin{minipage}{0.33\linewidth}
%   \begin{tikzpicture}[auto,
%     specification/.style ={circle, draw, thick, inner sep = 0pt, minimum size=2mm}]
%    \node[specification] (A)  at (18:1.5cm)  {};
%    \node[specification] (B)  at (90:1.5cm)  {};
%    \node[specification] (C)  at (162:1.5cm)  {};
%    \node[specification] (D) at (234:1.5cm)  {};
%    \node[specification] (E)  at (306:1.5cm)  {};
   
%    \draw[thick] (A) to  (B);
%    \draw[thick] (B) to  (C);
%    \draw[thick] (C) to  (D);
%    \draw[thick] (D) to  (E);
%    \draw[thick] (E) to  (A);

%    \draw[thick] (A) to  (C);
%    \draw[thick] (A) to  (D);
%    \draw[thick] (B) to  (D);
%    \draw[thick] (B) to  (E);
%    \draw[thick] (C) to  (E);

%  \end{tikzpicture} \end{minipage}
% \hspace{0.5cm}
%  \begin{minipage}[c]{0.2\linewidth}
%     \begin{tikzpicture}[auto,
%     specification/.style ={circle, draw, thick, inner sep = 0pt, minimum size=2mm}]
%    \node[specification] (A)  at (0,0)  {};
%  \end{tikzpicture}   
%  \end{minipage}

\end{frame}

\begin{frame}
  \frametitle{1. 顶点连通度和边连通度}
  \begin{Def}
    图$G$的\alert{边连通度}是指为了产生一个不连通图或者平凡图所需要从$G$中去掉的最少边的数目, 记为$\lambda (G)$。
  \end{Def}\pause
    \centering
  \begin{minipage}{0.24\linewidth}
    \centering
    \begin{tikzpicture}[auto,
    specification/.style ={circle, draw, thick}]
   \node[specification] (A) at (0,0)  {};
   \node[specification] (B)  at (0,1)  {};
   \node[specification] (C)  at (1,1)  {};
   \node[specification] (D) at (1,0)  {};
 \end{tikzpicture}\\
 A
\end{minipage}\hfill 
  \begin{minipage}{0.24\linewidth}
    \centering
    \begin{tikzpicture}[auto,
    specification/.style ={circle, draw, thick}]
   \node[specification] (A) at (0,0)  {};
   \node[specification] (B) at (0,1)  {};
   \node[specification] (C) at (1,1)  {};
   \node[specification] (D) at (1,0)  {};
   \draw[thick] (B) to  (C);
 \end{tikzpicture}\\
 B
\end{minipage}\hfill 
  \begin{minipage}{0.24\linewidth}
    \centering
    \begin{tikzpicture}[auto,
    specification/.style ={circle, draw, thick}]
   \node[specification] (A) at (0,0)  {};
   \node[specification] (B) at (0,1)  {};
   \node[specification] (C) at (1,1)  {};
   \node[specification] (D) at (1,0)  {};
   \draw[thick] (A) to  (B);
   \draw[thick] (B) to  (C);
 \end{tikzpicture}\\
 C
\end{minipage}\hfill 
  \begin{minipage}{0.24\linewidth}
    \centering
    \begin{tikzpicture}[auto,
    specification/.style ={circle, draw, thick}]
   \node[specification] (A)  at (0,0)  {};
   \node[specification] (B)  at (0,1)  {};
   \node[specification] (C)  at (1,1)  {};
   \node[specification] (D) at (1,0)  {};
   \draw[thick] (B) to  (C);
   \draw[thick] (D) to  (A);
 \end{tikzpicture}\\
 D
\end{minipage}\hfill

\vspace*{0.5cm}
  \begin{minipage}{0.24\linewidth}
    \centering
    \begin{tikzpicture}[auto,
    specification/.style ={circle, draw, thick}]
   \node[specification] (A) at (0,0)  {};
   \node[specification] (B)  at (0,1)  {};
   \node[specification] (C)  at (1,1)  {};
   \node[specification] (D) at (1,0)  {};
   \draw[thick] (A) to (B);
   \draw[thick] (B) to (C);
      \draw[thick] (B) to (D);
 \end{tikzpicture}\\
 E
\end{minipage}\hfill
  \begin{minipage}{0.24\linewidth}
    \centering
    \begin{tikzpicture}[auto,
    specification/.style ={circle, draw, thick}]
   \node[specification] (A) at (0,0)  {};
   \node[specification] (B) at (0,1)  {};
   \node[specification] (C) at (1,1)  {};
   \node[specification] (D) at (1,0)  {};
   \draw[thick] (A) to  (B);
   \draw[thick] (B) to (C);
   \draw[thick] (C) to (A);
 \end{tikzpicture}\\
 F
\end{minipage}\hfill
  \begin{minipage}{0.24\linewidth}
    \centering
    \begin{tikzpicture}[auto,
    specification/.style ={circle, draw, thick}]
   \node[specification] (A) at (0,0)  {};
   \node[specification] (B) at (0,1)  {};
   \node[specification] (C) at (1,1)  {};
   \node[specification] (D) at (1,0)  {};
   \draw[thick] (A) to  (B);
   \draw[thick] (B) to  (C);
      \draw[thick] (C) to (D);
 \end{tikzpicture}\\
 G
\end{minipage}\hfill 
  \begin{minipage}{0.24\linewidth}
    \centering
    \begin{tikzpicture}[auto,
    specification/.style ={circle, draw, thick}]
   \node[specification] (A)  at (0,0)  {};
   \node[specification] (B)  at (0,1)  {};
   \node[specification] (C)  at (1,1)  {};
   \node[specification] (D) at (1,0)  {};
   \draw[thick] (A) to  (C);
   \draw[thick] (C) to  (D);
   \draw[thick] (D) to (A);
   \draw[thick] (B) to (D);
 \end{tikzpicture}\\
 H
\end{minipage}\hfill 

\vspace*{0.5cm}
\flushleft
  \begin{minipage}{0.24\linewidth}
    \centering
    \begin{tikzpicture}[auto,
    specification/.style ={circle, draw, thick}]
   \node[specification] (A) at (0,0)  {};
   \node[specification] (B)  at (0,1)  {};
   \node[specification] (C)  at (1,1)  {};
   \node[specification] (D) at (1,0)  {};
   \draw[thick] (A) to (B);
   \draw[thick] (B) to (D);
   \draw[thick] (D) to (C);
      \draw[thick] (C) to (A);
 \end{tikzpicture}\\
 I
\end{minipage}
  \begin{minipage}{0.24\linewidth}
    \centering
    \begin{tikzpicture}[auto,
    specification/.style ={circle, draw, thick}]
   \node[specification] (A) at (0,0)  {};
   \node[specification] (B) at (0,1)  {};
   \node[specification] (C) at (1,1)  {};
   \node[specification] (D) at (1,0)  {};
   \draw[thick] (A) to  (B);
      \draw[thick] (C) to (D);
   \draw[thick] (D) to (A);
   \draw[thick] (A) to (C);
   \draw[thick] (B) to (D);
 \end{tikzpicture}\\
 J
\end{minipage} 
  \begin{minipage}{0.24\linewidth}
    \centering
    \begin{tikzpicture}[auto,
    specification/.style ={circle, draw, thick}]
   \node[specification] (A) at (0,0)  {};
   \node[specification] (B) at (0,1)  {};
   \node[specification] (C) at (1,1)  {};
   \node[specification] (D) at (1,0)  {};
   \draw[thick] (A) to  (B);
   \draw[thick] (B) to  (C);
      \draw[thick] (C) to (D);
   \draw[thick] (D) to (A);
   \draw[thick] (A) to (C);
   \draw[thick] (B) to (D);
 \end{tikzpicture}\\
 K
\end{minipage}  

\end{frame}

\begin{frame}
  \frametitle{1. 顶点连通度和边连通度}

  \begin{Thm}
    对任一图$G$,有 $\kappa (G) \leq \lambda (G) \leq \delta (G)$。
  \end{Thm}\pause
    \centering
  \begin{minipage}{0.24\linewidth}
    \centering
    \begin{tikzpicture}[auto,
    specification/.style ={circle, draw, thick}]
   \node[specification] (A) at (0,0)  {};
   \node[specification] (B)  at (0,1)  {};
   \node[specification] (C)  at (1,1)  {};
   \node[specification] (D) at (1,0)  {};
 \end{tikzpicture}\\
 A
\end{minipage}\hfill 
  \begin{minipage}{0.24\linewidth}
    \centering
    \begin{tikzpicture}[auto,
    specification/.style ={circle, draw, thick}]
   \node[specification] (A) at (0,0)  {};
   \node[specification] (B) at (0,1)  {};
   \node[specification] (C) at (1,1)  {};
   \node[specification] (D) at (1,0)  {};
   \draw[thick] (B) to  (C);
 \end{tikzpicture}\\
 B
\end{minipage}\hfill 
  \begin{minipage}{0.24\linewidth}
    \centering
    \begin{tikzpicture}[auto,
    specification/.style ={circle, draw, thick}]
   \node[specification] (A) at (0,0)  {};
   \node[specification] (B) at (0,1)  {};
   \node[specification] (C) at (1,1)  {};
   \node[specification] (D) at (1,0)  {};
   \draw[thick] (A) to  (B);
   \draw[thick] (B) to  (C);
 \end{tikzpicture}\\
 C
\end{minipage}\hfill 
  \begin{minipage}{0.24\linewidth}
    \centering
    \begin{tikzpicture}[auto,
    specification/.style ={circle, draw, thick}]
   \node[specification] (A)  at (0,0)  {};
   \node[specification] (B)  at (0,1)  {};
   \node[specification] (C)  at (1,1)  {};
   \node[specification] (D) at (1,0)  {};
   \draw[thick] (B) to  (C);
   \draw[thick] (D) to  (A);
 \end{tikzpicture}\\
 D
\end{minipage}\hfill

\vspace*{0.5cm}
  \begin{minipage}{0.24\linewidth}
    \centering
    \begin{tikzpicture}[auto,
    specification/.style ={circle, draw, thick}]
   \node[specification] (A) at (0,0)  {};
   \node[specification] (B)  at (0,1)  {};
   \node[specification] (C)  at (1,1)  {};
   \node[specification] (D) at (1,0)  {};
   \draw[thick] (A) to (B);
   \draw[thick] (B) to (C);
      \draw[thick] (B) to (D);
 \end{tikzpicture}\\
 E
\end{minipage}\hfill
  \begin{minipage}{0.24\linewidth}
    \centering
    \begin{tikzpicture}[auto,
    specification/.style ={circle, draw, thick}]
   \node[specification] (A) at (0,0)  {};
   \node[specification] (B) at (0,1)  {};
   \node[specification] (C) at (1,1)  {};
   \node[specification] (D) at (1,0)  {};
   \draw[thick] (A) to  (B);
   \draw[thick] (B) to (C);
   \draw[thick] (C) to (A);
 \end{tikzpicture}\\
 F
\end{minipage}\hfill
  \begin{minipage}{0.24\linewidth}
    \centering
    \begin{tikzpicture}[auto,
    specification/.style ={circle, draw, thick}]
   \node[specification] (A) at (0,0)  {};
   \node[specification] (B) at (0,1)  {};
   \node[specification] (C) at (1,1)  {};
   \node[specification] (D) at (1,0)  {};
   \draw[thick] (A) to  (B);
   \draw[thick] (B) to  (C);
      \draw[thick] (C) to (D);
 \end{tikzpicture}\\
 G
\end{minipage}\hfill 
  \begin{minipage}{0.24\linewidth}
    \centering
    \begin{tikzpicture}[auto,
    specification/.style ={circle, draw, thick}]
   \node[specification] (A)  at (0,0)  {};
   \node[specification] (B)  at (0,1)  {};
   \node[specification] (C)  at (1,1)  {};
   \node[specification] (D) at (1,0)  {};
   \draw[thick] (A) to  (C);
   \draw[thick] (C) to  (D);
   \draw[thick] (D) to (A);
   \draw[thick] (B) to (D);
 \end{tikzpicture}\\
 H
\end{minipage}\hfill 

\vspace*{0.5cm}
\flushleft
  \begin{minipage}{0.24\linewidth}
    \centering
    \begin{tikzpicture}[auto,
    specification/.style ={circle, draw, thick}]
   \node[specification] (A) at (0,0)  {};
   \node[specification] (B)  at (0,1)  {};
   \node[specification] (C)  at (1,1)  {};
   \node[specification] (D) at (1,0)  {};
   \draw[thick] (A) to (B);
   \draw[thick] (B) to (D);
   \draw[thick] (D) to (C);
      \draw[thick] (C) to (A);
 \end{tikzpicture}\\
 I
\end{minipage}
  \begin{minipage}{0.24\linewidth}
    \centering
    \begin{tikzpicture}[auto,
    specification/.style ={circle, draw, thick}]
   \node[specification] (A) at (0,0)  {};
   \node[specification] (B) at (0,1)  {};
   \node[specification] (C) at (1,1)  {};
   \node[specification] (D) at (1,0)  {};
   \draw[thick] (A) to  (B);
      \draw[thick] (C) to (D);
   \draw[thick] (D) to (A);
   \draw[thick] (A) to (C);
   \draw[thick] (B) to (D);
 \end{tikzpicture}\\
 J
\end{minipage} 
  \begin{minipage}{0.24\linewidth}
    \centering
    \begin{tikzpicture}[auto,
    specification/.style ={circle, draw, thick}]
   \node[specification] (A) at (0,0)  {};
   \node[specification] (B) at (0,1)  {};
   \node[specification] (C) at (1,1)  {};
   \node[specification] (D) at (1,0)  {};
   \draw[thick] (A) to  (B);
   \draw[thick] (B) to  (C);
      \draw[thick] (C) to (D);
   \draw[thick] (D) to (A);
   \draw[thick] (A) to (C);
   \draw[thick] (B) to (D);
 \end{tikzpicture}\\
 K
\end{minipage}  
  
\end{frame}
\begin{frame}
  \frametitle{1. 顶点连通度和边连通度}

  \begin{Thm1.1}
    对任一图$G$,有 $\kappa (G) \leq \lambda (G) \leq \delta (G)$。
  \end{Thm1.1}\pause
  \begin{proof}[证明]\justifying\let\raggedright\justifying
          先证$\lambda (G) \leq \delta (G)$。\pause如果$\delta(G) = 0$,\pause则$G$不连通或者为平凡图,\pause此时$\lambda(G) = 0$,\pause $\lambda(G)\leq \delta(G)$成立。\pause如果$\delta(G)>0$,\pause不妨设$\deg v = \delta(G)$,\pause从$G$中去掉与$v$关联的$\delta(G)$条边之后,\pause得到的图中$v$为孤立顶点,\pause所以$\lambda(G) \leq \delta(G)$。\pause因此,\pause对任意的图$G$,$\lambda(G)\leq \delta(G)$。
  \end{proof}
\end{frame}

\begin{frame}
  \frametitle{1. 顶点连通度和边连通度}

  \begin{Thm1.1}
    对任一图$G$,有 $\kappa (G) \leq \lambda (G) \leq \delta (G)$。
  \end{Thm1.1}\pause
    \centering
  \begin{minipage}{0.24\linewidth}
    \centering
    \begin{tikzpicture}[auto,
    specification/.style ={circle, draw, thick}]
   \node[specification] (A) at (0,0)  {};
   \node[specification] (B)  at (0,1)  {};
   \node[specification] (C)  at (1,1)  {};
   \node[specification] (D) at (1,0)  {};
 \end{tikzpicture}\\
 A
\end{minipage}\hfill 
  \begin{minipage}{0.24\linewidth}
    \centering
    \begin{tikzpicture}[auto,
    specification/.style ={circle, draw, thick}]
   \node[specification] (A) at (0,0)  {};
   \node[specification] (B) at (0,1)  {};
   \node[specification] (C) at (1,1)  {};
   \node[specification] (D) at (1,0)  {};
   \draw[thick] (B) to  (C);
 \end{tikzpicture}\\
 B
\end{minipage}\hfill 
  \begin{minipage}{0.24\linewidth}
    \centering
    \begin{tikzpicture}[auto,
    specification/.style ={circle, draw, thick}]
   \node[specification] (A) at (0,0)  {};
   \node[specification] (B) at (0,1)  {};
   \node[specification] (C) at (1,1)  {};
   \node[specification] (D) at (1,0)  {};
   \draw[thick] (A) to  (B);
   \draw[thick] (B) to  (C);
 \end{tikzpicture}\\
 C
\end{minipage}\hfill 
  \begin{minipage}{0.24\linewidth}
    \centering
    \begin{tikzpicture}[auto,
    specification/.style ={circle, draw, thick}]
   \node[specification] (A)  at (0,0)  {};
   \node[specification] (B)  at (0,1)  {};
   \node[specification] (C)  at (1,1)  {};
   \node[specification] (D) at (1,0)  {};
   \draw[thick] (B) to  (C);
   \draw[thick] (D) to  (A);
 \end{tikzpicture}\\
 D
\end{minipage}\hfill

\vspace*{0.5cm}
  \begin{minipage}{0.24\linewidth}
    \centering
    \begin{tikzpicture}[auto,
    specification/.style ={circle, draw, thick}]
   \node[specification] (A) at (0,0)  {};
   \node[specification] (B)  at (0,1)  {};
   \node[specification] (C)  at (1,1)  {};
   \node[specification] (D) at (1,0)  {};
   \draw[thick] (A) to (B);
   \draw[thick] (B) to (C);
      \draw[thick] (B) to (D);
 \end{tikzpicture}\\
 E
\end{minipage}\hfill
  \begin{minipage}{0.24\linewidth}
    \centering
    \begin{tikzpicture}[auto,
    specification/.style ={circle, draw, thick}]
   \node[specification] (A) at (0,0)  {};
   \node[specification] (B) at (0,1)  {};
   \node[specification] (C) at (1,1)  {};
   \node[specification] (D) at (1,0)  {};
   \draw[thick] (A) to  (B);
   \draw[thick] (B) to (C);
   \draw[thick] (C) to (A);
 \end{tikzpicture}\\
 F
\end{minipage}\hfill
  \begin{minipage}{0.24\linewidth}
    \centering
    \begin{tikzpicture}[auto,
    specification/.style ={circle, draw, thick}]
   \node[specification] (A) at (0,0)  {};
   \node[specification] (B) at (0,1)  {};
   \node[specification] (C) at (1,1)  {};
   \node[specification] (D) at (1,0)  {};
   \draw[thick] (A) to  (B);
   \draw[thick] (B) to  (C);
      \draw[thick] (C) to (D);
 \end{tikzpicture}\\
 G
\end{minipage}\hfill 
  \begin{minipage}{0.24\linewidth}
    \centering
    \begin{tikzpicture}[auto,
    specification/.style ={circle, draw, thick}]
   \node[specification] (A)  at (0,0)  {};
   \node[specification] (B)  at (0,1)  {};
   \node[specification] (C)  at (1,1)  {};
   \node[specification] (D) at (1,0)  {};
   \draw[thick] (A) to  (C);
   \draw[thick] (C) to  (D);
   \draw[thick] (D) to (A);
   \draw[thick] (B) to (D);
 \end{tikzpicture}\\
 H
\end{minipage}\hfill 

\vspace*{0.5cm}
\flushleft
  \begin{minipage}{0.24\linewidth}
    \centering
    \begin{tikzpicture}[auto,
    specification/.style ={circle, draw, thick}]
   \node[specification] (A) at (0,0)  {};
   \node[specification] (B)  at (0,1)  {};
   \node[specification] (C)  at (1,1)  {};
   \node[specification] (D) at (1,0)  {};
   \draw[thick] (A) to (B);
   \draw[thick] (B) to (D);
   \draw[thick] (D) to (C);
      \draw[thick] (C) to (A);
 \end{tikzpicture}\\
 I
\end{minipage}
  \begin{minipage}{0.24\linewidth}
    \centering
    \begin{tikzpicture}[auto,
    specification/.style ={circle, draw, thick}]
   \node[specification] (A) at (0,0)  {};
   \node[specification] (B) at (0,1)  {};
   \node[specification] (C) at (1,1)  {};
   \node[specification] (D) at (1,0)  {};
   \draw[thick] (A) to  (B);
      \draw[thick] (C) to (D);
   \draw[thick] (D) to (A);
   \draw[thick] (A) to (C);
   \draw[thick] (B) to (D);
 \end{tikzpicture}\\
 J
\end{minipage} 
  \begin{minipage}{0.24\linewidth}
    \centering
    \begin{tikzpicture}[auto,
    specification/.style ={circle, draw, thick}]
   \node[specification] (A) at (0,0)  {};
   \node[specification] (B) at (0,1)  {};
   \node[specification] (C) at (1,1)  {};
   \node[specification] (D) at (1,0)  {};
   \draw[thick] (A) to  (B);
   \draw[thick] (B) to  (C);
      \draw[thick] (C) to (D);
   \draw[thick] (D) to (A);
   \draw[thick] (A) to (C);
   \draw[thick] (B) to (D);
 \end{tikzpicture}\\
 K
\end{minipage}  
  
\end{frame}

\begin{frame}
  \frametitle{1. 顶点连通度和边连通度}

  \begin{Thm1.1}
    对任一图$G$,有 $\kappa (G) \leq \lambda (G) \leq \delta (G)$。
  \end{Thm1.1}\pause
  \begin{proof}[证明]\justifying\let\raggedright\justifying
   \pause接下来证明$\kappa (G) \leq \lambda (G)$。\pause如果$G$不连通或者为平凡图,\pause则$\kappa(G)=\lambda(G)=0$。\pause如果$G$是连通的且有一座桥$x$,\pause则$\lambda(G)=1$。\pause因为在这种情况下$G$或者有一个割点关联于$x$或者$G$为$K_2$,\pause所以$\kappa(G)=1$。\pause最后假定$\lambda(G)\geq 2$,\pause则$G$中有$\lambda(G)$条边,\pause移去它们后所得到的图不连通。\pause显然,\pause移去这些边中的$\lambda(G)-1$条边后得到一个图,\pause它有一条桥$x=uv$。\pause对于这$\lambda(G)-1$条边中每一条,\pause选取一个关联于它但与$u$和$v$都不同的顶点。\pause移去这些顶点之后就移去了这$\lambda(G)-1$条边。\pause如果这样产生的图是不连通的,\pause则$\kappa(G) < \lambda(G)$。\pause否则,\pause$x$是这样产生的图的一条桥,\pause从而移去$u$或$v$就产生了一个不连通图或平凡图。\pause所以,\pause在任何情况下,\pause$\kappa(G) \leq \lambda(G)$。
  \end{proof}
\end{frame}


\begin{frame}
  \frametitle{1. 顶点连通度和边连通度}

  \begin{Thm}
    对任何整数$a,b,c$, $0 < a \leq b \leq c$, 存在一个图$G$使得\[\kappa (G)
      = a, \lambda (G) = b, \delta (G) = c\]
  \end{Thm}

  \pause
  \begin{center}
    \begin{tikzpicture}[auto,
      specification/.style ={circle, draw, thick, inner sep = 0pt, minimum size=2mm}]

      \draw (-2,0) circle (1cm);
      \draw (2,0) circle (1cm);
      \draw (-2,0) node {$K_{c+1}$};
      \draw (2,0) node {$K_{c+1}$};

     \draw (0.5,0.7) node {$\vdots$};

      \draw (0,-0.3) node {$\vdots$};
   \node[specification] (A)   at (-1.3, 0.7)  {};
   \node[specification] (B)  at (1.3, 0.7)  {};
   \node[specification] (D) at (1.1,0.3)  {};

   \node[specification] (C)  at (-1.1,0.0)  {};
   \node[specification] (G)  at (1.1,0.0)  {};
   
   \node[specification] (E)  at (-1.3,-0.7)  {};      
   \node[specification] (F)  at (1.3,-0.7)  {};      
   
   
   \draw[thick] (A) to  (B);
   
   \draw[thick] (A) to  (D);

   \draw[thick] (C) to  (G);
   
   \draw[thick] (E) to  (F);
   \draw (-0.2,0.9) node {$b-a+1$条边};

   \draw (-0.1,-0.9) node {$a-1$条边};
 \end{tikzpicture}
\end{center}

  \end{frame}
  \begin{frame}
  \frametitle{1. 顶点连通度和边连通度}

  \begin{Thm}
    设$G=(V,E)$有$p$个顶点且$\delta(G) \geq [ \frac{p}{2} ]$,则$\lambda(G) = \delta(G)$。
  \end{Thm}\small{\pause
  \begin{proof}[证明]\justifying\let\raggedright\justifying
    \pause $\lambda(G) \leq \delta(G)$显然成立,\pause 只需要证明$\lambda(G) \geq \delta(G)$。

     

   \pause 因为$\delta(G) \geq [\frac{p}{2}]$,所以$G$是连通的。\pause 如果$G$为平凡图,则$\lambda (G) = \delta(G) = 0$。如果$G$不是平凡图,则$\lambda(G) > 0$,从而存在$V$的真子集$A$使得$G$中联结$A$中的一个顶点与$V\setminus A$中的一个顶点的边恰有$\lambda(G)$条。\pause 所有这些边的集合记为$F$。

  \pause  由$|A| + |V\setminus A| = p$知必有$|A| \leq [\frac{p}{2}]$或者$|V\setminus A| \leq [\frac{p}{2}]$。\pause 不妨设$|A| \leq [\frac{p}{2}]$。 \pause 由于$\delta(G) \geq [\frac{p}{2}]$,$A$中的每个顶点至少与$V\setminus A$中的一个顶点邻接。\pause 否则,如果$A$中的某个顶点$u$只与$A$中的顶点邻接,则$\deg u \leq |A|-1 \leq [\frac{p}{2}] - 1 < \delta(G)$,矛盾。 

\pause    设$v$为$A$中的任意一个顶点, $v$与$V\setminus A$中的$x$个顶点邻接,与$A$中的$y$个顶点邻接,则$\deg v = x + y$。 \pause $v$与$V\setminus A$中的$x$个顶点邻接,所对应的边的集合记为$F_1$,则$F_1 \subseteq F$;
 \pause    $v$与$A$中的$y$个顶点邻接,而这$y$个顶点中的每个顶点都至少与$V\setminus A$中的一个顶点邻接,所对应的边的集合记为$F_2$,则$F_2 \subseteq F$ 并且$F_1 \cap F_2 = \phi$,\pause 从而
    \[\lambda(G) \geq |F_1| + |F_2| = x + y = \deg v \geq  \delta(G)\]
  \end{proof}}
\end{frame}


\begin{frame}
  \frametitle{1. 顶点连通度和边连通度}
  \begin{Def}
    设$G$为一个图,如果$\kappa (G) \geq n$,则称$G$为\alert{$n$-顶点连通}的,简称$n$-连通;如果$\lambda (G) \geq n$,则称$G$为\alert{$n$-边连通}的。
  \end{Def}\pause
    \centering
  \begin{minipage}{0.24\linewidth}
    \centering
    \begin{tikzpicture}[auto,
    specification/.style ={circle, draw, thick}]
   \node[specification] (A) at (0,0)  {};
   \node[specification] (B)  at (0,1)  {};
   \node[specification] (C)  at (1,1)  {};
   \node[specification] (D) at (1,0)  {};
 \end{tikzpicture}\\
 A
\end{minipage}\hfill 
  \begin{minipage}{0.24\linewidth}
    \centering
    \begin{tikzpicture}[auto,
    specification/.style ={circle, draw, thick}]
   \node[specification] (A) at (0,0)  {};
   \node[specification] (B) at (0,1)  {};
   \node[specification] (C) at (1,1)  {};
   \node[specification] (D) at (1,0)  {};
   \draw[thick] (B) to  (C);
 \end{tikzpicture}\\
 B
\end{minipage}\hfill 
  \begin{minipage}{0.24\linewidth}
    \centering
    \begin{tikzpicture}[auto,
    specification/.style ={circle, draw, thick}]
   \node[specification] (A) at (0,0)  {};
   \node[specification] (B) at (0,1)  {};
   \node[specification] (C) at (1,1)  {};
   \node[specification] (D) at (1,0)  {};
   \draw[thick] (A) to  (B);
   \draw[thick] (B) to  (C);
 \end{tikzpicture}\\
 C
\end{minipage}\hfill 
  \begin{minipage}{0.24\linewidth}
    \centering
    \begin{tikzpicture}[auto,
    specification/.style ={circle, draw, thick}]
   \node[specification] (A)  at (0,0)  {};
   \node[specification] (B)  at (0,1)  {};
   \node[specification] (C)  at (1,1)  {};
   \node[specification] (D) at (1,0)  {};
   \draw[thick] (B) to  (C);
   \draw[thick] (D) to  (A);
 \end{tikzpicture}\\
 D
\end{minipage}\hfill

\vspace*{0.5cm}
  \begin{minipage}{0.24\linewidth}
    \centering
    \begin{tikzpicture}[auto,
    specification/.style ={circle, draw, thick}]
   \node[specification] (A) at (0,0)  {};
   \node[specification] (B)  at (0,1)  {};
   \node[specification] (C)  at (1,1)  {};
   \node[specification] (D) at (1,0)  {};
   \draw[thick] (A) to (B);
   \draw[thick] (B) to (C);
      \draw[thick] (B) to (D);
 \end{tikzpicture}\\
 E
\end{minipage}\hfill
  \begin{minipage}{0.24\linewidth}
    \centering
    \begin{tikzpicture}[auto,
    specification/.style ={circle, draw, thick}]
   \node[specification] (A) at (0,0)  {};
   \node[specification] (B) at (0,1)  {};
   \node[specification] (C) at (1,1)  {};
   \node[specification] (D) at (1,0)  {};
   \draw[thick] (A) to  (B);
   \draw[thick] (B) to (C);
   \draw[thick] (C) to (A);
 \end{tikzpicture}\\
 F
\end{minipage}\hfill
  \begin{minipage}{0.24\linewidth}
    \centering
    \begin{tikzpicture}[auto,
    specification/.style ={circle, draw, thick}]
   \node[specification] (A) at (0,0)  {};
   \node[specification] (B) at (0,1)  {};
   \node[specification] (C) at (1,1)  {};
   \node[specification] (D) at (1,0)  {};
   \draw[thick] (A) to  (B);
   \draw[thick] (B) to  (C);
      \draw[thick] (C) to (D);
 \end{tikzpicture}\\
 G
\end{minipage}\hfill 
  \begin{minipage}{0.24\linewidth}
    \centering
    \begin{tikzpicture}[auto,
    specification/.style ={circle, draw, thick}]
   \node[specification] (A)  at (0,0)  {};
   \node[specification] (B)  at (0,1)  {};
   \node[specification] (C)  at (1,1)  {};
   \node[specification] (D) at (1,0)  {};
   \draw[thick] (A) to  (C);
   \draw[thick] (C) to  (D);
   \draw[thick] (D) to (A);
   \draw[thick] (B) to (D);
 \end{tikzpicture}\\
 H
\end{minipage}\hfill 

\vspace*{0.5cm}
\flushleft
  \begin{minipage}{0.24\linewidth}
    \centering
    \begin{tikzpicture}[auto,
    specification/.style ={circle, draw, thick}]
   \node[specification] (A) at (0,0)  {};
   \node[specification] (B)  at (0,1)  {};
   \node[specification] (C)  at (1,1)  {};
   \node[specification] (D) at (1,0)  {};
   \draw[thick] (A) to (B);
   \draw[thick] (B) to (D);
   \draw[thick] (D) to (C);
      \draw[thick] (C) to (A);
 \end{tikzpicture}\\
 I
\end{minipage}
  \begin{minipage}{0.24\linewidth}
    \centering
    \begin{tikzpicture}[auto,
    specification/.style ={circle, draw, thick}]
   \node[specification] (A) at (0,0)  {};
   \node[specification] (B) at (0,1)  {};
   \node[specification] (C) at (1,1)  {};
   \node[specification] (D) at (1,0)  {};
   \draw[thick] (A) to  (B);
      \draw[thick] (C) to (D);
   \draw[thick] (D) to (A);
   \draw[thick] (A) to (C);
   \draw[thick] (B) to (D);
 \end{tikzpicture}\\
 J
\end{minipage} 
  \begin{minipage}{0.24\linewidth}
    \centering
    \begin{tikzpicture}[auto,
    specification/.style ={circle, draw, thick}]
   \node[specification] (A) at (0,0)  {};
   \node[specification] (B) at (0,1)  {};
   \node[specification] (C) at (1,1)  {};
   \node[specification] (D) at (1,0)  {};
   \draw[thick] (A) to  (B);
   \draw[thick] (B) to  (C);
      \draw[thick] (C) to (D);
   \draw[thick] (D) to (A);
   \draw[thick] (A) to (C);
   \draw[thick] (B) to (D);
 \end{tikzpicture}\\
 K
\end{minipage}  

\end{frame}

\begin{frame}[t]
  \frametitle{1. 顶点连通度和边连通度}
  \begin{Thm}
    设$G=(V,E)$为有$p$个顶点的图,$p \geq 3$,则$G$为2-连通的,当且仅当$G$的任意两个不同的顶点在同一个圈上。
  \end{Thm}
  \pause\begin{proof}[证明]
    \pause设$G$的任意两个不同的顶点在同一个圈上,\pause则$G$为没有割点的连通图,\pause所以$G$为2-连通的。
  \end{proof}
\end{frame}

\begin{frame}[t]
  \frametitle{1. 顶点连通度和边连通度}
  \begin{Thm1.4}
    设$G=(V,E)$为有$p$个顶点的图,$p \geq 3$,则$G$为2-连通的,当且仅当$G$的任意两个不同的顶点在同一个圈上。
  \end{Thm1.4}
  \begin{proof}[证明]\justifying\let\raggedright\justifying
    \pause设$G$为2-连通的,\pause$u$和$v$为$G$的两个不同的顶点,\pause以下施归纳于$u$与$v$之间的距离$d(u,v)$来证明$u$与$v$在同一个圈上。\pause当$d(u,v)=1$时,\pause由于$\kappa(G)\geq 2$,\pause所以$uv$不是桥,\pause于是$uv$必在某个圈上,\pause所以$u$与$v$在同一个圈上。


  \end{proof}
\end{frame}

\begin{frame}[t]
  \frametitle{1. 顶点连通度和边连通度}
  \begin{Thm1.4}
    设$G=(V,E)$为有$p$个顶点的图,$p \geq 3$,则$G$为2-连通的,当且仅当$G$的任意两个不同的顶点在同一个圈上。
  \end{Thm1.4}
  \begin{proof}[证明]
    \justifying\let\raggedright\justifying
\pause设对于$G$中的任意两个顶点$u$和$v$,\pause当$d(u,v)=k$时,\pause$u$与$v$必在同一个圈上。\pause以下证明对于$G$中的任意两个顶点$u$和$v$,\pause当$d(u,v)=k+1$时, \pause$u$与$v$必在同一个圈上。\pause由$d(u,v)=k+1$知$u$与$v$之间有一条长为$k+1$的路$P:uv_1v_2\cdots v_kv$。\pause显然$d(u,v_k)=k$。\pause由归纳假设,\pause$u$与$v_k$在同一个圈上,\pause于是,\pause$u$与$v_k$间有两条没有内部公共顶点(即除u与$v_k$外)的两条路$Q,W$。\pause由于$\kappa(G)\geq 2$,\pause所以$G$没有割点,\pause从而$G-v_k$为连通图。\pause于是,\pause$G-v_k$中存在从$u$到$v$的路$S$。\pause$u$为$Q,W,S$的公共顶点。\pause设$w$为$S$上从$u$到$v$且在$Q$或$W$上的最后一个顶点。\pause不妨设$w$在$Q$上,\pause则在$G$中存在包含$u$和$v$的圈:$Q$上的$u$与$w$间一段后接$S$上$w$与$v$间的那一段,然后是边$vv_k$,最后是$W$。
  \end{proof}
\end{frame}
\section{门格尔定理}

\begin{frame}
  \frametitle{2. 门格尔定理}
    \begin{Def}
    设$u$与$v$为图$G$中的两个不同的顶点。两条联结$u$与$v$的路,如果除了$u$与$v$外没有公共顶点,则称这两条路为联结$u$与$v$的\alert{不相交路};如果联结$u$与$v$的两条路上没有公共边,则称这两条路为联结$u$与$v$的\alert{边不相交路}。
  \end{Def}
  \begin{Thm}
    图$G$为$n-$连通的当且仅当每一对不同顶点间至少有$n$条不相交路。
  \end{Thm}
  \begin{Thm}
    图$G$为$n-$边连通的当且仅当$G$的任一对不同的顶点间至少有$n$条边不相交路。
  \end{Thm}
\end{frame}

\begin{frame}
\centering
  \begin{minipage}{0.24\linewidth}
    \centering
    \begin{tikzpicture}[auto,
    specification/.style ={circle, draw, thick}]
   \node[specification] (A) at (0,0)  {};
   \node[specification] (B)  at (0,1)  {};
   \node[specification] (C)  at (1,1)  {};
   \node[specification] (D) at (1,0)  {};
 \end{tikzpicture}\\
 \vspace*{0.1cm}
 A
\end{minipage}\hfill 
  \begin{minipage}{0.24\linewidth}
    \centering
    \begin{tikzpicture}[auto,
    specification/.style ={circle, draw, thick}]
   \node[specification] (A) at (0,0)  {};
   \node[specification] (B) at (0,1)  {};
   \node[specification] (C) at (1,1)  {};
   \node[specification] (D) at (1,0)  {};
   \draw[thick] (B) to  (C);
 \end{tikzpicture}\\
 \vspace*{0.1cm}
 B
\end{minipage}\hfill 
  \begin{minipage}{0.24\linewidth}
    \centering
    \begin{tikzpicture}[auto,
    specification/.style ={circle, draw, thick}]
   \node[specification] (A) at (0,0)  {};
   \node[specification] (B) at (0,1)  {};
   \node[specification] (C) at (1,1)  {};
   \node[specification] (D) at (1,0)  {};
   \draw[thick] (A) to  (B);
   \draw[thick] (B) to  (C);
 \end{tikzpicture}\\
 \vspace*{0.1cm}
 C
\end{minipage}\hfill 
  \begin{minipage}{0.24\linewidth}
    \centering
    \begin{tikzpicture}[auto,
    specification/.style ={circle, draw, thick}]
   \node[specification] (A)  at (0,0)  {};
   \node[specification] (B)  at (0,1)  {};
   \node[specification] (C)  at (1,1)  {};
   \node[specification] (D) at (1,0)  {};
   \draw[thick] (B) to  (C);
   \draw[thick] (D) to  (A);
 \end{tikzpicture}\\
 \vspace*{0.1cm}
 D
\end{minipage}\hfill

\vspace*{0.5cm}
  \begin{minipage}{0.24\linewidth}
    \centering
    \begin{tikzpicture}[auto,
    specification/.style ={circle, draw, thick}]
   \node[specification] (A) at (0,0)  {};
   \node[specification] (B)  at (0,1)  {};
   \node[specification] (C)  at (1,1)  {};
   \node[specification] (D) at (1,0)  {};
   \draw[thick] (A) to (B);
   \draw[thick] (B) to (C);
      \draw[thick] (B) to (D);
 \end{tikzpicture}\\
 \vspace*{0.1cm}
 E
\end{minipage}\hfill
  \begin{minipage}{0.24\linewidth}
    \centering
    \begin{tikzpicture}[auto,
    specification/.style ={circle, draw, thick}]
   \node[specification] (A) at (0,0)  {};
   \node[specification] (B) at (0,1)  {};
   \node[specification] (C) at (1,1)  {};
   \node[specification] (D) at (1,0)  {};
   \draw[thick] (A) to  (B);
   \draw[thick] (B) to (C);
   \draw[thick] (C) to (A);
 \end{tikzpicture}\\
 \vspace*{0.1cm}
 F
\end{minipage}\hfill
  \begin{minipage}{0.24\linewidth}
    \centering
    \begin{tikzpicture}[auto,
    specification/.style ={circle, draw, thick}]
   \node[specification] (A) at (0,0)  {};
   \node[specification] (B) at (0,1)  {};
   \node[specification] (C) at (1,1)  {};
   \node[specification] (D) at (1,0)  {};
   \draw[thick] (A) to  (B);
   \draw[thick] (B) to  (C);
      \draw[thick] (C) to (D);
 \end{tikzpicture}\\
 \vspace*{0.1cm}
 G
\end{minipage}\hfill 
  \begin{minipage}{0.24\linewidth}
    \centering
    \begin{tikzpicture}[auto,
    specification/.style ={circle, draw, thick}]
   \node[specification] (A)  at (0,0)  {};
   \node[specification] (B)  at (0,1)  {};
   \node[specification] (C)  at (1,1)  {};
   \node[specification] (D) at (1,0)  {};
   \draw[thick] (A) to  (C);
   \draw[thick] (C) to  (D);
   \draw[thick] (D) to (A);
   \draw[thick] (B) to (D);
 \end{tikzpicture}\\
 \vspace*{0.1cm}
 H
\end{minipage}\hfill 

\vspace*{0.5cm}
\flushleft
  \begin{minipage}{0.24\linewidth}
    \centering
    \begin{tikzpicture}[auto,
    specification/.style ={circle, draw, thick}]
   \node[specification] (A) at (0,0)  {};
   \node[specification] (B)  at (0,1)  {};
   \node[specification] (C)  at (1,1)  {};
   \node[specification] (D) at (1,0)  {};
   \draw[thick] (A) to (B);
   \draw[thick] (B) to (D);
   \draw[thick] (D) to (C);
      \draw[thick] (C) to (A);
 \end{tikzpicture}\\
 \vspace*{0.1cm}
 I
\end{minipage}
  \begin{minipage}{0.24\linewidth}
    \centering
    \begin{tikzpicture}[auto,
    specification/.style ={circle, draw, thick}]
   \node[specification] (A) at (0,0)  {};
   \node[specification] (B) at (0,1)  {};
   \node[specification] (C) at (1,1)  {};
   \node[specification] (D) at (1,0)  {};
   \draw[thick] (A) to  (B);
      \draw[thick] (C) to (D);
   \draw[thick] (D) to (A);
   \draw[thick] (A) to (C);
   \draw[thick] (B) to (D);
 \end{tikzpicture}\\
 \vspace*{0.1cm}
 J
\end{minipage} 
  \begin{minipage}{0.24\linewidth}
    \centering
    \begin{tikzpicture}[auto,
    specification/.style ={circle, draw, thick}]
   \node[specification] (A) at (0,0)  {};
   \node[specification] (B) at (0,1)  {};
   \node[specification] (C) at (1,1)  {};
   \node[specification] (D) at (1,0)  {};
   \draw[thick] (A) to  (B);
   \draw[thick] (B) to  (C);
      \draw[thick] (C) to (D);
   \draw[thick] (D) to (A);
   \draw[thick] (A) to (C);
   \draw[thick] (B) to (D);
 \end{tikzpicture}\\
 \vspace*{0.1cm}
 K
\end{minipage}  
  
\end{frame}


\section{匹配、霍尔定理}
\begin{frame}
  \begin{Exercise}
    设$G$为图。证明:若$\delta(G)\geq 2$,则$G$包含长度至少为$\delta(G)+1$的圈。  
    \end{Exercise}
    \begin{proof}[证明]\justifying\let\raggedright\justifying
      \pause设$P=v_0v_1\ldots v_n$为$G$中的一条最长路,\pause则$v_0$只能与$P$中的顶点相邻接,\pause否则假设$v_0$与不在$P$中的顶点$u$邻接,\pause则$uv_0v_1\ldots v_n$构成了$G$中一条更长的路,\pause与$P$为$G$中的最长路矛盾。\pause取最大的$s$使得$v_0$与$v_s$相邻接,\pause则$C=v_0v_1\ldots v_sv_0$为长度至少为$\delta(G)+1$的圈,\pause这是因为$v_0$至少与$\delta(G)$个顶点相邻接,\pause而所有这些与$v_0$邻接的顶点均在圈$C$中。
    \end{proof}
\end{frame}
\begin{frame}
  \begin{Exercise}
    设$T$为一棵包含$k+1$个顶点的树。证明:如果图$G$的最小度$\delta(G)\geq k$,则$G$有一个同构于$T$的子图。
  \end{Exercise}
  \begin{proof}[证明]\justifying\let\raggedright\justifying
    \pause用数学归纳法证明,\pause施归纳于$k$。
  
  \pause(1)当$k=0$时,\pause$T$是一棵包含$1$个顶点的树,\pause在$G$中取任意一个顶点$u$,\pause该顶点自身为$G$的一个与$T$同构的子图。
  
    \pause(2)假设当$k=n$时结论成立,\pause往证当$k=n+1$时结论也成立。\pause设$T$是一棵$n+1+1$个顶点的树,\pause去掉一个叶子顶点$v$,\pause得到一棵树$T'$,\pause则$T'$是一棵有$n+1$个顶点的树。\pause图$G$的最小度$\delta(G)\geq n+1\geq n$,\pause由归纳假设,\pause$G$中存在一个同构于$T'$的子图$G'$。\pause设在$T$中与其叶子顶点$v$邻接的顶点为$u$,\pause在$T'$与$G'$的同构中,\pause与$u$对应的顶点为$u'$。\pause在$G$中,\pause$\deg u'\geq n+1$,\pause由于$G'$中有$n+1$个顶点,\pause$u'$在$G'$中至多有$n$条与之关联的边,\pause因此$u'$与$G$中除去$G'$中的顶点之外的其他某个顶点$v'$邻接,\pause在$G'$中添加顶点$v'$和边$u'v'$,\pause则得到一个与$T$同构的子图。
  \end{proof}
\end{frame}
\begin{frame}
    \frametitle{3. 匹配}\pause\centering
    \begin{minipage}{0.24\linewidth}
    \centering
    \begin{tikzpicture}[auto,
    specification/.style ={circle, draw, thick}]
   \node[specification] (A) [label=180:$v_1$] at (0,0)  {};
   \node[specification] (B) [label=180:$v_2$] at (0,1)  {};
   \node[specification] (C) [label=0:$v_3$] at (1,1)  {};
   \node[specification] (D) [label=0:$v_4$] at (1,0)  {};
   \draw[thick] (A) to  (B);
   \draw[thick] (B) to  (C);
 \end{tikzpicture}\\
C
\end{minipage}\pause
  \begin{minipage}{0.24\linewidth}
    \centering
    \begin{tikzpicture}[auto,
    specification/.style ={circle, draw, thick}]
   \node[specification] (A)  [label=180:$v_1$] at (0,0)  {};
   \node[specification] (B) [label=180:$v_2$] at (0,1)  {};
   \node[specification] (C) [label=0:$v_3$] at (1,1)  {};
   \node[specification] (D) [label=0:$v_4$] at (1,0)  {};
   \draw[thick] (A) to  (C);
   \draw[thick] (C) to  (D);
   \draw[thick] (D) to (A);
   \draw[thick] (B) to (D);
 \end{tikzpicture}\\
 H
\end{minipage}

\end{frame}
\begin{frame}
    \frametitle{3. 匹配}\centering
    \begin{minipage}{0.24\linewidth}
    \centering
    \begin{tikzpicture}[auto,
    specification/.style ={circle, draw, thick}]
   \node[specification] (A) [label=180:$v_1$] at (0,0)  {};
   \node[specification] (B) [label=180:$v_2$] at (0,1)  {};
   \node[specification] (C) [label=0:$v_3$] at (1,1)  {};
   \node[specification] (D) [label=0:$v_4$] at (1,0)  {};
   \draw[thick] (A) to  (B);
   \draw[thick] (B) to  (C);
 \end{tikzpicture}\\
C
\end{minipage}
  \begin{minipage}{0.24\linewidth}
    \centering
    \begin{tikzpicture}[auto,
    specification/.style ={circle, draw, thick}]
   \node[specification] (A) [label=180:$v_1$] at (0,0)  {};
   \node[specification] (B) [label=180:$v_2$] at (0,1)  {};
   \node[specification] (C) [label=0:$v_3$] at (1,1)  {};
   \node[specification] (D) [label=0:$v_4$] at (1,0)  {};
   \draw[thick,red] (A) to  (C);
   \draw[thick] (C) to  (D);
   \draw[thick] (D) to (A);
   \draw[thick,red] (B) to (D);
 \end{tikzpicture}\\
 H
\end{minipage}

\end{frame}

\begin{frame}
  \frametitle{3. 匹配}

  \begin{Def}\justifying\let\raggedright\justifying
    设$G=(V,E)$为一个图,$G$的任意两条不邻接的边$x$与$y$称为互相\alert{独立}的边。
  \end{Def}\pause
    \centering
  \begin{minipage}{0.24\linewidth}
    \centering
    \begin{tikzpicture}[auto,
    specification/.style ={circle, draw, thick}]
   \node[specification] (A) [label=180:$v_1$] at (0,0)  {};
   \node[specification] (B) [label=180:$v_2$] at (0,1)  {};
   \node[specification] (C) [label=0:$v_3$] at (1,1)  {};
   \node[specification] (D) [label=0:$v_4$] at (1,0)  {};
   \draw[thick,red] (A) to  (C);
   \draw[thick] (C) to  (D);
   \draw[thick] (D) to (A);
   \draw[thick,red] (B) to (D);
 \end{tikzpicture}\\
 H
\end{minipage} \pause
  \begin{minipage}{0.24\linewidth}
    \centering
    \begin{tikzpicture}[auto,
    specification/.style ={circle, draw, thick}]
   \node[specification] (A) [label=180:$v_1$] at (0,0)  {};
   \node[specification] (B) [label=180:$v_2$] at (0,1)  {};
   \node[specification] (C) [label=0:$v_3$] at (1,1)  {};
   \node[specification] (D) [label=0:$v_4$] at (1,0)  {};
   \draw[thick,red] (A) to  (C);
   \draw[thick,red] (C) to  (D);
   \draw[thick] (D) to (A);
   \draw[thick] (B) to (D);
 \end{tikzpicture}\\
 H
\end{minipage}


\end{frame}

\begin{frame}
  \frametitle{3. 匹配}

  \begin{Def}\justifying\let\raggedright\justifying
    图$G$的边集$E$的子集$Y$称为$G$的一个\alert{匹配},如果$Y$中任意两条不同的边都是互相独立的。
  \end{Def}
\pause
      \centering
  \begin{minipage}{0.24\linewidth}
    \centering
    \begin{tikzpicture}[auto,
    specification/.style ={circle, draw, thick}]
   \node[specification] (A) [label=180:$v_1$] at (0,0)  {};
   \node[specification] (B) [label=180:$v_2$] at (0,1)  {};
   \node[specification] (C) [label=0:$v_3$] at (1,1)  {};
   \node[specification] (D) [label=0:$v_4$] at (1,0)  {};
   \draw[thick,red] (A) to  (C);
   \draw[thick] (C) to  (D);
   \draw[thick] (D) to (A);
   \draw[thick,red] (B) to (D);
 \end{tikzpicture}\\
 H
\end{minipage} \pause
  \begin{minipage}{0.24\linewidth}
    \centering
    \begin{tikzpicture}[auto,
    specification/.style ={circle, draw, thick}]
   \node[specification] (A) [label=180:$v_1$] at (0,0)  {};
   \node[specification] (B) [label=180:$v_2$] at (0,1)  {};
   \node[specification] (C) [label=0:$v_3$] at (1,1)  {};
   \node[specification] (D) [label=0:$v_4$] at (1,0)  {};
   \draw[thick,red] (A) to  (C);
   \draw[thick,red] (C) to  (D);
   \draw[thick] (D) to (A);
   \draw[thick] (B) to (D);
 \end{tikzpicture}\\
 H
\end{minipage}\pause
  \begin{minipage}{0.24\linewidth}
    \centering
    \begin{tikzpicture}[auto,
    specification/.style ={circle, draw, thick}]
   \node[specification] (A) [label=180:$v_1$] at (0,0)  {};
   \node[specification] (B) [label=180:$v_2$] at (0,1)  {};
   \node[specification] (C) [label=0:$v_3$] at (1,1)  {};
   \node[specification] (D) [label=0:$v_4$] at (1,0)  {};
   \draw[thick] (A) to  (C);
   \draw[thick,red] (C) to  (D);
   \draw[thick] (D) to (A);
   \draw[thick] (B) to (D);
 \end{tikzpicture}\\
 H
\end{minipage}

\end{frame}



\begin{frame}
  \frametitle{3. 匹配}
  \begin{Def}\justifying\let\raggedright\justifying
    设$Y$为图$G=(V,E)$的一个匹配,如果$2|Y|=|V|$,则称$Y$为$G$的一个\alert{完美匹配}。
  \end{Def}
\pause
      \centering
  \begin{minipage}{0.24\linewidth}
    \centering
    \begin{tikzpicture}[auto,
    specification/.style ={circle, draw, thick}]
   \node[specification] (A) [label=180:$v_1$]  at (0,0)  {};
   \node[specification] (B) [label=180:$v_2$] at (0,1)  {};
   \node[specification] (C) [label=0:$v_3$] at (1,1)  {};
   \node[specification] (D) [label=0:$v_4$] at (1,0)  {};
   \draw[thick,red] (A) to  (C);
   \draw[thick] (C) to  (D);
   \draw[thick] (D) to (A);
   \draw[thick,red] (B) to (D);
 \end{tikzpicture}\\
 H
\end{minipage}
\end{frame}

\begin{frame}
  \frametitle{3. 匹配}
  \begin{Def}\justifying\let\raggedright\justifying
    设$Y$为图$G=(V,E)$的一个匹配,如果$2|Y|=|V|$,则称$Y$为$G$的一个\alert{完美匹配}。
  \end{Def}
  \pause
    \begin{minipage}{0.24\linewidth}
    \centering
    \begin{tikzpicture}[auto,
    specification/.style ={circle, draw, thick}]
   \node[specification] (A) at (0,0)  {};
   \node[specification] (B)  at (0,1)  {};
   \node[specification] (C)  at (1,1)  {};
   \node[specification] (D) at (1,0)  {};
 \end{tikzpicture}\\
 A
\end{minipage}\hfill 
  \begin{minipage}{0.24\linewidth}
    \centering
    \begin{tikzpicture}[auto,
    specification/.style ={circle, draw, thick}]
   \node[specification] (A) at (0,0)  {};
   \node[specification] (B) at (0,1)  {};
   \node[specification] (C) at (1,1)  {};
   \node[specification] (D) at (1,0)  {};
   \draw[thick] (B) to  (C);
 \end{tikzpicture}\\
 B
\end{minipage}\hfill 
  \begin{minipage}{0.24\linewidth}
    \centering
    \begin{tikzpicture}[auto,
    specification/.style ={circle, draw, thick}]
   \node[specification] (A) at (0,0)  {};
   \node[specification] (B) at (0,1)  {};
   \node[specification] (C) at (1,1)  {};
   \node[specification] (D) at (1,0)  {};
   \draw[thick] (A) to  (B);
   \draw[thick] (B) to  (C);
 \end{tikzpicture}\\
 C
\end{minipage}\hfill 
  \begin{minipage}{0.24\linewidth}
    \centering
    \begin{tikzpicture}[auto,
    specification/.style ={circle, draw, thick}]
   \node[specification] (A)  at (0,0)  {};
   \node[specification] (B)  at (0,1)  {};
   \node[specification] (C)  at (1,1)  {};
   \node[specification] (D) at (1,0)  {};
   \draw[thick] (B) to  (C);
   \draw[thick] (D) to  (A);
 \end{tikzpicture}\\
 D
\end{minipage}\hfill

\vspace*{0.5cm}
  \begin{minipage}{0.24\linewidth}
    \centering
    \begin{tikzpicture}[auto,
    specification/.style ={circle, draw, thick}]
   \node[specification] (A) at (0,0)  {};
   \node[specification] (B)  at (0,1)  {};
   \node[specification] (C)  at (1,1)  {};
   \node[specification] (D) at (1,0)  {};
   \draw[thick] (A) to (B);
   \draw[thick] (B) to (C);
      \draw[thick] (B) to (D);
 \end{tikzpicture}\\
 E
\end{minipage}\hfill
  \begin{minipage}{0.24\linewidth}
    \centering
    \begin{tikzpicture}[auto,
    specification/.style ={circle, draw, thick}]
   \node[specification] (A) at (0,0)  {};
   \node[specification] (B) at (0,1)  {};
   \node[specification] (C) at (1,1)  {};
   \node[specification] (D) at (1,0)  {};
   \draw[thick] (A) to  (B);
   \draw[thick] (B) to (C);
   \draw[thick] (C) to (A);
 \end{tikzpicture}\\
 F
\end{minipage}\hfill
  \begin{minipage}{0.24\linewidth}
    \centering
    \begin{tikzpicture}[auto,
    specification/.style ={circle, draw, thick}]
   \node[specification] (A) at (0,0)  {};
   \node[specification] (B) at (0,1)  {};
   \node[specification] (C) at (1,1)  {};
   \node[specification] (D) at (1,0)  {};
   \draw[thick] (A) to  (B);
   \draw[thick] (B) to  (C);
      \draw[thick] (C) to (D);
 \end{tikzpicture}\\
 G
\end{minipage}\hfill 
  \begin{minipage}{0.24\linewidth}
    \centering
    \begin{tikzpicture}[auto,
    specification/.style ={circle, draw, thick}]
   \node[specification] (A)  at (0,0)  {};
   \node[specification] (B)  at (0,1)  {};
   \node[specification] (C)  at (1,1)  {};
   \node[specification] (D) at (1,0)  {};
   \draw[thick] (A) to  (C);
   \draw[thick] (C) to  (D);
   \draw[thick] (D) to (A);
   \draw[thick] (B) to (D);
 \end{tikzpicture}\\
 H
\end{minipage}\hfill 

\vspace*{0.5cm}
\flushleft
  \begin{minipage}{0.24\linewidth}
    \centering
    \begin{tikzpicture}[auto,
    specification/.style ={circle, draw, thick}]
   \node[specification] (A) at (0,0)  {};
   \node[specification] (B)  at (0,1)  {};
   \node[specification] (C)  at (1,1)  {};
   \node[specification] (D) at (1,0)  {};
   \draw[thick] (A) to (B);
   \draw[thick] (B) to (D);
   \draw[thick] (D) to (C);
      \draw[thick] (C) to (A);
 \end{tikzpicture}\\
 I
\end{minipage}
  \begin{minipage}{0.24\linewidth}
    \centering
    \begin{tikzpicture}[auto,
    specification/.style ={circle, draw, thick}]
   \node[specification] (A) at (0,0)  {};
   \node[specification] (B) at (0,1)  {};
   \node[specification] (C) at (1,1)  {};
   \node[specification] (D) at (1,0)  {};
   \draw[thick] (A) to  (B);
      \draw[thick] (C) to (D);
   \draw[thick] (D) to (A);
   \draw[thick] (A) to (C);
   \draw[thick] (B) to (D);
 \end{tikzpicture}\\
 J
\end{minipage} 
  \begin{minipage}{0.24\linewidth}
    \centering
    \begin{tikzpicture}[auto,
    specification/.style ={circle, draw, thick}]
   \node[specification] (A) at (0,0)  {};
   \node[specification] (B) at (0,1)  {};
   \node[specification] (C) at (1,1)  {};
   \node[specification] (D) at (1,0)  {};
   \draw[thick] (A) to  (B);
   \draw[thick] (B) to  (C);
      \draw[thick] (C) to (D);
   \draw[thick] (D) to (A);
   \draw[thick] (A) to (C);
   \draw[thick] (B) to (D);
 \end{tikzpicture}\\
 K
\end{minipage}  
\end{frame}



\begin{frame}
  \frametitle{3. 匹配}

    \begin{Def}\justifying\let\raggedright\justifying
   设$Y$为图$G=(V,E)$的一个匹配,如果对于$G$的任一匹配$Y'$,恒有$|Y'|\leq |Y|$, 则称$Y$为$G$的一个\alert{最大匹配}。
  \end{Def}
\pause
\centering
  \begin{minipage}{0.24\linewidth}
    \centering
    \begin{tikzpicture}[auto,
    specification/.style ={circle, draw, thick}]
   \node[specification] (A) at (0,0)  {};
   \node[specification] (B)  at (0,1)  {};
   \node[specification] (C)  at (1,1)  {};
   \node[specification] (D) at (1,0)  {};
 \end{tikzpicture}\\
 A
\end{minipage}\hfill 
  \begin{minipage}{0.24\linewidth}
    \centering
    \begin{tikzpicture}[auto,
    specification/.style ={circle, draw, thick}]
   \node[specification] (A) at (0,0)  {};
   \node[specification] (B) at (0,1)  {};
   \node[specification] (C) at (1,1)  {};
   \node[specification] (D) at (1,0)  {};
   \draw[thick] (B) to  (C);
 \end{tikzpicture}\\
 B
\end{minipage}\hfill 
  \begin{minipage}{0.24\linewidth}
    \centering
    \begin{tikzpicture}[auto,
    specification/.style ={circle, draw, thick}]
   \node[specification] (A) at (0,0)  {};
   \node[specification] (B) at (0,1)  {};
   \node[specification] (C) at (1,1)  {};
   \node[specification] (D) at (1,0)  {};
   \draw[thick] (A) to  (B);
   \draw[thick] (B) to  (C);
 \end{tikzpicture}\\
 C
\end{minipage}\hfill 
  \begin{minipage}{0.24\linewidth}
    \centering
    \begin{tikzpicture}[auto,
    specification/.style ={circle, draw, thick}]
   \node[specification] (A)  at (0,0)  {};
   \node[specification] (B)  at (0,1)  {};
   \node[specification] (C)  at (1,1)  {};
   \node[specification] (D) at (1,0)  {};
   \draw[thick] (B) to  (C);
   \draw[thick] (D) to  (A);
 \end{tikzpicture}\\
 D
\end{minipage}\hfill

\vspace*{0.5cm}
  \begin{minipage}{0.24\linewidth}
    \centering
    \begin{tikzpicture}[auto,
    specification/.style ={circle, draw, thick}]
   \node[specification] (A) at (0,0)  {};
   \node[specification] (B)  at (0,1)  {};
   \node[specification] (C)  at (1,1)  {};
   \node[specification] (D) at (1,0)  {};
   \draw[thick] (A) to (B);
   \draw[thick] (B) to (C);
      \draw[thick] (B) to (D);
 \end{tikzpicture}\\
 E
\end{minipage}\hfill
  \begin{minipage}{0.24\linewidth}
    \centering
    \begin{tikzpicture}[auto,
    specification/.style ={circle, draw, thick}]
   \node[specification] (A) at (0,0)  {};
   \node[specification] (B) at (0,1)  {};
   \node[specification] (C) at (1,1)  {};
   \node[specification] (D) at (1,0)  {};
   \draw[thick] (A) to  (B);
   \draw[thick] (B) to (C);
   \draw[thick] (C) to (A);
 \end{tikzpicture}\\
 F
\end{minipage}\hfill
  \begin{minipage}{0.24\linewidth}
    \centering
    \begin{tikzpicture}[auto,
    specification/.style ={circle, draw, thick}]
   \node[specification] (A) at (0,0)  {};
   \node[specification] (B) at (0,1)  {};
   \node[specification] (C) at (1,1)  {};
   \node[specification] (D) at (1,0)  {};
   \draw[thick] (A) to  (B);
   \draw[thick] (B) to  (C);
      \draw[thick] (C) to (D);
 \end{tikzpicture}\\
 G
\end{minipage}\hfill 
  \begin{minipage}{0.24\linewidth}
    \centering
    \begin{tikzpicture}[auto,
    specification/.style ={circle, draw, thick}]
   \node[specification] (A)  at (0,0)  {};
   \node[specification] (B)  at (0,1)  {};
   \node[specification] (C)  at (1,1)  {};
   \node[specification] (D) at (1,0)  {};
   \draw[thick] (A) to  (C);
   \draw[thick] (C) to  (D);
   \draw[thick] (D) to (A);
   \draw[thick] (B) to (D);
 \end{tikzpicture}\\
 H
\end{minipage}\hfill 

\vspace*{0.5cm}
\flushleft
  \begin{minipage}{0.24\linewidth}
    \centering
    \begin{tikzpicture}[auto,
    specification/.style ={circle, draw, thick}]
   \node[specification] (A) at (0,0)  {};
   \node[specification] (B)  at (0,1)  {};
   \node[specification] (C)  at (1,1)  {};
   \node[specification] (D) at (1,0)  {};
   \draw[thick] (A) to (B);
   \draw[thick] (B) to (D);
   \draw[thick] (D) to (C);
      \draw[thick] (C) to (A);
 \end{tikzpicture}\\
 I
\end{minipage}
  \begin{minipage}{0.24\linewidth}
    \centering
    \begin{tikzpicture}[auto,
    specification/.style ={circle, draw, thick}]
   \node[specification] (A) at (0,0)  {};
   \node[specification] (B) at (0,1)  {};
   \node[specification] (C) at (1,1)  {};
   \node[specification] (D) at (1,0)  {};
   \draw[thick] (A) to  (B);
      \draw[thick] (C) to (D);
   \draw[thick] (D) to (A);
   \draw[thick] (A) to (C);
   \draw[thick] (B) to (D);
 \end{tikzpicture}\\
 J
\end{minipage} 
  \begin{minipage}{0.24\linewidth}
    \centering
    \begin{tikzpicture}[auto,
    specification/.style ={circle, draw, thick}]
   \node[specification] (A) at (0,0)  {};
   \node[specification] (B) at (0,1)  {};
   \node[specification] (C) at (1,1)  {};
   \node[specification] (D) at (1,0)  {};
   \draw[thick] (A) to  (B);
   \draw[thick] (B) to  (C);
      \draw[thick] (C) to (D);
   \draw[thick] (D) to (A);
   \draw[thick] (A) to (C);
   \draw[thick] (B) to (D);
 \end{tikzpicture}\\
 K
\end{minipage}  


\end{frame}
\begin{frame}
  \frametitle{3. 匹配}

  \begin{Def}\justifying\let\raggedright\justifying
    设$G=(V,E)$为一个偶图且$V=V_1\cup V_2$,
    $\forall x \in
    E$,$x$为联结$V_1$的一个顶点与$V_2$的一个顶点的边。如果存在$G$的一个匹配$Y$使得$|Y|=min\{|V_1|,|V_2|\}$,则称$Y$是偶图$G$的一个\alert{完全匹配}。
  \end{Def}

\end{frame}


\begin{frame}
  \frametitle{匹配}
  \begin{Thm}\justifying\let\raggedright\justifying
    设$G=((V_1,V_2),E)$为偶图,存在$G$的一个完全匹配$M$且$|M| = |V_1|$的充分必要条件是对$V_1$的任意子集$A$, $|N(A)| \geq |A|$,其中\[N(A) = \{y\in V_2|\exists x \in A \{x,y\} \in E\}\]
  \end{Thm}

\end{frame}

\begin{frame}
  \frametitle{匹配}
  \begin{Def}\justifying\let\raggedright\justifying
  设$M$为图$G=(V,E)$的一个匹配,如果一条路$P$上的边在$M$与$E\setminus M$中交错出现,则称路$P$为图$G$中的一条\alert{M-交错路}。\pause进一步,如果$P$的两个端点都不与$M$中的边相关联,则称$P$为一条\alert{M-增广路}。
  \end{Def}
\end{frame}
\begin{frame}
  \frametitle{匹配}
  \begin{Thm3.1}\justifying\let\raggedright\justifying
    设$G=((V_1,V_2),E)$为偶图,存在$G$的一个完全匹配$M$且$|M| = |V_1|$的充分必要条件是对$V_1$的任意子集$A$, $|N(A)| \geq |A|$,其中\[N(A) = \{y\in V_2|\exists x \in A \{x,y\} \in E\}\]
  \end{Thm3.1}\pause
  \begin{proof}[证明]
\justifying\let\raggedright\justifying
\pause设$G=((V_1,V_2),E)$为偶图,\pause如果存在$G$的一个完全匹配$Y$且$|Y| = |V_1|$,\pause则显然对$V_1$的任意子集$A$, \pause$|N(A)| \geq |A|$。
    \renewcommand{\qedsymbol}{}    
\end{proof}  
\end{frame}

% \begin{frame}
%   \frametitle{匹配}
%   \begin{Thm}
%     设$G=((V_1,V_2),E)$为偶图,存在$G$的一个完全匹配$M$且$|M| = |V_1|$的充分必要条件是对$V_1$的任意子集$A$, $|N(A)| \geq |A|$,其中\[N(A) = \{y\in V_2|\exists x \in A \{x,y\} \in E\}\]
%   \end{Thm}\pause
%     \begin{tikzpicture}[auto,
%     specification/.style ={circle, draw, thick}]
%    \node[specification] (A) at (0,1)  {};
%    \node[specification] (B) at (1,1)  {};
%    \node[specification] (C) at (2,1)  {};
%    \node[specification] (D) at (3,1)  {};
%    \node[specification] (E) at (4,1)  {};
%    \node[specification] (F) at (5,1)  {};
%    \node[specification] (G) at (6,1)  {};
%    \node[specification] (H) at (7,1)  {};
   
%    \node[specification] (I) at (0,0)  {};
%    \node[specification] (J) at (1,0)  {};
%    \node[specification] (K) at (2,0)  {};
%    \node[specification] (L) at (3,0)  {};
%    \node[specification] (M) at (4,0)  {};
%    \node[specification] (N) at (5,0)  {};
%    \node[specification] (O) at (6,0)  {};
%    \node[specification] (P) at (7,0)  {};
%    \node[specification] (Q) at (8,0)  {};

%    \draw[thick] (A) to  (B);
%    \draw[thick] (B) to  (C);
%       \draw[thick] (C) to (D);
%    \draw[thick] (D) to (A);
%    \draw[thick] (A) to (C);
%    \draw[thick] (B) to (D);
%  \end{tikzpicture}\\
    
% \end{frame}


\begin{frame}
  \begin{proof}[证明]\justifying\let\raggedright\justifying
    \pause设$G=((V_1,V_2),E)$为偶图,\pause对$V_1$的任意子集$A$, \pause$|N(A)| \geq |A|$。\pause用反证法证明$G$一定存在一个完全匹配。\pause假设$G$中不存在完全匹配。\pause设$M^*$为$G$的一个最大匹配,\pause则$M^*$不是$G$的完全匹配,\pause从而存在顶点$u\in V_1$,\pause$u$不与$M^*$中的任意一条边相关联。\pause设$Z$为所有可以从顶点$u$经由一条$M^*-$交错路到达的顶点构成的集合。\pause由$M^*$为一个最大匹配知$u$为$Z$中唯一没有与$M^*$中的边相关联的顶点。\pause记$R=V_1\cap Z$,\pause$B=V_2\cap Z$。\pause显然$f=\{(x,y)\in R\setminus \{u\}\times B | \{x,y\}\in M^*\}$为从集合$R\setminus \{u\}$到$B$的双射,\pause因此$|B|=|R|-1$。\pause以下证明$N(R)=B$。\pause显然$B\subseteq N(R)$。\pause由$N(R)$中的每个顶点都在从$u$出发的一条$M^*$交错路上知$N(R)\subseteq B$。\pause由$|B|=|R|-1$及$B= N(R)$知$|N(R)|=|R|-1$,\pause与已知条件矛盾。
  \end{proof}
\end{frame}

% \begin{frame}
%     \begin{codebox}
%     \Procname{$\proc{Augmenting-Path-Search}(G,M,u)$}
%     \zi \Comment $G$ is a bipartite graph statisfying the Hall condition,
%     \zi \Comment $M$ is a match in $G$,
%     \zi \Comment and $u$ is a vertex which is not incident to any edge in $M$
%     \li $V(T) \gets \{u\}$
%     \li $E(T) \gets \phi$
%     \li $R(T) \gets \{u\}$
%     \li \While there is an edge $xy$ whith $x\in R(T)$ and $y\in V(G)\setminus V(T)$
%     \li \Do
%         $V(T) \gets V(T)\cup \{y\}$
%     \li $E(T) \gets E(T)\cup \{xy\}$
%     \li \If y is incident with an edge $yz$ in $M$
%     \li \Do
%     $V(T) \gets V(T)\cap \{z\}$
%     \li $E(T) \gets E(T) \cup \{yz\}$
%     \li $R(T) \gets R(T) \cup \{z\}$
%     \Else 
%     \li $M \gets M \bigtriangleup E(P)$, where $P=uTy$
%     \li \Return M
%     \End
%     \End
%   \end{codebox}

% \end{frame}

% \begin{frame}
%   \frametitle{匹配}
%   \begin{proof}[证明]
% \justifying\let\raggedright\justifying
% 设$G=((V_1,V_2),E)$为偶图,如果存在$G$的一个完全匹配$Y$且$|Y| = |V_1|$,则显然对$V_1$的任意子集$A$, $|N(A)| \geq |A|$。

% 设$G=((V_1,V_2),E)$为偶图,对$V_1$的任意子集$A$, $|N(A)| \geq |A|$,以下用数学归纳法证明存在$G$的一个完全匹配$Y$使得$|Y| = |V_1|$,施归纳于$|V_1|$。

% (1)当$|V_1|=1$时,设$V_1$中唯一的一个元素为$u$,由$|N(V_1)| \geq |V_1|$知
% $N(V_1)$中至少含有一个元素$v$,则$\{\{u,v\}\}$构成了$G$的一个满足条件的完全匹配。

% (2)假设当$|V_1|<k$时结论成立,往证当$|V_1|=k$时结论也成立。
% 设$|V_1|=k$,分以下两种情况讨论:

% (i)对$V_1$的任意真子集$A$,$|N(A)| > |A| + 1$。取$V_1$中的任意一个元素$u$,由于$|N(\{u\})| \geq 1$, 可取
% $N(\{u\})$中的一个元素$v$使得$uv \in E$。 考虑偶图$G-\{u,v\}$,对任意的
% $V_1\setminus \{u\}$的子集$B$, $|N(B)| \geq |B|$。由归纳假设,偶图$G-\{u,v\}$
% 有一个完全匹配$Y'$且$|Y'| = |V_1\setminus \{u\}|$。$Y' \cup \{\{u,v\}\}$即为$G$的
% 一个完全匹配,且$|Y' \cup \{\{u,v\}\}| = |V_1|$。
%   \end{proof}
% \end{frame}
% \begin{frame}
%   \frametitle{匹配}
%   \begin{proof}[证明(续上页)]
% \justifying\let\raggedright\justifying
% (ii)存在$V_1$的真子集$A$,$|N(A)| = |A|$。

% 考虑图$G$中由$A \cup N(A)$导出的子图$G_1$以及由$(V_1\setminus A) \cup
% (N(V_1\setminus A)\setminus N(A))$导出的子图$G_2$。 $G_1$为偶图,且在$G_1$中对$A$的任意子集$B$,$|N(B)| \geq |B|$。
% $G_2$为偶图,且在$G_2$中对集合$V_1\setminus A$的任意子集$C$, $|N(C)| \geq |C|$, 这是因为如果
% $|N(C)| < |C|$, 则$|N(C \cup A)| < |C \cup A|$, 与前提条件矛盾。由归纳假设,
% $G_1$有完全匹配$M_1$,$|M_1|=|A|$,$G_2$有完全匹配$M_2$,$|M_2|=|V_1\setminus A|$。于是
% $M_1\cup M_2$构成了$G$的完全匹配,且$|M_1\cup M_2| = |V_1|$。

%   \end{proof}
% \end{frame}

\begin{frame}
  \frametitle{3. 匹配}

  \begin{Def}\justifying\let\raggedright\justifying
    设$X$为一个有穷集合,$(A_1,A_2,\ldots,A_n)$为$X$的子集的一个序列,由$X$的互不相同的元素构成的集合$\{s_1,s_2,\ldots,s_n\}$称为系统\[T:A_1,A_2,\ldots,A_n\]的\alert{相异代表系},如果$s_i\in A_i$,$i=1,2,\ldots,n$。
  \end{Def}

\end{frame}
\begin{frame}
  \begin{Exercise}\justifying\let\raggedright\justifying
    设集合$X=\{1,2,3,4,5,6\}$,则$\{1,2\},\{2,3\},\{3,4\},\{4,5\},\{5,6\},\{6,1\}$有()个相异代表系。

    A. 1

    B. 2

    C. 3

    D. 4
  \end{Exercise}
\end{frame}
\begin{frame}
  \frametitle{匹配}
    \begin{Thm}\justifying\let\raggedright\justifying
    设$X$为一个有限集,系统$T:A_1,A_2,\cdots,A_n$为$X$的一些子集组成的,则$T$有相异代表系的充分必要条件是$\forall I \subseteq \{1,2,\cdots, n\}$有
    \[|\bigcup_{i\in I}A_i|\geq |I|\]
  \end{Thm}
\end{frame}


\begin{frame}
  \begin{Exercise}
  设图$G$的顶点$u$与$v$之间有一条通道,那么$u$与$v$之间有一条路。
\end{Exercise}
\pause
\begin{proof}[证明]\justifying\let\raggedright\justifying
\pause用数学归纳法证明,\pause施归纳于顶点$u$与$v$之间通道的长$l$。

  \pause(1)当$l=0$时,\pause顶点$u$与$v$之间有一条长为$0$的通道,\pause此时$u=v$,\pause显然$u$与$v$之间有一条路。

  \pause (2)\pause假设当$l=k$时结论成立,\pause往证当$l=k+1$时结论也成立。\pause设顶点$u$与$v$之间有一条长为$k+1$的通道$uv_1v_2\cdots v_kv$,\pause则顶点$u$与顶点$v_k$之间有一条长为$k$的通道。\pause由归纳假设,\pause$u$与$v_k$之间有一条路$P:uu_1u_2\ldots u_sv_k$,\pause此时如果$v$不在路$P$中出现,\pause则路$P$之后接顶点$v$就构成了$u$与$v$之间的一条路$uu_1u_2\ldots u_sv_kv$;\pause如果$v$在路$P$中出现,\pause设$v$在路$P$中的第一次出现记为$u_i$,\pause那么路$P$中从$u$到$u_i$之间的路$uu_1u_2\ldots u_i$就是$u$与$v$之间的一条路。
\end{proof}

\end{frame}

\begin{frame}
  \begin{Exercise}
      设$u$与$v$为图$G$的两个不同的顶点。如果$u$与$v$间有两条不同的通道(迹),则$G$中是否有圈?
    \end{Exercise}
    \pause\begin{proof}[答]\justifying\let\raggedright\justifying
  \pause设$u$与$v$是图$G$的两个不同顶点。\pause如果$u$与$v$间有两条不同的通道,\pause则$G$中不一定有圈。\pause举例如下:\pause考虑$G=(\{u,v\},\{\{u,v\}\})$,\pause则$uv$和$uvuv$为$u$与$v$间两条不同的通道,\pause但$G$中没有圈。

  \pause如果$u$与$v$间有两条不同的迹,\pause则$G$中一定有圈。\pause证明如下:\pause设$u$与$v$间有两条不同的迹$T_1$和$T_2$。\pause如果$T_1$和$T_2$都为路,\pause则$G$中有圈;\pause如果$T_1=uv_1v_2\ldots v_nv$不是路,\pause设$v_j=v_i(i<j)$为第一个重复的顶点,\pause则$v_iv_{i+1}\ldots v_j$构成$G$中的一个圈;\pause同理,\pause如果$T_2$不是路,\pause$G$中有圈。
\end{proof}
\end{frame}

\begin{frame}  
  \frametitle{参考文献}

    \begin{thebibliography}{99}
  \bibitem[Gale, 1962]{Gale1962}D. Gale and L. S. Shapley.
\newblock College Admissions and the Stability of Marriage.
\newblock The American Mathematical Monthly,  1962.
  \end{thebibliography}

%  \cite{Gale:Marriage}
\end{frame}

\end{CJK}


%\bibliographystyle{acm}
%\bibliography{ref}

\end{document}

%%% Local Variables:
%%% mode: latex
%%% TeX-master: t
%%% End:
