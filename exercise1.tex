\documentclass{article}
\usepackage{CJKutf8}
\usepackage{amsmath}
\usepackage{amsthm}
\begin{document}
\begin{CJK}{UTF8}{gbsn}
\newtheorem{Exercise}{习题}
\begin{Exercise}
  写出方程$x^2+2x+1=0$的根所构成的集合。
\end{Exercise}

\begin{Exercise}
  设有$n$个集合$A_1$,$A_2$,$\ldots$,$A_n$,且$A_1\subseteq A_2\subseteq \ldots \subseteq A_n \subseteq A_1$,试证
  \[A_1 = A_2 = \ldots = A_n\]
\end{Exercise}


\begin{Exercise}
  设集合$S=\{\phi, \{\phi\}\}$,则$2^S=\underline{\quad\quad\quad}$。
\end{Exercise}
\begin{Exercise}
  设集合$S$有$n$个元素,证明$2^S$有$2^n$个元素。
\end{Exercise}

\begin{Exercise}
  设$A$,$B$为集合,试证
  \[(A\setminus B)\cup B = (A\cup B)\setminus B \Leftrightarrow B = \phi\]
\end{Exercise}
\begin{Exercise}
  设$A$,$B$为集合,试证$A=\phi \Leftrightarrow B= A\bigtriangleup B$。
\end{Exercise}
\begin{Exercise}
  设$A$,$B$为集合,证明$A\setminus (B\cup C) = A\setminus B \setminus C$。
\end{Exercise}
\begin{Exercise}
  设$A,B,C$为集合,证明$(A\cup B)\setminus C = (A\setminus C) \cup (B\setminus C)$。
\end{Exercise}

\begin{Exercise}
  设$A,B,C$为集合,证明$(A\cap B)\setminus C = (A\setminus C) \cap (B\setminus C)$。
\end{Exercise}
\begin{Exercise}
  设$A,B,C$都是集合,若$A\cup B = A\cup C$且$A\cap B = A\cap C$,试证$B=C$。
\end{Exercise}
\begin{Exercise}
  下列等式是否成立?如果成立,请给出证明;如果不成立,请说明理由。

  a) $(A\setminus B)\cup C = A\setminus (B\setminus C)$;

  b)$A\cup(B\setminus C) = (A\cup B)\setminus C$;

  c)$A\setminus (B\cup C) = (A\cup B)\setminus C$。
\end{Exercise}
\begin{Exercise}
  下列命题中哪个是真的?(    )

A. 对任意集合$A$,$B$,$2^{A\cup B} = 2^A \cup 2^B$。

B. 对任意集合$A$,$B$,$2^{A\cap B} = 2^A \cap 2^B$。

C. 对任意集合$A$,$B$,$2^{A\setminus B} = 2^A \setminus 2^B$。

D. 对任意集合$A$,$B$,$2^{A\bigtriangleup B} = 2^A \bigtriangleup 2^B$。

\end{Exercise}
\begin{Exercise}
  填空:设$A$,$B$为两个集合。

  a) $x\notin A\cup B\Leftrightarrow \underline{\quad\quad\quad\quad\quad\quad\quad\quad\quad}$

  b) $x\notin A\cap B\Leftrightarrow \underline{\quad\quad\quad\quad\quad\quad\quad\quad\quad}$

  c) $x\notin A\setminus B\Leftrightarrow \underline{\quad\quad\quad\quad\quad\quad\quad\quad\quad}$

  d) $x\notin A\bigtriangleup B\Leftrightarrow \underline{\quad\quad\quad\quad\quad\quad\quad\quad\quad}$
\end{Exercise}
\begin{Exercise}
  设$A$,$B$,$C$为任意三个集合,下列集合表达式中哪一个等于$A\setminus (B\cap C)$?(   )

  A. $(A\setminus B)\cap (A\setminus C)$

  B. $(A\setminus B)\cup (A\setminus C)$

  C. $(A\cap B)\setminus (A\cap C)$

  D. $(A\cup B)\setminus (A\cup C)$
\end{Exercise}


\begin{Exercise}
  设$A,B,C$为集合,并且$A\cup B=A\cup C$,则下列哪个等式成立?(  )

  A.$B=C$

  B.$A\cap B = A\cap C$

  C.$A\cap B^c = A\cap C^c$

  D.$A^c\cap B = A^c\cap C$
\end{Exercise}



\begin{Exercise}
  设$A,B,C$为集合,化简:

  \begin{equation*}
    \begin{split}
     & (A\cap B \cap C)\cup(A^c\cap B \cap C)\cup(A\cap B^c \cap C)\cup \\
     & (A\cap B \cap C^c)\cup(A^c\cap B^c \cap C)\cup(A\cap B^c \cap C^c)\cup \\
     & (A^c\cap B \cap C^c)
    \end{split}
  \end{equation*}
\end{Exercise}

\begin{Exercise}
  设$V$为一个集合,证明:$\forall S,T,W \in 2^V$有$S \subseteq T \subseteq W$当且仅当$S \bigtriangleup T \subseteq S \bigtriangleup W$且$S \subseteq W$。
\end{Exercise}

\begin{Exercise}
  设$A=\{a,b,c\}, B=\{e,f,g,h\}, C=\{x,y,z\}$。求$A\times B, B\times A, A\times C, A\times B \times C, A^2\times B$。
\end{Exercise}

\begin{Exercise}
  设$A,B$为集合,试证:$A\times B= B\times A$的充分必要条件是下列三个条件至少一个成立:

  (1) $A=\phi$;(2) $B=\phi$;3 $A=B$。
\end{Exercise}

\begin{Exercise}
  设$A$,$B$,$C$,$D$为任意四个集合,证明
  \[(A\cap B) \times (C \cap D) = (A\times C)\cap (B \times D)\]
\end{Exercise}

\begin{Exercise}
  设$A,B,C$为集合,证明:$A\times(B\bigtriangleup C) = (A\times B)\bigtriangleup(A\times C)$。
\end{Exercise}

\begin{Exercise}
  设$A$有$m$个元素,$B$有$n$个元素,则$A\times B$是多少个序对组成的?$A\times B$有多少个不同的元素?
\end{Exercise}

\begin{Exercise}
  设$A,B$为集合,$B\neq \phi$。试证:如果$A\times B= B\times B$,则$A=B$。
\end{Exercise}

\begin{Exercise}
  某班学生中有$45\%$正在学德文,$65\%$正在学法文,问此班中至少有百分之几的学生正在同时学德文和法文?
\end{Exercise}

\begin{Exercise}
  设$A,B$为两个有穷集合,则$|2^{2^{A\times B}}|=\underline{\quad\quad\quad}$。
\end{Exercise}

\begin{Exercise}
  毕业舞会上,小伙子与姑娘跳舞。已知每个小伙子至少与一个姑娘跳过舞,但未能与所有的姑娘跳过舞。同样的,每个姑娘也至少与一个小伙子跳过舞,但也未能与所有的小伙子跳过舞。证明:在所有参加舞会的小伙子与姑娘中,必可找到两个小伙子与两个姑娘,这两个小伙子中的每一个只与这两个姑娘中的一个跳过舞,而这两个姑娘中的每一个也只与这两个小伙子中的一个跳过舞。
\end{Exercise}
\end{CJK}
\end{document}


%%% Local Variables:
%%% mode: latex
%%% TeX-master: t
%%% End:
