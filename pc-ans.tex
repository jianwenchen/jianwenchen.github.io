\documentclass{article}
\usepackage{CJKutf8}
\usepackage{amsmath}
\usepackage{amssymb}
\usepackage{amsfonts}
\usepackage{amsthm}
\usepackage{titlesec}
\usepackage{titletoc}
\usepackage{xCJKnumb}
\usepackage{tikz}
\usepackage{mathrsfs}
\usepackage{indentfirst}

\newtheorem{Def}{定义}
\newtheorem{Thm}{定理}
\newtheorem{Exercise}{练习}

\newtheorem*{Example}{例}


\begin{document}
\begin{CJK*}{UTF8}{gbsn}
  \title{命题逻辑演算形式系统}
  \author{陈建文}
  \maketitle

  课后作业题

  \begin{Exercise}
    在$PC$中证明下列事实:
  \end{Exercise}

  $1.\vdash (A\to(A\to B))\to (A\to B)$

\begin{proof}[证明]$\quad$
  
  $(1)(A\to(A\to B))\to ((A\to A)\to (A\to B))$ 公理2

  $(2)((A\to(A\to B))\to ((A\to A)\to (A\to B)))\to ((A\to A) \to ((A\to (A\to B))\to (A\to B)))

  $(3)(A\to A) \to ((A\to (A\to B))\to (A\to B))(1)(2)r_{mp}$

  $(4)A\to A$

  $(5)(A\to(A\to B))\to (A\to B) (3)(4)r_{mp}$


\end{proof}

$9.((A\to B)\to A)\to A$

\begin{proof}[证明]$\squad$

  $(1)(\lnot(A\to B)\to A)\to((A\to A)\to (((A\to B)\to A)\to A))$

  $(2)((\lnot(A\to B)\to A)\to((A\to A)\to (((A\to B)\to A)\to A)))\to ((A\to A)\to ((\lnot(A\to B)\to A)\to (((A\to B)\to A)\to A)))$

  $(3)(A\to A)\to ((\lnot(A\to B)\to A)\to (((A\to B)\to A)\to A))(1)(2)r_{mp}$

  $(4)A\to A$

  $(5)(\lnot(A\to B)\to A)\to (((A\to B)\to A)\to A)(3)(4)r_{mp}$

  $(6)\lnot A\to (A\to B)$

  $(7)(\lnot A\to (A\to B))\to(\lnot(A\to B)\to A)$

  $(8)((A\to B)\to A)\to A(5)(7)r_{mp}$

\end{proof}

$10.((A\to B)\to C)\to ((C\to A)\to A)$

\begin{proof}[证明]$\squad$

  $(1)((A\to B)\to C)\to ((C\to A)\to ((A\to B)\to A))$

  $(2)((A\to B)\to A)\to A$

  $(3)(((A\to B)\to A)\to A)\to (((C\to A)\to ((A\to B)\to A))\to ((C\to A)\to A))$

$(4)((C\to A)\to ((A\to B)\to A))\to ((C\to A)\to A)(2)(3)r_{mp}$

$(5)(((A\to B)\to C)\to ((C\to A)\to ((A\to B)\to A)))\to ((((C\to A)\to ((A\to B)\to A))\to ((C\to A)\to A)) \to (((A\to B)\to C)\to ((C\to A)\to A)))$

$(6)(((C\to A)\to ((A\to B)\to A))\to ((C\to A)\to A)) \to (((A\to B)\to C)\to ((C\to A)\to A))(1)(5)r_{mp}$

$(7)((A\to B)\to C)\to ((C\to A)\to A)(4)(6)r_{mp}$
\end{proof}

$11.((A\to B)\to C)\to ((A\to C)\to C)$
\begin{proof}[证明]$\squad$

  $(1)(\lnot A\to C)\to ((C\to C)\to ((A\to C)\to C))$

  $(2)((\lnot A\to C)\to ((C\to C)\to ((A\to C)\to C)))\to ((C\to C)\to ((\lnot A\to C)\to ((A\to C)\to C)))$

  $(3)(C\to C)\to ((\lnot A\to C)\to ((A\to C)\to C))(1)(2)r_{mp}$

  $(4)C\to C$

  $(5)(\lnot A\to C)\to ((A\to C)\to C)(3)(4)r_{mp}$

  $(6)(\lnot A\to (A\to B))\to (((A\to B)\to C)\to (\lnot A\to C))$

  $(7)\lnot A\to (A\to B)$

  $(8)((A\to B)\to C)\to (\lnot A\to C)(6)(7)r_{mp}$

  $(9)(((A\to B)\to C)\to (\lnot A\to C))\to (((\lnot A\to C)\to ((A\to C)\to C))\to(((A\to B)\to C)\to ((A\to C)\to C)) )$

  $(10)((\lnot A\to C)\to ((A\to C)\to C))\to(((A\to B)\to C)\to ((A\to C)\to C))(8)(9)r_{mp}$

  $(11)((A\to B)\to C)\to ((A\to C)\to C)(5)(10)r_{mp}$

\end{proof}
\begin{Exercise}
  利用演绎定理在$PC$中证明:
\end{Exercise}

$1.\vdash ((A\to B)\to A)\to A$

\begin{proof}[证明]
  只需证$(A\to B)\to A\vdash A$。

  $(1)(A\to B)\to A$前提

  $(2)\lnot A\to (A\to B)$

  $(3)((A\to B)\to A)\to ((\lnot A\to (A\to B))\to (\lnot A\to A))$

  $(4)(\lnot A\to (A\to B))\to (\lnot A\to A)(1)(3)r_{mp}$

  $(5)\lnot A\to A(2)(4)r_{mp}$

  $(6)(\lnot A\to A)\to A$

  $(7)A (5)(6)r_{mp}$


\end{proof}
\end{CJK*}
\end{document}





%%% Local Variables:
%%% mode: latex
%%% TeX-master: t
%%% End:
