\documentclass{article}
\usepackage{CJKutf8}
\usepackage{amsmath}
\usepackage{amssymb}
\usepackage{amsfonts}
\usepackage{amsthm}
\usepackage{titlesec}
\usepackage{titletoc}
\usepackage{xCJKnumb}
\usepackage{tikz}
\usepackage{mathrsfs}
\usepackage{indentfirst}

\newtheorem{Def}{定义}
\newtheorem{Thm}{定理}
\newtheorem{Exercise}{练习}

\newtheorem*{Example}{例}


\begin{document}
\begin{CJK*}{UTF8}{gbsn}
  \title{第六讲 循环群}
  \author{陈建文}
  \maketitle
  % \tableofcontents
  
\begin{Def}
  群$G$称为循环群,如果$G$是由其中的某个元素$a$生成的,即$G=(a)=\{\cdots,a^{-2},a^{-1},e,a,a^2,\cdots\}$。
\end{Def}

\begin{Example}
  整数加法群$(Z,+)$为循环群,其生成元为$1$。
\end{Example}

\begin{Example}
  模$n$同余类加群$Z_n=\{[0],[1],\cdots,[n-1]\}$为一个阶为$n$的有限循环群,其生成元为$[1]$。
\end{Example}

\begin{Thm}
  (1)循环群$G=(a)$为无穷循环群的充分必要条件是$a$的阶为无穷大,此时$G=\{\cdots,a^{-2},a^{-1},e,a,a^2,\cdots\}$;
  
  (2)循环群$G=(a)$为$n$阶循环群的充分必要条件是$a$的阶为$n$,此时$G=\{e,a,a^2,\cdots,a^{n-1}\}$。$\forall m\in Z$,$a^m=a^{m\bmod n}$。
\end{Thm}
\begin{proof}[证明]
  $G=(a)=\{\cdots,a^{-2},a^{-1},e,a,a^2,\cdots\}$

  分两种情况讨论:

  (1)$a$的阶为无穷大

  以下证明$\forall i,j\in Z,i\neq j\to a^i\neq a^j$。

  用反证法。不妨设$j>i$。假设$a^i=a^j$,则$a^{j-i}=e$,与$a$的阶为无穷大矛盾。

  (2)$a$的阶为$n$

  要证$G=\{e,a,a^2,\cdots,a^{n-1}\}$。

  $\forall m\in Z$,$\exists i,0\leq i \leq n-1$,$a^m=a^i$

  $m=qn+r,0\leq r < n$,$a^m=a^{qn+r}=(a^n)^qa^r=a^r$,$\forall m\in Z$,$a^m=a^{m\bmod n}$。

  $\forall i,j$,$0\leq i\leq n-1$,$0\leq j \leq n-1$,$a^i\neq a^j$。

  用反证法。假设$a^i=a^j$,不妨设$j>i$,则$a^{j-i}=e$,$0<j-i\leq n-1$,这与$a$的阶为$n$矛盾。
\end{proof}
\begin{Thm}
  (1)无穷循环群同构于整数加群$(Z,+)$,即如果不计同构,无穷循环群只有一个,就是整数加群;
  
  (2)阶为$n$的有限循环群同构于模$n$同余类加群$(Z_n,+)$,即如果不计同构,$n$阶循环群只有一个,就是模$n$同余类加群。
\end{Thm}
\begin{proof}[证明]
  (1)设$G=(a)=\{\cdots,a^{-2},a^{-1},e,a,a^2,\cdots\}$

  $\phi:G\to Z$,$\forall i\in Z, \phi(a^i)=i$

  $\phi(a^i\circ a^j)=\phi(a^{i+j})=i+j=\phi(a^i)\circ\phi(a^j)$

  (2)设$G=(a)=\{e,a,a^2,\cdots,a^{n-1}\}$

  $\phi:G\to Z_n$,$\forall i\in Z, 0\leq i\leq n-1,\phi(a^i)=[i]$

  $\forall i,j, 0\leq i\leq n-1,0\leq j\leq n-1,\phi(a^i\circ a^j)=\phi(a^{i+j})=\phi(a^{(i+j)\bmod n})=[(i+j)\bmod n]=[i+j]=[i]+[j]=\phi(a^i)\circ \phi(a^j)$。

\end{proof}
\begin{Thm}
  设$G=(a)$为由$a$生成的循环群,则

  (1)循环群的子群仍为循环群;

  (2)如果$G$为无限循环群,则$H_0=\{e\},H_m=(a^m),m=1,2,\cdots$为$G$的所有子群,这里$H_m,m=1,2,\cdots$都同构于$G$;

  (3)如果$G$为$n$阶循环群,则$H_0=\{e\},H_m=(a^m)$,$m|n$,为$G$的所有子群。每个子群$H_m$的阶为$\frac{n}{m}$。
\end{Thm}
\begin{proof}[证明]
  (1)设$H$为循环群$G=(a)$的子群,令$m=\min \{i\in Z^+|a^i\in H\}$,以下证明$H=(a^{m})$。$\forall j\in Z$,如果$a^j\in H$,$j=qm+r$,$0\leq r < m$,
  则$a^j=a^{qm+r}=(a^m)^qa^r$,从而$a^r=a^j(a^{m})^{-q}\in H$,此时必有$r=0$,于是$a^j=(a^m)^q$,从而$H\subseteq (a^m)$。$(a^m)\subseteq H$显然成立,于是$H=(a^m)$。

  (2)显然$H_0$,$H_m,m=1,2,\cdots$都为$G$的子群。设$H$为$G$的任意一个子群,由(1)知$H$为循环群,从而$\exists m\in Z$,使得$H=(a^m)=(a^{-m})$。

  (3)由(2)知,$H_0=\{e\}$,$(a^k),k=1,2,\cdots$为$G$的所有子群。

  $\forall k,k=1,2,\cdots$,令$m=\min \{i\in Z^+|a^i\in (a^k)\}$。由(1)的证明过程知$(a^k)=(a^m)$,以下证明$m|n$。

  设$n=qm+r$,$0\leq r < m$,则$a^n=a^{qm+r}=(a^m)^qa^r$,由$a$的阶为$n$知$a^n=e$,从而$e=(a^m)^qa^r$,于是$a^r=(a^m)^{-q}\in (a^m)=(a^k)$,此时必有$r=0$,否则与$m$的定义矛盾,所以$m|n$。

  由$(a^m)^{\frac{n}{m}}=e$,当$0<k<\frac{n}{m}$时,$(a^m)^k\neq e$知$a^m$的阶为$\frac{n}{m}$,此时$(a^m)=\{e,a^m,a^{2m},\cdots,a^{\frac{n}{m}-1}m\}$,$|(a^m)|=\frac{n}{m}$。
\end{proof}
\begin{Example}
  设$a\in Z$,$b\in Z$,$a$和$b$不全为$0$,则在整数加群$(Z,+)$中集合$\{a,b\}$的生成子群为$H=\{ma+nb|m\in Z,n\in Z\}$。
  由于循环群$(Z,+)$的每个子群都为循环群,因此存在正整数$d$,使得$H=(d)$。这里$d$为$a$和$b$的最大公约数$(a,b)$。
  这是因为由$a\in H$知存在$p\in Z$,使得$a=pd$,即$d|a$;由$b\in H$知存在$q\in Z$,使得$b=qd$,即$d|b$;又因为$d\in H$,从而存在$m\in Z,n\in Z$,使得$d=ma+nb$,从而$\forall d'\in Z$,由$d'|a$并且$d'|b$,可以得到$d'|d$。
\end{Example}
\begin{Thm}
  设$a,b\in Z$,$a$和$b$不全为$0$,则$\exists m,n\in Z$使得$(a,b)=ma+nb$。
\end{Thm}
\begin{Thm}
  设$a,b\in Z$,$b>0$,$a=qb+r$,$0\leq r < b$,则$(a,b)=(b,r)$。
\end{Thm}
\begin{proof}[证明]
  设$A=\{x\in Z|x>0,x|a,x|b\}$,$B=\{x\in Z|x>0,x|b,x|r\}$,以下证明$A=B$,从而集合$A$中最大的数等于集合$B$中最大的数,
  即$(a,b)=(b,r)$。

  $\forall x\in A$,则$x>0$,$x|a$并且$x|b$,由$a=qb+r$知$x|r$,从而$x>0$,$x|b$且$x|r$,即$x\in B$;$\forall x\in B$,则$x>0$,$x|b$并且$x|r$,由$a=qb+r$知$x|a$,从而$x>0$,$x|a$且$x|b$,即$x\in A$。
\end{proof}
\begin{Example}
  计算$(266,112)$,并将其表示成$m\cdot 266 + n\cdot 112$的形式。
\end{Example}
\begin{proof}[解]
  由
  \begin{align*}
    266&=2*112+42\\
    112&=2*42+28\\
    42&=1*28+14\\
    28&=2*14+0\\
  \end{align*}
可得$(266,112)=14$。

由
\begin{align*}
  42&=266+(-2)*112\\
  28&=112+(-2)*42\\
  &=112+(-2)*(266+(-2)*112)\\
  &=(-2)*266+5*112\\
  14&=42+(-1)*28\\
  &=(266+(-2)*112)+(-1)*((-2)*266+5*112)\\
  &=3*266+(-7)*112
\end{align*}
得$(266,112)=3*266+(-7)*112$。
\end{proof}
课后作业题:
\begin{Exercise}
证明:$n$次单位根之集对数的通常乘法构成一个循环群。
\end{Exercise}

\begin{Exercise}
找出模$12$的同余类加群的所有子群。
\end{Exercise}

\begin{Exercise}
  设$G=(a)$为一个$n$阶循环群。证明:如果$(r,n)=1$,则$(a^r)=G$。
\end{Exercise}


\begin{Exercise}
  设群$G$中元素$a$的阶为$n$,$(r,n)=d$。证明:$a^r$的阶为$n/d$。
\end{Exercise}
\end{CJK*}
\end{document}





%%% Local Variables:
%%% mode: latex
%%% TeX-master: t
%%% End:



