\documentclass{article}
\usepackage{tikz}
\usepackage{CJKutf8}
\usepackage{amsmath}
\usepackage{amsthm}
\begin{document}
\begin{CJK}{UTF8}{gbsn}
\newtheorem{Exercise}{习题}
\begin{Exercise}
    设$G$为一个有$k$个支的平面图。如果$G$的顶点数、边数、面数分别为$p$,$q$和$f$,试证:
  \[p-q+f=k+1\]
\end{Exercise}
\begin{proof}[证明]
    设$G$的的$k$个支分别为$G_1$,$G_2$,$\ldots$,$G_k$,其中$G_i$有$p_i$个顶点,$q_i$条边,$f_i$个面($1\leq i \leq k$)。
  
    由欧拉公式知,
    \begin{align*}
      p_1-q_1+f_1&=2\\
      p_2-q_2+f_2&=2\\
      \cdots&\\
      p_k-q_k+f_k&=2
    \end{align*}
    以上各式相加得:
    \[(p_1+p_2+\ldots+p_k)-(q_1+q_2+\ldots+q_k)+(f_1+f_2+\ldots+f_k)=2k\]
  
  由$G$只有一个外部面知
  \[f_1 + f_2 + \ldots + f_k = f + (k-1)\]
  从而
  \[p-q+f+(k-1)=2k\]
  即
  \[p-q+f = k + 1\]
  \end{proof}  
\begin{Exercise}
    如果$G$为顶点数$p\geq 11$的可平面图,试证$G^c$不是可平面图。
\end{Exercise}
\begin{proof}[证明]
    用反证法,假设$G^c$也是可平面图。设$G$有$q$条边,由$G$为可平面图知
  \[q \leq 3p - 6\]
  设$G^c$有$q_1$条边,由$G^c$为有$p$个顶点的可平面图知
  \[q_1 \leq 3p - 6\]
  于是
  \[q + q _1 \leq 6p-12\]
  即
  \[\frac{p(p-1)}{2} \leq 6p-12\]
  \[p^2-p \leq 12p-24\]
  \[p^2-13p+24 \leq 0\]
  当$p\geq 11$时,
  \begin{equation*}
    \begin{split}
     &p^2-13p+24\\
     =&(p-\frac{13}{2})^2-\frac{169}{4}+24\\
     \geq&(11-\frac{13}{2})^2-\frac{169}{4} + 24\\
     =&\frac{81}{4} - \frac{169}{4} + 24\\
     =&2 > 0
    \end{split}
  \end{equation*}
  矛盾。
\end{proof}
\begin{Exercise}
    不存在$7$条棱的凸多面体。
\end{Exercise}
\begin{proof}[证明]
    用反证法,假设存在$7$条棱的凸多面体,其对应的平面图有$p$个顶点,则
    \[7 \leq 3p - 6\]
    于是
    \[p \geq \frac{13}{3}\]
    又由每个顶点的度大于等于$3$知
    \[3p \leq 2 * 7\]
    于是
    \[p \leq \frac{14}{3}\]
    由于不存在正整数$p$使得$\frac{13}{3} \leq p \leq \frac{14}{3}$
    结论得证。
  \end{proof}
\begin{Exercise}
    设$G$为一个没有三角形的可平面图。证明$G$中存在一个顶点$v$使得$\deg v \leq 3$。
\end{Exercise}
\begin{proof}[证明]
    当图$G$的顶点数$p=1,2$时,结论显然成立。
    当$p \geq 3$时,用反证法证明结论也成立。假设$\delta (G) \geq 4$,设$G$有$q$条边,则
    \[2q \geq 4p\]
    于是\[q \geq 2p\]
    由$G$为可平面图知
    \[q \leq 2p - 4\]
  矛盾。  
  \end{proof}
\begin{Exercise}
    设$G$为一个没有三角形的可平面图。应用数学归纳法证明$G$为4-可着色的。
\end{Exercise}
\begin{proof}[证明]用数学归纳法证明,施归纳于顶点数$p$。

    (1)当$p=1$时,结论显然成立。

    (2)假设当$p=k(k\geq 1)$时结论成立,往证当$p=k+1$时结论也成立。设$G$为包含$k+1$个顶点,没有三角形的可平面图知,$G$中存在一个顶点$v$,$\deg v \leq 3$。显然,$G-v$为包含$k$个顶点,没有三角形的可平面图,由归纳假设,$G-v$为$4$可着色的。假设已经用至多$4$种颜色对$G-v$进行了顶点着色,使得任意相邻的顶点着不同的颜色,那么此时在$G$中与$v$邻接的顶点用了至多$3$种颜色,用另外一种不同的颜色对顶点$v$进行着色,从而用至多$4$种颜色就可以对$G$的顶点进行着色使得相邻的顶点着不同的颜色,即$G$为$4$可着色的。
    
  \end{proof}

\end{CJK}
\end{document}


%%% Local Variables:
%%% mode: latex
%%% TeX-master: t
%%% End:
