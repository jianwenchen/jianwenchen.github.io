\documentclass{article}
\usepackage{tikz}
\usepackage{CJKutf8}
\usepackage{amsmath}
\usepackage{amsthm}
\usepackage{enumitem}
\newtheorem{Def}{定义}
\newtheorem{Exercise}{练习}
\newtheorem*{Thm}{定理}
\newtheorem*{Example}{例}
\setlist[enumerate,1]{label=(\arabic*)}
\begin{document}
\begin{CJK}{UTF8}{gbsn}
  \title{第一讲 命题与联结词}
  \author{陈建文}
  \maketitle

  课后作业题:
  \begin{Exercise}
    判定下列逻辑蕴含和逻辑等价是否成立,其中$A,B,C$为任意公式。
    \begin{enumerate}
      \item $A\Rightarrow  B\to A$
      \item $A\to (B\to C)\Rightarrow (A\to B)\to (A\to C)$
      \item $\lnot A\to \lnot B\Leftrightarrow B\to A$
      \item $A\to (B\to C)\Leftrightarrow (A\land B)\to C$
      \item $(A\lor B)\to C\Leftrightarrow (A\to C)\land (B\to C)$
      \item $\lnot A\lor B,A\to (B\land C),D\to B\Rightarrow \lnot B\to C$   
    \end{enumerate}

    \begin{proof}[解]
      \begin{enumerate}
        \item 成立。这是因为对任意的指派$\alpha$,如果$\alpha(A)=T$,则$\alpha(B\to A)=T$。
        \item 成立。这是因为对任意的指派$\alpha$,如果$\alpha((A\to B)\to(A\to C))=F$,则$\alpha(A\to B)=T$并且$\alpha(A\to C)=F$,从而$\alpha(A)=T,\alpha(C)=F,\alpha(B)=T$,于是$\alpha(A\to(B\to C))=F$。
        \item 成立。这是因为对任意的指派$\alpha$,$\alpha(\lnot A \to \lnot B)=F$等价于$\alpha(A)=F$并且$\alpha(B)=T$,等价于$\alpha(B\to A)=F$。
        \item 成立。这是因为对任意的指派$\alpha$,$\alpha(A\to(B\to C))=F$等价于$\alpha(A)=T,\alpha(B)=T,\alpha(C)=F$,等价于$\alpha((A\land B)\to C)=F$。
        \item 成立。这是因为对任意的指派$\alpha$,$\alpha((A\lor B)\to C)=F$等价于$\alpha(A\lor B)=T$并且$\alpha(C)=F$,等价于$\alpha(A)=T$或者$\alpha(B)=T$,并且$\alpha(C)=F$,等价于$\alpha(A\to C)=F$或者$\alpha(B\to C)=F$,等价于$\alpha((A\to C)\land (B\to C))=F$。
        \item 不成立。这是因为对于指派$\alpha$,$\alpha(A)=F,\alpha(B)=F,\alpha(C)=F,\alpha(D)=F$,则$\alpha(\lnot A\lor B)=T,\alpha(A\to (B\land C))=T,\alpha(D\to B)=T$,而$\alpha(\lnot B\to C)=F$。
      \end{enumerate}
    \end{proof}
  \end{Exercise}
  % 我们还可以利用真值表检验$(p\to q) \land (p\to \lnot q) \to \lnot p$是永真的。
  
  %     \begin{tabular}{cc|cc}
  %   $p$& $q$&$(p\to q) \land (p\to \lnot q) \to \lnot p$\\
  %   \hline
  %   T&T&T\\
  %   T&F&T\\
  %   F&T&T\\
  %   F&F&T\\      
  %     \end{tabular}

  %     假设我约定"$\to$"的真值表如下:

  %   \begin{tabular}{cc|c}
  %   p& q& p $\to$ q\\
  %   \hline
  %   T&T&T\\
  %   T&F&F\\
  %   F&T&F\\
  %   F&F&T\\
  %   \end{tabular}\hspace{0.87cm}

  %   我们会发现复合命题$(p\to q) \land (p\to \lnot q) \to \lnot p$不是永真的,这将与我们关于“蕴含”的思维不相符。

  %   同时我们还会发现$p \to q$和$q\to p$在逻辑上是等价的。

  %   \begin{tabular}{cc|cc}
  %   p& q& p $\to$ q&q $\to$ p\\
  %   \hline
  %   T&T&T&T\\
  %   T&F&F&F\\
  %   F&T&F&F\\
  %   F&F&T&T\\
  %   \end{tabular}\hspace{0.87cm}

  %   这也与我们的思维习惯不相符。

    

  
  % 有些逻辑术语从外文翻译成中文时产生了不同的称谓,在本门课程中关于逻辑术语我们做如下的约定:


  % The negation of a proposition $P$: $\lnot P$

  % 命题$P$的否定:$\lnot P$

  % The converse of $P\to Q$: $Q \to P$

  % 命题$P\to Q$的逆命题:$Q\to P$



  % The inverse of $P\to Q$: $\lnot P \to \lnot Q$
  
  % 在较深入的探讨数理逻辑的教材中,该概念用的很少,因此我们不给出具体的翻译称谓,在需要表达该概念时明确说明为$\lnot P \to \lnot Q$即可。

  
  % The contrapositive of $P\to Q$: $\lnot Q \to \lnot P$

  % 命题$P\to Q$的逆否命题:$\lnot Q \to \lnot P$

  % 需要特别说明的是,命题$P\to Q$的否定为$\lnot (P \to Q)$,而不是$P \to \lnot Q$。 

  
  

  
\end{CJK}
\end{document}


%%% Local Variables:
%%% mode: latex
%%% TeX-master: t
%%% End:
