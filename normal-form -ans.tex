\documentclass{article}
\usepackage{CJKutf8}
\usepackage{amsmath}
\usepackage{amssymb}
\usepackage{amsfonts}
\usepackage{amsthm}
\usepackage{titlesec}
\usepackage{titletoc}
\usepackage{xCJKnumb}
\usepackage{tikz}
\usepackage{mathrsfs}
\usepackage{indentfirst}
\usepackage{enumitem}
\newtheorem{Def}{定义}
\newtheorem{Thm}{定理}
\newtheorem{Exercise}{练习}

\newtheorem*{Example}{例}
\setlist[enumerate,1]{label=(\arabic*)}

\begin{document}
\begin{CJK*}{UTF8}{gbsn}
  \title{第二讲 范式}
  \author{陈建文}
  \maketitle
  % \tableofcontents


  课后作业题:

  \begin{Exercise}
    求下列公式的主合取范式与主析取范式:
    \begin{enumerate}
      \item $p\to (p\land q)$
      \item $(p\to q)\to (q\to r)$
      \item $(p\to (p\land q))\lor r$
    \end{enumerate}
  \end{Exercise}
  \begin{proof}[解]
    (1)
    \begin{align*}
      &p\to(p\land q)\\
      \Leftrightarrow & \lnot p \lor (p\land q)\\
      \Leftrightarrow & (\lnot p \lor p)\land (\lnot p \lor q)\\
      \Leftrightarrow & \lnot p \lor q\quad \text{主合取范式}\\
      &p\to(p\land q)\\
      \Leftrightarrow & \lnot p \lor (p\land q)\\
      \Leftrightarrow & (\lnot p \land (q \lor \lnot q))\lor (p\land q)\\
      \Leftrightarrow &(\lnot p \land q)\lor (\lnot p \land \lnot q)\lor (p\land q)\quad \text{主析取范式}\\
    \end{align*}


    (2)画出$(p\to q)\to (q\to r)$的真值表如下:

    \begin{tabular}{ccc|c}
      $p$& $q$& $r$& $(p\to q)\to (q\to r)$\\
      \hline
     T& T&T&T\\
      T&T&F&F\\
      T&F&T&T\\
       T& F&F&T\\
      F&T&T&T\\
      F&T&F&F\\
     F& F&T&T\\
      F&  F&F&T\\      
    \end{tabular}

    主合取范式为$(\lnot p \lor \lnot q \lor r)\land (p \lor \lnot q \lor r)$

    主析取范式为$(p \land q \land r) \lor (p \land \lnot q \land r) \lor (p \land \lnot q \land \lnot r) \lor (\lnot p \land q \land r) \lor (\lnot p \land \lnot q \land r) \lor (\lnot p \land \lnot q \land \lnot r)$

    

    (3)画出$(p\to (p\land q))\lor r$的真值表如下:

    \begin{tabular}{ccc|c}
      $p$& $q$& $r$& $(p\to (p\land q))\lor r$\\
      \hline
     T& T&T&T\\
      T&T&F&T\\
      T&F&T&T\\
       T& F&F&F\\
      F&T&T&T\\
      F&T&F&T\\
     F& F&T&T\\
      F&  F&F&T\\      
    \end{tabular}

    主合取范式为$\lnot p \lor q \lor r$

    主析取范式为$(p \land q \land r) \lor (p \land q \land \lnot r) \lor (p \land \lnot q \land r) \lor (\lnot p \land q \land r) \lor (\lnot p \land q \land \lnot r) \lor (\lnot p \land \lnot q \land r)  \lor (\lnot p \land \lnot q \land \lnot r)$

  \end{proof}
\end{CJK*}
\end{document}





%%% Local Variables:
%%% mode: latex
%%% TeX-master: t
%%% End: