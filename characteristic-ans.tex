\documentclass{article}
\usepackage{CJKutf8}
\usepackage{amsmath}
\usepackage{amssymb}
\usepackage{amsfonts}
\usepackage{amsthm}
\usepackage{titlesec}
\usepackage{titletoc}
\usepackage{xCJKnumb}
\usepackage{tikz}
\usepackage{mathrsfs}
\usepackage{indentfirst}

\newtheorem{Def}{定义}
\newtheorem{Thm}{定理}
\newtheorem{Cor}{推论}
\newtheorem{Exercise}{练习}

\newtheorem*{Example}{例}


\begin{document}
\begin{CJK*}{UTF8}{gbsn}
  \title{第十讲 无零因子环的特征数}
  \author{陈建文}
  \maketitle
  % \tableofcontents
  


课后作业题:
\begin{Exercise}
设$F$为一个域,$|F|=4$,证明:

(1)$F$的特征数为$2$;

(2)$F$的任意一个非零元并且非单位元$1$的元素$x$均满足方程$x^2=x+1$;

(3)列出$F$的加法表和乘法表。
\end{Exercise}
\begin{proof}[证明]

  (1)$F$中任一非零元关于加法的阶必整除$4$,因此可能为$1$,$2$或$4$,又因为$F$的特征数必为素数,因此$F$的特征数为$2$。

  (2)设$x$为$F$的任意一个非零元并且非单位元$1$的元素,

  则$x+1\neq 0$(否则$x+1=1+1$,可得$x=1$,与$x$非单位元$1$矛盾),$x+1\neq 1$(否则$x=0$,与$x$非零元矛盾),$x+1\neq x$(否则$1=0$,与$1\neq 0$矛盾),
  于是$x$和$x+1$为$F$中与$0$和$1$不同的其他两个元素,此时必有$x(x+1)=1$,这是因为$x(x+1)\neq x$(否则$x+1=1$,与$x+1\neq 1$矛盾),$x(x+1)\neq x+1$(否则$x=1$,与$x$非单位元$1$矛盾)。

  于是$x^2+x=1$,两边同时加$x$得$x^2=x+1$(由于$x$得阶为$2$,这里$x+x=0$)。

  (3)设$F=\{0,1,a,b\}$,加法表如下:

  \begin{tabular}{c|cccc}
    $+$&0&1&a&b\\
    \hline
  0&0&1&a&b\\
  1&1&0&b&a\\
  a&a&b&0&1\\
  b&b&a&1&0\\
  \end{tabular}

  乘法表如下:

  \begin{tabular}{c|cccc}
    $\circ$&0&1&a&b\\
    \hline
  0&0&0&0&0\\
  1&0&1&a&b\\
  a&0&a&b&1\\
  b&0&b&1&a\\
  \end{tabular}

\end{proof}


\end{CJK*}
\end{document}





%%% Local Variables:
%%% mode: latex
%%% TeX-master: t
%%% End:



