\documentclass{article}
\usepackage{CJKutf8}
\usepackage{amsmath}
\usepackage{amssymb}
\usepackage{amsfonts}
\usepackage{amsthm}
\usepackage{titlesec}
\usepackage{titletoc}
\usepackage{xCJKnumb}
\usepackage{tikz}
\usepackage{mathrsfs}
\usepackage{indentfirst}

\newtheorem{Def}{定义}
\newtheorem{Thm}{定理}
\newtheorem{Exercise}{练习}

\newtheorem*{Example}{例}


\begin{document}
\begin{CJK*}{UTF8}{gbsn}
  \title{第七讲 正规子群、商群}
  \author{陈建文}
  \maketitle
  % \tableofcontents
  



课后作业题:
\begin{Exercise}
设$A$和$B$为群$G$的两个有限子群,证明:
\[|AB|=\frac{|A||B|}{|A\cap B|}\]
\end{Exercise}
\begin{proof}[证明]
  因为$A\cap B$为$A$的子群,因此存在$a_1,a_2,\cdots,a_n\in A$,使得
  \[A=a_1(A\cap B)\cup a_2(A\cap B)\cup \cdots \cup a_n(A\cap B)\]
  这里$n=\frac{|A|}{|A\cap B|}$。以下验证$AB=a_1B\cup a_2B\cup \cdots \cup a_nB$,并且对任意的$i,j$,$1\leq i \leq n$,$1\leq j \leq n$,
  $a_iB\cap a_jB=\phi$,于是$|AB|=n|B|=\frac{|A||B|}{|A\cap B|}$。

  $\forall g\in AB$,存在$a\in A$,$b\in B$使得$g=ab$。进一步,存在$i$,$1\leq i\leq n$,$x\in A\cap B$使得$a=a_ix$,于是$g=a_ixb\in a_iB$(因为$x\in A\cap B\subseteq B$,$b\in B$,从而$xb\in B$)。

以下用反正法证明对任意的$i,j$,$1\leq i \leq n$,$1\leq j \leq n$,
$a_iB\cap a_jB=\phi$。假设存在$i,j$,$1\leq i \leq n$,$1\leq j \leq n$,使得$a_iB\cap a_jB\neq \phi$,则存在$x$,$x\in a_iB\cap a_jB$。
设$x=a_ib_1=a_jb_2$,这里$b_1\in B$,$b_2\in B$,则$a_i^{-1}b_j=b_1b_2^{-1}\in A\cap B$,从而$a_i(A\cap B)=a_j(A\cap B)$,与$a_i(A\cap B)\cap a_j(A\cap B)=\phi$矛盾。

\end{proof}
\begin{Exercise}
  利用上题的结论证明:六阶群中有唯一的一个三阶子群。
\end{Exercise}
\begin{proof}[证明]
  设$A$和$B$为六阶群$G$的两个三阶子群,由练习1结论可得:
  \[|AB|=\frac{|A||B|}{|A\cap B|}\]

  由于$A\cap B$为$A$的子群,所以必有$|A\cap B|||A|$,从而$|A\cap B|=1$或$3$。如果$|A\cap B|=1$,则$|AB|=9$,这与$G$为一个六阶群,$AB$为$G$的群子集矛盾,从而$|A\cap B|=3$,此时必有$A=B$,结论得证。
\end{proof}
\begin{Exercise}
设$G$为一个$n^2$阶的群,$H$为$G$的一个$n$阶子群。证明:$\forall x\in G, x^{-1}Hx\cap H \neq \{e\}$。
\end{Exercise}
\begin{proof}[证明]
  用反证法,假设存在$x\in G$,$x^{-1}Hx\cap H = \{e\}$。由练习1结论可得,
  \[|H(x^{-1}Hx)|=\frac{|H||x^{-1}Hx|}{|H\cap (x^{-1}Hx)|}=n^2\]
  又由于$G$为一个$n^2$阶的群,所以$H(x^{-1}Hx)=G$,由教材例题结论可得$x^{-1}Hx\cap H \neq \{e\}$,矛盾。
\end{proof}
\begin{Exercise}
证明:指数为2的子群为正规子群。
\end{Exercise}
\begin{proof}[证明]
  设$H$为群$G$的指数为2的子群,则存在$a\in G$使得$G=H\cup aH$。


  $\forall g\in G$,如果$g\in H$,则显然$gHg^{-1}\subseteq H$;如果$g\in aH$,则存在$h\in H$使得$g=ah$,
  以下证明$gHg^{_1}\subseteq H$,从而可得$H$为$G$的正规子群。

  $\forall x\in gHg^{-1}$,存在$h_1\in H$使得$x=gh_1g^{-1}$,再由$g=ah$得$x=ahh_1(ah)^{-1}=ahh_1h^{-1}a^{-1}$。
  此时必有$x\in H$,否则$x\in aH$,从而存在$h_2\in H$使得$x=ah_2$,于是$ahh_1h^{-1}a^{-1}=ah_2$,由此可得$a=h_2^{-1}hh_1h^{-1}\in H$,与$a\in aH$矛盾。
\end{proof}
\begin{Exercise}
证明:两个正规子群的交还是正规子群。
\end{Exercise}
\begin{proof}[证明]
  设$N_1$和$N_2$为群$G$的两个正规子群,显然$N_1\cap N_2$为$G$的子群。$\forall a\in G$,易得$a(N_1\cap N_2)a^{-1}\subseteq aN_1a^{-1}\subseteq N_1$,$a(N_1\cap N_2)a^{-1}\subseteq aN_2a^{-1}\subseteq N_2$,由此可得$a(N_1\cap N_2)a^{-1}\subseteq N_1\cap N_2$,这证明了$N_1\cap N_2$为$G$的正规子群。
\end{proof}
\begin{Exercise}
设$H$为群$G$的子群,$N$为群$G$的正规子群,试证:$NH$为群$G$的子群。
\end{Exercise}
\begin{proof}[证明]
  只需证$NH=HN$。

  $\forall g \in NH$,$\exists n\in N, h\in H$使得$g=nh$,由$N$为正规子群知$hN=Nh$,从而$\exists n_1\in N$使得$nh=hn_1$,于是$g=nh=hn_1\in HN$。

  $\forall g \in HN$,$\exists h\in H, n\in N$使得$g=hn$,由$N$为正规子群知$hN=Nh$,从而$\exists n_1\in N$使得$hn=n_1h$,于是$g=hn=n_1h\in NH$。
\end{proof}
\begin{Exercise}
设$G$为一个阶为$2n$的交换群,试证:$G$必有一个$n$阶商群。
\end{Exercise}
\begin{proof}[证明]
  由以前作业题知$G$中存在一个阶为$2$的元素$a$,则$G/(a)$为$G$的一个$n$阶商群。
\end{proof}
\begin{Exercise}
设$H$为群$G$的子群,证明:$H$为群$G$的正规子群的充分必要条件是$\forall a,b\in G,(aH)(bH)=(ab)H$。
\end{Exercise}

\begin{proof}[证明]
  由教材定理知如果$H$为群$G$的正规子群,则$\forall a,b\in G,(aH)(bH)=(ab)H$。

  以下假设$\forall a,b\in G,(aH)(bH)=(ab)H$,往证$H$为群$G$的正规子群。

  $\forall a\in G$,$(aH)(a^{-1}H)=(aa^{-1})H=eH=H$,从而$\forall h\in H$,$aha^{-1}h\in H$,于是$\exists h_1\in H, aha^{-1}h=h_1$,由此可得$aha^{-1}=h_1h^{-1}\in H$,
  这证明了$aHa^{-1}\subseteq H$,即$H$为群$G$的正规子群。
\end{proof}
\end{CJK*}
\end{document}





%%% Local Variables:
%%% mode: latex
%%% TeX-master: t
%%% End:



