\documentclass{article}
\usepackage{CJKutf8}
\usepackage{amsmath}
\usepackage{amssymb}
\usepackage{amsfonts}
\usepackage{amsthm}
\usepackage{titlesec}
\usepackage{titletoc}
\usepackage{xCJKnumb}
\usepackage{tikz}
\usepackage{mathrsfs}
\usepackage{indentfirst}

\newtheorem{Def}{定义}
\newtheorem{Thm}{定理}
\newtheorem{Exercise}{练习}

\newtheorem*{Example}{例}


\begin{document}
\begin{CJK*}{UTF8}{gbsn}
  \title{第七讲 正规子群、商群}
  \author{陈建文}
  \maketitle
  % \tableofcontents
  

\begin{Def}
  设$G$为一个群,$G$的任意子集称为群子集。在$2^G$中借助于$G$的乘法引入一个代数运算,称为群子集的乘法:$\forall A,B\in 2^G$,
  \[AB=\{ab|a\in A \text{且} b\in B\}\]
  对任意的$A\in 2^G$,定义
\[A^{-1}=\{a^{-1}|a\in A\}\]
\end{Def}

\begin{Thm}
  设$G$为一个群,则$\forall A,B,C\in 2^G$,$(AB)C=A(BC)$。
\end{Thm}

\begin{Thm}
  设$G$为一个群,则$\forall A,B\in 2^G$,$(AB)^{-1}=B^{-1}A^{-1}$。
\end{Thm}

\begin{Thm}
  设$G$为一个群,$H$为$G$的一个子群,则$HH=H,H^{-1}=H,HH^{-1}=H$。
\end{Thm}

\begin{Thm}
设$A$,$B$为群$G$的子群,则$AB$为$G$的子群的充分必要条件为$AB=BA$。
\end{Thm}

\begin{Example}
设$H$为$G$的一个子群且$H\neq \{e\}$。如果存在一个元素$x_0\in G$使得$H(x_0^{-1}Hx_0)=G$,则$H\cap (x_0^{-1}Hx_0)\neq \{e\}$。
\end{Example}

\begin{Def}
  设$H$为群$G$的子群,如果$\forall a\in G$有$aH=Ha$,则称$H$为$G$的正规子群。
\end{Def}
\begin{Thm}
 设$H$为群$G$的一个子群,则下列三个命题等价:

 (1)$H$为群$G$的正规子群;

 (2)$\forall a\in G, aHa^{-1}=H$;

 (3)$\forall a\in G, aHa^{-1}\subseteq H$。
\end{Thm}

\begin{Thm}
设$H$为群$G$的正规子群,则$H$的所有左陪集构成的集族$S_l$对群子集乘法形成一个群。
\end{Thm}
\begin{Def}
  群$G$的正规子群$H$的所有左陪集构成的集族,对群子集乘法构成的群称为$G$对$H$的商群,记为$G/H$。
\end{Def}


课后作业题:
\begin{Exercise}
设$A$和$B$为群$G$的两个有限子群,证明:
\[|AB|=\frac{|A||B|}{|A\cap B|}\]
\end{Exercise}
\begin{Exercise}
  利用上题的结论证明:六阶群中有唯一的一个三阶子群。
\end{Exercise}
\begin{Exercise}
设$G$为一个$n^2$阶的群,$H$为$G$的一个$n$阶子群。证明:$\forall x\in G, x^{-1}Hx\cap H \neq \{e\}$。
\end{Exercise}
\begin{Exercise}
证明:指数为2的子群为正规子群。
\end{Exercise}
\begin{Exercise}
证明:两个正规子群的交还是正规子群。
\end{Exercise}
\begin{Exercise}
设$H$为群$G$的子群,$N$为群$G$的正规子群,试证:$NH$为群$G$的子群。
\end{Exercise}
\begin{Exercise}
设$G$为一个阶为$2n$的交换群,试证:$G$必有一个$n$阶商群。
\end{Exercise}
\begin{Exercise}
设$H$为群$G$的子群,证明:$H$为群$G$的正规子群的充分必要条件是$H$的任意两个左陪集的乘积还是$H$的一个左陪集。
\end{Exercise}
\end{CJK*}
\end{document}





%%% Local Variables:
%%% mode: latex
%%% TeX-master: t
%%% End:



