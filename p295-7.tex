\documentclass{article}
\usepackage{tikz}
\usepackage{CJKutf8}
\usepackage{amsmath}
\usepackage{amsthm}
\begin{document}
\begin{CJK}{UTF8}{gbsn}
\newtheorem*{Ex}{习题}
\begin{Ex}
证明:每个哈密顿可平面图都是$4$-面可着色的。
\end{Ex}
\begin{proof}[证明]
将圈内的每个面内画一个顶点,两邻的面对应的顶点之间连接一条边,所得到的图连通无圈,所以是树。树是2可着色的,从而哈密顿圈内部是2-面可着色的。
同理,哈密顿圈外部是2-面可着色的。因此,整个哈密顿可平面图是$4$-面可着色的。
\end{proof}
\end{CJK}
\end{document}


%%% Local Variables:
%%% mode: latex
%%% TeX-master: t
%%% End:
