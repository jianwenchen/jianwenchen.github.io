\documentclass{article}
\usepackage{CJKutf8}
\usepackage{amsmath}
\usepackage{amssymb}
\usepackage{amsfonts}
\usepackage{amsthm}
\usepackage{titlesec}
\usepackage{titletoc}
\usepackage{xCJKnumb}
\usepackage{tikz}
\usepackage{mathrsfs}
\usepackage{indentfirst}

\newtheorem{Def}{定义}
\newtheorem{Thm}{定理}
\newtheorem{Exercise}{练习}

\newtheorem*{Example}{例}

\newtheorem{Cor}{推论}
\begin{document}
\begin{CJK*}{UTF8}{gbsn}
  \title{第七讲 陪集、拉格朗日定理}
  \author{陈建文}
  \maketitle
  % \tableofcontents
  

课后作业题:
\begin{Exercise}
证明:六阶群里必有一个三阶子群。
\end{Exercise}
\begin{proof}[证明]
  设$G$为任意一个六阶群。在$G$中如果存在一个阶为$3$的元素$a$,则$(a)$为$G$的一个三阶子群;如果存在一个阶为$6$的元素$b$,则$(b^2)$为$G$的一个三阶子群;
否则,由于$G$中每个元素的阶均整除$6$,此时$G$中除了单位元外每个元素的阶都为$2$,因此$G$为交换群。取$G$中的元素$e,x,y$,这里$e$为$G$的单位元,$x$和$y$为不是单位元的互不相同的其他两个元素,
易验证此时必有$xy=e$,这是因为如果$xy\neq e$,则可以验证$\{e,x,y,xy\}$构成一个四阶群,但$4$不整除$6$,矛盾。于是,$\{e,x,y\}$构成了$G$中的一个三阶群。
\end{proof}
\begin{Exercise}
设$p$为一个素数,证明:在阶为$p^m$的群里一定含有一个$p$阶子群,其中$m\geq 1$。
\end{Exercise}
\begin{proof}[证明]
  设$G$为任意一个$p^m$阶群。在$G$中任取一个不是单位元的元素$a$,则$a$的阶整除$p^m$。由于$a$不是单位元,因此$a$的阶不为$1$,从而存在$i$,$1\leq i\leq m$,使得$a$的阶为$p^i$。
  如果$i=1$,则$(a)$为$G$的一个$p$阶子群;如果$i>1$,则$(a^{i-1})$为$G$的一个$p$阶子群。
\end{proof}
\begin{Exercise}
在三次对称群$S_3$中,找一个子群$H$,使得$H$的左陪集不等于$H$的右陪集。
\end{Exercise}
\begin{proof}[证明]
  设$H=\{(1),(12)\}$,则$(13)H=\{(13),(132)\}$,$H(13)=\{(13),(123)\}$, 
  
  $(13)H\neq H(13)$。
\end{proof}
\begin{Exercise}
设$H$为群$G$的一个子群,如果左陪集$aH$等于右陪集$Ha$,即$aH=Ha$,则$\forall h\in H, ah=ha$一定成立吗?
\end{Exercise}
\begin{proof}[证明]
  不一定成立。

  例如,$H=\{(1),(123),(132)\}$为$S_3$的一个子群,$(12)H=\{(12),(13),(23)\}$,
  $H(12)=\{(12),(23),(13)\}$,$(12)H=H(12)$。$(12)(123)=(13)$,$(123)(12)=(23)$,$(12)(123)\neq (123)(12)$。
\end{proof}
\end{CJK*}
\end{document}





%%% Local Variables:
%%% mode: latex
%%% TeX-master: t
%%% End:



