\documentclass{article}
\usepackage{CJKutf8}
\usepackage{amsmath}
\usepackage{amssymb}
\usepackage{amsfonts}
\usepackage{amsthm}
\usepackage{titlesec}
\usepackage{titletoc}
\usepackage{xCJKnumb}
\usepackage{tikz}
\usepackage{mathrsfs}
\usepackage{indentfirst}

\newtheorem{Def}{定义}
\newtheorem{Thm}{定理}
\newtheorem{Exercise}{练习}

\newtheorem*{Example}{例}


\begin{document}
\begin{CJK*}{UTF8}{gbsn}
  \title{第二讲 半群、幺半群与群}
  \author{陈建文}
  \maketitle



  课后作业题:

\begin{Exercise}
  给出一个半群,它有无穷多个右单位元素。
\end{Exercise}
\begin{proof}[解]
  设$S$为一切形如
  \[\begin{bmatrix}
    a&0\\
    b&0
  \end{bmatrix},a,b\in N\]
的$2\times 2$矩阵之集,则$S$对矩阵的乘法构成一个半群。

$\forall d\in N$,$2\times 2$矩阵
\[\begin{bmatrix}1&0\\d&0\end{bmatrix}\]
为右单位元素。于是,$(S,*)$有无穷多个右单位元素。  
\end{proof}
\begin{Exercise}
  设$(S,\circ)$为一个半群,$a\in S$称为左消去元素,如果$\forall x, y\in S$,有$a\circ x=a\circ y$,则一定有$x=y$。试证:
  如果$a$和$b$均为左消去元,则$a\circ b$也是左消去元。
\end{Exercise}
\begin{proof}[证明]
  如果$\forall x, y\in S$,$(a\circ b)\circ x=(a\circ b)\circ y$,由结合律知$a\circ (b\circ x)=a\circ (b\circ y)$,由$a$为左消去元知$b\circ x=b\circ y$,由$b$为左消去元知$x=y$,这证明了$a\circ b$也是左消去元。
\end{proof}
\begin{Exercise}
  设$Z$为整数集合,$M=Z\times Z$。在$M$上定义二元运算$\circ$如下:

  $\forall (x_1,x_2), (y_1,y_2)\in M, (x_1,x_2)\circ (y_1,y_2)=(x_1y_1+2x_2y_2,x_1y_2+x_2y_1)$

  试证:

  (1)$M$对上述定义的代数运算构成一个幺半群。

  (2)如果$(x_1,x_2)\neq (0,0)$,则$(x_1,x_2)$是左消去元。

  (3)运算“$\circ$”满足交换率。
\end{Exercise}
\begin{proof}[证明]
  (1)$\forall (x_1,x_2), (y_1,y_2),(z_1,z_2)\in M$,
  \begin{align*}
    &((x_1,x_2)\circ (y_1,y_2))\circ (z_1,z_2) = (x_1y_1+2x_2y_2,x_1y_2+x_2y_1)\circ (z_1,z_2)\\
     =& (x_1y_1z_1+2x_2y_2z_1+2x_1y_2z_2+2x_2y_1z_2,x_1y_1z_2+2x_2y_2z_2+x_1y_2z_1+x_2y_1z_1)\\
     &(x_1,x_2)\circ ((y_1,y_2)\circ (z_1,z_2)) = (x_1,x_2)\circ (y_1z_1+2y_2z_2,y_1z_2+y_2z_1)\\
     =&(x_1y_1z_1+2x_1y_2z_2+2x_2y_1z_2+2x_2y_2z_1,x_1y_1z_2+x_1y_2z_1+x_2y_1z_1+2x_2y_2z_2)\\
  &((x_1,x_2)\circ (y_1,y_2))\circ (z_1,z_2) = (x_1,x_2)\circ ((y_1,y_2)\circ (z_1,z_2))\\
    \end{align*}
    这验证了“$\circ$”运算满足结合律。

  \[\forall (x,y)\in M,(1,0)\circ (x,y) = (x,y)\circ (1,0) = (x,y)\]
  
  这验证了$(1,0)$为“$\circ$”运算的单位元。
    
    
    于是,$M$对“$\circ$”运算构成一个幺半群。

(2) $\forall (y_1,y_2),(z_1,z_2)\in M$,如果$(x_1,x_2)\circ (y_1,y_2)=(x_1,x_2)\circ (z_1,z_2)$,
则$(x_1y_1+2x_2y_2,x_1y_2+x_2y_1)=(x_1z_1+2x_2z_2,x_1z_2+x_2z_1)$,

当$(x_1,x_2)\neq 0$时,以下证明关于$y_1$和$y_2$的线性方程组
\[
\begin{cases}
  x_1y_1+2x_2y_2=x_1z_1+2x_2z_2\\
  x_1y_2+x_2y_1=x_1z_2+x_2z_1
\end{cases}\]
即
\[
\begin{cases}
  x_1y_1+2x_2y_2=x_1z_1+2x_2z_2\\
  x_2y_1+x_1y_2=x_2z_1+x_1z_2
\end{cases}\]
有唯一解$y_1=z_1,y_2=z_2$,因此$(x_1,x_2)$是左消去元。

以上线性方程组系数矩阵的行列式
\[\begin{vmatrix}x_1&2x_2\\x_2&x_1\end{vmatrix}=x_1^2-2x_2^2\neq 0\]

这是因为如果$x_1^2-2x_2^2=0$,则$x_1^2=2x_2^2$,此时如果$x_2=0$,则$x_1=0$,与$(x_1,x_2)\neq 0$矛盾;如果$x_2\neq 0$,则$(\frac{x_1}{x_2})^2=2$,与$x_1$和$x_2$都为整数矛盾。

(3)$\forall (x_1,x_2), (y_1,y_2)\in M$,

\begin{align*}
  (x_1,x_2)\circ (y_1,y_2) &= (x_1y_1+2x_2y_2,x_1y_2+x_2y_1)\\
  (y_1,y_2)\circ (x_1,x_2) &= (y_1x_1+2y_2x_2,y_1x_2+x_1y_2)\\
  (x_1,x_2)\circ (y_1,y_2) &= (y_1,y_2)\circ (x_1,x_2)\\
\end{align*}
这证明了“$\circ$”运算满足交换律。
\end{proof}

\begin{Exercise}
  证明:有限半群中一定有一个元素$a$使得$a\circ a=a$。
\end{Exercise}

\begin{proof}[证明]
  设$(S,\circ)$为有限半群,$x\in S$。由$S$为有限半群知,存在正整数$i$和$j$,使得$x^i=x^j$。
  不妨设$j>i$。$\forall m\in Z^+$,如果$m\geq j$,则$x^m=x^{m-j}x^j=x^{m-j}x^i=x^{m-(j-i)}$。取正整数$n$,使得$(n+1)(j-i)\geq j$,则$x^{2n(j-i)}=x^{n(j-i)}$,于是$(x^{n(j-i)})^2=x^{n(j-i)}$。
\end{proof}


%%% Local Variables:
%%% mode: latex
%%% TeX-master: "chapter1"
%%% End:

\end{CJK*}
\end{document}





%%% Local Variables:
%%% mode: latex
%%% TeX-master: t
%%% End:



