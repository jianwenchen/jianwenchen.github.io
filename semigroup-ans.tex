\documentclass{article}
\usepackage{CJKutf8}
\usepackage{amsmath}
\usepackage{amssymb}
\usepackage{amsfonts}
\usepackage{amsthm}
\usepackage{titlesec}
\usepackage{titletoc}
\usepackage{xCJKnumb}
\usepackage{tikz}
\usepackage{mathrsfs}
\usepackage{indentfirst}

\newtheorem{Def}{定义}
\newtheorem{Thm}{定理}
\newtheorem{Exercise}{练习}

\newtheorem*{Example}{例}


\begin{document}
\begin{CJK*}{UTF8}{gbsn}
  \title{第二讲 半群、幺半群与群}
  \author{陈建文}
  \maketitle
  % \tableofcontents
  \begin{Def}
    代数系$(S,\circ)$称为一个半群,如果二元代数运算“$\circ$”满足结合律,即$\forall a,b,c \in S$,
    \[(a\circ b) \circ c = a \circ (b\circ c)\text{。}\] 
  \end{Def}
  
  \begin{Example}
    正整数集合$Z^+$对“$+$”运算构成一个半群。
  \end{Example}
  
  $\forall a,b,c \in Z^+ (a+b)+c = a + (b + c)$
  
  \begin{Def}
    如果一个半群中的二元代数运算满足交换律,则称此半群为交换半群。
  \end{Def}
  
  \begin{Example}
    设$S$为一切形如
    \[\begin{bmatrix}
      a&b\\
      0&0
    \end{bmatrix},a,b\in N\]
  的$2\times 2$矩阵之集,则$S$对矩阵的乘法构成一个不可交换的半群。
  
  $\forall d\in N$,$2\times 2$矩阵
  \[\begin{bmatrix}1&d\\0&0\end{bmatrix}\]
  为左单位元素。于是,$(S,*)$有无穷多个左单位元素,然而它却没有右单位元素。
  \end{Example}
  
  \begin{Thm}
    设$(S,\circ)$为一个代数系,如果二元代数运算$\circ$既有左单位元$e_l$,又有右单位元$e_r$,则$e_l=e_r$,从而有单位元且单位元是唯一的。
  \end{Thm}
  \begin{proof}[证明]
    $e_r=e_l\circ e_r=e_l$
  \end{proof}
  \begin{Def}
    有单位元素的半群称为独异点(monoid),或称为幺半群。
  \end{Def}
  
  \begin{Example}
    自然数集合$N$对加法运算“$+$”构成幺半群,单位元为$0$。正整数集合$Z^+$对乘法运算“$\times$”构成幺半群,单位元为$1$。
  \end{Example}
  
  \begin{Example}
    设$S$为任意一个集合,则$(2^S,\cup,\phi)$和$(2^S,\cap,S)$都为幺半群。
  \end{Example}
  
  \begin{Def}
    如果一个幺半群中的二元代数运算满足交换律,则称此幺半群为交换幺半群。
  \end{Def}
  
  \begin{Example}
    设$S$为非空集合,$S^S=\{f|f:S\to S\}$,则$S^S$对映射的合成构成了一个以$I_S$为单位元的幺半群$(S^S,\circ,I_S)$,它是不可交换幺半群。
  \end{Example}
  \begin{Example}
    设$M_n$为所有$n\times n$实矩阵构成的集合,则$M_n$对矩阵的乘法构成了一个以$I_n$为单位元的幺半群$(M_n,*,I_n)$。
  \end{Example}
  
  
  \begin{Def}
    设$(S,\circ,e)$为一个幺半群,$a\in S$。如果存在$a_l\in S$使得$a_l\circ a=e$,则称$a_l$为$a$的左逆元素;如果存在$a_r
    \in S$使得$a\circ a_r=e$,则称$a_r$为$a$的右逆元素;如果存在$b\in S$使得$b\circ a=a\circ b=e$,则称$b$为$a$的逆元素。
  \end{Def}
  \begin{Thm}
    如果幺半群$(S,\circ,e)$中的元素$a$既有左逆元素$a_l$,又有右逆元素$a_r$,则$a_l=a_r$。于是,$a$有逆元素且$a$的逆元素是唯一的,记为$a^{-1}$。
  \end{Thm}
  \begin{proof}[证明]
    $a_r=e\circ a_r=(a_l\circ a)\circ a_r=a_l\circ(a\circ a_r)=a_l\circ e=a_l$
  \end{proof}
  \begin{Def}
    每个元素都有逆元素的幺半群称为群。
  \end{Def}
  
  \begin{Def}
    设$G$为一个非空集合,“$\circ$”为$G$上的一个二元代数运算。如果下列各个条件成立,则称$G$对“$\circ$”运算构成一个群(group):
  
    I. “$\circ$”运算满足结合律,即$\forall a,b,c \in G$ $(a\circ b)\circ c = a\circ(b\circ c)$;
  
    II. 对“$\circ$”运算,$G$中有一个单位元$e$,即$\forall a\in G$ $e\circ a =a\circ e= a$;
  
    III. 对$G$中的每个元素,关于$\circ$运算有一个逆元,
    即$\forall a\in G \exists b\in G b\circ a = a\circ b= e$。
  \end{Def}
  群$G$中的“$\circ$”运算通常称为乘法,$a\circ b$简写为$ab$。$a$的逆元记为$a^{-1}$。
  \begin{Example}
    整数集合$Z$,有理数集合$Q$,实数集合$R$,复数集合$C$对通常的加法运算构成群;
    非零有理数集合$Q^*$,非零实数集合$R^*$,非零复数集合$C^*$对通常的乘法运算构成群。
  \end{Example}
  \begin{Def}
    如果一个群中的二元代数运算满足交换律,则称此群为交换群,又称为Abel群。
  \end{Def}
  \begin{Example}
  设$S$为一个非空集合,从$S$到$S$的所有双射构成的集合对映射的合成构成一个群,称为$S$上的对称群,记为$Sym(S)$。当$S=\{1,2,\cdots,n\}$时,$Sym(S)=S_n$。
  \end{Example}
    \begin{Example}
      设$M_n$为所有可逆$n\times n$实矩阵构成的集合,则$M_n$对矩阵的乘法构成了一个以$I_n$为单位元的群$(M_n,*,I_n)$。
    \end{Example}
  \begin{Def}
    群$(G,\circ)$称为有限群,如果$G$为有限集。$G$的基数称为群$G$的阶,记为$|G|$。如果$G$含有无穷多个元素,则称$G$为无限群。
  \end{Def}
  
  \begin{Def}
    设$a,b\in Z$,如果存在$q\in Z$使得$a=qb$,则称$b$整除$a$,记为$b|a$。
  \end{Def}
  \begin{Def}
    设$a,b\in Z$,$b>0$,$a=qb+r$,$q\in Z$,$0\leq r<b$,则称$r$为$a$除以$b$所得到的余数,记为$a\bmod b$。
  \end{Def}
  
  \begin{Def}
    设$a,b,n\in Z$,$n>0$,如果$a\bmod n=b\bmod n$,则称$a$与$b$模$n$同余,记为$a \equiv b\pmod{n}$。
  \end{Def}
  
  
  \begin{Thm}
    $\forall a,b,n\in Z, n>0, a\equiv b\pmod{n}$等价于$n|(a-b)$。
  \end{Thm}
  
  \begin{Thm}
    1. $\forall a\in Z, a \equiv a \pmod{n}$;
  
    2. $\forall a, b\in Z$,如果$a\equiv b \pmod{n}$,则$b\equiv a\pmod{n}$;
  
    3. $\forall a,b,c\in Z$,如果$a\equiv b\pmod{n}$并且$b\equiv c\pmod{n}$,则$a\equiv c\pmod{n}$;
  
    4. $\forall a,b,k\in Z$,如果$a\equiv b\pmod{n}$,则$a+k\equiv b+k\pmod{n}$;
  
    5. $\forall a,b,c,d\in Z$,如果$a\equiv b\pmod{n}$并且$c\equiv d\pmod{n}$,则$a+c\equiv b+d \pmod{n}$;
  
    6. $\forall a,b,k\in Z$,如果$a\equiv b\pmod{n}$,则$ak\equiv bk\pmod{n}$;
  
    7. $\forall a,b,c,d\in Z$,如果$a\equiv b\pmod{n}$并且$c\equiv d\pmod{n}$,则$ac\equiv bd \pmod{n}$
  
    8. $\forall a,b\in Z$,$ab \bmod n=(a\bmod n)(b\bmod n) \bmod n$。
  \end{Thm}
  
  \begin{Def}
    设$n\in Z$,$n>0$,$\forall x\in Z$,定义$[x]=\{y|y\equiv x \pmod{n}\}$,称为整数集$Z$上在模$n$同余的等价关系下的一个等价类。
  \end{Def}
  \begin{Example}
  模$4$同余关系的所有等价类为:
  
  $[0]=\{\cdots,-8,-4,0,4,8,\cdots\}$
  
  $[1]=\{\cdots,-7,-3,1,5,9,\cdots\}$
  
  $[2]=\{\cdots,-6,-2,2,6,10,\cdots\}$
  
  $[3]=\{\cdots,-5,-1,3,7,11,\cdots\}$
  \end{Example}
  \begin{Thm}
    设$n\in Z$,$n>0$,$\forall x,y\in Z$,$[x]=[y]$当且仅当$x\equiv y\pmod{n}$。
  \end{Thm}
  \begin{proof}[证明]
    $\Rightarrow$
  
    $x\in [x]=[y]\Rightarrow x\equiv y(\bmod n)$。
  
    $\Leftarrow$
  
    $t\in [x]\Leftrightarrow t\equiv x \pmod{n} \Leftrightarrow t\equiv y \pmod{n} \Leftrightarrow t\in [y]$。
  \end{proof}
  \begin{Example}
    设$Z_n=\{[0],[1],\cdots,[n-1]\}$为整数集$Z$上在模$n$同余的等价关系下所有等价类之集,
    在$Z_n$上定义加法运算“$+$”如下:$\forall [i],[j]\in Z_n,[i]+[j]=[i+j]$,则$(Z_n,+)$构成一个交换群。
    
  %  在$Z_n$上定义乘法运算“$*$”如下:$\forall [i],[j]\in Z_n,[i]*[j]=[i*j]$,则$(Z_n,*)$构成一个交换幺半群;
  \end{Example}
  \begin{proof}[证明]
    $\forall i,j,i',j'\in Z$,如果$[i]=[i']$,$[j]=[j']$,则$[i+j]=[i'+j']$,这验证了“$+$”为一个运算。
  
    $\forall i,j,k\in Z$,$([i]+ [j])+ [k]=[i+j]+ [k]=[(i+j)+k]$,$[i]+ ([j]+ [k])=[i]+ [j+k]=[i+(j+k)]$,$([i]+ [j])+ [k]=[i]+ ([j]+ [k])$,这验证了加法运算$+$满足结合律。
  
    $\forall i\in Z$,$[0]+[i]=[i]+[0]=[i]$,这验证了$[0]$为单位元。
  
    $\forall i\in Z$,$[n-i]+[i]=[i]+[n-i]=[n]=[0]$,这说明$[i]$有逆元。
  
    以上验证了$Z_n$对于加法运算“$+$”构成一个群。  
  \end{proof}
  
  \begin{Example}
    设$Z'_n=\{0,1,2,\cdots,n-1\}$,在$Z'_n$上定义运算"$\oplus$"如下:$i\oplus j=(i+j)\bmod n$,则$(Z'_n,\oplus)$构成一个群。
  \end{Example}
  \begin{proof}[证明]
    $\forall a,b,c\in Z'_n,(a\oplus b)\oplus c=a\oplus (b\oplus c)$
  
    $((a+b)\bmod n+c)\bmod n=(a+(b+c)\bmod n)\bmod n$
  
    $((a+b)\bmod n+c)\bmod n=(a+b+c)\bmod n$
  
    $(a+(b+c)\bmod n)\bmod n=(a+b+c)\bmod n$
  
    $0\oplus a = (0+a)\bmod n = a$
  
    如果$a\neq 0$,则$(n-a)\oplus a = (n-a+a)\bmod n = 0$;$0\oplus 0=(0+0)\bmod n=0$。
  \end{proof}
  
  
  设用$n$个二进制位表示一个整数$x$,$x$的补码定义为
  
  如果$x\geq 0$,则$x$的补码为$x$的原码;
  
  如果$x < 0$, 则$x$的补码为$x+2^n$的原码。
  
  例:设用8个二进制位表示一个整数,计算7和-7的补码。
  
  解:
  
  因为$7\geq 0$,因此7的补码为7的原码,即7的补码为00000111。
  
  因为$-7 < 0$,因此-7的补码为$-7+2^8$的原码,即-7的补码为11111001。
  
  $-7$的补码还可以这样求解:先计算7的原码,得到00000111,然后取反加1,得到$-7$的补码为11111001。
  
  例:设用8个二进制位表示一个整数,计算-128的补码。
  
  解:因为$-128 < 0$,因此-128的补码为$-128+2^8$的原码,即-128的补码为$10000000$。同样的,$-128$的补码还可以这样求解:先计算128的原码,得到$10000000$,然后取反加1,得到$-128$的补码为$10000000$。
  
  如果用$n$个二进制位表示一个整数,用补码表示的数字的范围为$-2^{n-1}\sim 2^{n-1}-1$。对于补码而言,如果首位为0,其表示的是大于等于0的整数,如果首位为1,其表示的是负数。
  
  例:如果用8个二进制位表示一个整数,00001010为哪个整数的补码?10001010为哪个整数的补码?
  
  解:因为00001010的首位为0,它为一个大于等于0的整数的补码,这个整数为10。因为10001010的首位为1,它为一个负数的补码,这个负数为$138-2^8=-118$。
  
  计算机中普遍采用补码表示数字的原因是对于负数的加法可以采用与自然数的加法一样的加法器。下面简略介绍其思想。
  
  设$x$和$y$为任意的两个整数,分以下4种情况讨论:
  
  $x\geq 0$,$y\geq 0$:此时$x$的补码为$x$的原码,$y$的补码为$y$的原码,按照自然数相加计算得到$x+y$,恰为$x+y$的补码。
  
  $x < 0$,$y \geq 0$:此时$x$的补码为$x+2^n$的原码,$y$的补码为$y$的原码,按照自然数相加计算得到$x+2^n+y=(x+y)+2^n$。如果$x+y<0$,则得到的恰为$x+y$的补码;如果$x+y\geq0$,计算结果的第$n$位(从最右边数起,依次为第0位,第1位,$\cdots$,第$n-1$位,第$n$位)会自动抛掉。
  
  $x \geq 0$,$y < 0$:此时$x$的补码为$x$的原码,$y$的补码为$y+2^n$的原码,按照自然数相加计算得到$x+(y+2^n)=(x+y)+2^n$。如果$x+y<0$,则得到的恰为$x+y$的补码;如果$x+y\geq0$,计算结果的第$n$位会自动抛掉。
  
  $x < 0$,$y < 0$:此时$x$的补码为$x+2^n$的原码,$y$的补码为$y+2^n$的原码,按照自然数相加计算得到$(x+2^n)+(y+2^n)=(x+y)+2^n + 2^n$,计算结果的第$n$位会自动抛掉,于是最终得到的计算结果为$(x+y)+2^n$,恰为$x+y$的补码。
  
  \begin{Def}
    设$G$为一个非空集合,“$\circ$”为$G$上的一个二元代数运算。如果下列各个条件成立,则称$G$对“$\circ$”运算构成一个群(group):
  
    I. “$\circ$”运算满足结合律,即$\forall a,b,c \in G$ $(a\circ b)\circ c = a\circ(b\circ c)$;
  
    II. 对“$\circ$”运算,$G$中有一个左单位元$e$,即$\forall a\in G$ $e\circ a = a$;
  
    III. 对$G$中的每个元素,关于$\circ$运算有一个左逆元,
    即$\forall a\in G \exists b\in G b\circ a = e$,其中$e$为II中的同一个左单位元素。
  \end{Def}
  \begin{Thm}
    设$G$为一个群,则$\forall a\in G$,$a$的左逆元也是$a$的右逆元。
  \end{Thm}
  \begin{proof}[证明]
    $\forall a\in G$,设$a_l$为$a$的一个左逆元,则
    \[a_la=e\]
    两边同时右乘以$a_l$得
    \[(a_la)a_l=ea_l\]
    从而
    \[a_l(aa_l)=a_l\]
    两边同时左乘以$a_l$的左逆元得
    \[aa_l=e\]
  \end{proof}
  \begin{Thm}
    设$G$为一个群,则$G$的左单位元$e$也是右单位元。
  \end{Thm}
  \begin{proof}[证明]
    $\forall a\in G$,设$a_l$为$a$的左逆元,则$ae=a(a_la)=(aa_l)a=ea=a$,所以$e$也是右单位元。
  \end{proof}
  \begin{Thm}
    设$a$与$b$为群$G$的任意两个元素,则$(a^{-1})^{-1}=a$,$(ab)^{-1}=b^{-1}a^{-1}$。
  \end{Thm}
  \begin{proof}[证明]
    由\[aa^{-1}=e\]
    得\[(a^{-1})^{-1}=a\]
  由\[(b^{-1}a^{-1})(ab)=b^{-1}(a^{-1}a)b=b^{-1}eb=e\]
  得\[(ab)^{-1}=b^{-1}a^{-1}\]
  \end{proof}
    




  课后作业题:

\begin{Exercise}
  给出一个半群,它有无穷多个右单位元素。
\end{Exercise}
\begin{proof}[解]
  设$S$为一切形如
  \[\begin{bmatrix}
    a&0\\
    b&0
  \end{bmatrix},a,b\in N\]
的$2\times 2$矩阵之集,则$S$对矩阵的乘法构成一个半群。

$\forall d\in N$,$2\times 2$矩阵
\[\begin{bmatrix}1&0\\d&0\end{bmatrix}\]
为右单位元素。于是,$(S,*)$有无穷多个右单位元素。  
\end{proof}
\begin{Exercise}
  设$(S,\circ)$为一个半群,$a\in S$称为左消去元素,如果$\forall x, y\in S$,有$a\circ x=a\circ y$,则一定有$x=y$。试证:
  如果$a$和$b$均为左消去元,则$a\circ b$也是左消去元。
\end{Exercise}
\begin{proof}[证明]
  如果$\forall x, y\in S$,$(a\circ b)\circ x=(a\circ b)\circ y$,由结合律知$a\circ (b\circ x)=a\circ (b\circ y)$,由$a$为左消去元知$b\circ x=b\circ y$,由$b$为左消去元知$x=y$,这证明了$a\circ b$也是左消去元。
\end{proof}
\begin{Exercise}
  设$Z$为整数集合,$M=Z\times Z$。在$M$上定义二元运算$\circ$如下:

  $\forall (x_1,x_2), (y_1,y_2)\in M, (x_1,x_2)\circ (y_1,y_2)=(x_1y_1+2x_2y_2,x_1y_2+x_2y_1)$

  试证:

  (1)$M$对上述定义的代数运算构成一个幺半群。

  (2)如果$(x_1,x_2)\neq (0,0)$,则$(x_1,x_2)$是左消去元。

  (3)运算“$\circ$”满足交换率。
\end{Exercise}
\begin{proof}[证明]
  (1)$\forall (x_1,x_2), (y_1,y_2),(z_1,z_2)\in M$,
  \begin{align*}
    &((x_1,x_2)\circ (y_1,y_2))\circ (z_1,z_2) = (x_1y_1+2x_2y_2,x_1y_2+x_2y_1)\circ (z_1,z_2)\\
     =& (x_1y_1z_1+2x_2y_2z_1+2x_1y_2z_2+2x_2y_1z_2,x_1y_1z_2+2x_2y_2z_2+x_1y_2z_1+x_2y_1z_1)\\
     &(x_1,x_2)\circ ((y_1,y_2)\circ (z_1,z_2)) = (x_1,x_2)\circ (y_1z_1+2y_2z_2,y_1z_2+y_2z_1)\\
     =&(x_1y_1z_1+2x_1y_2z_2+2x_2y_1z_2+2x_2y_2z_1,x_1y_1z_2+x_1y_2z_1+x_2y_1z_1+2x_2y_2z_2)\\
  &((x_1,x_2)\circ (y_1,y_2))\circ (z_1,z_2) = (x_1,x_2)\circ ((y_1,y_2)\circ (z_1,z_2))\\
    \end{align*}
    这验证了“$\circ$”运算满足结合律。

  \[\forall (x,y)\in M,(1,0)\circ (x,y) = (x,y)\circ (1,0) = (x,y)\]
  
  这验证了$(1,0)$为“$\circ$”运算的单位元。
    
    
    于是,$M$对“$\circ$”运算构成一个幺半群。

(2) $\forall (y_1,y_2),(z_1,z_2)\in M$,如果$(x_1,x_2)\circ (y_1,y_2)=(x_1,x_2)\circ (z_1,z_2)$,
则$(x_1y_1+2x_2y_2,x_1y_2+x_2y_1)=(x_1z_1+2x_2z_2,x_1z_2+x_2z_1)$,

当$(x_1,x_2)\neq 0$时,以下证明关于$y_1$和$y_2$的线性方程组
\[
\begin{cases}
  x_1y_1+2x_2y_2=x_1z_1+2x_2z_2\\
  x_1y_2+x_2y_1=x_1z_2+x_2z_1
\end{cases}\]
即
\[
\begin{cases}
  x_1y_1+2x_2y_2=x_1z_1+2x_2z_2\\
  x_2y_1+x_1y_2=x_2z_1+x_1z_2
\end{cases}\]
有唯一解$y_1=z_1,y_2=z_2$,因此$(x_1,x_2)$是左消去元。

以上线性方程组系数矩阵的行列式
\[\begin{vmatrix}x_1&2x_2\\x_2&x_1\end{vmatrix}=x_1^2-2x_2^2\neq 0\]

这是因为如果$x_1^2-2x_2^2=0$,则$x_1^2=2x_2^2$,此时如果$x_2=0$,则$x_1=0$,与$(x_1,x_2)\neq 0$矛盾;如果$x_2\neq 0$,则$(\frac{x_1}{x_2})^2=2$,与$x_1$和$x_2$都为整数矛盾。

(3)$\forall (x_1,x_2), (y_1,y_2)\in M$,

\begin{align*}
  (x_1,x_2)\circ (y_1,y_2) &= (x_1y_1+2x_2y_2,x_1y_2+x_2y_1)\\
  (y_1,y_2)\circ (x_1,x_2) &= (y_1x_1+2y_2x_2,y_1x_2+x_1y_2)\\
  (x_1,x_2)\circ (y_1,y_2) &= (y_1,y_2)\circ (x_1,x_2)\\
\end{align*}
这证明了“$\circ$”运算满足交换律。
\end{proof}

\begin{Exercise}
  证明:有限半群中一定有一个元素$a$使得$a\circ a=a$。
\end{Exercise}

\begin{proof}[证明]
  设$(S,\circ)$为有限半群,$x\in S$。由$S$为有限半群知,存在正整数$i$和$j$,使得$x^i=x^j$。
  不妨设$j>i$。$\forall m\in Z^+$,如果$m\geq j$,则$x^m=x^{m-j}x^j=x^{m-j}x^i=x^{m-(j-i)}$。取正整数$n$,使得$(n+1)(j-i)\geq j$,则$x^{2n(j-i)}=x^{n(j-i)}$,于是$(x^{n(j-1)})^2=x^{n(j-i)}$。
\end{proof}
\begin{Exercise}
  设$R$为实数集合,$S=\{(a,b)|a\neq 0,a,b\in R\}$。在$S$上利用通常的加法和乘法定义二元运算“$\circ$”如下:
  \[(a,b)\circ (c,d) = (ac, ad + b)\]
  验证$(S,\circ)$为群。
\end{Exercise}
\begin{proof}[证明]
  $\forall (a,b),(c,d),(e,f)\in S$,
  \begin{align*}
    &((a,b)\circ (c,d))\circ (e,f) = (ac,ad+b)\circ (e,f)\\
    =&(ace,acf+ad+b)\\
    &(a,b)\circ ((c,d)\circ (e,f)) = (a,b)\circ (ce,cf+d)\\
    =&(ace,acf+ad+b)\\
    &((a,b)\circ (c,d))\circ (e,f) = (a,b)\circ ((c,d)\circ (e,f))\\
  \end{align*}
  这验证了“$\circ$”运算满足结合律。

  $\forall (c,d)\in S, (1,0)\circ (c,d) = (1\cdot c, 1\cdot d + 0) = (c,d)$,
这验证了$(1,0)$为$\circ$运算的左单位元。

$\forall (c,d)\in S,(1/c,-d/c)\circ (c,d) = (1,0)$,这验证了$(1/c,-d/c)$为$(c,d)$关于左单位元$(1,0)$的左逆元。
\end{proof}
\begin{Exercise}
  $n$次方程$x^n=1$的根称为$n$次单位根,所有$n$次单位根之集记为$U_n$。证明:$U_n$对通常的复数乘法构成一个群。
\end{Exercise}
\begin{proof}[证明]
  对任意的$x_1\in U_n$,$x_2\in U_n$,则$x^n_1=1$,$x_2^n=1$,从而$(x_1x_2)^n=1$,即$x_1x_2\in U_n$。

  结合律显然成立。

  $1^n=1$,从而$1\in U_n$。 $\forall x\in U_n, 1\cdot x = x$,这说明$1$为左单位元。

  $\forall x\in U_n$,则$x^n=1$,从而$(\frac{1}{x})^n=1$,即$\frac{1}{x}\in U_n$。$\frac{1}{x}\cdot x =1$说明$\frac{1}{x}$为$x$关于左单位元$1$的左逆元。

  以上验证了$U_n$对通常的复数乘法构成一个群。
\end{proof}

\begin{Exercise}
 令
 \[G=\bigg\{\begin{bmatrix}
  1&0\\0&1
 \end{bmatrix},
 \begin{bmatrix}
  -1&0\\0&-1
 \end{bmatrix},
 \begin{bmatrix}
  1&0\\0&-1
 \end{bmatrix},
 \begin{bmatrix}
  -1&0\\0&1
 \end{bmatrix}\bigg\}\] 
 试证:$G$对矩阵乘法构成一个群。
\end{Exercise}

\begin{proof}[证明]
  易验证$G$对矩阵的乘法满足封闭性。

  结合律显然成立。

  $\begin{bmatrix}
    1&0\\0&1
   \end{bmatrix}$为左单位元。

   $\begin{bmatrix}
    1&0\\0&1
   \end{bmatrix}$的左逆元为
   $\begin{bmatrix}
    1&0\\0&1
   \end{bmatrix}$,
   $\begin{bmatrix}
    -1&0\\0&-1
   \end{bmatrix}$的左逆元为
   $\begin{bmatrix}
    -1&0\\0&-1
   \end{bmatrix}$,
   $\begin{bmatrix}
    1&0\\0&-1
   \end{bmatrix}$的左逆元为$\begin{bmatrix}
    1&0\\0&-1
   \end{bmatrix}$,
   $\begin{bmatrix}
    -1&0\\0&1
   \end{bmatrix}$的左逆元为$\begin{bmatrix}
    -1&0\\0&1
   \end{bmatrix}$。

   以上验证了$G$对矩阵的乘法构成一个群。
  
\end{proof}

%%% Local Variables:
%%% mode: latex
%%% TeX-master: "chapter1"
%%% End:

\end{CJK*}
\end{document}





%%% Local Variables:
%%% mode: latex
%%% TeX-master: t
%%% End:



