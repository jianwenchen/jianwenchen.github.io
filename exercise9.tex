\documentclass{article}
\usepackage{tikz}
\usepackage{CJKutf8}
\usepackage{amsmath}
\usepackage{amsthm}
\begin{document}
\begin{CJK}{UTF8}{gbsn}
\newtheorem{Exercise}{习题}
\begin{Exercise}
    设$G$为一个有$k$个支的平面图。如果$G$的顶点数、边数、面数分别为$p$,$q$和$f$,试证:
  \[p-q+f=k+1\]
\end{Exercise}
\vspace{10cm}
\begin{Exercise}
    如果$G$为顶点数$p\geq 11$的可平面图,试证$G^c$不是可平面图。
\end{Exercise}
\vspace{10cm}
\begin{Exercise}
    不存在$7$条棱的凸多面体。
\end{Exercise}
\vspace{10cm}
\begin{Exercise}
    设$G$为一个没有三角形的可平面图。证明$G$中存在一个顶点$v$使得$\deg v \leq 3$。
\end{Exercise}
\vspace{10cm}
\begin{Exercise}
    设$G$为一个没有三角形的可平面图。应用数学归纳法证明$G$为4-可着色的。
\end{Exercise}
\end{CJK}
\end{document}


%%% Local Variables:
%%% mode: latex
%%% TeX-master: t
%%% End:
