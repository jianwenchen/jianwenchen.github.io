\documentclass{article}
\usepackage{CJKutf8}
\usepackage{amsmath}
\usepackage{amssymb}
\usepackage{amsfonts}
\usepackage{amsthm}
\usepackage{titlesec}
\usepackage{titletoc}
\usepackage{xCJKnumb}
\usepackage{tikz}
\usepackage{mathrsfs}
\usepackage{indentfirst}

\newtheorem{Def}{定义}
\newtheorem{Thm}{定理}
\newtheorem{Exercise}{练习}

\newtheorem*{Example}{例}


\begin{document}
\begin{CJK*}{UTF8}{gbsn}

  课后作业题

  \begin{Exercise}
    设$(S,\circ)$为一个代数系,如果二元代数运算“$\circ$”满足结合律和交换律,则$\forall a_i\in S$,$i=1,2,\cdots,n$,$n$个元素$a_1,a_2,\cdots,a_n$的乘积仅与这$n$个元素有关而与它们的次序无关。
  \end{Exercise}

\begin{proof}[证明]
    设$\pi$为从集合$\{1,2,\cdots,n\}$到$\{1,2,\cdots,n\}$的一个双射,
    以下用数学归纳法证明$a_{\pi(1)}\circ a_{\pi(2)}\circ \cdots \circ a_{\pi(n)}=(((a_1\circ a_2)\circ a_3)\circ \cdots )\circ a_n$。

    这里$a_{\pi(1)}\circ a_{\pi(2)}\circ \cdots \circ a_{\pi(n)}$表示按照$a_{\pi(1)},a_{\pi(2)}, \cdots, a_{\pi(n)}$的次序进行“$\circ$”运算时任意加括号所得到的运算结果。

    当$n=1$时,结论显然成立。

    假设当$n=k$时结论成立,往证当$n=k+1$时,结论也成立。

    设$\pi(i)=k+1$,则
    \begin{align*}
        &a_{\pi(1)}\circ a_{\pi(2)}\circ \cdots \circ a_{\pi(k+1)}\\
        =&((((a_{\pi(1)}\circ a_{\pi(2)})\circ a_{\pi(3)}) \cdots )\circ a_{\pi(i-1)})\circ (a_{\pi(i)} \circ ((((a_{\pi(i+1)}\circ a_{\pi(i+2)})\circ a_{\pi(i+3)})\circ \cdots )\circ a_{\pi(k+1)}))\\
        =&((((a_{\pi(1)}\circ a_{\pi(2)})\circ a_{\pi(3)}) \cdots )\circ a_{\pi(i-1)})\circ (((((a_{\pi(i+1)}\circ a_{\pi(i+2)})\circ a_{\pi(i+3)})\circ \cdots )\circ a_{\pi(k+1)})\circ a_{\pi(i)} )\\
        =&((((a_{\pi(1)}\circ a_{\pi(2)})\circ a_{\pi(3)}) \cdots )\circ a_{\pi(i-1)})\circ ((((a_{\pi(i+1)}\circ a_{\pi(i+2)})\circ a_{\pi(i+3)})\circ \cdots )\circ a_{\pi(k+1)}))\circ a_{\pi(i)}\\
        =&((((a_1\circ a_2)\circ a_3)\circ \cdots \circ) a_{k})\cdots \circ a_{k+1}
    \end{align*}
\end{proof}
\end{CJK*}
\end{document}





%%% Local Variables:
%%% mode: latex
%%% TeX-master: t
%%% End:
