\documentclass{article}
\usepackage{CJKutf8}
\usepackage{amsmath}
\usepackage{amssymb}
\usepackage{amsfonts}
\usepackage{amsthm}
\usepackage{titlesec}
\usepackage{titletoc}
\usepackage{xCJKnumb}
\usepackage{tikz}
\usepackage{mathrsfs}
\usepackage{indentfirst}
\usepackage{enumitem}
\newtheorem{Def}{定义}
\newtheorem{Thm}{定理}
\newtheorem{Exercise}{练习}

\newtheorem*{Example}{例}
\setlist[enumerate,1]{label=(\arabic*)}

\begin{document}
\begin{CJK*}{UTF8}{gbsn}
  \title{第二讲 范式}
  \author{陈建文}
  \maketitle
  % \tableofcontents

  \section{范式}
\begin{Def}[文字]
一个命题符号或者其否定称为一个文字。
\end{Def}
\begin{Example}
  $p$,$\lnot p$都是文字。
\end{Example}
\begin{Def}[合取式]
  $B_1\land B_2 \land \cdots \land B_n(n\geq 1)$称为一个合取式,其中$B_i (1\leq i \leq n)$为一个文字。单独的一个文字也称为一个合取式。
\end{Def}
\begin{Example}
  $p\land \lnot q$,$\lnot p\land \lnot q$都是合取式。
\end{Example}
\begin{Def}[析取式]
  $B_1\lor B_2 \lor \cdots \lor B_n(n\geq 1)$称为一个析取式,其中$B_i (1\leq i \leq n)$为一个文字。单独的一个文字也称为一个析取式。
\end{Def}
\begin{Example}
  $p\lor \lnot q$,$\lnot p\lor \lnot q$都是析取式。
\end{Example}
\begin{Def}[合取范式]
$A_1\land A_2\land \cdots \land A_n(n\geq 1)$称为一个合取范式,其中$A_i(1\leq i \leq n)$为析取式,称为一个合取项。
\end{Def}
\begin{Example}
  $(\lnot p\lor q)\land(r \lor s)$,$\lnot p\lor r \lor s$都是合取范式。
\end{Example}
\begin{Def}[析取范式]
  $A_1\lor A_2\lor \cdots \lor A_n(n\geq 1)$称为一个析取范式,其中$A_i(1\leq i \leq n)$为合取式,称为一个析取项。
  \end{Def}
  \begin{Example}
    $(\lnot p\land q)\lor(r \land s)$,$\lnot p\land r \land s$都是析取范式。
  \end{Example}
\begin{Example}
  求公式$(p\land q)\to (\lnot q \land r)$的合取范式和析取范式。
\end{Example}
\begin{proof}[解]
  \begin{equation*}
    \begin{split}
      &(p\land q)\to (\lnot q \land r)\\
      \Leftrightarrow&\lnot(p\land q)\lor (\lnot q \land r)\\ 
      \Leftrightarrow&\lnot p\lor \lnot q \lor (\lnot q \land r)\quad\text{析取范式}\\
      \Leftrightarrow&(\lnot p\lor \lnot q \lor \lnot q)\land (\lnot p\lor \lnot q \lor   r)\\
      \Leftrightarrow&(\lnot p\lor \lnot q)\land (\lnot p\lor \lnot q \lor   r)\quad \text{合取范式}\\
      \Leftrightarrow&\lnot p\lor \lnot q \quad \text{既是析取范式又是合取范式}
    \end{split}
  \end{equation*}
\end{proof}

合取范式和析取范式的求解过程:

\begin{enumerate}
  \item 消去“$\leftrightarrow$”:$A\leftrightarrow B\Leftrightarrow (A\to B)\land(B\to A)$
  \item 消去“$\to$”:$A\to B\Leftrightarrow \lnot A\lor B$
  \item 进行公式变形:$\lnot (A\lor B)\Leftrightarrow \lnot A\land \lnot B,\lnot (A\land B)\Leftrightarrow \lnot A\lor \lnot B,\lnot\lnot A \Leftrightarrow A,A\land (B\lor C)\Leftrightarrow (A\land B)\lor (A\land C),A\lor (B\land C)\Leftrightarrow (A\lor B)\land (A\lor C)$
  \item 公式化简:$A\lor A\Leftrightarrow A,A\land A\Leftrightarrow A$
\end{enumerate}

  \section{主范式}

\begin{Def}[主合取范式]
  设命题公式$A(p_1,p_2,\cdots, p_n)$的合取范式为$A_1\land A_2\land \cdots \land A_k(k\geq 1)$,如果其中每一个合取项$A_j(1\leq j \leq k)$的形式为$A_j=Q_1\lor Q_2\lor \cdots \lor Q_n$,这里$Q_i=p_i$或者$\lnot p_i(1\leq i \leq n)$,则称$A_1\land A_2\land \cdots \land A_k(k\geq 1)$为$A$的主合取范式。
\end{Def}  
\begin{Def}[主析取范式]
  设命题公式$A(p_1,p_2,\cdots, p_n)$的析取范式为$A_1\lor A_2\lor \cdots \lor A_k(k\geq 1)$,如果其中每一个析取项$A_j(1\leq j \leq k)$的形式为$A_j=Q_1\land Q_2\land \cdots \land Q_n$,这里$Q_i=p_i$或者$\lnot p_i(1\leq i \leq n)$,则称$A_1\lor A_2\lor \cdots \lor A_k(k\geq 1)$为$A$的主析取范式。
\end{Def} 
  \begin{Example}
    求公式$(p\land q)\to (\lnot q \land r)$的主合取范式和主析取范式。
  \end{Example}
  \begin{proof}[解法一]
    \begin{equation*}
      \begin{split}
        &(p\land q)\to (\lnot q \land r)\\
        \Leftrightarrow&\lnot p\lor \lnot q\\
        \Leftrightarrow&\lnot p\lor \lnot q \lor (r\land \lnot r)\\
        \Leftrightarrow&(\lnot p\lor \lnot q \lor r)\land (\lnot p\lor \lnot q \lor \lnot r)\quad\text{主合取范式}\\
        &(p\land q)\to (\lnot q \land r)\\
        \Leftrightarrow&\lnot p\lor \lnot q\\
        \Leftrightarrow&(\lnot p \land (q\lor \lnot q))\lor ((p\lor \lnot p)\land \lnot q)\\
        \Leftrightarrow&(\lnot p \land q)\lor(\lnot p \land \lnot q)\lor (p\land \lnot q)\lor (\lnot p\land \lnot q)\\
        \Leftrightarrow&(\lnot p \land q)\lor(\lnot p \land \lnot q)\lor (p\land \lnot q)\\
        \Leftrightarrow&(\lnot p \land q \land (r\lor \lnot r))\lor(\lnot p \land \lnot q\land (r\lor \lnot r))\lor (p\land \lnot q\land (r\lor \lnot r))\\
        \Leftrightarrow&(\lnot p \land q \land r)\lor(\lnot p \land q \land \lnot r)\\
        &\lor(\lnot p \land \lnot q\land r)\lor( \lnot p \land \lnot q\land\lnot r)\\
        &\lor (p\land \lnot q\land r)\lor(p\land \lnot q\land \lnot r)\quad \text{主析取范式}\\
      \end{split}
    \end{equation*}
  \end{proof}
  \begin{proof}[解法二]$\quad$

    \begin{tabular}{ccc|c}
      $p$& $q$& $r$& $(p\land q)\to (\lnot q \land r)$\\
      \hline
     T& T&T&F\\
      T&T&F&F\\
      T&F&T&T\\
       T& F&F&T\\
      F&T&T&T\\
      F&T&F&T\\
     F& F&T&T\\
      F&  F&F&T\\      
    \end{tabular}

    主析取范式为$(p\land \lnot q\land r)\lor(p\land \lnot q\land \lnot r)\lor(\lnot p \land q \land r)\lor(\lnot p \land q \land \lnot r)
    \lor(\lnot p \land \lnot q\land r)\lor( \lnot p \land \lnot q\land\lnot r)
     $

    主合取范式为$ (\lnot p\lor \lnot q \lor \lnot r)\land (\lnot p\lor \lnot q \lor r)$
  \end{proof}
  课后作业题:

  \begin{Exercise}
    求下列公式的主合取范式与主析取范式:
    \begin{enumerate}
      \item $p\to (p\land q)$
      \item $(p\to q)\to (q\to r)$
      \item $(p\to (p\land q))\lor r$
    \end{enumerate}
  \end{Exercise}
\end{CJK*}
\end{document}





%%% Local Variables:
%%% mode: latex
%%% TeX-master: t
%%% End: