\documentclass{article}
\usepackage{CJKutf8}
\usepackage{amsmath}
\usepackage{amssymb}
\usepackage{amsfonts}
\usepackage{amsthm}
\usepackage{titlesec}
\usepackage{titletoc}
\usepackage{xCJKnumb}
\usepackage{tikz}
\usepackage{mathrsfs}
\usepackage{indentfirst}
\usepackage{enumitem}
\newtheorem{Def}{定义}
\newtheorem{Thm}{定理}
\newtheorem{Exercise}{练习}

\newtheorem*{Example}{例}
\newtheorem*{thm}{定理}
\setlist[enumerate,1]{label=(\arabic*)}

\begin{document}
\begin{CJK*}{UTF8}{gbsn}
  \title{第五讲 命题逻辑演算形式系统}
  \author{陈建文}
  \maketitle
  % \tableofcontents
  \section{命题逻辑演算形式系统(PC)的组成}

  PC:Propositional Calculus

    1. 字符集
    \begin{enumerate}
      \item 表示命题的符号:$p_1,p_2,\cdots,p_n,\cdots$
      \item 联结词完备集:$\{\lnot, \to\}$
      \item 辅助符号:$()$
    \end{enumerate}
   
    2. 命题公式:
    \begin{enumerate}
      \item 任意一个表示命题的符号为命题公式;
      \item 若A,B是命题公式,则$(\lnot A), (A\to B)$是命题公式;
      \item 有限次使用(1)与(2)所得到的结果均是命题公式。
    \end{enumerate}

    3. 公理  
  
    $A_1:A\to(B\to A)$

    $A_2:(A\to(B\to C))\to((A\to B)\to (A\to C))$

    $A_3:(\lnot A\to \lnot B)\to (B \to A)$

    4. 推理规则 
    
    $r_{mp}:A,A\to B, B$

    5. 定理推导
    

    证明:称下列公式序列为公式$A$在$PC$中的一个证明:

    \[A_1,A_2,\cdots,A_m(=A)\]

    其中$A_i(i=1,2,\cdots,m)$或为$PC$的公理,或为$A_j(j<i)$,或为$A_j,A_k(j,k<i)$使用$r_{mp}$导出的公式。


    定理:如果公式$A$在$PC$中有一个证明序列,则称$A$为$PC$的定理,记为$\vdash_{PC}A$,简记为$\vdash A$。


    演绎:设$\Gamma$为$PC$中若干公式构成的公式集,称下列公式序列为公式$A$以$\Gamma$为前提的演绎:
    \[A_1,A_2,\cdots, A_m(=A)\]
    其中$A_i(i=1,2,\cdots,m)$或为$PC$的公理,或为$\Gamma$中的成员,或为$A_j(j<i)$,或为$A_j$,$A_k(j,k<i)$使用$r_{mp}$导出的公式。


  如果$\Gamma=\{B\}$,则$\Gamma\vdash A$简记为$B\vdash A$,表示公式$A$可由前提$B$在$PC$中演绎出来。如果此时还有$A\vdash B$,则称公式$A$和$B$演绎等价,记为$A\vdash \dashv B$。
  \begin{Thm}$\vdash A\to A$\end{Thm}
  \begin{proof}[证明]$\quad$

    \begin{enumerate}
      \item $(A\to ((B\to A) \to A))\to ((A\to (B\to A)) \to (A\to A)) \quad\quad A_2$
      \item $A\to ((B\to A) \to A) \quad\quad A_1$
      \item $(A\to (B\to A)) \to (A\to A) \quad\quad (1)(2)r_{mp}$
      \item $A\to (B\to A) \quad\quad A_1$
      \item $A\to A \quad\quad (3)(4)r_{mp}$  
    \end{enumerate}
  \end{proof}

  \begin{Thm}若$\Gamma\vdash P$,则$\Gamma\vdash A\to P$。\end{Thm}
  \begin{proof}[证明]$\quad$
    \begin{enumerate}
      \item $P \quad\quad$ 
      \item $P\to (A\to P) \quad\quad A_1$
      \item $A\to P \quad\quad (1)(2)r_{mp}$
    \end{enumerate}
  \end{proof}

  \begin{Thm}若$\Gamma\vdash A\to B$,$\Gamma \vdash B\to C$,则$\Gamma\vdash A\to C$\end{Thm}
  \begin{proof}[证明]$\quad$
    \begin{enumerate}
      \item $\Gamma\vdash B \to C\qquad$已知条件
      \item $\Gamma\vdash A\to (B\to C)\qquad$定理
      \item $\Gamma\vdash (A\to (B\to C)) \to ((A\to B)\to (A\to C))\quad \quad A_2$
      \item $\Gamma \vdash(A\to B)\to (A\to C) \quad\quad(2)(3)r_{mp}$
      \item $\Gamma \vdash A\to B\qquad$已知条件
      \item $\Gamma \vdash A\to C \quad\quad(4)(5)r_{mp}$
    \end{enumerate}
  \end{proof}

  \begin{Thm}$\vdash \lnot A\to (A\to B)$
  \end{Thm}
  \begin{proof}[证明]$\quad$
    \begin{enumerate}
      \item $ \lnot A \to (\lnot B \to \lnot A) \quad\quad A_1$ 
      \item $ (\lnot B \to \lnot A)\to (A\to B)\quad\quad A_3$
      \item  $\lnot A \to (A\to B)\qquad$定理 
    \end{enumerate}
  \end{proof}

  \begin{Thm}$\lnot \lnot A\vdash A$\end{Thm}
  \begin{proof}[证明]$\quad$
    \begin{enumerate}
      \item $\lnot \lnot A\qquad$(前提)
      \item $\lnot \lnot A\to (\lnot \lnot \lnot \lnot A \to \lnot \lnot A)\qquad A_1$
      \item $\lnot \lnot \lnot \lnot A \to \lnot \lnot A\qquad (1)(2)r_{mp}$
      \item $(\lnot \lnot \lnot \lnot A \to \lnot \lnot A) \to ( \lnot A \to \lnot \lnot \lnot A) \quad \quad A_3$
      \item $ \lnot A \to \lnot \lnot \lnot A \quad\quad (3)(4)r_{mp}$
      \item  $( \lnot A \to \lnot \lnot \lnot A)\to (\lnot \lnot A \to A) \quad \quad A_3$
      \item $\lnot \lnot A \to A \quad\quad (5)(6)r_{mp}$
      \item $A\qquad(1)(7)r_{mp}$
    \end{enumerate}
  \end{proof}

  \begin{Thm}$\vdash(B\to C)\to ((A\to B)\to (A\to C))$\end{Thm}
  \begin{proof}[证明]$\quad$
    \begin{enumerate}
      \item $(A\to (B\to C)) \to ((A\to B)\to (A\to C)) \quad\quad A_2$
      \item $(B\to C) \to (A\to (B\to C)) \quad\quad A_1$
      \item $(B\to C)\to ((A\to B)\to (A\to C))\qquad (1)(2)\text{定理}$
    \end{enumerate}
  \end{proof}

  \begin{Thm}$\vdash (A\to B) \to ((B\to C)\to (A\to C))$\end{Thm}
  \begin{proof}[证明]$\quad$
    \begin{enumerate}
      \item  $(B\to C)\to ((A\to B)\to (A\to C))\qquad \text{定理}$
      \item $((B\to C)\to (A\to B))\to ((B\to C)\to (A\to C))\quad\quad (1)A_2 r_{mp}$
      \item $(A\to B) \to ((B\to C)\to (A\to B)) A_1$
      \item $(A\to B) \to ((B\to C)\to (A\to C)) \quad\quad(2)(3)\text{定理}$
    \end{enumerate}
  \end{proof}

  \begin{Thm}$\vdash (A\to(B\to C))\to (B\to (A\to C))$\end{Thm}
  \begin{proof}[证明]$\quad$
    \begin{enumerate}
      \item $(A\to(B\to C))\to ((A\to B)\to (A\to C))\quad\quad A_2$
      \item $B\to (A\to B) \quad\quad A_1$
      \item $(B\to (A\to B)) \to (((A\to B)\to (A\to C))\to (B\to (A\to C)))\qquad\text{定理}$
      \item $((A\to B)\to (A\to C))\to (B\to (A\to C))\quad\quad (2)(3)r_{mp}$
      \item $(A\to(B\to C))\to (B\to (A\to C))\quad\quad (1)(4)\text{定理}$
    \end{enumerate}
  \end{proof}

  \begin{Thm}$\vdash(\lnot A\to A)\to A$\end{Thm}
  \begin{proof}[证明]$\quad$
    \begin{enumerate}
      \item $\lnot A \to (A \to \lnot(\lnot A\to A))\qquad$定理
      \item $(\lnot A\to A)\to (\lnot A \to \lnot(\lnot A\to A))\qquad (1)A_2r_{mp}$
      \item $(\lnot A \to \lnot(\lnot A\to A))\to ((\lnot A\to A)\to A)\qquad A_3$
      \item  $(\lnot A\to A)\to ((\lnot A\to A)\to A)\qquad (2)(3)$定理
      \item $((\lnot A\to A)\to (\lnot A\to A))\to ((\lnot A\to A)\to A)\qquad (4)A_2r_{mp}$
      \item $(\lnot A\to A)\to (\lnot A\to A)\qquad$定理
      \item $(\lnot A\to A)\to A\qquad (5)(6)r_{mp}$
    \end{enumerate}
  \end{proof}

  \begin{Thm}$\vdash \lnot\lnot A\to A$\end{Thm}
  \begin{proof}[证明]$\quad$
    \begin{enumerate}
      \item $(\lnot A\to A)\to A\qquad$定理
      \item $(\lnot \lnot A)\to ((\lnot A\to A)\to A)\qquad(1)$定理
      \item $(\lnot \lnot A\to (\lnot A\to A))\to (\lnot \lnot A\to A)\qquad (2)A_2r_{mp}$
      \item $\lnot \lnot A\to (\lnot A\to A)\qquad$定理
      \item $\lnot\lnot A\to A\qquad (3)(4)r_{mp}$
    \end{enumerate}
  \end{proof}

  \begin{Thm}$\vdash A\to \lnot\lnot A$\end{Thm}
  \begin{proof}[证明]$\quad$
    \begin{enumerate}
      \item $(\lnot\lnot\lnot A \to \lnot A)\to (A\to \lnot\lnot A)\qquad A_3$
      \item $\lnot\lnot\lnot A \to \lnot A\qquad$定理
      \item $A\to \lnot\lnot A\qquad (1)(2)r_{mp}$
    \end{enumerate}
  \end{proof}

  \begin{Thm}$\vdash (A\to \lnot B)\to (B\to \lnot A)$\end{Thm}
  \begin{proof}[证明]$\quad$
    \begin{enumerate}
      \item $(\lnot \lnot A \to A)\to ((A\to \lnot B)\to (\lnot\lnot A\to \lnot B))$
      \item $\lnot \lnot A \to A$
      \item $(A\to \lnot B)\to (\lnot\lnot A\to \lnot B) (1)(2)r_{mp}$
      \item $(\lnot\lnot A\to \lnot B)\to (B\to \lnot A)\qquad A_3$
      \item $(A\to \lnot B)\to (B\to \lnot A)$
    \end{enumerate}
  \end{proof}

  \begin{Thm}$\vdash (A\to B)\to (\lnot B\to \lnot A)$\end{Thm}
  \begin{proof}[证明]$\quad$
    \begin{enumerate}
      \item $(B\to \lnot\lnot B)\to ((A\to B)\to (A\to \lnot\lnot B))$
      \item $B\to \lnot\lnot B$
      \item $(A\to B)\to (A\to \lnot\lnot B)\qquad (1)(2)r_{mp}$
      \item $(A\to \lnot\lnot B)\to (\lnot B\to \lnot A)$
      \item $(A\to B)\to (\lnot B\to \lnot A)$
    \end{enumerate}
  \end{proof}

  \begin{Thm}$\vdash (\lnot A \to B)\to (\lnot B\to A)$\end{Thm}
  \begin{proof}[证明]$\quad$
    \begin{enumerate}
      \item $(B\to \lnot\lnot B)\to ((\lnot A\to B)\to (\lnot A\to \lnot\lnot B))$
      \item $B\to \lnot\lnot B$
      \item $(\lnot A\to B)\to (\lnot A\to \lnot\lnot B) (1)(2)r_{mp}$
      \item $(\lnot A\to \lnot\lnot B)\to (\lnot B\to A)\qquad A_3$
      \item $(\lnot A \to B)\to (\lnot B\to A)$
    \end{enumerate}
  \end{proof}

  \begin{thm}
    设$A$,$B$为命题公式,且满足$\vdash A\to B$,$\vdash B\to A$,公式$D$是将公式$C$中$A$的某次出现替换为公式$B$所得到的公式,则$\vdash C\to D$,$\vdash D\to C$。
  \end{thm}
  \begin{proof}[证明]
    根据定义,每个命题公式都有一个形成规则,例如公式$\lnot(A\to B)$可以如下形成:
    \begin{enumerate}
      \item A是命题公式;
      \item B是命题公式;
      \item $(A\to B)$是命题公式;
      \item $(\lnot(A\to B))$是命题公式。
    \end{enumerate}
    这里我们称$\lnot(A\to B)$可以由$4$步形成。

    以下对命题公式$C$形成的步数$n$归纳证明结论成立。

    当$n=1$时,此时$C=A$,$D=B$,由$\vdash A\to B$,$\vdash B\to A$知$\vdash C\to D$,$\vdash D\to C$。

    假设当$n<k$时结论成立,往证当$n=k$时结论也成立。

    如果$C=\lnot C_1$,此时如果$C=A$,则$D=B$,由$\vdash A\to B$,$\vdash B\to A$知$\vdash C\to D$,$\vdash D\to C$。如果$C\neq A$,假设$C_1$中对应$A$的出现替换为$B$后所得到的公式为$D_1$,则$D=\lnot D_1$。
由归纳假设$\vdash C_1\to D_1, \vdash D_1\to C_1$。
从以下证明序列知
\begin{enumerate}
  \item $C_1\to D_1$
  \item $D_1\to C_1$
  \item $(C_1\to D_1) \to (\lnot D_1\to \lnot C_1)\qquad$定理
  \item $\lnot D_1\to \lnot C_1\qquad (1)(3)r_{mp}$
  \item $(D_1\to C_1) \to (\lnot C_1\to \lnot D_1)\qquad$定理
  \item $\lnot C_1\to \lnot D_1\qquad (2)(5)r_{mp}$
\end{enumerate}
$\vdash C\to D$,$\vdash D\to C$。

如果$C=C_1\to C_2$,此时如果$C=A$,则$D=B$,由$\vdash A\to B$,$\vdash B\to A$知$\vdash C\to D$,$\vdash D\to C$。如果$C\neq A$,假设$C_1$中对应$A$的出现替换为$B$后所得到的公式为$D_1$,则$D=D_1\to D_2$,这里$D_2=C_2$。由归纳假设,$\vdash C_1\to D_1, \vdash D_1\to C_1$,同时$\vdash C_2\to D_2, \vdash D_2\to C_2$也成立。同理,假设$C_2$中对应$A$的出现替换为$B$后所得到的公式为$D_2$,则$D=D_1\to D_2$,这里$D_1=C_1$。此时亦有$\vdash C_1\to D_1, \vdash D_1\to C_1$,$\vdash C_2\to D_2, \vdash D_2\to C_2$。
从以下证明序列知

\begin{enumerate}
  \item $D_1\to C_1$
  \item $(D_1\to C_1)\to ((C_1\to C_2)\to (D_1\to C_2))\qquad$定理
  \item $(C_1\to C_2)\to (D_1\to C_2)\qquad (1)(2)r_{mp}$
  \item $C_2\to D_2$
  \item $(C_2\to D_2)\to ((D_1\to C_2)\to (D_1\to D_2))\qquad$定理
  \item $(D_1\to C_2)\to (D_1\to D_2)\qquad (4)(5)r_{mp}$
  \item $(C_1\to C_2)\to (D_1\to D_2)\qquad $定理
\end{enumerate}
$\vdash C\to D$,同理可证$\vdash D\to C$。
  \end{proof}
  \begin{Thm}$\vdash (A\to C)\to ((B\to C)\to ((\lnot A\to B)\to C))$\end{Thm}
  \begin{proof}[证明]$\quad$
    \begin{enumerate}
      \item $(\lnot A\to B)\to (\lnot A\to B)\qquad$定理
      \item $((\lnot A\to B)\to (\lnot A\to B))\to (\lnot A\to ((\lnot A\to B)\to B))\qquad$定理
      \item $\lnot A\to ((\lnot A\to B)\to B)\qquad (1)(2)r_{mp}$
      \item $\lnot A\to (\lnot B\to \lnot(\lnot A\to B))\qquad (3)\text{定理}r_{mp}$
      \item $\lnot C\to (\lnot A\to (\lnot B\to \lnot(\lnot A\to B)))\qquad$定理
      \item $(\lnot C\to \lnot A)\to(\lnot C\to  (\lnot B\to \lnot(\lnot A\to B)))\qquad(5)A_2r_{mp}$
      \item $(\lnot C\to  (\lnot B\to \lnot(\lnot A\to B)))\to ((\lnot C\to \lnot B)\to (\lnot C \to \lnot (\lnot A\to B)))\qquad A_2$
      \item $(\lnot C\to \lnot A)\to((\lnot C\to \lnot B)\to (\lnot C \to \lnot (\lnot A\to B)))\qquad$定理
      \item $(A\to C)\to ((B\to C)\to ((\lnot A\to B)\to C))\qquad (8)\text{定理}r_{mp}$
    \end{enumerate}
  \end{proof}
\begin{thm}设$\Gamma$为任意公式的集合,$A$,$B$为任意两个公式,则$\Gamma,A\vdash B$当且仅当$\Gamma \vdash A\to B$。
\end{thm}
\begin{proof}[证明]充分性:由$\Gamma\vdash A\to B$知从前提集$\Gamma$出发,在$PC$中能够得到公式$A\to B$的一个演绎序列,即$A_1,A_2,\cdots, A_n(=A\to B)$,则从$\Gamma\cup \{A\}$出发,可以得到如下的演绎序列:$A_1,A_2,\cdots,A_n(=A\to B), A, B$,即第$n+2$步的结论可由已知的第$n$步的结论$A\to B$加上第$n+1$步的已知前提条件$A$通过$r_{mp}$所得,即$\Gamma,A\vdash B$。

  必要性:由$\Gamma,A\vdash B$知从前提集$\Gamma\cup \{A\}$出发,在$PC$中能够得到公式$B$的一个演绎序列,即$B_1,B_2,\cdots,B_k(=B)$,下面通过数学归纳法来证明$\Gamma\vdash A\to B$,施归纳于此演绎序列的长度$k$:

  (1)当$k=1$时,根据演绎的定义知此时$B$或者为公理,或者$B\in \Gamma\cup\{A\}$。
  如果$B$为公理,则从前提集$\Gamma$出发存在如下的演绎序列:

  $B(\text{公理}),B\to(A\to B)(\text{公理}),A\to B(r_{mp})$

  即$\Gamma\vdash A\to B$;

  如果$B\in \Gamma$,则从前提集$\Gamma$出发存在如下的演绎序列:

  $B(\text{前提}),B\to(A\to B)(\text{公理}),A\to B(r_{mp})$

  即$\Gamma\vdash A\to B$;

  如果$B\in \{A\}$,则$B=A$,此时$A\to B$即为$A\to A$,而$A\to A$为$PC$中已证的定理,从而$\Gamma\vdash A\to A$。

  (2)假设当$k<n$时结论成立,即对上述演绎序列中的公式$B_i(i<n)$,均有$\Gamma\vdash A\to B_i$。则当$k=n$时,$B_k=B_n=B$或为公理,或者$B\in \Gamma\cup \{A\}$,或由$B_i,B_j(i,j<n)$通过$r_{mp}$所得。如果此时$B_k=B_n=B$为公理,或者$B\in \Gamma\cup\{A\}$,则讨论情况同$(1)$;如果$B$由$B_i$,$B_j(i,j<n)$通过$r_{mp}$导出,则不妨设$B_j=B_i\to B$,根据归纳假设,
  \[\Gamma\vdash A\to B_i, \Gamma\vdash A\to B_j\]
  即
  \[\Gamma\vdash A\to B_i, \Gamma\vdash A\to (B_i\to B)\]
  又
  \[(A\to(B_i\to B))\to ((A\to B_i)\to (A\to B))\]

  所以
  \[\Gamma\vdash (A\to B_i)\to(A\to B)\]
  从而
  \[\Gamma\vdash A\to B\]
\end{proof}
\begin{Example}
  利用演绎定理证明

  \[\vdash (A\to (B\to C))\to ((C\to D)\to (A\to (B\to D)))\]
\end{Example}
\begin{proof}[证明]
根据演绎定理只需证:

$A\to (B\to C)\vdash (C\to D)\to (A\to (B\to D))$

只需证:

$A\to (B\to C),(C\to D)\vdash (A\to (B\to D))$

只需证:

$A\to (B\to C),(C\to D), A\vdash (B\to D)$

只需证:

$A\to (B\to C),(C\to D), A, B\vdash D$

\begin{enumerate}
  \item $A\qquad$前提
  \item $A\to (B\to C)\qquad$前提
  \item $B\to C\qquad (1)(2)r_{mp}$
  \item $B\qquad$前提
  \item $C\qquad(3)(4)r_{mp}$
  \item $C\to D$前提
  \item $D\qquad(5)(6)r_{mp}$
\end{enumerate}
\end{proof}
\begin{Example}
  利用演绎定理证明

  \[\vdash ((A\to B)\to (A\to C))\to (A\to (B\to C))\]
\end{Example}
\begin{proof}[证明]

  根据演绎定理只需证:
  
  $(A\to B)\to (A\to C)\vdash (A\to (B\to C))$

  只需证:

  $(A\to B)\to (A\to C), A\vdash (B\to C)$

  只需证:

  $(A\to B)\to (A\to C), A, B\vdash C$

  \begin{enumerate}
    \item $B\qquad$前提
    \item $B\to (A\to B)\qquad A_1$
    \item $A\to B\qquad (1)(2)r_{mp}$
    \item $(A\to B)\to (A\to C)\qquad$前提
    \item $A\to C\qquad (3)(4)r_{mp}$
    \item $A\qquad$前提
    \item $C\qquad(5)(6)r_{mp}$
  \end{enumerate}
\end{proof}
  


\begin{Example}
  利用演绎定理证明

  \[\vdash (A\to C)\to ((B\to C)\to ((\lnot A\to B)\to C))\]
\end{Example}

\begin{proof}[证明]
根据演绎定理只需证:

$A\to C\vdash ((B\to C)\to ((\lnot A\to B)\to C))$

只需证:

$A\to C, B\to C\vdash ((\lnot A\to B)\to C)$

只需证:

$A\to C, B\to C, \lnot A\to B\vdash C$

\begin{enumerate}
  \item $\lnot A\to B\qquad$前提
  \item $B\to C\qquad$前提
  \item $\lnot A\to C\qquad$定理
  \item $(\lnot A\to C)\to (\lnot C\to A)\qquad$定理
  \item $\lnot C\to A\qquad(3)(4)r_{mp}$
  \item $A\to C\qquad$前提
  \item $\lnot C\to C(5)(6)$前提
  \item $(\lnot C\to C)\to C$
  \item $C\qquad (7)(8)r_{mp}$
\end{enumerate}
\end{proof}
课后作业题

  \begin{Exercise}
    在$PC$中证明下列事实:
  \end{Exercise}

$1.\vdash (A\to(A\to B))\to (A\to B)$




$2.\lnot A\vdash A\to B$



$3.A\to B, \lnot(B\to C)\to \lnot A\vdash A\to C$



$4.\vdash (A\to (B\to C))\to ((A\to (D\to B))\to (A\to (D\to C)))$



$5.\vdash (A\to (B\to C))\to ((C\to D)\to (A\to (B\to D)))$


$6.\vdash ((A\to B)\to C)\to (B\to C)$


$7.\vdash ((A\to B)\to (B\to A))\to (B\to A)$


$8.\vdash A\to ((A\to B)\to (C\to B))$

$9.\vdash ((A\to B)\to A)\to A$



$10.\vdash ((A\to B)\to C)\to ((C\to A)\to A)$



$11.\vdash ((A\to B)\to C)\to ((A\to C)\to C)$


$12.\vdash (((A\to B)\to C)\to D)\to((B\to D)\to (A\to D))$

$13.\vdash (A\to C)\to ((B\to C)\to (((A\to B)\to B)\to C))$

$14.\vdash (A\to C)\to ((B\to C)\to (((B\to A)\to A)\to C))$
\begin{Exercise}
  利用演绎定理在$PC$中证明:
\end{Exercise}

$1.\vdash (B\to A)\to (\lnot A\to \lnot B)$

$2.\vdash (A\to B)\to ((B\to C)\to (A\to C))$

$3.\vdash ((A\to B)\to A)\to A$

$4.\vdash \lnot (A\to B)\to (B\to A)$



\begin{Exercise}
  将$PC$中公理$A_3$改为$(\lnot A\to \lnot B)\to ((\lnot A\to B)\to A)$,记所得系统为$PC'$。证明:
\begin{enumerate}
  \item $\vdash_{PC} (\lnot A\to \lnot B)\to ((\lnot A\to B)\to A)$
  \item $\vdash_{PC'}(\lnot A\to \lnot B)\to (B\to A)$
\end{enumerate}
\end{Exercise}

\begin{Exercise}
  在$PC$中证明:

  (1)如果$\vdash A\to (B\to C),\vdash B$,则$\vdash A\to C$

  (2)如果$\Gamma, \lnot A\vdash B, \Gamma, \lnot A\vdash \lnot B$,则$\Gamma\vdash A$
\end{Exercise}

\begin{Exercise}
  证明$(\lnot A\to B)\to (A\to \lnot B)$不是$PC$的定理。
\end{Exercise}

\end{CJK*}
\end{document}





%%% Local Variables:
%%% mode: latex
%%% TeX-master: t
%%% End: