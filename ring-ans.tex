\documentclass{article}
\usepackage{CJKutf8}
\usepackage{amsmath}
\usepackage{amssymb}
\usepackage{amsfonts}
\usepackage{amsthm}
\usepackage{titlesec}
\usepackage{titletoc}
\usepackage{xCJKnumb}
\usepackage{tikz}
\usepackage{mathrsfs}
\usepackage{indentfirst}

\newtheorem{Def}{定义}
\newtheorem{Thm}{定理}
\newtheorem{Exercise}{练习}

\newtheorem*{Example}{例}


\begin{document}
\begin{CJK*}{UTF8}{gbsn}
  \title{第十讲 环的定义及简单性质}
  \author{陈建文}
  \maketitle

课后作业题:
\begin{Exercise}
  设$Z(\sqrt{2})=\{m+n\sqrt{2}|m,n\in Z\}$,其中$Z$为全体整数之集合。试证:$Z(\sqrt{2})$对数的通常加法和乘法构成一个环。
\end{Exercise}
\begin{proof}[证明]
  $\forall m_1,n_1,m_2,n_2\in Z$,$(m_1+n_1\sqrt{2})+(m_2+n_2\sqrt{2})=(m_1+m_2)+(n_1+n_2)\sqrt{2}$,$(m_1+n_1\sqrt{2})(m_2+n_2\sqrt{2})=(m_1m_2+2n_1n_2)+(m_1n_2+m_2n_1)\sqrt{2}$,
  这验证了加法和乘法满足封闭性。

  加法的结合律显然成立。

  加法的单位元为$0+0\sqrt{2}=0$。

  $\forall m,n\in Z$,$m+n\sqrt{2}$对加法的逆元为$(-m)+(-n)\sqrt{2}$。

  乘法的结合律,乘法对加法的分配律显然成立。
\end{proof}
\begin{Exercise}
  设$Q(\sqrt[3]{2})=\{a+b\sqrt[3]{2}|a,b\in Q\}$,其中$Q$为全体有理数之集合。试证:$Q(\sqrt[3]{2})$对数的通常加法和乘法不构成一个环。
\end{Exercise}
\begin{proof}[证明]
  $Q(\sqrt[3]{2})$对乘法不满足封闭性。否则如果$\sqrt[3]{2}\sqrt[3]{2}=a+b\sqrt[3]{2}$,则$\sqrt[3]{4}=a+b\sqrt[3]{2}$,
  从而$2=\sqrt[3]{2}\sqrt[3]{4}=\sqrt[3]{2}(a+b\sqrt[3]{2})=a\sqrt[3]{2}+b\sqrt[3]{4}=a\sqrt[3]{2}+b(a+b\sqrt[3]{2})=ab+(a+b^2)\sqrt[3]{2}$,
  于是$2-ab=(a+b^2)\sqrt[3]{2}$。此时如果$a+b^2=0$,则$2-ab=0$,可得$b^3=-2$,与$b$为有理数矛盾。如果$a+b^2\neq 0$,则$\sqrt[3]{2}=\frac{2-ab}{a+b^2}$,等式的左边是一个无理数,右边是一个有理数,也矛盾。
\end{proof}
\begin{Exercise}
  环$R$如果对于乘法有左单位元,则环$R$对于乘法的左单位元称为它的左单位元;环$R$如果对于乘法有单位元,则环$R$对于乘法的单位元称为它的单位元。设$e$为环$R$的唯一左单位元,试证$e$为$R$的单位元。
\end{Exercise}
\begin{proof}[证明]
  $\forall a,b\in R$,
  \[(ae-a+e)b=(ae)b-ab+eb=ab-ab+b=b\]
  从而$ae-a+e$也为$R$的左单位元。又由于$e$为环$R$的唯一左单位元,从而$ae-a+e=e$,于是$ae=a$,这说明$e$也为$R$的右单位元,从而为$R$的单位元。
\end{proof}
\begin{Exercise}
  设$(R,+,\circ)$为一个有单位元$1$的环,如果$R$中的元素$a$,$b$及$ab-1$均有逆元素,试证$a-b^{-1}$及$(a-b^{-1})^{-1}-a^{-1}$也有逆元素,并且
\[((a-b^{-1})^{-1}-a^{-1})^{-1}=aba-a\]
\end{Exercise}
\begin{proof}[证明]
  由$ab-1$可逆及$a$可逆知$(ab-1)a=aba-a$可逆。

  欲证
  \[((a-b^{-1})^{-1}-a^{-1})^{-1}=aba-a\]
  只需证
  \[((a-b^{-1})^{-1}-a^{-1})(aba-a)=1\]
  只需证
  \[(a-b^{-1})^{-1}(aba-a)-ba+1=1\]
  只需证
  \[(a-b^{-1})^{-1}(aba-a)=ba\]
  只需证
  \[(a-b^{-1})(ba)=aba-a\]
  如果$a-b^{-1}$可逆,该等式显然成立。

  以上证明了$(a-b^{-1})^{-1}-a^{-1}$为$aba-a$的左逆元,又由于$aba-a$可逆,因此$(a-b^{-1})^{-1}-a^{-1}$为$aba-a$的逆元。


  我们还需要证明$a-b^{-1}$可逆。

  由
  \[(a-b^{-1})(ba)=aba-a\]
  可得
  \[(a-b^{-1})=(ab-1)b^{-1}\]
这里$ab-1$和$b^{-1}$均可逆,从而$a-b^{-1}$可逆。
\end{proof}
\begin{Exercise}
  试证:有单位元素的环$R$中零因子没有逆元素。
\end{Exercise}
\begin{proof}[证明]
  设$a$为$R$的左零因子,则存在一个$b\in R$,$b\neq 0$,使得$ab=0$。以下用反证法证明$a$没有逆元素。假设$a$有逆元素,
  则$a^{-1}(ab)=a^{-1}0=0$,即$(a^{-1}a)b=b=0$,与$b\neq 0$矛盾。同理可证$a$的右零因子也没有逆元素。
\end{proof}
\begin{Exercise}
  在交换环中二项式定理
\[(a+b)^n=a^n+\binom{n}{1}a^{n-1}b+\binom{n}{2}a^{n-2}b^2+\cdots+\binom{n}{n-1}ab^{n-1}+b^n\]
  成立。
\end{Exercise}
\begin{proof}[证明]
用数学归纳法证明,施归纳于$n$。

当$n=1$时,结论显然成立。

假设当$n=k$时结论成立,往证当$n=k+1$时结论也成立。
\begin{align*}
  &(a+b)^{k+1}\\
  =&(a+b)^k(a+b)\\
  =&(a^k+\binom{k}{1}a^{k-1}b+\binom{k}{2}a^{k-2}b^2+\cdots+\binom{k}{k-1}ab^{k-1}+b^k)(a+b)\\
  =&a^{(k+1)}+(\binom{k}{0}+\binom{k}{1})a^{(k+1)-1}b+(\binom{k}{1}+\binom{k}{2})a^{(k+1)-2}b^2+\cdots+(\binom{k}{k-1}+\binom{k}{k})ab^{(k+1)-1}+b^{(k+1)}\\
  =&a^{(k+1)}+\binom{k+1}{1}a^{(k+1)-1}b+\binom{k+1}{2}a^{(k+1)-2}b^2+\cdots+\binom{k+1}{(k+1)-1}ab^{(k+1)-1}+b^{(k+1)}\\
\end{align*}
\end{proof}

\begin{Exercise}
  给出一个环$R$的例子,环$R$中有单位元,但$R$的某个子环的单位元与$R$的单位元不同。
\end{Exercise}
\begin{proof}[证明]
  在环$Z_6=\{[0],[1],[2],[3],[4],[5]\}$中,$S=\{[0],[2],[4]\}$对于$Z_6$中的加法和乘法构成$Z_6$的子环,$Z_6$的单位元为$[1]$,$S$的单位元为$[4]$。
\end{proof}
\begin{Exercise}
证明:对于一个有单位元的环,加法的交换律是环的定义里其他条件的结果。
\end{Exercise}
\begin{proof}[证明]
  设$1$为环$R$的单位元。$\forall a,b\in R$,$(a+b)-(b+a)=(a+b)-1\cdot(b+a)=(a+b)+(-1)(b+a)=(1a+1b)+((-1)b+(-1)a)=1a+(1+(-1))b+(-1)a=1a+0+(-1)a=1a+(-1)a=(1+(-1))a=0a=0$,故$a+b=b+a$,即$R$中的加法满足交换律。
\end{proof}

\begin{Exercise}
  设$Q(\sqrt[3]{2},\sqrt[3]{4})=\{a+b\sqrt[3]{2}+c\sqrt[3]{4}|a,b,c\in Q\}$,其中$Q$为全体有理数之集合。试证:$Q(\sqrt[3]{2},\sqrt[3]{4})$对数的通常加法和乘法构成一个域。  
\end{Exercise}
$\forall a_1,b_1,c_1,a_2,b_2,c_2\in Q$,$(a_1+b_1\sqrt[3]{2}+c_1\sqrt[3]{4})+(a_2+b_2\sqrt[3]{2}+c_2\sqrt[3]{4})=(a_1+a_2)+(b_1+b_2)\sqrt[3]{2}+(c_1+c_2)\sqrt[3]{4}$,$(a_1+b_1\sqrt[3]{2}+c_1\sqrt[3]{4})(a_2+b_2\sqrt[3]{2}+c_2\sqrt[3]{4})=(a_1a_2+2b_1c_2+2c_1b_2)+(a_1b_2+b_1a_2+2c_1c_2)\sqrt[3]{2}+(a_1c_2+b_1b_2+c_1a_2)\sqrt[3]{4}$,
  这验证了加法和乘法满足封闭性。

  加法和乘法的结合律显然成立。

  加法的单位元为$0+0\sqrt{2}=0$;乘法的单位元为$1+0\sqrt[3]{2}+0\sqrt[3]{4}=1$。

  $\forall m,n\in Q$,$m+n\sqrt{2}$对加法的逆元为$(-m)+(-n)\sqrt{2}$。


  $\forall a,b,c\in Z$,$a,b,c$不全为$0$,以下计算$a+b\sqrt[3]{2}+c\sqrt[3]{4}$对乘法的逆元。

  设$x^3-1=(cx^2+bx+a)(px+q)+r$,这里$p,q,r$为有理数,则$(\sqrt[3]{2})^3-1=(c(\sqrt[3]{2})^2+b\sqrt[3]{2}+a)(p\sqrt[3]{2}+q)+r$,于是$(a+b\sqrt[3]{2}+c\sqrt[3]{4})\frac{p\sqrt[3]{2}+q}{1-r}=1$,$a+b\sqrt[3]{2}+c\sqrt[3]{4}$的逆元为$\frac{p\sqrt[3]{2}+q}{1-r}$。

  乘法对加法的分配律显然成立。
\begin{Exercise}
  环$R$称为布尔环,如果$R$中任一非零元都是幂等的,即$\forall a\in R, a^2=a$。试证:

  (1)$\forall a\in R, a+a =0$;

  (2)布尔环是交换环;

  (3)给出一个布尔环的例子。
\end{Exercise}

\begin{proof}[(1)证明]
  $\forall a,b\in R$,$(a+b)^2=(a+b)(a+b)=a+b$,即$a^2+ab+ba+b^2=a+b$,由$a^2=a$和$b^2=b$可得$a+ab+ba+b=a+b$,于是$ab+ba=0$。

  当$b=a$时,$a^2+a^2=0$,即$a+a=0$。
\end{proof}

\begin{proof}[(2)证明]
  由(1)的证明过程知$\forall a,b\in R, ab+ba=0$。因为$a+a=0$,所以$a=-a$。于是,$ab=-ba=b(-a)=ba$。
\end{proof}

\begin{proof}[(3)]
  设$S$为全集,则$(2^S,\bigtriangleup,\cap)$为一个布尔环。

  $(\{T,F\},\oplus, \land)$为一个布尔环,同时也为一个域。
这里,$T\oplus T= F, T\oplus F = T, F\oplus T=T, F\oplus F=F$。
\end{proof}

\end{CJK*}
\end{document}





%%% Local Variables:
%%% mode: latex
%%% TeX-master: t
%%% End:



