\documentclass{article}
\usepackage{CJKutf8}
\usepackage{amsmath}
\usepackage{amssymb}
\usepackage{amsfonts}
\usepackage{amsthm}
\usepackage{titlesec}
\usepackage{titletoc}
\usepackage{xCJKnumb}
\usepackage{tikz}
\usepackage{mathrsfs}
\usepackage{indentfirst}

\newtheorem{Def}{定义}
\newtheorem{Thm}{定理}
\newtheorem{Exercise}{练习}

\newtheorem*{Example}{例}


\begin{document}
\begin{CJK*}{UTF8}{gbsn}
  \title{第十讲 环的定义及简单性质}
  \author{陈建文}
  \maketitle
  % \tableofcontents
  \begin{Def}
    设$R$为一个非空集合,$R$中有两个代数运算,一个叫做加法并用“$+$”表示,另一个叫做乘法并用“$\circ$”表示,如果
  
    (1)$(R,+)$为一个$Abel$群:
  
    I.$\forall a,b,c\in R(a\circ b)\circ c=a\circ (b\circ c)$;
  
    II.$\exists 0\in R \forall a\in R 0+a=a+0=a$;
  
    III.$\forall a\in R \exists b\in R b+a = a+b =0$,$a$的逆元记为$-a$;
  
    IIII.$\forall a,b\in R a+b=b+a$。
  
    (2)$(R,\circ)$为一个半群;$\forall a,b,c\in R (a\circ b)\circ c=a\circ (b\circ c)$
  
    (3)乘法对加法满足左、右分配律:$\forall a,b,c\in R$
  \begin{align*}
    a\circ(b+c)=(a\circ b)+(a\circ c)\\
    (b+c)\circ a=(b\circ a) + (c\circ a)
  \end{align*}
  则称代数系$(R,+,\circ)$为一个环(ring)。
  \end{Def}
  在环$R$中,$a\circ b$简写为$ab$。
  \begin{Example}
    整数集合$Z$对通常数的加法和乘法构成一个环$(R,+,\cdot)$,称为整数环。
  \end{Example}
  
  \begin{Example}
    文字$x$的整系数多项式之集$Z[x]$对多项式的加法和乘法构成一个环。
  \end{Example}
  

  \begin{Def}
  环$(R,+,\circ)$称为交换环或可换环,如果其中的乘法满足交换律,即$\forall a,b\in R$,$ab=ba$。
  \end{Def}
  
  \begin{Example}
    设$M_n$为一切$n\times n$实矩阵之集,则$M_n$对矩阵的加法和乘法构成一个非交换环$(M_n,+,\cdot)$,称为$n$阶矩阵环。
  \end{Example}
  
  \begin{Def}
    如果环$R$的乘法有单位元,则环$R$对于乘法运算的单位元简称为环$R$的单位元。
  \end{Def}

  \begin{Example}
整数环$(Z,+,\cdot)$有单位元$1$。偶数集合对于通常数的乘法和加法构成一个没有单位元的环。
  \end{Example}

  \begin{Def}
    环$(R,+,\circ)$称为有限环,如果$R$为有限非空的集合。
  \end{Def}
  \begin{Example}
    令$S=\{0\}$,则$S$对数的通常加法和乘法构成一个环,称为零环,它仅有一个元素。
  \end{Example}
  \begin{Example}
    全体整数集$Z$对模$n$同余类之集$Z_n=\{[0],[1],\cdots,[n-1]\}$(n为正整数),对其上定义的同余类加法和同余类乘法构成一个环。
    同余类加法定义为
    \[[i]+[j]=[i+j]\]
    同余类乘法定义为
    \[[i]\cdot [j]=[i\cdot j]\]
  
    $\forall i,j,i',j'\in Z$,如果$[i]=[i']$,$[j]=[j']$,则$[ij]=[i'j']$,这验证了“$\cdot$”为一个运算。
  
    $\forall i,j,k\in Z$,验证$([i]\cdot [j])\cdot [k]=[i]\cdot ([j]\cdot [k])$:$([i]\cdot [j])\cdot [k]=[ij]\cdot [k]=[(ij)k]$,$[i]\cdot ([j]\cdot [k])=[i]\cdot [jk]=[i(jk)]$。
  
  $\forall i,j,k\in Z$,验证$[i]\cdot ([j]+[k])=[i]\cdot [j] + [i]\cdot [k]$:$[i]\cdot([j]+[k])=[i]\cdot[j+k]=[i(j+k)]$,$[i]\cdot [j]+[i]\cdot [k]=[ij]+[ik]=[ij+ik]$。
  
  $\forall i,j,k\in Z$,验证$([j]+[k])\cdot [i]=[j]\cdot [i]+[j]\cdot [k]$:$([j]+[k])\cdot [i]=[j+k]\cdot[i]=[(j+k)\cdot i],[j]\cdot [i]+[j]\cdot [k]=[j\cdot i]+[j\cdot k]=[j\cdot i+j\cdot k]$。
  
  \end{Example}
  \begin{Def}
    设$(R,+,\circ)$为一个环,$\forall a,b\in R,a-b$定义为$a+(-b)$。
  \end{Def}
  \begin{Thm}
    设$(R,+,\circ)$为一个环,$\forall a,b,c\in R$,
  
    1. $-(a+b)=-a-b$
  
    这是因为$(-a-b)+(a+b)=((-a)+(-b))+(a+b)=((-a)+a)+((-b)+b)=0+0=0$。
  
    2. $0\circ a = a\circ 0 = 0$
  
    这是因为$0\circ a = (0+0)\circ a=0\circ a+0\circ a$,两边同时加上$0\circ a$的逆元得$0=0\circ a$;同理,$a\circ 0=a\circ (0+0)=a\circ 0 + a\circ 0$,两边同时加上$a\circ 0$的逆元得$a\circ 0=0$。
  
    3. $(-a)b = -(ab)$,$a(-b)=-(ab)$
  
    这是因为$(-a)b+ab=((-a)+a)b=0b=0$,$a(-b)+ab=a((-b)+b)=a0=a$。
  
  
    4. $(-a)(-b)=ab$
  
    这是因为$(-a)(-b)=-(a(-b))=-(-(ab))=ab$。
  
    5. $a(b-c)=ab-ac$
  
    这是因为$a(b-c)=a(b+(-c))=ab+a(-c)=ab+(-ac)=ab-ac$。
  \end{Thm}
  
  \begin{Def}
    在环$(R,+,\circ)$中,$\forall a\in R$,定义$0a=0$,$(n+1)a=na+a(n\geq 0)$,$(-n)a=n(-a)(n\geq 1)$。
  \end{Def}
  
  \begin{Thm}
    设$(R,+,\circ)$为一个环,$\forall a,b \in R$,$m, n \in Z$,
  
    1. $n(-a)=-(na)$
  
    2. $(m+n)a=ma + na$
  
    3. $m(na)=(mn)a$
  
    4. $m(a+b) = ma + mb$
  
    5. $n(a-b) = na - nb$
  
    这是因为$n(a-b)=n(a+(-b))=na+n(-b)=na+(-(nb))=na-nb$。
  
    6. $(na)b=a(nb)=n(ab)$
  
    
  \end{Thm}
  
  \begin{Def}
    在环$(R,+,\circ)$中,$\forall a\in R$,定义
    $a^1=a$,$a^{m+1}=a^m\circ a(m\geq 1)$。
  \end{Def}
  
  \begin{Thm}
    设$(R,+,\circ)$为一个环,$\forall a,b \in R$,$m, n \in Z^+$,
  
    1. $a^{m+n}=a^m \circ a^n$
  
    2. $(a^{m})^n=a^{mn}$
  
    3. 如果$ab=ba$,则二项式定理成立,即当$n>0$时
  
    \[(a+b)^n=\sum_{i=0}^{n}\binom{n}{i}a^ib^{n-i}\]
  \end{Thm}
  
  \begin{Example}
    在环$(M_2,+,\cdot)$中,$\begin{bmatrix}
      1&0\\
      0&0\\
    \end{bmatrix}$
    和$\begin{bmatrix}
      0&0\\
      0&1\\
    \end{bmatrix}$
    是$M_2$中的两个非零元素,但是\[\begin{bmatrix}
      1&0\\
      0&0\\
    \end{bmatrix}\begin{bmatrix}
      0&0\\
      0&1\\
    \end{bmatrix}=\begin{bmatrix}
      0&0\\
      0&0\\
    \end{bmatrix}\]
  \end{Example}
  
  \begin{Def}
    设$(R,+,\circ)$为一个环,$a\in R$,如果存在一个元素$b\in R$,$b\neq 0$,使得$ab=0$,则称$a$为$R$的一个左零因子;如果存在一个元素$c\in R$,$c\neq 0$,使得$ca=0$,则称$a$为$R$的一个右零因子;如果$a$既是$R$的左零因子,又是$R$的右零因子,则称$a$为$R$的零因子。
  \end{Def}
  
  \begin{Def}
    没有非零的左零因子,也没有非零的右零因子的环称为无零因子环。可换的无零因子环称为整环。
  \end{Def}
  
  \begin{Thm}
    环$R$为无零因子环的充分必要条件是$\forall a,b \in R$,如果$a\neq 0$并且$b\neq 0$,则$ab\neq 0$。
  \end{Thm}
  \begin{proof}[证明]
    环$R$不是无零因子环等价于$\exists a,b\in R, a\neq 0\land b\neq 0\land ab=0$,等价于$\lnot (\forall a,b\in R,a\neq 0\land b\neq 0 \to ab\neq 0)$。
  \end{proof}
  \begin{Thm}
    环$R$为无零因子环的充分必要条件是$\forall a,b\in R$,如果$ab=0$,则$a=0$或者$b=0$。
  \end{Thm}
  \begin{Thm}
    环$R$为无零因子环的充分必要条件是在$R$中乘法满足左消去律或右消去律,即
  
    $\forall a,b,c\in R$,如果$a\neq 0$,$ab=ac$,则$b=c$;
  
    或者
  
    $\forall a,b,c\in R$,如果$a\neq 0$,$ba=ca$,则$b=c$。
  
  \end{Thm}
  \begin{proof}[证明]
    由$R$为无零因子环,往证在$R$中乘法满足左消去律。
  
    $\forall a,b,c\in R$,如果$a\neq 0$,$ab=ac$,则$a(b-c)=0$(这是因为$ab+(-(ac))=0$,从而$ab+a(-c)=0$,于是$a(b+(-c))=0$),由$R$为无零因子环知$b-c=0$,因此$b=c$。
  
    由在环$R$中乘法满足左消去律,往证$R$为无零因子环。
  
  
    $\forall a,b\in R$,如果$a\neq 0$并且$b\neq 0$,用反证法证明$ab\neq 0$。如果$ab=0$,则$ab=a0$,于是$b=0$,与$b\neq 0$矛盾。
  
    同理可证$R$为无零因子环的充分必要条件是在$R$中乘法满足右消去律。
  \end{proof}
  \begin{Def}
    一个环称为一个体,如果它满足以下两个条件:
  
    (1)它至少含有一个非零元素;
  
    (2)非零元素的全体对乘法构成一个群。
  \end{Def}
  
  \begin{Def}
    如果一个体中的乘法满足交换律,则称之为域。
  \end{Def}
  \begin{Def}
    有理数集$Q$、实数集$R$、复数集$C$对通常的乘法和加法都构成域。
  \end{Def}
  \begin{Thm}
    至少有一个非零元素的无零因子有限环是体。
  \end{Thm}
  \begin{Def}
    仅有有限个元素的体(域)称为有限体(域)。
  \end{Def}
  \begin{Example}
    设$p$为一个素数,则模$p$同余类环$(Z_p,+,\circ)$为一个有限域。
  \end{Example}
  \begin{proof}[证明]
    只需证$Z_p$为无零因子环:$\forall [a],[b]\in Z_p$,如果$[a][b]=[0]$,则$[ab]=[0]$,从而$p|ab$,由$p$为素数知$p|a$或者$p|b$,所以$[a]=[0]$或者$[b]=[0]$。
  \end{proof}
  \begin{Def}
    设$(F,+,\circ)$为一个域,$\forall a,b\in F$,$b$除以$a$的商$\frac{b}{a}$定义为$ba^{-1}$。
  \end{Def}
  \begin{Thm}
    在域$F$中,商有以下性质:
  
    (1)$\forall a,b,c,d\in F, b\neq 0, d\neq 0, ad=bc \Leftrightarrow \frac{a}{b} = \frac{c}{d}$;
  
    (2)$\forall a,b,c,d\in F, b\neq 0, d\neq 0, \frac{a}{b}\circ \frac{c}{d}=\frac{ac}{bd}, \frac{a}{b}\pm\frac{c}{d}=\frac{ad\pm bc}{bd}$;
  
    (3)$\forall a,b,c,d\in F, b\neq 0, d\neq 0,c\neq 0, \frac{\frac{a}{b}}{\frac{c}{d}}=\frac{ad}{bc}$。
  \end{Thm}
  
  \begin{Def}
    设$(R,+,\circ)$为一个环,$S\subseteq R$,如果$S$对$R$的加法和乘法也构成一个环,则称$S$为$R$的一个子环。
  \end{Def}
  
  \begin{Def}
    设$(F,+,\circ)$为一个体(域),$E\subseteq F$,如果$E$对$F$的加法和乘法也构成一个体(域),则称$E$为$F$的一个子体(域)。
  \end{Def}
  \begin{Thm}
    环$R$的非空子集$S$为$R$的一个子环的充分必要条件是:
  
    (1)$\forall a,b\in S, a-b\in S$;
  
    (2)$\forall a,b\in S, ab\in S$。
  
    体(域)$F$的非空子集$E$为$F$的一个子体(子域)的充分必要条件是:
  
    (1)$|E|\geq 2$;
  
    (2)$\forall a,b \in E, a-b\in E$;
  
    (3)$\forall a,b \in E, a\neq 0, b\neq 0, ab^{-1}\in E$。
  \end{Thm}  



课后作业题:
\begin{Exercise}
  设$Z(\sqrt{2})=\{m+n\sqrt{2}|m,n\in Z\}$,其中$Z$为全体整数之集合。试证:$Z(\sqrt{2})$对数的通常加法和乘法构成一个环。
\end{Exercise}
\begin{proof}[证明]
  $\forall m_1,n_1,m_2,n_2\in Z$,$(m_1+n_1\sqrt{2})+(m_2+n_2\sqrt{2})=(m_1+m_2)+(n_1+n_2)\sqrt{2}$,$(m_1+n_1\sqrt{2})(m_2+n_2\sqrt{2})=(m_1m_2+2n_1n_2)+(m_1n_2+m_2n_1)\sqrt{2}$,
  这验证了加法和乘法满足封闭性。

  加法的结合律显然成立。

  加法的单位元为$0+0\sqrt{2}=0$。

  $\forall m,n\in Z$,$m+n\sqrt{2}$对加法的逆元为$(-m)+(-n)\sqrt{2}$。

  乘法的结合律,乘法对加法的分配律显然成立。
\end{proof}
\begin{Exercise}
  设$Q(\sqrt[3]{2})=\{a+b\sqrt[3]{2}|a,b\in Q\}$,其中$Q$为全体有理数之集合。试证:$Q(\sqrt[3]{2})$对数的通常加法和乘法不构成一个环。
\end{Exercise}
\begin{proof}[证明]
  $Q(\sqrt[3]{2})$对乘法不满足封闭性。否则如果$\sqrt[3]{2}\sqrt[3]{2}=a+b\sqrt[3]{2}$,则$\sqrt[3]{4}=a+b\sqrt[3]{2}$,
  从而$2=\sqrt[3]{2}\sqrt[3]{4}=\sqrt[3]{2}(a+b\sqrt[3]{2})=a\sqrt[3]{2}+b\sqrt[3]{4}=a\sqrt[3]{2}+b(a+b\sqrt[3]{2})=ab+(a+b^2)\sqrt[3]{2}$,
  于是$2-ab=(a+b^2)\sqrt[3]{2}$。此时如果$a+b^2=0$,则$2-ab=0$,可得$b^3=-2$,与$b$为有理数矛盾。如果$a+b^2\neq 0$,则$\sqrt[3]{2}=\frac{2-ab}{a+b^2}$,等式的左边是一个无理数,右边是一个有理数,也矛盾。
\end{proof}
\begin{Exercise}
  环$R$如果对于乘法有左单位元,则环$R$对于乘法的左单位元称为它的左单位元;环$R$如果对于乘法有单位元,则环$R$对于乘法的单位元称为它的单位元。设$e$为环$R$的唯一左单位元,试证$e$为$R$的单位元。
\end{Exercise}
\begin{proof}[证明]
  $\forall a,b\in R$,
  \[(ae-a+e)b=(ae)b-ab+eb=ab-ab+b=b\]
  从而$ae-a+e$也为$R$的左单位元。又由于$e$为环$R$的唯一左单位元,从而$ae-a+e=e$,于是$ae=a$,这说明$e$也为$R$的右单位元,从而为$R$的单位元。
\end{proof}
\begin{Exercise}
  环$R$如果对于乘法有单位元,则环$R$对于乘法的单位元称为它的单位元。设$(R,+,\circ)$为一个有单位元$1$的环,如果$R$中的元素$a$,$b$及$ab-1$均有逆元素,试证$a-b^{-1}$及$(a-b^{-1})^{-1}-a^{-1}$也有逆元素,并且
\[((a-b^{-1})^{-1}-a^{-1})^{-1}=aba-a\]
\end{Exercise}
\begin{proof}[证明]
  由$ab-1$可逆及$a$可逆知$(ab-1)a=aba-a$可逆。

  欲证
  \[((a-b^{-1})^{-1}-a^{-1})^{-1}=aba-a\]
  只需证
  \[((a-b^{-1})^{-1}-a^{-1})(aba-a)=1\]
  只需证
  \[(a-b^{-1})^{-1}(aba-a)-ba+1=1\]
  只需证
  \[(a-b^{-1})^{-1}(aba-a)=ba\]
  只需证
  \[(a-b^{-1})(ba)=aba-a\]
  如果$a-b^{-1}$可逆,该等式显然成立。

  以上证明了$(a-b^{-1})^{-1}-a^{-1}$为$aba-a$的左逆元,又由于$aba-a$可逆,因此$(a-b^{-1})^{-1}-a^{-1}$为$aba-a$的逆元。


  我们还需要证明$a-b^{-1}$可逆。

  由
  \[(a-b^{-1})(ba)=aba-a\]
  可得
  \[(a-b^{-1})=(ab-1)b^{-1}\]
这里$ab-1$和$b^{-1}$均可逆,从而$a-b^{-1}$可逆。
\end{proof}
\begin{Exercise}
  环$R$如果对于乘法有单位元,则环$R$对于乘法的单位元称为它的单位元。试证:有单位元素的环$R$中零因子没有逆元素。
\end{Exercise}
\begin{proof}[证明]
  设$a$为$R$的左零因子,则存在一个$b\in R$,$b\neq 0$,使得$ab=0$。以下用反证法证明$a$没有逆元素。假设$a$有逆元素,
  则$a^{-1}(ab)=a^{-1}0=0$,即$(a^{-1}a)b=b=0$,与$b\neq 0$矛盾。同理可证$a$的右零因子也没有逆元素。
\end{proof}
\begin{Exercise}
  在交换环中二项式定理
\[(a+b)^n=a^n+\binom{n}{1}a^{n-1}b+\binom{n}{2}a^{n-2}b^2+\cdots+\binom{n}{n-1}ab^{n-1}+b^n\]
  成立。
\end{Exercise}
\begin{proof}[证明]
用数学归纳法证明,施归纳于$n$。

当$n=1$时,结论显然成立。

假设当$n=k$时结论成立,往证当$n=k+1$时结论也成立。
\begin{align*}
  &(a+b)^{k+1}\\
  =&(a+b)^k(a+b)\\
  =&(a^k+\binom{k}{1}a^{k-1}b+\binom{k}{2}a^{k-2}b^2+\cdots+\binom{k}{k-1}ab^{k-1}+b^k)(a+b)\\
  =&a^{(k+1)}+(\binom{k}{0}+\binom{k}{1})a^{(k+1)-1}b+(\binom{k}{1}+\binom{k}{2})a^{(k+1)-2}b^2+\cdots+(\binom{k}{k-1}+\binom{k}{k})ab^{(k+1)-1}+b^{(k+1)}\\
  =&a^{(k+1)}+\binom{k+1}{1}a^{(k+1)-1}b+\binom{k+1}{2}a^{(k+1)-2}b^2+\cdots+\binom{k+1}{(k+1)-1}ab^{(k+1)-1}+b^{(k+1)}\\
\end{align*}
\end{proof}

\begin{Exercise}
  给出一个环$R$的例子,环$R$中有单位元,但$R$的某个子环的单位元与$R$的单位元不同。
\end{Exercise}
\begin{proof}[证明]
  在环$Z_6=\{[0],[1],[2],[3],[4],[5]\}$中,$S=\{[0],[2],[4]\}$对于$Z_6$中的加法和乘法构成$Z_6$的子环,$Z_6$的单位元为$[1]$,$S$的单位元为$[4]$。
\end{proof}
\begin{Exercise}
证明:对于一个有单位元的环,加法的交换律是环的定义里其他条件的结果。
\end{Exercise}
\begin{proof}[证明]
  设$1$为环$R$的单位元。$\forall a,b\in R$,$(a+b)-(b+a)=(a+b)-1\cdot(b+a)=(a+b)+(-1)(b+a)=(1a+1b)+((-1)b+(-1)a)=1a+(1+(-1))b+(-1)a=1a+0+(-1)a=1a+(-1)a=(1+(-1))a=0a=0$,故$a+b=b+a$,即$R$中的加法满足交换律。
\end{proof}

\begin{Exercise}
  设$Q(\sqrt[3]{2},\sqrt[3]{4})=\{a+b\sqrt[3]{2}+c\sqrt[3]{4}|a,b,c\in Q\}$,其中$Q$为全体有理数之集合。试证:$Q(\sqrt[3]{2},\sqrt[3]{4})$对数的通常加法和乘法构成一个域。  
\end{Exercise}
$\forall a_1,b_1,c_1,a_2,b_2,c_2\in Q$,$(a_1+b_1\sqrt[3]{2}+c_1\sqrt[3]{4})+(a_2+b_2\sqrt[3]{2}+c_2\sqrt[3]{4})=(a_1+a_2)+(b_1+b_2)\sqrt[3]{2}+(c_1+c_2)\sqrt[3]{4}$,$(a_1+b_1\sqrt[3]{2}+c_1\sqrt[3]{4})(a_2+b_2\sqrt[3]{2}+c_2\sqrt[3]{4})=(a_1a_2+2b_1c_2+2c_1b_2)+(a_1b_2+b_1a_2+2c_1c_2)\sqrt[3]{2}+(a_1c_2+b_1b_2+c_1a_2)\sqrt[3]{4}$,
  这验证了加法和乘法满足封闭性。

  加法和乘法的结合律显然成立。

  加法的单位元为$0+0\sqrt{2}=0$;乘法的单位元为$1+0\sqrt[3]{2}+0\sqrt[3]{4}=1$。

  $\forall m,n\in Q$,$m+n\sqrt{2}$对加法的逆元为$(-m)+(-n)\sqrt{2}$。


  $\forall a,b,c\in Z$,$a,b,c$不全为$0$,以下计算$a+b\sqrt[3]{2}+c\sqrt[3]{4}$对乘法的逆元。

  设$x^3-1=(cx^2+bx+a)(px+q)+r$,这里$p,q,r$为有理数,则$(\sqrt[3]{2})^3-1=(c(\sqrt[3]{2})^2+b\sqrt[3]{2}+a)(p\sqrt[3]{2}+q)+r$,于是$(a+b\sqrt[3]{2}+c\sqrt[3]{4})\frac{p\sqrt[3]{2}+q}{1-r}=1$,$a+b\sqrt[3]{2}+c\sqrt[3]{4}$的逆元为$\frac{p\sqrt[3]{2}+q}{1-r}$。

  乘法对加法的分配律显然成立。
\begin{Exercise}
  环$R$称为布尔环,如果$R$中任一非零元都是幂等的,即$\forall a\in R, a^2=a$。试证:

  (1)$\forall a\in R, a+a =0$;

  (2)布尔环是交换环;

  (3)给出一个布尔环的例子。
\end{Exercise}

\begin{proof}[(1)证明]
  $\forall a,b\in R$,$(a+b)^2=(a+b)(a+b)=a+b$,即$a^2+ab+ba+b^2=a+b$,由$a^2=a$和$b^2=b$可得$a+ab+ba+b=a+b$,于是$ab+ba=0$。

  当$b=a$时,$a^2+a^2=0$,即$a+a=0$。
\end{proof}

\begin{proof}[(2)证明]
  由(1)的证明过程知$\forall a,b\in R, ab+ba=0$。因为$a+a=0$,所以$a=-a$。于是,$ab=-ba=b(-a)=ba$。
\end{proof}

\begin{proof}[(3)]
  设$S$为全集,则$(2^S,\bigtriangleup,\cap)$为一个布尔环。

  $(\{T,F\},\oplus, \land)$为一个布尔环,同时也为一个域。
这里,$T\oplus T= F, T\oplus F = T, F\oplus T=T, F\oplus F=F$。
\end{proof}

\end{CJK*}
\end{document}





%%% Local Variables:
%%% mode: latex
%%% TeX-master: t
%%% End:



