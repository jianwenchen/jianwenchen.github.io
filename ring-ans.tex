\documentclass{article}
\usepackage{CJKutf8}
\usepackage{amsmath}
\usepackage{amssymb}
\usepackage{amsfonts}
\usepackage{amsthm}
\usepackage{titlesec}
\usepackage{titletoc}
\usepackage{xCJKnumb}
\usepackage{tikz}
\usepackage{mathrsfs}
\usepackage{indentfirst}

\newtheorem{Def}{定义}
\newtheorem{Thm}{定理}
\newtheorem{Exercise}{练习}

\newtheorem*{Example}{例}


\begin{document}
\begin{CJK*}{UTF8}{gbsn}
  \title{第九讲 环的定义及简单性质}
  \author{陈建文}
  \maketitle
  % \tableofcontents
  



课后作业题:
\begin{Exercise}
  设$Z(\sqrt{2})=\{m+n\sqrt{2}|m,n\in Z\}$,其中$Z$为全体整数之集合。试证:$Z(\sqrt{2})$对数的通常加法和乘法构成一个环。
\end{Exercise}
\begin{proof}[证明]
  $\forall m_1,n_1,m_2,n_2\in Z$,$(m_1+n_1\sqrt{2})+(m_2+n_2\sqrt{2})=(m_1+m_2)+(n_1+n_2)\sqrt{2}$,$(m_1+n_1\sqrt{2})(m_2+n_2\sqrt{2})=(m_1m_2+2n_1n_2)+(m_1n_2+m_2n_1)\sqrt{2}$,
  这验证了加法和乘法满足封闭性。

  加法的结合律显然成立。

  加法的单位元为$0+0\sqrt{2}=0$。

  $\forall m,n\in Z$,$m+n\sqrt{2}$对加法的逆元为$(-m)+(-n)\sqrt{2}$。

  乘法的结合律,乘法对加法的分配律显然成立。
\end{proof}
\begin{Exercise}
  设$Q(\sqrt[3]{2})=\{a+b\sqrt[3]{2}|a,b\in Q\}$,其中$Q$为全体有理数之集合。试证:$Q(\sqrt[3]{2})$对数的通常加法和乘法不构成一个环。
\end{Exercise}
\begin{proof}[证明]
  $Q(\sqrt[3]{2})$对乘法不满足封闭性。否则如果$\sqrt[3]{2}\sqrt[3]{2}=a+b\sqrt[3]{2}$,则$\sqrt[3]{4}=a+b\sqrt[3]{2}$,
  从而$2=\sqrt[3]{2}\sqrt[3]{4}=\sqrt[3]{2}(a+b\sqrt[3]{2})=a\sqrt[3]{2}+b\sqrt[3]{4}=a\sqrt[3]{2}+b(a+b\sqrt[3]{2})=ab+(a+b^2)\sqrt[3]{2}$,
  于是$\sqrt[3]{2}=\frac{2-ab}{a+b^2}$。等式的右边是一个有理数,左边是一个无理数,矛盾。
\end{proof}
\begin{Exercise}
设$e$为环$R$的唯一左单位元,试证$e$为$R$的单位元。
\end{Exercise}
\begin{proof}[证明]
  $\forall a,b\in R$,
  \[(ae-a+e)b=(ae)b-ab+eb=ab-ab+b=b\]
  从而$ae-a+e$也为$R$的左单位元,又由于$e$为环$R$的唯一左单位元,从而$ae-a+e=e$,于是$ae=a$,这说明$e$也为$R$的右单位元,从而为$R$的单位元。
\end{proof}
\begin{Exercise}
设$(R,+,\circ)$为一个有单位元$1$的环,如果$R$中的元素$a$,$b$及$ab-1$均有逆元素,试证$a-b^{-1}$及$(a-b^{-1})^{-1}-a^{-1}$也有逆元素,并且
\[((a-b^{-1})^{-1}-a^{-1})^{-1}=aba-a\]
\end{Exercise}
\begin{proof}[证明]
  欲证
  \[((a-b^{-1})^{-1}-a^{-1})^{-1}=aba-a\]
  只需证
  \[((a-b^{-1})^{-1}-a^{-1})(aba-a)=1\]
  只需证
  \[(a-b^{-1})^{-1}(aba-a)-ba+1=1\]
  只需证
  \[(a-b^{-1})^{-1}(aba-a)=ba\]
  只需证
  \[(a-b^{-1})(ba)=aba-a\]
  该等式显然成立。

  我们还需要证明$a-b^{-1}$可逆。

  在
  \[(a-b^{-1})(ba)=aba-a\]
  的启发下,
  计算
  \[(a-b^{-1})(b(ab-1)^{-1})=1\]
  得$a-b^{-1}$可逆。

\end{proof}
\begin{Exercise}
  有单位元素的环$R$中零因子没有逆元素。
\end{Exercise}
\begin{proof}[证明]
  设$a$为$R$的零因子,则存在一个$b\in R$,$b\neq 0$,使得$ab=0$。以下用反证法证明$a$没有逆元素。假设$a$有逆元素,
  则$a^{-1}(ab)=a^{-1}0=0$,即$(a^{-1}a)b=b=0$,与$b\neq 0$矛盾。
\end{proof}
\begin{Exercise}
  在交换环中二项式定理
\[(a+b)^n=a^n+\binom{n}{1}a^{n-1}b+\binom{n}{2}a^{n-2}b^2+\cdots+\binom{n}{n-1}ab^{n-1}+b^n\]
  成立。
\end{Exercise}
\begin{proof}[证明]
用数学归纳法证明,施归纳于$n$。

当$n=1$时,结论显然成立。

假设当$n=k$时结论成立,往证当$n=k+1$时结论也成立。
\begin{align*}
  &(a+b)^{k+1}\\
  =&(a+b)^k(a+b)\\
  =&(a^k+\binom{k}{1}a^{k-1}b+\binom{k}{2}a^{k-2}b^2+\cdots+\binom{k}{k-1}ab^{k-1}+b^k)(a+b)\\
  =&a^{(k+1)}+(\binom{k}{0}+\binom{k}{1})a^{(k+1)-1}b+(\binom{k}{1}+\binom{k}{2})a^{(k+1)-2}b^2+\cdots+(\binom{k}{k-1}+\binom{k}{k})ab^{(k+1)-1}+b^{(k+1)}\\
  =&a^{(k+1)}+\binom{k+1}{1}a^{(k+1)-1}b+\binom{k+1}{2}a^{(k+1)-2}b^2+\cdots+\binom{k+1}{(k+1)-1}ab^{(k+1)-1}+b^{(k+1)}\\
\end{align*}
\end{proof}
\end{CJK*}
\end{document}





%%% Local Variables:
%%% mode: latex
%%% TeX-master: t
%%% End:



